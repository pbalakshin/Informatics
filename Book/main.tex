\documentclass[14pt, fleqn]{extarticle}

\usepackage[T1,T2A]{fontenc}
\usepackage[utf8]{inputenc}
\usepackage[english,russian]{babel}
% \usepackage[left=3cm,right=2.5cm,top=2cm,bottom=2cm]{geometry}
\usepackage[left=2.5cm,right=2.5cm,top=2.5cm,bottom=2.5cm]{geometry}
\usepackage{graphicx}
\usepackage{indentfirst} %indent first par

% Шрифт Times в тексте как основной
\usepackage{tempora}
% Шрифт Times в формулах как основной
\usepackage[varg, cmbraces, cmintegrals]{newtxmath}

\usepackage[breaklinks]{hyperref}
\usepackage{xurl}

\usepackage{xr} % cross-file references
\usepackage{minted} % code listings

\usepackage{csquotes}
\usepackage[font=normalsize,labelfont=it,textfont=it,justification=centering]{caption}
\usepackage{subcaption}
\usepackage{float} %for figure position
\usepackage{amsmath} %for math equations
\usepackage{lastpage}
\usepackage{ulem} %зачёркнутый текст
\usepackage{svg}
\usepackage{wrapfig}
\usepackage{iflang}

\usepackage{pgfplots}
\pgfplotsset{compat=1.9}

% \usepackage{siunitx} % for B/kB/MB etc.

\usepackage{tikz}
\usetikzlibrary{graphs}

\usepackage{todonotes}
\presetkeys{todonotes}{inline,backgroundcolor=yellow}{}

\usepackage{enumitem} %for list customization
% \setlist[itemize]{leftmargin=2cm,labelsep=0.5cm}
% \setlist[enumerate]{leftmargin=1.5cm}

\usepackage{longtable}
\usepackage{tabularx}

\usepackage{makecell}
\renewcommand\theadfont{\normalsize\bfseries}

\newcolumntype{C}{>{\centering\arraybackslash}X}
\renewcommand\tabularxcolumn[1]{m{#1}}% for vertical centering text in X column
\usepackage{multirow}

%% Многостраничные таблицы в гостовском формате. Аргументы:
%% 1 - имя, по которому ссылаемся
%% 2 - подпись
%% 3 - строка форматирования столбцов (как в tabular)
%% 4 - заголовки столбцов (первая строчка, которая будет везде повторяться)
%% 5 - число столбцов (не осилил вычислить его из значения 3)

\newenvironment{nirtable}[5]{
    \begin{longtable}{#3}
        \caption{#2}\label{#1}
        \\\hline #5 \\\hline \endfirsthead
        \multicolumn{#4}{r}{\normalsize{Продолжение таблицы~\thetable}} \\\hline #5 \\\hline \endhead
}{\end{longtable}}

% \newcommand{\innesr}[2][]{\renewcommand*{\arraystretch}{1}\begin{tabular}{>{#1}c}#2\end{tabular}}

% footnotes will start from 1 on each page
\usepackage[perpage]{footmisc}

%% Перечисления по умолчанию слишком разрежены.
% \usepackage{enumitem}
% \setlist{nosep}

% Использовать полужирное начертание для векторов
\let\vec=\mathbf

%% Не используем буллеты.
% \renewcommand\labelitemi{---}

% \newcommand{\specialcell}[2][c]{%command \specialcell for auto-carry in cells of table
% \begin{tabular}[#1]{@{}c@{}}#2\end{tabular}}

\usepackage{titlesec}
\titleformat*{\section}{\large\bfseries}
\titleformat*{\subsection}{\normalsize\bfseries}
\titleformat*{\subsubsection}{\normalsize\bfseries}
% \titleformat*{\paragraph}{\normalsize\bfseries}

\titleformat{\paragraph}
{\normalfont\normalsize\bfseries}{\theparagraph}{1em}{}
\titlespacing*{\paragraph}
{0pt}{3.25ex plus 1ex minus .2ex}{1.5ex plus .2ex}

\titleformat*{\subparagraph}{\normalsize\bfseries}

\usepackage{listings} %listings
\usepackage{color} %colors for listings


%% "Тонкая" настройка теховских штрафов при формировании абзацев.
\sloppy
\binoppenalty=10000
\relpenalty=10000
\clubpenalty=10000
\widowpenalty=10000

%% Подписи к рисункам, таблицам, листингам.
\floatstyle{plaintop}

\captionsetup[figure]{
    labelsep=endash,
    singlelinecheck=false,
    labelfont={normalsize,md},
    justification=centering,
    position=bottom
}

\captionsetup[table]{
    labelsep=endash,
    singlelinecheck=false,
    labelfont={normalsize,md},
    justification=justified,
    position=top
}

% \floatsetup[algorithm]{style=plain, capposition=top}
% \captionsetup[algorithm]{
%     labelsep=endash,
%     singlelinecheck=false,
%     labelfont={normalsize,md},
%     justification=justified,
%     position=top
% }

% \floatsetup[lstlisting]{style=plain, capposition=top}
% \captionsetup[lstlisting]{
%     labelsep=endash,
%     singlelinecheck=false,
%     labelfont={normalsize,md},
%     justification=justified,
%     position=top
% }




\usepackage[
    backend=biber,
    bibencoding=utf8,
    language=auto,
    autolang=other,
    movenames=false,
    sorting=ntvy,
    isbn=false,
    style=gost-numeric,
]{biblatex}

\renewcommand*{\mkgostheading}[1]{#1}
\providecommand{\datecircaprint}{}
\providecommand{\dateeraprintpre}[1]{}
\providecommand{\mkyearzeros}{}
\providecommand{\dateeraprint}[1]{}
\providecommand{\dateuncertainprint}{}

\renewbibmacro*{//}{\nopunct\printtext{\addspace\mbox{//}\addnbspace}}
\renewcommand*{\newblockpunct}{\textemdash\addnbspace\bibsentence}
\DeclareFieldFormat*{pages}{\mkpageprefix[bookpagination][\mbox]{#1}}
\DeclareFieldFormat*{labelnumberwidth}{#1}
\DefineBibliographyStrings{english}{pages={p\adddot}}
\DefineBibliographyExtras{russian}{\protected\def\bibrangedash{\textendash\penalty\hyphenpenalty}}

\DeclareSourcemap{
    \maps[datatype=bibtex]{
        \map{
            \step[fieldsource=language, match=russian, final]
            \step[fieldset=presort, fieldvalue={a}]
        }
        \map{
            \step[fieldsource=language, notmatch=russian, final]
            \step[fieldset=presort, fieldvalue={z}]
        }
    }
}

\newcommand{\hiddensection}[1]{
\phantomsection
\section*{#1}
\addcontentsline{toc}{section}{#1}
}

\newcommand{\hiddensubsection}[1]{
\phantomsection
\subsection*{#1}
\addcontentsline{toc}{subsection}{#1}
}

\newcommand{\hiddensubsubsection}[1]{
\phantomsection
\subsubsection*{#1}
\addcontentsline{toc}{subsubsection}{#1}
}

% Сделаем ссылки URL нормальным шрифтомs
\renewcommand\UrlFont{\rmfamily}

% Уберём пробел перед разделяющим двоеточием
\renewcommand*{\addcolondelim}{\addcolon\space}

\DeclareBibliographyDriver{online}{%
    \usebibmacro{bibindex}%
    \usebibmacro{begentry}%
    \usebibmacro{heading}%
    \newunit
    \usebibmacro{author/editor}%
    \setunit*{\labelnamepunct}%
    \usebibmacro{title}%
    \addspace\foreignlanguage{russian}{[Электронный ресурс]}
    \setunit{\addcolondelim}%
    \usebibmacro{translation}%
    \def\bbx@gost@respdelim{\setunit{\respdelim}}% ----- Resp starts -----
    \usebibmacro{byauthor}%
    \setunit*{\resppunct}%
    \iflistundef{organization}
    {}
    {%
        \setrespdelim%
        \printlist{organization}%
        \setunit*{\resppunct}%
    }%
    \usebibmacro{credits}%
    \setunit*{\resppunct}%
    \usebibmacro{byeditor}%
    \setunit*{\resppunct}%
    \usebibmacro{bytranslator+others}%
    \newunit\newblock
    \printfield{version}%
    \newunit\newblock
    \printlist[semicolondelim]{specdata}%
    \newunit\newblock
    \usebibmacro{date}%
    \newunit\newblock
    \printupdate%
    \newunit\newblock
    \printfield{systemreq}%
    \newunit\newblock
    \usebibmacro{doi+eprint+url+note}%
    \newunit\newblock
    \usebibmacro{addendum+pubstate}%
    \setunit{\bibpagerefpunct}\newblock
    \usebibmacro{pageref}%
    \newunit\newblock
    \usebibmacro{related:init}%
    \usebibmacro{related}%
    \usebibmacro{finentry}
}


\addbibresource{bibliography.bib}

\newcommand{\addbibliography}[1]{
    \phantomsection
    \addcontentsline{toc}{section}{#1}
    \printbibliography[title={#1}]
}


\setminted{
    fontsize=\small,
    frame=single,
    encoding=utf8,
    samepage,
    autogobble,
    tabsize=4
}

\setmintedinline{
    fontsize=\normalsize,
    encoding=utf8,
    breaklines,
    breakafter=-/
}

% Disabling italic font in minted
\usepackage{xpatch}
\xpatchcmd{\mint}{\begingroup}{\begingroup\let\itshape\relax}{}{}
\xpatchcmd{\inputminted}{\begingroup}{\begingroup\let\itshape\relax}{}{}
\xpatchcmd{\mintinline}{\begingroup}{\begingroup\let\itshape\relax}{}{}
\xpatchcmd{\minted}{\VerbatimEnvironment}{\VerbatimEnvironment\let\itshape\relax}{}{}

\newcommand{\code}[2][text]{\mintinline{#1}{#2}}

% \setlength{\marginparwidth}{2cm}

%% Закомментить при дебаге и раскомменить при релизе
%% наличие данной команды убирает все блоки туду с документа
% \presetkeys{todonotes}{disable}{}

\providecommand\phantomsection{}

\graphicspath{ {./images/} }

\pagestyle{plain} %for counting pages
\setcounter{page}{2}

%% Абзацный отступ по ГОСТу - пять букв. Это примерно столько.
\setlength{\parindent}{1.25cm}

\newcommand{\example}[1]{\bigskip\noindent\emph{\textbf{Пример #1:}} }
\newcommand{\task}{\noindent\emph{\textit{Задание:}} }
\newcommand{\solution}{\noindent\emph{\textit{Решение:}} }
\newcommand{\answer}{\noindent\emph{\textit{Ответ:}} }
\newcommand{\explain}{\noindent\emph{\textit{Пояснение:}} }

\begin{document}
    \nocite{*}

    % \todo{Если видно это сообщение - раскомментировать строку в main.tex}
    
\thispagestyle{empty}
\begin{center}
    \includegraphics[scale=0.15]{itmo_logo_black_2022}

    \vspace*{\fill}

    \begin{bfseries}
        \begin{Large}
            П.В.~Балакшин, В.В.~Соснин, Е.А.~Машина, А.В.~Мишенёв
        
            \vspace*{\fill}
        
            \centerline{ИНФОРМАТИКА}
        \end{Large}
     
        \vspace*{\fill}
        
        \includegraphics[width=10cm, height=9cm]{_i_Computer}
        
        \vspace*{\fill}
        
        \begin{large}
            Санкт-Петербург \\ \the\year
        \end{large}
    \end{bfseries}   
\end{center}
 %обложка, не идёт в счёт страниц
    \include{firstPage}
    \include{description}

    \setcounter{page}{3}
    \addtocontents{toc}{\protect\sloppy}
    \tableofcontents

    \hiddensection{О курсе}
\label{sec:about_course}

Цель данного методического пособия состоит в изучении общих принципов работы компьютера и получении навыков работы с рядом пакетов. Студентам предлагается рассмотреть для получения базовых знаний и умений по работе с компьютером такие темы как:
\begin{itemize}[noitemsep]
    \item основы теории информации;
    \item сжатие компьютерных данных;
    \item помехоустойчивое кодирование;
    \item архитектура ЭВМ;
    \item организация компьютерных сетей;
    \item работа с офисными пакетами;
    \item программное обеспечение профессионального программиста.
\end{itemize}

Наряду с этими темами, авторы этого пособия предоставляют возможность более детально ознакомиться с рядом современных пакетов, в их числе такие широко известные продукты Microsoft, как Microsoft Office Word и Microsoft Excel, и система компьютерной верстки \TeX, имеющая свой собственный язык разметки. Освоение указанных пакетов позволит студентам получить полезные навыки по подготовке презентаций, научно-технических отчетов о результатах выполненной работы, в оформлении результатов исследований в виде статей и докладов на научно-технических конференциях. Также авторы пособия продемонсрируют не-которые методики использования программных средств для решения практических задач.

    \section{Основы теории информации}
\label{sec:theory}

\subsection{Терминология информатики}

Начать изучение информатики невозможно, не разобравшись в точном значении термина «информатика». Однако, до сих пор в мировой научной общественности не сложилось четкого понимания этого термина. Рассмотрим одно из популярных определений:

\textbf{Информатика} --- дисциплина, изучающая свойства и структуру информации, закономерности ее создания, преобразования, накопления, передачи и использования. За рубежом сложилась чуть более узкая трактовка термина информатика. Там под этим понимают пересечение сразу трех областей науки – это информационные технологии, теория информации и computer science. Всё обозначенное выше подходит под определение самого курса "Информатика".

Изучая некоторую науку важно представлять основные даты, вехи её развития:
\begin{itemize}
\item 1956(57) – появление термина <<информатика>> (\textit{нем.} Informatik, Штейнбух).
\item 1968 – первое упоминание в СССР (информология, Харкевич).
\item 197Х – информатика стала отдельной наукой.
\item 4 декабря – день российской информатики.
\end{itemize}
\subsection{Терминология теории информации}

Рассмотрим некоторые терминологические тонкости. В обыденном языке, слова <<информация>> и <<данные>> считаются синонимами. Они, как правило, употребляются взаимозаменяемо. И так обстоит дело в информатике и в целом, в компьютерных науках.

Понятие \textit{``информация''} имеет различные трактовки в различных предметных областях. Например, \textit{информация} может пониматься как:
\begin{itemize}[noitemsep]
    \item сигналы для управления, приспособления рассматриваемой системы (в кибернетике);
    \item мера хаоса в рассматриваемой системе (в физике);
    \item вероятность выбора в рассматриваемой системе (в теории вероятностей);
    \item мера разнообразия в рассматриваемой системе (в биологии) и др.
\end{itemize}
Но мы остановимся на понятиях, близких к информатике.

\begin{description}
    \item [Информация] --- это некоторая упорядоченная последовательность сообщений, отражающих, передающих и увеличивающих наши знания.
    \item [Информация] --- это сведения об окружающем мире (объекте, процессе, явлении, событии), которые являются объектом преобразования (включая хранение, передачу и т.д.) и используются для выработки поведения, для принятия решения, для управления или для обучения.
    \item [Информация] --- это новые сведения, подлежащие передаче, хранению и обработке.
\end{description}

Рассмотрим это фундаментальное понятие информатики на основе понятия \textit{``алфавит''} (``алфавитный'', формальный подход). Дадим формальное определение \textit{алфавита}.

\begin{description}
    \item [Алфавит] --- конечное множество различных знаков (букв), символов, для которых определена операция \emph{конкатенации} (присоединения символа к символу или цепочке символов); с ее помощью по определенным правилам соединения символов и слов можно получать слова (цепочки знаков) и словосочетания (цепочки \textit{слов}) в этом \textit{алфавите} (над этим \textit{алфавитом}).
    \item [Знак (буква)] --- любой элемент алфавита (элемент $x$ алфавита $X$, где $x \in X$). Понятие знака неразрывно связано с тем, что им обозначается (``со смыслом''), они вместе могут рассматриваться как пара элементов ($x$, $y$), где $x$ – сам знак, а $y$ – обозначаемое этим знаком.
\end{description}

\example{1}
\noindent
Примеры \emph{алфавитов:} множество из десяти цифр, множество из знаков русского языка, точка и тире в азбуке Морзе и др. В \emph{алфавите} цифр знак 5 связан с понятием ``быть в количестве пяти элементов''.

\textbf{Слово} в алфавите (или над алфавитом) - конечная последовательность знаков (букв) алфавита.

\textbf{Длина} |p| некоторого слова $p$ в алфавите (над алфавитом) - число составляющих его букв.

\textbf{Словарь (словарный запас)} - множество различных слов в алфавите (над алфавитом).
В отличие от конечного \emph{алфавита}, словарный запас может быть и бесконечным.
\emph{Слова} над некоторым заданным \emph{алфавитом} и определяют так называемые \emph{сообщения}.

\example{2}
\noindent
\emph{Слова} над \emph{алфавитом} кириллицы --- ``Информатика'',``инто'', ``ииии'', ``и''.

\noindent
\emph{Слова} над \emph{алфавитом} десятичных цифр и знаков арифметических операций --- ``1256'', ``23+78'', ``35–6+89'', ``4''.

\noindent
\emph{Слова} над \emph{алфавитом} азбуки Морзе --- ``.'', ``. . –'', ``– – –''.

В \emph{алфавите} должен быть определен порядок следования \emph{букв} (порядок типа ``предыдущий элемент --- последующий элемент''), то есть любой \emph{алфавит} имеет упорядоченный вид $X = {x_1, x_2, \ldots, x_n}$ .

Таким образом, \emph{алфавит} должен позволять решать задачу лексикографического (алфавитного) упорядочивания, или задачу расположения \emph{слов} над этим \emph{алфавитом}, в соответствии с порядком, определенным в \emph{алфавите} (то есть по символам \emph{алфавита}).

\subsection{Признаки классификации информации}

Рассмотрим две классификации информации. Первая из них --- классификация по форме \emph{сообщений} --- определенного вида сигналов, символов:
\begin{itemize}[noitemsep]
  \item отношение к источнику или приемнику (входная, выходная и внутренняя);
  \item отношение к конечному результату (исходная, промежуточная и результирующая);
  \item актуальность;
  \item адекватность;
  \item доступность (открытая, закрытая);
  \item понятность;
  \item полнота (достаточная, недостаточная, избыточная);
  \item достоверность;
  \item массовость;
  \item изменчивость (постоянная, переменная, смешанная);
  \item объективность;
  \item точность;
  \item стадия использования (первичная, вторичная);
  \item ценность.
\end{itemize}

Вторая классификация --- по форме преставления информации, способам ее кодирования и хранения:
\begin{itemize}[noitemsep]
  \item графическая;
  \item звуковая;
  \item текстовая;
  \item числовая;
  \item видеоинформация.
\end{itemize}

\subsection{Измерение количества информации}

Любые сообщения измеряются в \emph{байтах, килобайтах, мегабайтах, гигабайтах, терабайтах, петабайтах} и \emph{эксабайтах}, а кодируются, например, в компьютере, с помощью \emph{алфавита} из нулей и единиц, записываются и реализуются в ЭВМ в \emph{битах}.

Приведем основные соотношения между единицами измерения \emph{сообщений}:
\begin{itemize}[noitemsep]
    \item 1 бит (\textbf{bi}nary digi\textbf{t} - двоичное число) = 0 или 1;
    \item 1 байт = 8 бит;
    \item 1 килобайт (1 Кб) = $2^{13}$ бит;
    \item 1 мегабайт (1 Мб) = $2^{23}$ бит;
    \item 1 гигабайт (1 Гб) = $2^{33}$ бит;
    \item 1 терабайт (1 Тб) = $2^{43}$ бит;
    \item 1 петабайт (1 Пб) = $2^{53}$ бит;
    \item 1 эксабайт (1 Эб) = $2^{63}$ бит.
\end{itemize}

Теперь нам известно понятие информации, но необходимо еще конкретно знать сколько этой информации. Поэтому есть два важных определения:
\begin{description}
    \item [Количество информации] --- число, адекватно характеризующее разнообразие (структурированность, определённость,выбор состояний и т.д.) в оцениваемой системе. Количество информации часто оценивается в битах, причем такая оценка может выражаться и в долях бит (так как речь идет не об измерении или кодировании сообщений).
    \item [Мера информации] --- численная оценка количества информации, которая обычно задана неотрицательной, определенной на множестве событий и являющейся аддитивной функцией (то есть, мера информации объединения событий (множеств) равна сумме мер каждого события). Заметим, что функция меры информации монотонна (при уменьшении или увеличении вероятности некоторого события количество иноформации в системе монотонно уменьшается или увеличивается). 
    \\\textbf{Важно:} мера вероятности всегда находится в диапазоне от 0 до 1
\end{description}

Для измерения информации используются различные подходы и методы, например, с использованием меры информации по Р. Хартли и К. Шеннону.

\newpage
\subsubsection{Мера Хартли}

\begin{wrapfigure}{l}{0.21\textwidth}
    \centering
    \includegraphics[width=0.2\textwidth]{hartley}
    \caption*{Ральф Хартли\\1888 -- 1970}
\end{wrapfigure}

Пусть известны $N$ состояний системы $S$ ($N$  опытов с различными, равновозможными, последовательными состояниями системы). Если каждое состояние системы закодировать двоичными кодами, то минимальная длина $d$ полученного кода определяется из условия:

$$
2^{d} \ge N \qquad \mbox{\emph{или}} \qquad  d \ge \log_{2}N
$$

Значит, для однозначного описания системы требуется $\log_{2}N$ бит. В общем случае количество информации в системе $S$ равно:
$$
H_{s} = \log_{k}N
$$

Единицы измерения количества информации:
\begin{itemize}[noitemsep]
  \item Бит ($k = 2$)
  \item Трит ($k = 3$)
  \item Дит (харт) ($k = 10$)
  \item Нит (нат) ($k = e$)
\end{itemize}

\paragraph{Примеры использования меры Хартли}

\example{1}

\task мальчик загадывает число от 1 до 64. Какое количество вопросов типа "да-нет" понадобится, чтобы гарантированно угадать число?

\solution

\begin{itemize}[noitemsep]
    \item Первый вопрос: "Загаданное число меньше 32?". Ответ: "Да".
    \item Второй вопрос: "Загаданное число меньше 16?". Ответ: "Нет".
    \item [] \dots
    \item Шестой вопрос точно приведет к правильному ответу.
\end{itemize}

\noindentЗначит, в соответствии с мерой Хартли в загадке мальчика содержится $\log_{2}64 = 6$ бит информации ($N = 64$ так как возможно 64 вариантов загаданного числа).

\answer 6 бит.


\example{2}

\task Мальчик держит за спиной шахматного ферзя и собирается поставить его на произвольную клетку пустой доски. Какое количество информации содержится в его действии?

\solution Шахматная доска имеет размеры $8\times 8$ клеток.
Ферзь может быть как белым, так и черным, поэтому количество равновероятных состояний будет равно $8 \times 8 \times 2 = 128$.
Получается, количество информации по мере Хартли равно $\log_{2}128 = 7$ бит.

\answer 7 бит.

\bigskip

Если во множестве $X = {x_1,x_2, ..., x_n}$ искать произвольный элемент, то для его нахождения (по Хартли) необходимо иметь не менее $\log_{a}n$ (единиц) информации. 

Уменьшение $H$ говорит об уменьшении разнообразия состояний $N$ системы, а увеличение $H$ говорит об увеличении разнообразия состояний $N$ системы.

Мера Хартли подходит лишь для идеальных, абстрактных систем, так как в реальных системах состояния системы неодинаково осуществимы (неравновероятны).
\subsubsection{Мера Шеннона}

\begin{wrapfigure}{l}{0.21\textwidth}
    \centering
    \includegraphics[width=0.2\textwidth]{shannon}
    \caption*{Клод Шеннон\\1916 -- 2001}
\end{wrapfigure}

Если состояния системы не равновероятны, используют меру Шеннона. Мера Шеннона оценивает информацию отвлеченно от ее смысла:
$$I = - \sum^{N}_{i=1}p_{i}\times \log_{2}p_{i},$$ где:
\begin{description}[noitemsep]
    \item [$I$] -- количество информации, выраженное в битах (в $\log_{k}p_{i}$ $k = 2$);
    \item [$N$] --- число состояний системы;
    \item [$p_{i}$] --- вероятность (относительная частота) перехода системы в $i$-е состояние (вероятность того, что система находится в состоянии $i$)
\end{description}

Сумма всех $p_{i}$ должна быть равна единице.

Если все состояния рассматриваемой системы равновозможны, равновероятны, то есть $p_i = 1/n$, то из \emph{формулы Шеннона} можно получить (как частный случай) \emph{формулу Хартли}:
$$I = \log_{2}n.$$

Обозначим величину:
$$f_i = -n\log_{2}p_i.$$

Тогда из \emph{формулы К. Шеннона} следует, что количество информации I можно понимать как среднеарифметическое величин $f_i$ , то есть величину $f_i$ можно интерпретировать как \emph{информационное содержание символа алфавита} с индексом i и величиной $p_i$ вероятности появления этого символа в любом сообщении (слове), передающем информацию.

В термодинамике известен так называемый коэффициент Больцмана $k = 1.38 * 10^{-16} \mbox{(эрг.град)}$ и выражение (\emph{формула Больцмана}) для энтропии или меры хаоса в термодинамической системе:
$$
S = -k \sum^{N}_{i=1}p_{i}\times \ln{p_{i}}
$$

Сравнивая выражения для I и S, можно заключить, что величину I можно понимать как энтропию из-за нехватки информации в системе (о системе).

Формулы энтропии и информации идентичны, но смысл разный. Энтропия априорная характеристика (до передачи), информация – апостериорная (после передачи).

Из этой формулы следуют важные выводы:
\begin{itemize}
    \item увеличение меры Шеннона свидетельствует об уменьшении энтропии (увеличении порядка) системы;
    \item уменьшение меры Шеннона свидетельствует об увеличении энтропии (увеличении беспорядка) системы.
\end{itemize}

Положительная сторона \emph{формулы Шеннона} --- ее отвлеченность от смысла информации. Кроме того, в отличие от \emph{формулы Хартли}, она учитывает различность состояний, что делает ее пригодной для практических вычислений. Основная отрицательная сторона \emph{формулы Шеннона} – она не распознает различные состояния системы с одинаковой вероятностью.

\paragraph{Примеры использования меры Шеннона}

\example{1}

\task девочка наугад вытаскивает из мешка мяч. Известно, что в мешке всего 8 мячей, из них: 4 красных, 2 синих, 1 зеленый и 1 белый. Какое количество информации содержится в этом событии?

\solution
\begin{itemize}
    \item Вероятность вытащить красный мяч равна $\displaystyle\frac{4}{8} = 0,5$
    \item Вероятность вытащить синий мяч равна $\displaystyle\frac{4}{8} = 0,25$
    \item Вероятность вытащить зеленый мяч равна $\displaystyle\frac{1}{8} = 0,125$
    \item Вероятность вытащить белый мяч равна $\displaystyle\frac{1}{8} = 0,125$
\end{itemize}

\noindentЗначит количество информации, выраженное в битах равно:
\begin{flalign*}
I = {} & -( \frac{1}{2} \times \log_{2}\frac{4}{8}  + \frac{1}{4} \times \log_{2}\frac{1}{8} + \frac{1}{8} \times \log_{2}\frac{1}{8} + \frac{1}{8} \times \log_{2}\frac{1}{8})  \\
         & = -(-0,5 \times 1 - 0,25 \times 2 - 0,125 \times 3 - 0,125 \times 3) \\
         & =  -(-0,5 - 0,5 - 0,375 - 0,375) \\
         & = 1,7 \mbox{ бит}
\end{flalign*}
% $I = -(0,5 \times \log_{2}0,5  + 0,25 \times \log_{2}0,25 + 0,125 \times \log_{2}0,125 + 0,125 \times \log_{2}0,125) = -(-0,5\times 1 - 0,25\times 2 - 0,125\times 3 - 0,125\times 3) = -(-0,5 - 0,5\hm - 0,375\hm - 0,375\hm) = 1,75$ бит.

\answer 1,75 бит
\subsection{Методы получения информации}

Методы получения информации можно разбить на три большие группы:
\begin{itemize}
    \item \emph{Эмпирические};
    \item \emph{Теоретические}; 
    \item \emph{Эмпирико-теоретические}.
\end{itemize}

Кратко рассмотрим и охарактеризуем все три метода по отдельности. 

\subsubsection{Эмпирические методы}
Эмпирические методы или методы получения эмпирических данных.
\begin{description}
    \item [Наблюдение] --- сбор первичной информации об объекте, процессе, явлении.
    \item [Сравнение] --- обнаружение и соотнесение общего и различного.
    \item [Измерение] --- поиск с помощью измерительных приборов эмпирических фактов.
    \item [Эксперимент] --- преобразование, рассмотрение объекта, процесса, явления с целью выявления каких-то новых свойств.
\end{description}

\noindent
Кроме классических форм их реализации, в последнее время используются опрос, интервью, тестирование и другие.

\subsubsection{Теоретические методы}
Теоретические методы или методы построения различных теорий.
\begin{description}
    \item [Восхождение от абстрактного к конкретному] --- получение знаний о целом или о его частях на основе знаний об абстрактных проявлениях в сознании, в мышлении.
    \item [Идеализация] --- получение знаний о целом или его частях путем представления в мышлении целого или частей, не существующих в действительности.
    \item [Формализация] --- получение знаний о целом или его частях с помощью языков искусственного происхождения (формальное описание, представление).
    \item [Аксиоматизация] --- получение знаний о целом или его частях с помощью некоторых аксиом (не доказываемых в данной теории утверждений) и правил получения из них (и из ранее полученных утверждений) новых верных утверждений.
    \item [Виртуализация] --- получение знаний о целом или его частях с помощью искусственной среды, ситуации.
\end{description}

\subsubsection{Эмпирико-теоретические методы}
tЭмпирико-теоретические методы (смешанные) или методы построения теорий на основе полученных эмпирических данных об объекте, процессе, явлении.

\begin{itemize}
    \item \textbf{Абстрагирование} --- выделение наиболее важных для исследования свойств, сторон исследуемого объекта, процесса, явления и игнорирование несущественных и второстепенных.
    \item \textbf{Анализ} --- разъединение целого на части с целью выявления их связей.
    \item \textbf{Декомпозиция} --- разъединение целого на части с сохранением их связей с окружением.
    \item \textbf{Синтез} --- соединение частей в целое с целью выявления их взаимосвязей.
    \item \textbf{Композиция} --- соединение частей целого с сохранением их взаимосвязей с окружением.
    \item \textbf{Индукция} --- получение знания о целом по знаниям о частях.
    \item \textbf{Дедукция} --- получение знания о частях по знаниям о целом.
    \item \textbf{Эвристики, использование эвристических процедур} --- получение знания о целом по знаниям о частях и по наблюдениям, опыту, интуиции, предвидению.
    \item \textbf{Моделирование (простое моделирование)}, использование приборов -- получение знания о целом или о его частях с помощью модели или приборов.
    \item \textbf{Исторический метод} --- поиск знаний с использованием предыстории, реально существовавшей или же мыслимой.
    \item \textbf{Логический метод} --- поиск знаний путем воспроизведения частей, связей или элементов в мышлении.
    \item \textbf{Макетирование} --- получение информации по макету, представлению частей в упрощенном, но целостном виде.
    \item \textbf{Актуализация} --- получение информации с помощью перевода целого или его частей (а следовательно, и целого) из статического состояния в динамическое состояние.
    \item \textbf{Визуализация} --- получение информации с помощью наглядного или визуального представления состояний объекта, процесса, явления.
\end{itemize}

\noindent
Кроме указанных классических форм реализации теоретико-эмпирических методов часто используются и мониторинг (система наблюдений и анализа состояний), деловые игры и ситуации, экспертные оценки (экспертное оценивание), имитация (подражание) и другие формы.

\example{1}

\noindent
Для построения модели планирования и управления производством в рамках страны, региона или крупной отрасли нужно решить следующие проблемы:
\begin{enumerate}
    \item Определить структурные связи, уровни управления и принятия решений, ресурсы; при этом чаще используются методы наблюдения, сравнения, измерения, эксперимента, анализа и синтеза, дедукции и индукции, эвристический, исторический и логический методы, макетирование и др.;
    \item Определить гипотезы, цели, возможные проблемы планирования; наиболее используемые методы --- наблюдение, сравнение, эксперимент, абстрагирование, анализ, синтез, дедукция, индукция, эвристический, исторический, логический и др.;
    \item Конструирование эмпирических моделей; наиболее используемые методы --- абстрагирование, анализ, синтез, индукция, дедукция, формализация, идеализация и др.;
    \item Поиск решения проблемы планирования и просчет различных вариантов, директив планирования, поиск оптимального решения; используемые чаще методы – измерение, сравнение, эксперимент, анализ, синтез, индукция, дедукция, актуализация, макетирование, визуализация, виртуализация и др.
\end{enumerate}

    \include{src/02}
    \section{Системы счисления}
\label{sec:notation-systems}

В истории разные народы использовали системы счисления с разными основаниями.
Каждый народ руководствовался своими доводами в пользу того или иного числа в основании.
Например, африканские племена использовали 5-ричную систему счисления, потому что на руке 5 пальцев.
Тибетцы и нигерийцы использовали 12-ричную, это количество фаланг на четырех пальцах.
Привычная нам система счисления с основанием равным 10 появилась в Европе в 16, а в России в 17 веке.

\subsection{Позиционная система счисления}
\label{subsec:positional_notation_systems}

Рассмотрим формулу записи числа в позиционной системе счисления:

\noindent
$$X_{(q)} = x_{n-1} \times q^{n-1} + x_{n-2} \times q^{n-2} + \ldots + x_{1} \times q^{1} + x_{0}\times q^{0} + x_{-1} \times q^{-1} + \ldots + x_{-m} \times q^{-m}$$

Или
\[
X_{(q)} = \sum_{i=-m}^{n-1} x_{i} \times q^{i}
\]

\noindent
Где:

\begin{description}[noitemsep]
    \item [$X_{(q)}$] --- запись числа в системе счисления с основанием $q$
    \item [$x_{i}$] --- натуральные числа меньше $q$, то есть цифры
    \item [$n$] --- число разрядов целой части
    \item [$m$] --- число разрядов дробной части
    \item [$q$] --- показатель системы счисления
\end{description}

\noindentСамо число $X_{(q)}$ имеет следующий вид: 
$X_{(q)} = x_{n-1}x_{n-2} \ldots x_{1}x_{0}x_{-1} \ldots x_{1-m}x_{-m}$


\noindent
Рассмотрим данную формулу на примере:

$$123,45_{10} = 1\times 10^{2} + 2\times 10^{1} + 3\times 10^{0} + 4\times 10^{-1} + 5\times 10^{-2}$$

\noindent
Мы разложили число $123,45$ по этой формуле. В данном случае $q$ = 10, $n = 3$, $m = 2$, $X_{(q)} = 123,45$, а $x_{3-1} = 1$ ($x_{2} = 1$), $x_{1} = 2$ и так далее.

В позиционной системе счисления важную роль имеет порядок цифр, то есть значение каждого числового знака (цифры) в записи числа зависит от его позиции (разряда).

\subsection{Перевод чисел из одной системы счисления в другую}
\label{subsec:notations_conversion}

Существуют три способа перевода из одной системы счисления в другую:

\begin{enumerate}
    \item Из десятичной системы счисления в систему счисления с основанием $N$
    \item Из системы счисления с основанием $N$ в десятичную систему счисления
    \item Из системы счисления с основанием $N$ в систему счисления с основанием $N^{k}$ и обратно, при условии $k \in \mathbb{N}$
\end{enumerate}

\subsubsection{Перевод числа из десятичной системы счисления в систему счисления с основанием \texorpdfstring{$N$}{N}}
\label{subsubsec:notation_conversion_10_to_n}

Чтобы перевести дробное число в систему счисления с основанием N необходимо разделить его на две части: целую и дробную, и каждую часть переводить отдельно.

\paragraph{Преобразования целой части числа}
Для перевода целой части числа из десятичной системы счисления в другую необходимо:

\begin{enumerate}
    \item Разделить целую часть десятичного числа на основание новой системы счисления;
    \item Записать остаток деления;
    \item Разделить получившийся результат деления (п.1) на основание новой системы счисления (при необходимости);
    \item Записать остаток деления;
    \item Повторять, пока целая часть десятичного числа не будет равна 0.
\end{enumerate}

Получившиеся в ходе деления остатки и есть цифры искомого числа в новой системе счисления. Записать остатки в обратном порядке (начиная с последнего полученного). Стоит заметить, что в данном способе очень удобно применять деление столбиком.

\bigskip

\example{1}
\task Перевести число $45_{10}$ в троичную систему счисления.
\solution Последовательно разделим $45_{10}$ на $3$, записывая остатки:

\begin{figure}[H]
    \includegraphics[scale=0.5]{3.2.1-example1}
\end{figure}

\begin{figure}[H]
\centering
\includegraphics[width=4cm]{1_2_1(1)}
\end{figure}

Полученные остатки: 0, 0, 2, 1. Записываем их в обратном порядке.

\answer $45_{10} = 1200_{3}$

\paragraph{Преобразования дробной части числа}
Для перевода дробной части числа из десятичной системы счисления в другую необходимо:
\begin{enumerate}
    \item Умножить дробную часть десятичного числа на основание новой системы счисления;
    \item Отделить и записать целую часть;
    \item Умножить дробную часть результата умножения (п.1) на основание новой системы счисления (при необходимости);
    \item Отделить и записать целую часть;
    \item Повторять, пока дробная часть десятичного числа не будет равна 0.
\end{enumerate}

Получившиеся в ходе умножения целые части и есть цифры искомого числа в новой системе счисления.
Записать целые части в прямом порядке (начиная с первого полученного). Первая записанная целая часть (0) идет в целую часть нового числа, а в дробную записываются полученные целые части, начиная со второй.
Стоит заметить, что в данном способе очень удобно применять умножение столбиком.

\bigskip
\example{2}
\task перевести число $0,625_{10}$ в четверичную систему счисления.
\solution умножим дробную часть $0,625_{10}$ на $4$, записывая целые части, пока не получим в дробной части 0:

\begin{figure}[H]
    \includegraphics[scale=0.5]{3.2.1-example2}
\end{figure}

\begin{figure}[H]
\centering
\includegraphics[width=1.7cm]{1_2_1(2)}
\end{figure}

\explain сначала умножаем 0,625 на 4, получаем 2,5. 2 записываем в целые части, а далее используем дробную часть - 0,5. Умножаем 0,5 на 4, получаем 2, записываем в целые части. Так как дробная часть равна 0, то перевод окончен.
Полученные целые части: 0, 2, 2. Первая полученная целая часть (0) идет в целую часть нового числа. Остальные (2, 2) в дробную часть.

\answer $0,625_{10} = 0,22_{4}$

\bigskip

\example{3}
\task перевести число $43,52_{10}$ в пятеричную систему счисления.
\solution разделим $43,52_{10}$ на две части: целую ($43_{10}$) и дробную ($0,52_{10}$). Переведем целую и дробную части по отдельности:

\begin{figure}[H]
    \includegraphics[scale=0.5]{3.2.1-example3}
\end{figure}

\begin{figure}[H]
\centering
\includegraphics[width=6cm]{1_2_1(3)}
\end{figure}

Полученные остатки (3, 1, 1) запишем в обратном порядке: $43_{10} = 113_{5}$.
Полученные целые части (0, 2, 3) запишем в прямом порядке: $0,52_{10} = 0,23_{5}$.
Объеденим полученные части

\answer $43,52_{10} = 113,23_{5}$

\subsubsection{Перевод числа из системы счисления с основанием \texorpdfstring{$N$}{N} в десятичную систему счисления}
\label{subsubsec:notation_conversion_n_to_10}

Формула перевода числа из системы счисления с основанием N в десятичную систему счисления это практически формула записи числа в позиционной системе счисления.

\[
X_{(10)} = \sum_{i=-m}^{n-1} x_{i} \times q^{i}
\]

\noindentГде:
\begin{description}[noitemsep]
    \item [$X_{(10)}$] --- запись числа в системе счисления с основанием $q$
    \item [$x_{i}$] --- натуральные числа меньше $q$, то есть цифры
    \item [$n$] --- число разрядов целой части
    \item [$m$] --- число разрядов дробной части
    \item [$q$] --- показатель системы счисления
\end{description}

\example{1}
\task Перевести число $1101,111_{2}$ в десятичную систему счисления.
\solution
% $1101,111_{2} = 1 \times 2^{3} + 1\times 2^{2} + 0\times 2^{1} + 1\times 2^{0} + 1\times 2^{-1} + 1\hm\times 2^{-2}\hm + 1\hm\times 2^{-3}\hm = 1\times 8 + 1\times 4 + 0\times 2 + 1\times 1 + 1\times 0,5 + 1\times 0,25 + 1\times 0,125\hm = 8 + 4 + 1 + 0,5 + 0,25 + 0,125 = 13,875_{10} $
\begin{flalign*}
1101,111_{2} & = 1 \times 2^{3} + 1 \times 2^{2} + 0 \times 2^{1} + 1 \times 2^{0} + 1 \times 2^{-1} +  1 \times 2^{-2} + 1 \times 2^{-3} \\
             & = 1 \times 8 + 1 \times 4 + 0 \times 2 + 1 \times 1 + 1 \times 0,5 + 1 \times 0,25 + 1 \times 0,125 \\
             & = 8 + 4 + 1 + 0,5 + 0,25 + 0,125 \\
             & = 13,875_{10}
\end{flalign*}

\answer $1101,111_{2} = 13,875_{10}$

\subsubsection{Перевод числа из системы счисления с основанием \texorpdfstring{$N$}{N} в систему счисления с основанием \texorpdfstring{$N^{k}$}{Nk} и обратно, при условии \texorpdfstring{$k \in \mathbb{N}$}{k in N}}
\label{subsubsec:notation_conversion_n_to_nk}

Если основание системы счисления первого числа является степенью основания системы счисления второго числа ($N = N^{k}$), при условии $k \in \mathbb{N}$, то можно использовать следующий алгоритм.

\paragraph{Преобразование $N \to N^{k}$}
\begin{enumerate}
    \item Дополнить число (записанное в системе счисления $N$) незначащими нулями так, чтобы количество цифр было кратно $k$ (если число дробное, то дополнить так, чтобы и в целой и в дробной частях количество цифр было кратно $k$).
    \item Разбить это число на группы по $k$ цифр, начиная от нуля (если число дробное, то целую часть разбивать, начиная от запятой в левую сторону, а дробную часть, начиная от запятой в правую сторону).
    \item Заменить каждую такую группу эквивалентным числом, записанным в системе $N^{k}$.
\end{enumerate}

\paragraph{Преобразование $N^{k} \to N$}

\begin{enumerate}
    \item Заменить каждую цифру числа, записанного в системе счисления $N^{k}$, эквивалентным набором из $k$ цифр системы счисления $N$.
\end{enumerate}

Рассмотрим данный метод на системах счисления с основанием $N = 2^{k}$. Для этого воспользуемся таблицей~\ref{tab:2k_bases}.

\begin{table}[H]
    \caption{Основания вида $2^{k}$}
    \label{tab:2k_bases}
    \centering
    \begin{tabular}{|c|c|c|c|}
    \hline
    Десятичная & Двоичная & Восмеричная & Шестнадцатиричная
    \\\hline
    0 & 0000 & 00 & 0 \\
    1 & 0001 & 01 & 1 \\
    2 & 0010 & 02 & 2 \\
    3 & 0011 & 03 & 3 \\
    4 & 0100 & 04 & 4 \\
    5 & 0101 & 05 & 5 \\
    6 & 0110 & 06 & 6 \\
    7 & 0111 & 07 & 7 \\
    8 & 1000 & 10 & 8 \\
    9 & 1001 & 11 & 9 \\
    10 & 1010 & 12 & A \\
    11 & 1011 & 13 & B \\
    12 & 1100 & 14 & C \\
    13 & 1101 & 15 & D \\
    14 & 1110 & 16 & E \\
    15 & 1111 & 17 & F \\
    \hline
    \end{tabular}
\end{table}

\example{1}

\task Пользуясь таблицей перевести число $1542,43_{8}$ в двоичную систему счисления

\solution По таблице находим чему равны цифры исходного числа в двоичной системе. $1_{8} = 001_{2}$, $5_{8} = 101_{2}$ (незначащие нули убираем, так как необходимо, чтобы количество цифр в эквивалентном наборе было равно степени $k$ из выражения $N = N^{k}$, где $N^{k}$ - исходная система счисления. В данном случае $k = 3$) и так далее. Заменяем каждую цифру числа эквивалентным набором.

\answer $1542,43_{8} = 001101100010,100011_{2}$

\example{2}

\task Пользуясь таблицей перевести число $11010,11_{2}$ в шестнадцатиричную систему счисления

 \solution Первым делом, добавим незначащие нули так, чтобы количество цифр было кратно $k$ (в данном случае $k = 4$). Так как число дробное, не забываем добавлять нули и в конце числа. Получим $00011010,1100_{2}$.
Теперь необходимо разбить число на группы по $k$ цифр (начинаем от запятой). Результат: $0001\ 1010\ ,\ 1100\ _{2}$.
Пользуясь таблицей, заменяем группы цифр эквивалентными числами, записанным в шестнадцатиричной системе счисления.

\answer $11010,11_{2} = 1A,C_{16}$.

\subsection{Оптимальная система счисления}
\label{subsec:optimal_notation_system}

Давайте представим, что Вы по несчастливой (или счастливой) случайности попали на необитаемый остров.
Обеспокоенный количеством дней, которые Вам суждено провести на острове в ожидании спасателей, Вы решаете вести счет дней с помощью камней.
Но вот незадача --- камней на острове нашлось всего 60 штук (небогатый на камни остров оказался).
И для того, чтобы вести учет дней как можно продуктивнее (учесть как можно больше дней) необходимо выбрать систему счисления, плотность записи числа которой максимальна при данных обстоятельствах.

Существует зависимость плотности записи информации от основания системы счисления.
Если взять $N$ камней, а за основание принять число $X$, то получится $^N/_X$ разрядов, которыми можно закодировать $X^{^N/_X}$ чисел.

То есть с помощью 60 камней мы можем закодировать: $2^{30}$, $3^{20}$, $4^{15}$, $5^{12}$, $6^{10}$, $10^{6}$, $12^{2}$, $15^{4}$, $20^{3}$, $30^{2}$ или $60^1$ чисел.
Все зависит от того, какую систему счисления мы выберем. Возведя все числа в степени, мы увидим, что самое большое из них это $3^{20} = 3486784401$.

Удельная натуральнологарифмическая плотность записи числа зависит от основания системы счисления $x$ и выражается функцией $y = \dfrac{\ln{x}}{x}$.
Эта функция имеет максимум при $x = e = 2.718281828\ldots$

\begin{center}
\begin{tikzpicture}
\begin{axis}[
    xlabel=$x$, ylabel=$y$,
    domain=0:20,
    xmin=-1, xmax=20,
    ymin=0.1, ymax=0.5,
    extra x ticks={2.71},
    extra x tick labels={$e$},
]
\addplot[red, smooth] {ln(x)/x};
\draw[dashed] (axis cs:2.71,0) -- (axis cs:2.71,0.368);
\end{axis}
\end{tikzpicture}
\end{center}


\begin{figure}[H]
\centering
\includegraphics[width=6cm]{1_2_3(1)}
\end{figure}

Таким образом, самая оптимальная система счисления имеет основание равное $e = 2.718281828\ldots$, то есть нецелочисленное.
Она обладает наибольшей плотностью записи информации.

Возвращаясь к нашему пребыванию на острове, мы не можем взять дробное основание для системы счисления. Поэтому мы берем самое близкое целое к $e$ - это 3.
% !TeX root = ../../main.tex
\subsection{Округление чисел}
\label{subsec:numbers_rounding}

Как происходит округление чисел в десятичной системе счисления?
Каждый учил в школе правила округления: 1, 2, 3, 4 округляется в меньшую сторону, а 5, 6, 7, 8 и 9 --- в большую.
Рассмотрим два примера:

\begin{figure}[H]
    \begin{minipage}[t]{.45\linewidth}
        \begin{tabular}{|c|c|c|}
            \multicolumn{3}{l}{Пример 1:} \\
            \hline
            & Число & Округление \\
            \cline{2-3}
            & 5,9 & 6,0 \\
            & 4,1 & 4,0 \\
            & 5,0 & 5,0 \\
            & 6,6 & 7,0 \\
            & 2,4 & 2,0 \\
            \hline
            Cумма & 24,0 & 24,0 \\
            \hline
        \end{tabular}
    \end{minipage}
    \hfill
    \begin{minipage}[t]{.45\linewidth}
        \begin{tabular}{|c|c|c|}
            \multicolumn{3}{l}{Пример 2:} \\
            \hline
            & Число & Округление \\
            \cline{2-3}
            & 5,5 & 6,0 \\
            & 3,5 & 4,0 \\
            & 2,5 & 3,0 \\
            & 8,5 & 9,0 \\
            & 1,5 & 2,0 \\
            \hline
            Cумма & 21,5 & 24,0 \\
            \hline
        \end{tabular}
    \end{minipage}
\end{figure}

В примере 1 мы видим, что сумма до и после округления одинакова.
Так случилось, потому что мы округляли то в большую сторону, то в меньшую, и в итоге количество округлений в большую сторону равно количеству округлений в меньшую.
Округление нам не помешало получить красивую сумму.

А теперь посмотрим на специально подобранный пример 2.
Разница сумм довольно большая.
Так получилось, потому что мы округляли всегда только в большую сторону, хотя мы округляли по правилам.

Если проделать такой эксперимент с большим количеством чисел (тысяча, две тысячи или даже больше), то ошибка будет небольшая, но она будет.

Работая с числами, у которых показатель системы счисления четный, мы натыкаемся на следующую проблему: у нас нечетное количество чисел для округления.
Разберем на примере десятичной системы счисления.
В ней всего 10 цифр - 0, 1, 2, 3, 4, 5, 6, 7, 8, 9.
Числа $X.0$ мы не округляем.
Остается 9 чисел: $X.1$, $X.2$, $X.3$, $X.4$ округляем в меньшую сторону (всего 4 числа), а $X.5$, $X.6$, $X.7$, $X.8$, $X.9$ - в большую (всего 5 чисел).
$X.5$ - середина между $X+1$ и $X$, однако мы округляем в пользу $X+1$, то есть в большую.
Таким образом, при работе с большим количеством чисел, мы всегда будем округлять в большую сторону чаще, чем в меньшую.
Отсюда и ошибка в итоговой сумме.

Чтобы этого избежать, некоторые программы при автоматическом округлении большого количества чисел округляют $X.5$ по очереди то в большую сторону, то в меньшую.

В системах счисления с нечетным основанием такой ошибки нет.
Возьмем, к примеру, пятиричную систему счисления.
Она содержит цифры 0, 1, 2, 3, 4.
Числа $X.0$ мы не округляем, остается 4 числа: $X.1$, $X.2$ округляем в меньшую сторону (всего 2 числа), а $X.3$, $X.4$ - в большую (всего 2 числа).
Таким образом, количество чисел, округленных в меньшую сторону, равно количеству чисел, округленных в большую.

\subsection{Нетрадиционные системы счисления}
\label{subsec:nontraditional_notation_systems}

% !TeX root = ../../../main.tex
\subsubsection{Факториальная система счисления}
\label{subsubsec:factorial_system}

Любое натуральное число можно представить в виде

\[
    X = \sum^{n}_{k = 1} d_{k}\times k! \quad \text{где  } 0 \leqslant d_{k} \leqslant k
\]

В основании факториальной системы счисления используется факториал.
Запись числа в факториальной системе счисления будет иметь вид:
\[
    X_{\text{ф}} = d_{n} d_{n-1} \ldots d_{1}
\]

\paragraph{Перевод из факториальной системы счисления в десятичную}

Алгоритм перевода из факториальной системы счисления в десятичную очень похож на алгоритм перевода из системы счисления с основанием $N$ в десятичную (раздел~\ref{subsubsec:notation_conversion_n_to_10}).

\[
    X_{10} = d_{n} \times n! + d_{n-1} \times (n-1)! + d_{n-2} \times (n-2)! + \ldots + d_{2}\times 2! + d_{1}\times 1!
\]

\noindent
Где:
\begin{description}[noitemsep]
    \item [$X_{10}$] --- искомое число в десятичной системе счисления;
    \item [$d_{i}$] --- натуральные числа меньше или равные $i$;
    \item [$n$] --- количество разрядов исходного числа
\end{description}

\example{1}
\task{} перевести число $221_{\text{ф}}$ в десятичную систему счисления
\solution{}
\begin{flalign*}
221_{\text{ф}} & = 2 \times 3! + 2\times 2! + 1\times 1! \\
               & = 2\times 6 + 2\times 2 + 1\times 1 \\
               & = 12 + 4 + 1 \\
               & = 17_{10}
\end{flalign*}

\answer{}  $221_{\text{ф}} = 17_{10}$

\paragraph{Перевод из десятичной системы счисления в факториальную}

Для перевода воспользуемся все той же формулой:

\[
    X_{10} = d_{n} \times n! + d_{n-1} \times (n-1)! + d_{n-2} \times (n-2)! + \ldots + d_{2}\times 2! + d_{1}\times 1!
\]

\noindent
Стоит обратить внимание, что $0 \leqslant d_{1}\leqslant 1$; $0 \leqslant d_{2}\leqslant 2$ и так далее.

\begin{enumerate}
    \item Находим факториал $k!$, значение которого больше $X_{10}$, но ближе всего к нему. Тогда $n = k - 1$, где $n$ - количество разрядов искомого числа в факториальной системе.
    \item Записываем \[ X_{10} = d_{n}\times n! + d_{n-1}\times (n-1)! + d_{n-2}\times (n-2)! + ... + d_{2}\times 2! + d_{1}\times 1!\] с уже полученным $n$.
    \item Начиная с $d_{n}$ с помощью ума и смекалки начинаем подбирать коэффициенты, помня, что $d_{1} \in \{0,1\}$, $d_{2} \in \{0,1,2\}$ и т.д.
\end{enumerate}

\example{2}
\task{} перевести число $54_{10}$ в факториальную систему счисления
\solution{} $54_{10} < 5! (5! = 120)$, значит количество разрядов равно $5 - 1 = 4$.

Запишем формулу для $n = 4$: $54_{10} = d_{4}\times 4! + d_{3}\times 3! + d_{2} \times 2! + d_{1}\times 1!$

Подберем коэффициенты: $54_{10} = 2\times 4! + 1\times 3! + 0\times 2! + 0 \times 1!$

\answer{}  $54_{10} = 2100_{\text{ф}}$

\noindent
\textbf{Применение факториальной системы счисления:} декодирование и кодирование перестановок.

\example{3}

\noindent
\begin{minipage}[t]{.75\linewidth}
    \flushleft%
    \task{} Имеется $n = 5$ чисел (1, 2, 3, 4, 5), нужно найти все их перестановки. Известно, что существует $n! = 5! = 120$ таких перестановок. Найти перестановку, если известен ее номер $k = 52$.

    \solution{} Переведем $k$ в факториальную систему: $52_{10} = 2 \times 4! + 0 \times 3! + 2\times 2! + 0 \times 1! = 2020_{\text{ф}}$
    Дополним результат до $n - 1$ разрядов (при необходимости), расставим символы по местам:

    \bigskip
    \begin{tabular}{rll}
        1. & Справа от $5$ есть $2$ меньшие цифры & \code{( - - 5 - - )}; \\
        2. & Справа от $4$ есть $0$ меньших цифр  & \code{( - - 5 - 4 )}; \\
        3. & Справа от $3$ есть $2$ меньшие цифры & \code{( 3 - 5 - 4 )}; \\
        4. & Справа от $2$ есть $0$ меньших цифр  & \code{( 3 - 5 2 4 )}; \\
    \end{tabular}

    \bigskip
    \begin{tabular}{rl|ccccc|}
        \cline{3-7}
        1. & Справа от $5$ есть $2$ меньшие цифры: & - & - & 5 & - & - \\ \cline{3-7}
        2. & Справа от $4$ есть $0$ меньших цифр:  & - & - & 5 & - & 4 \\ \cline{3-7}
        3. & Справа от $3$ есть $2$ меньшие цифры: & 3 & - & 5 & - & 4 \\ \cline{3-7}
        4. & Справа от $2$ есть $0$ меньших цифр:  & 3 & - & 5 & 2 & 4 \\ \cline{3-7}
        \cline{3-7}
    \end{tabular}

    \bigskip
    \begin{enumerate}[noitemsep]
        \item Справа от $5$ есть $2$ меньшие цифры \ $(- - 5 - -)$;
        \item Справа от $4$ есть $0$ меньших цифр \ \ $(-\ - 5\ -\ 4)$;
        \item Справа от $3$ есть $2$ меньшие цифры \  $(3\ -\ 5\ - 4)$;
        \item Справа от $2$ есть $0$ меньших цифр \ \ $(3\ -\ 5\ \ 2\ \ 4)$;
    \end{enumerate}

    \answer{} (3 1 5 2 4)
\end{minipage}
\hfill
\noindent
\begin{minipage}[t]{.2\linewidth}
    \vspace{0pt}
    \flushright%
    \begin{tabular}{|c|c|}
        \hline
        0   & 12345 \\
        1   & ????? \\
        ... & ..... \\
        52  & ????? \\
        ... & ..... \\
        119 & 54321 \\
        \hline
    \end{tabular}
\end{minipage}

% !TeX root = ../../../main.tex
\subsubsection{Система счисления Цекендорфа}\label{subsubsec:zeckendorfs-notation-system}
Любое натуральное число можно представить в виде

\[
    X = \sum^{n}_{k = 1} d_{k} \times F_{k} \quad \text{где  } d_{k} \in \{0,1\},\text{ а } F_{k} \text{ --- числа Фибоначчи}
\]
Каждое число Фибоначчи есть сумма двух предыдущих чисел: \\
$F_{k} = \{1, 1, 2, 3, 5, 8, 13, 21, \ldots\}$.

В записи чисел в системе счисления Цекендорфа первая единица из ряда чисел Фибоначчи \textbf{не используется} (т.к. первая единица это $F_{0}$).

Запись числа в системе счисления Цекендорфа будет иметь вид
\[
    X_{\text{ц}} = d_{n}d_{n-1} \ldots d_{1}
\]

В записи чисел в системе счисления Цекендорфа \textbf{не допускается использование двух единиц подряд}.

\paragraph{Перевод из системы счисления Цекендорфа в десятичную}\label{par:zeckendorfs-to-decimal}

Алгоритм перевода из системы счисления Цекендорфа в десятичную очень похож на алгоритм перевода из системы счисления с основанием $N$ в десятичную (раздел~\ref{subsubsec:notation-conversion-n-to-10}):

\[
    X_{10} = d_{n} \times F_{n} + d_{n-1}\times F_{n-1} + d_{n-2} \times F_{n-2} + \ldots + d_{2} \times F_{2} + d_{1} \times F_{1},
\]

\noindentгде:
\begin{description}
    \item [$X_{10}$] --- искомое число в десятичной системе счисления;
    \item [$d_{i}$] --- натуральные числа меньше или равные $i$;
    \item [$n$] --- количество разрядов исходного числа.
\end{description}

\example{1}

\task{} перевести число $100101_{\text{ц}}$ в десятичную систему счисления
\solution{} $100101_{\text{ц}} = 1 \times 13 + 0 \times 8 + 0 \times 5 + 1 \times 3 + 0 \times 2 + 1 \times 1 = \allowbreak 13 + 3 + 1 = \allowbreak 17_{10}$
\answer{} $100101_{\text{ц}} = 17_{10}$

\paragraph{Перевод из десятичной системы счисления в систему счисления Цекендорфа}\label{par:decimal-to-zeckendorfs}

Для перевода воспользуемся все той же формулой:
\[
    X_{10} = d_{n} \times F_{n} + d_{n-1} \times F_{n-1} + d_{n-2} \times F_{n-2} + \ldots + d_{2} \times F_{2} + d_{1} \times F_{1}
\]

\begin{enumerate}
    \item Находим число $F_{k}$ в ряду чисел Фибоначчи, которое больше $X_{10}$, но ближе всего к нему. Тогда $n = k - 1$, где $n$ - количество разрядов искомого числа в системе Цекендорфа.
    \item Записываем с уже полученным $n$. \[X_{10} = d_{n} \times F_{n} + d_{n-1} \times F_{n-1} + d_{n-2} \times F_{n-2} + ... + d_{2}\times F_{2} + d_{1} \times F_{1}\]
    \item Начиная с $d_{n}$ с помощью ума и смекалки начинаем подбирать коэффициенты, помня, что $d_{1} \in \{0,1\}$ и что две единицы не могут стоять рядом.
\end{enumerate}

\example{2}

\task{} перевести число $19_{10}$ в систему счисления Цекендорфа

\solution{} $19_{10} < 21 (21 = F_{7})$, значит количество разрядов равно $7 - 1 = 6$. \\
Запишем формулу для $n = 6$: \\
$d_{6}\times 13 + d_{5}\times 8 + d_{4}\times 5 + d_{3}\times 3 + d_{2}\times 2 + d_{1}\times 1$ \\
Подберем коэффициенты: \\
$19_{10} = 1 \times 13 + 0 \times 8 + 1 \times 5 + 0 \times 3 + 0 \times 2 + 1\times 1 = \allowbreak 101001_{\text{ц}}
$

\answer{} $19_{10} = 101001_{\text{ц}}$

\textbf{Применение системы счисления Цекендорфа:} кодирование данных с маркером завершения \enquote{11} , у некоторых народов в сельском хозяйстве --- минимизация необходимого числа зерен.

% !TeX root = ../../../main.tex
\subsubsection{Система счисления Бергмана}\label{subsubsec:bergmans}

Любое действительное неотрицательное число можно представить в виде
\[
    X = \sum^{\infty}_{k = -\infty} d_{k} \times z^{k} \quad \text{где } d_{k} \in \{0,1\}, \text{ а } z = \frac{1+\sqrt{5}}{2}
\]
Число $z$ - число золотой пропорции.

Запись числа в системе счисления Бергмана будет иметь вид: \[
    X_{\text{Б}} = d_{n}d_{n-1} \ldots d_{2}d_{1}d_{0},d_{-1}d_{-2} \ldots d_{-m}
\]

Чтобы исключить неоднозначность, используется запись с наибольшим количеством разрядов.

\paragraph{Перевод из системы счисления Бергмана в десятичную}
Алгоритм перевода из системы счисления Бергмана в десятичную очень похож на алгоритм перевода из системы счисления с основанием $N$ в десятичную (\cref{subsubsec:notation-conversion-n-to-10}).
\[
    X_{10} = d_{n}\times z^{n} + d_{n-1}\times z^{n-1} + \ldots + d_{1} \times z^{1} + d_{0} \times z^{0} + d_{-1}\times z^{-1}\ + \ldots + d_{-m}\times z^{-m}
\]

\noindentГде:
\begin{description}[noitemsep]
    \item [$X_{10}$] --- искомое число в десятичной системе счисления;
    \item [$d_{i}$] --- число, принимающее значение 0 или 1;
    \item [$n$] --- количество разрядов целой части
    \item [$m$] ---  количество разрядов целой части
\end{description}

\example{1}

\medskip
\task{} Перевести число $100,01_{\text{Б}}$ в десятичную систему счисления
\solution{} Пусть $\displaystyle X = \frac{1+\sqrt{5}}{2}$, тогда

\begin{flalign*}
    100,01_{\text{Б}} & = 1 \times X^2 + 0 \times X^1 + 0 \times X^0 + 0 \times X^{-1} + 1 \times X^{-2}
        = X^2 + X^{-2} && \\
        & = (\frac{1+\sqrt{5}}{2})^{2} + (\frac{1+\sqrt{5}}{2})^{-2}
        = \frac{6 + 2\sqrt{5}}{4} + \frac{4}{6 + 2\sqrt{5}} && \\
        & = \frac{9 + 3\sqrt{5}}{3 + \sqrt{5}}
        = \frac{3(3 + \sqrt{5})}{3 + \sqrt{5}}
        = 3_{10} &&
\end{flalign*}

\answer{} $100,01_{\text{Б}} = 3_{10}$

\paragraph{Перевод чисел из десятичной системы в систему счисления Бергмана происходит методом подбора}
\textbf{Применение системы счисления Бергмана:} запись иррациональных чисел конечным числом цифр, контроль арифметических операций, коррекция ошибок, самосинхронизация кодовых последовательностей при передаче по каналу связи.

\subsubsection{Нега-позиционная система счисления}
\label{subsubsec:negative_base_system}

Это системы счисления с отрицательным основанием. Числа в нега-позиционных системах счисления, которые содержат четное количество цифр - отрицательные.
Главная особенность таких систем в том, что не нужно никаких специальных знаков для обозначения отрицательных чисел.

\bigskip
\textbf{Перевод чисел из нега-позиционных систем счисления полностью описан в разделе~\ref{subsubsec:notation_conversion_n_to_10}. Обратный перевод происходит методом подбора.}

\bigskip
\emph{Примеры в нега-десятичной системе счисления:}
$$
15_{-10} = 1 \times (-10)^1 + 5 \times (-10)^0 = - 1\times 10 + 5\times 1 = -10 + 5 = -5_{10}
$$

\begin{flalign*}
532_{-10}   & = 5 \times (-10)^2 + 8 \times (-10)^1 + 2 \times (-10)^0
            = 5 \times 100 - 8\times 10 + 2\times 1 && \\
            & = 500 - 80 + 2
            = 422_{10} &&
\end{flalign*}

\subsubsection{Симметричная система счисления}
\label{subsubsec:symmetic_system}

Это системы счисления с отрицательными числами.
Центром симметрии является 0, поэтому основанием для таких систем счисления могут быть только нечетные числа.
Главная особенность таких систем, как и нега-позиционных, в том, что не нужно никаких специальных знаков для обозначения отрицательных чисел.

\bigskip
\textbf{Перевод чисел из симметричных систем счисления и полностью описан в разделе~\ref{subsubsec:notation_conversion_n_to_10} Обратный перевод происходит методом подбора.}

\bigskip
\emph{Примеры в симметричной пятеричной системе счисления:}
Если в обычной пятеричной системе используются цифры \{0, 1, 2, 3, 4\}, то в симметричной пятеричной системе используются \{-2, -1, 0, 1, 2\}.
\begin{flalign*}
10\overline{2}\overline{1}2_{5\mbox{С}} & = 1 \times 5^4 + 0 \times 5^3 + (-2) \times 5^2 + (-1) \times 5^1 + 2 \times 5^0  \\
                                        & = 1\times 625 + 0\times 125 - 2\times 25 - 1\times 5 + 2\times 1 \\
                                        & = 625 - 50 - 5 + 2 \\
                                        & = 572_{10}
\end{flalign*}

\begin{flalign*}
\overline{1}021\overline{2}_{5\mbox{С}} & = (-1) \times 5^4 + 0\times 5^3 + 2\times 5^2 + 1\times 5^1 + (-2)\times 5^0 \\
                                        & = - 1\times 625 + 0\times 125 + 2\times 25 + 1 \times 5 - 2\times 1 \\
                                        & = - 625 + 50 + 5 - 2 \\
                                        & = -572_{10}
\end{flalign*}

Стоит обратить внимание, что цифры с чертой сверху --- отрицательные.

    \section{Арифметика в ограниченной разрядной сетке}
\label{sec:limited_arithmetics}

\subsection{Представление отрицательных чисел в ЭВМ}

В электронных вычислительных машинах нет возможности обозначить знак "минус" перед числом. Существует несколько способов решения этой проблемы:

\begin{description}
    \item [Специальный знаковый бит] --- определенный бит означает знак числа.
    \example{1} $+5_{10} = 0101_{2}$, $-5_{10} = 1101_{2}$ \\
    В данном случае, знаковый бит --- старший.

    \item [Фиксированное смещение] --- все числа уменьшены на какое-то определенное число.
    \example{2} $-5_{10} = 0000_{2}$, $-4_{10} = 0001_{2}$, \ldots ,$+10_{10} = 1111_{2}$ \\
    В данном случае, все числа уменьшены на 5.

    \item [Нега-двоичная система счисления] --- основание системы счисления равно $-2$.
    \example{3} $-4_{10} = 1100_{-2}$, $+5_{10} = 0101_{-2}$

    \item [Обратный (инверсный) код] --- инвертируются все биты.
    \example{4} $+5_{10} = 0101_{2}$, $-5_{10} = 1010_{2}$

    \item [Дополнительный код] --- инверсия всех бит плюс единица.
    \example{5} $+5_{10} = 0101_{2}$, $-5_{10} = 1011_{2}$
\end{description}

Некоторые из этих способов были реализованы. В 50-х и 60-х годах широко использовался четвертый способ. И специалисты теории информации разбились на два лагеря: те, кто использовал четвертый способ, и те, кто использовал пятый. Долго не могли примириться и компьютеры существовали и в том и в другом виде. Это привело к тому, что в стандарте языка программирования Си не определено как именно представлять отрицательные числа. Если Вы не будете использовать стандартные конструкции языка (например, $a = b + c$), в которых язык Си сам считает значение из памяти и приведет к нужному отрицательному или положительному виду, а сразу обратитесь к внутреннему представлению памяти - к ячейке по адресу (возьмете значение напрямую из ячейки), то хранящиеся там биты будут различны. Все будет зависеть от того, как работает Ваш компьютер: по четвертому способу или по пятому. Так же при складывании некоторых чисел, интерпретированных в обратном коде, в ограниченной разрядной сетке, возникает ошибка и требуется корректировка результата. Пятый способ не требует корректировки. Поэтому было решено использовать для представления отрицательных чисел в ЭВМ дополнительный код.s

\subsubsection*{Алгоритм перевода двоичного числа в дополнительный код и из дополнительного кода в прямой}
\begin{enumerate}
    \item Инвертировать все биты;
    \item Прибавить единицу;
\end{enumerate}

Отличить, в каком коде представлено число в разрядной сетке - в дополнительном или прямом, можно по старшему значащему биту. Если старший бит равен нулю - число положительное, единице - число отрицательное (например, в числе $1010_{2}$ старший бит равен $1$ - число отрицательное и представлено в дополнительном коде).

\bigskip

\noindent\textbf{\emph{Важно! Нумерация бит в разрядной сетке начинается с нуля и идет справа налево.}}

\example{6}

\task перевести число $1101011011_{2}$ в дополнительный код.

\solution инвертируем биты: $1101011011_{2} \to 0010100100_{2}$;
Прибавляем единицу: $0010100100_{2} + 1_{2} = 0010100101_{2}$;

\answer $0010100101_{2}$

\example{7}

\task представить число $-18_{10}$ в 8-разрядной сетке.

\solution так как у нас число отрицательное, то сначала переведем модуль данного числа в двоичную систему счисления: $|-18_{10}| = 18_{10} = 10010_{2}$;
Теперь представим его в дополнительном коде: $10010_{2} \to 01101_{2} + 1_{2} = 01110_{2}$;
Так как у нас 8-разрядная сетка, то дополним число незначащими нулями: $0000\ 1110_{2}$

\answer $-18_{10} = 0000\ 1110_{2}$

\subsection{Диапазон значений}
Фиксированное значение разрядности хранимого числа определяет диапазон возможных значений, которые можно записать в отведенное количество байт.

Пусть в некотором компьютере переменная А хранится с использованием $k$ бит. Чтобы определить диапазон возможных значений, достаточно найти минимальное и максимальное значения А.

\subsubsection{Беззнаковые числа}
Если мы рассматриваем только неотрицательные числа, то минимальным значением $A$ будет 0. Это соответствует случаю, когда в каждом разряде числа $A$ записан ноль.

Максимально представимым число $A$ будет тогда, когда в каждый разряд записаны единицы. Это число, равное $2^k-1$.
Почему минус 1? Ответить на этот вопрос поможет простой пример: рассмотрим трехразрядное число в десятичной СС. Очевидно, что наибольшее такое число 999. Легко убедится, что число 999 можно получить, если вычесть единицу из минимального 4-разрядного числа ($1000 = 10^3$). Видим, что степень 10 соответствует количеству разрядов, которым будет ограничено представление числа. Аналогичное правило можно вывести и для двоичной СС. Тогда для расчёта диапазона представления целых неотрицательных чисел при наличии $k$-разрядной сетки компьютера можно применять следующую формулу: $A \in [0;2^k-1]$.

\subsubsection{Знаковые числа}
Так как при $k$-разрядном представлении отрицательного числа $A$ в дополнительным коде старший разряд выделяется для хранения знака числа, то непосредственное значение числа А может храниться в $k-1$ разрядах. Поэтому для знаковых чисел при наличии $k$-разрядной сетки компьютера можно применять следующую формулу: $A \in [-2^{k-1};2^{k-1}]$.

\subsection{Флаги состояния процессора}
\label{subsec:flags_registers}

\textbf{Регистр флагов} --- регистр процессора, отражающий текущее состояние процессора.

% Как nirtable только с возможностью расширить на всю страницу (сделать environment как nirtable не вышло)
\begin{xltabular}{\textwidth}{|c|c|C|C|}
    \caption{Регистр флагов Intel x86}\label{tab:flags}\\\hline

    \thead{№\\бита} & \thead{Обозна-\\чение} & \thead{Название} & \thead{Описание}\\\hline
    \endfirsthead

    \multicolumn{4}{r}{\small Продолжение таблицы~\thetable}\\\hline
    \thead{№\\бита} & \thead{Обозна-\\чение} & \thead{Название} & \thead{Описание}\\\hline
    \endhead

    \multicolumn{4}{|c|}{\textbf{FLAGS}} \\ \hline
    0 & CF & Carry Flag & Флаг переноса \\ \hline
    1 & - & - & Зарезервирован \\ \hline
    2 & PF & Parity Flag & Флаг четности \\ \hline
    3 & - & - & Зарезервирован \\ \hline
    4 & AF & Auxiliary Carry Flag & Вспомогательный флаг переноса \\ \hline
    5 & - & - & Зарезервирован \\ \hline
    6 & ZF & Zero Flag & Флаг нуля \\ \hline
    7 & SF & Sign Flag & Флаг знака \\ \hline
    8 & TF & Trap Flag & Флаг трассировки \\ \hline
    9 & IF & Interrupt Enable Flag & Флаг разрешения прерываний  \\ \hline
    10 & DF & Direction Flag & Флаг направления \\ \hline
    11 & OF & Overflow Flag & Флаг переполнения \\ \hline
    12 & \multirow{2}{*}{IOPL} & \multirow{2}{=}{\centering I/O Privilege Level} & \multirow{2}{=}{\centering Уровень приоритета ввода-вывода} \\ \cline{1-1}
    13 & & & \\ \hline
    14 & NT & Nested Task & Флаг вложенности задач \\ \hline
    15 & - & - & Зарезервирован \\ \hline

    \multicolumn{4}{|c|}{\textbf{EFLAGS}} \\ \hline
    16 & RF & Resume Flag & Флаг возобновления \\ \hline
    17 & VM & Virtual-8086 Mode & Режим виртуального процессора 8086 \\ \hline
    18 & AC & Alignment Check & Проверка выравнивания \\ \hline
    19 & VIF & Virtual Interrupt Flag & Виртуальный флаг разрешения прерывания \\ \hline
    20 & VIP & Virtual Interrupt Pending & Ожидающее виртуальное прерывание \\ \hline
    21 & ID & ID Flag & Проверка на доступность инструкции CPUID \\ \hline
    22 -- 31 & - & - & Зарезервированы \\ \hline
    \multicolumn{4}{|c|}{\textbf{RFLAGS}} \\ \hline
    32 -- 63 & - & - & Зарезервированы \\ \hline
    22 & \multirow{3}{*}{-} & \multirow{3}{*}{-} & \multirow{3}{*}{Зарезервированы}  \\ \cline{1-1}
    $\cdots$ & & & \\ \cline{1-1}
    31 & & & \\ \hline

    \multicolumn{4}{|c|}{\textbf{RFLAGS}} \\ \hline
    32 & \multirow{3}{*}{-} & \multirow{3}{*}{-} & \multirow{3}{*}{Зарезервированы} \\ \cline{1-1}
    $\cdots$ & & & \\ \cline{1-1}
    63 & & & \\\hline
\end{xltabular}

После любой арифметической операции процессор автоматически без участия программиста заполняет регистр флагов состояния. Состояние процессора меняется после каждой арифметической операции.

Таблица~\ref{tab:flags} представлена для ознакомления. Для курса <<Информатика>> необходимо знать следующие флаги:

\begin{description}
    \item [CF (Carry Flag) --- Флаг переноса.] Устанавливается (принимает значение $1$) в случае, если происходит перенос за пределы разрядов или заем извне.
    \item [PF (Parity Flag) --- Флаг четности] Устанавливается, если младший значащий байт результата содержит четное число единичных (ненулевых) бит. Изначально этот флаг был ориентирован на использование в коммуникационных программах: при передаче данных по линиям связи для контроля мог также передаваться бит четности.
    \item [AF (Auxiliary Carry Flag) --- Вспомогательный флаг переноса.] Устанавливается, если произошел заем или перенос между первым и вторым полубайтами (третьим и четвертым битами).
    \item [ZF (Zero Flag) --- Флаг нуля.] Устанавливается, если результат машинной операции по модулю 2 в степени $k$ (где $k$ - разрядность ячейки) равен нулю (другими словами, принимает значение $1$, если результат выполнения операции равен нулю).
    \item [SF (Sign Flag) --- Флаг знака.] Устанавливается, если результат выполнения операции отрицателен (равен значению старшего значащего бита).
    \item [OF (Overflow Flag) --- Флаг переполнения.] Устанавливается, если в результате выполнения операции со знаковыми числами появляется одна из ошибок: при сложении положительных чисел получается отрицательный результат или при сложении отрицательных чисел получается положительный результат.
\end{description}

\example{1}

\task сложить $14837_{10}$ и $5832_{10}$ в 16-разрядной сетке. Расставить флаги состояния процессора.

\solution для начала переведем исходные числа в двоичную систему счисления. $14837_{10} = 0011\ 1001\ 1111\ 0101_{2}$, $5832_{10} = 0001\ 0110\ 1100\ 1000_{2}$.
\begin{tabular}{r l c r c | c c |}
\\ \cline{6-7}
\multirow{2}{*}{+} & $0011\ 1001\ 1111\ 0101_{2}$ & = & $14837_{10}$ & & CF = 0 & ZF = 0 \\
& $0001\ 0110\ 1100\ 1000_{2}$ & = & $5832_{10}$ & & PF = 1 & SF = 0 \\ \cline{2-2}
& $0101\ 0000\ 1011\ 1101_{2}$ & = & $20669_{10}$ & & AF = 0 & OF = 0 \\ \cline{6-7}
\end{tabular}

\bigskip

\noindentТак как $0101\ 0000\ 1011\ 1101_{2} = 20669_{10}$, то результат операции сложения в 16-разрядной сетке корректен.

\example{2}

\task сложить $21324_{10}$ и $13543_{10}$ в 16-разрядной сетке. Расставить флаги состояния процессора.

\solution для начала переведем исходные числа в двоичную систему счисления. $21324_{10} = 0101\ 0011\ 0100\ 1100_{2}$, $13543_{10} = 0011\ 0100\ 1110\ 0111_{2}$.
\begin{tabular}{r l c r c | r r |}
\\ \cline{6-7}
\multirow{2}{*}{+} & $0101\ 0011\ 0100\ 1100_{2}$ & = & $21324_{10}$ & & CF = 0 & ZF = 0 \\
& $0011\ 0100\ 1110\ 0111_{2}$ & = & $13543_{10}$ & & PF = 0 & SF = 0 \\ \cline{2-2}
 & $1000\ 1000\ 0011\ 0011_{2}$ & = & $-30669_{10}$ & & AF = 0 & OF = 1 \\ \cline{6-7}
\end{tabular}

\bigskip

\noindentТак как $1000\ 1000\ 0011\ 0011_{2} = -30669_{10} \ne 34867_{10} = 21324_{10} + 13543_{10}$, то результат операции сложения в 16-разрядной сетке некорректен. \\OF = 1: при складывании положительных чисел получили отрицательное.

\example{3}

\task аналогично примерам 1 и 2 - сложить $-7453_{10}$ и $24732_{10}$.

\solution $24732_{10} = 0110\ 0000\ 1001\ 1100_{2}$, $-7453_{10} = 1110\ 0010\ 1110\ 0011_{2}$ ($-7453_{10}$ представляем в дополнительном коде для 16-разрядной сетки).

\begin{tabular}{r l c r c | r r |}
\\ \cline{6-7}
\multirow{2}{*}{+} & $0110\ 0000\ 1001\ 1100_{2}$ & = & $24732_{10}$ & & CF = 1 & ZF = 0 \\
& $1110\ 0010\ 1110\ 0011_{2}$ & = & $-7453_{10}$ & & PF = 0 & SF = 0 \\ \cline{2-2}
$1$& $0100\ 0011\ 0111\ 1111_{2}$ & = & $17279_{10}$ & & AF = 0 & OF = 0 \\ \cline{6-7}
\end{tabular}

\bigskip

\noindentТак как $0100\ 0011\ 0111\ 1111_{2} = 17279_{10}$, то результат операции сложения в 16-разрядной сетке корректен. Несмотря на то, что произошел выход за пределы разрядности сетки.

    % \section{Терминология теории информации}

Понятие \textit{"информация"} имеет различные трактовки в различных предметных областях. Например,\textit{информация} может пониматься как:
\begin{itemize}
\item сигналы для управления, приспособления рассматриваемой системы (в кибернетике);
\item мера хаоса в рассматриваемой системе (в физике);
\item вероятность выбора в рассматриваемой системе (в теории вероятностей);
\item мера разнообразия в рассматриваемой системе (в биологии) и др.
\end{itemize}
 Но мы остановимся на понятиях, близких к информатике.
\\
\\\textbf{Информация} - это некоторая упорядоченная последовательность сообщений, отражающих, передающих и увеличивающих наши знания.
\\
\\\textbf{Информация} - это сведения об окружающем мире (объекте, процессе, явлении, событии), которые являются объектом преобразования (включая хранение, передачу и т.д.) и используются для выработки поведения, для принятия решения, для управления или для обучения.
\\
\\\textbf{Информация} - это новые сведения, подлежащие передаче, хранению и обработке.s
\newpage 
Рассмотрим это фундаментальное понятие информатики на основе понятия \textit{"алфавит"} ("алфавитный", формальный подход). Дадим формальное определение \textit{алфавита}.
\\
\\\textbf{Алфавит} - конечное множество различных знаков (букв), символов, для которых определена операция \emph{конкатенации} (присоединения символа к символу или цепочке символов); с ее помощью по определенным правилам соединения символов и слов можно получать слова (цепочки знаков) и словосочетания (цепочки \textit{слов}) в этом \textit{алфавите} (над этим \textit{алфавитом}).
\\
\\\textbf{Знак (буква)} - любой элемент алфавита (элемент $x$ алфавита $X$, где $x \in X$). Понятие знака неразрывно связано с тем, что им обозначается ("со смыслом"), они вместе могут рассматриваться как пара элементов ($x$, $y$), где $x$ – сам знак, а $y$ – обозначаемое этим знаком.\\
\\\emph{\textbf{Пример 1:}}
\\Примеры \emph{алфавитов:} множество из десяти цифр, множество из знаков русского языка, точка и тире в азбуке Морзе и др. В \emph{алфавите} цифр знак 5 связан с понятием "быть в количестве пяти элементов".\\
\\\textbf{Слово} в алфавите (или над алфавитом) - конечная последовательность знаков (букв) алфавита.
\\
\\\textbf{Длина} |p| некоторого слова $p$ в алфавите (над алфавитом) - число составляющих его букв.
\\
\\\textbf{Словарь (словарный запас)} - множество различных слов в алфавите (над алфавитом).
\\В отличие от конечного \emph{алфавита}, словарный запас может быть и бесконечным.\\
\\\emph{Слова} над некоторым заданным \emph{алфавитом} и определяют так называемые \emph{сообщения}.\\
\\\emph{\textbf{Пример 2:}}
\\\emph{Слова} над \emph{алфавитом} кириллицы - "Информатика","инто", "ииии'', "и". 
\\\emph{Слова} над \emph{алфавитом} десятичных цифр и знаков арифметических операций – "1256", "23+78", "35–6+89", "4". 
\\\emph{Слова} над \emph{алфавитом} азбуки Морзе – ".", ". . –", "– – –".\\
\\В  \emph{алфавите} должен быть определен порядок следования \emph{букв} (порядок типа "предыдущий элемент – последующий элемент"), то есть любой \emph{алфавит} имеет упорядоченный вид $X = {x_1, x_2, …, x_n}$ .\\
\\Таким образом, \emph{алфавит} должен позволять решать задачу лексикографического (алфавитного) упорядочивания, или задачу расположения \emph{слов} над этим \emph{алфавитом}, в соответствии с порядком, определенным в \emph{алфавите} (то есть по символам \emph{алфавита}).

\section{Признаки классификации информации}

Рассмотрим две классификации информации. Первая из них - классификация по форме \emph{сообщений} - определенного вида сигналов, символов:
\begin{itemize}
  \item отношение к источнику или приемнику (входная, выходная и внутренняя);
  \item отношение к конечному результату (исходная, промежуточная и результирующая);
  \item актуальность;
  \item адекватность;
  \item доступность (открытая, закрытая);
  \item понятность;
  \item полнота (достаточная, недостаточная, избыточная);
  \item достоверность;
  \item массовость;
  \item изменчивость (постоянная, переменная, смешанная);
  \item объективность;
  \item точность;
  \item стадия использования (первичная, вторичная);
  \item ценность.
\end{itemize}

Вторая классификация - по форме преставления информации, способам ее кодирования и хранения:
\begin{itemize}
  \item графическая;
  \item звуковая;
  \item текстовая;
  \item числовая;
  \item видеоинформация.
\end{itemize}

\section{Измерение количества информации}
Любые сообщения измеряются в \emph{байтах, килобайтах, мегабайтах, гигабайтах, терабайтах, петабайтах} и \emph{эксабайтах}, а кодируются, например, в компьютере, с помощью \emph{алфавита} из нулей и единиц, записываются и реализуются в ЭВМ в \emph{битах}.

Приведем основные соотношения между единицами измерения \emph{сообщений}:
\begin{itemize}
\item 1 бит (\textbf{bi}nary digi\textbf{t} - двоичное число) = 0 или 1;
\item 1 байт = 8 бит;
\item 1 килобайт (1Кб) = $2^{13}$ бит;
\item 1 мегабайт (1Мб) = $2^{23}$ бит;
\item 1 гигабайт (1Гб) = $2^{33}$ бит;
\item 1 терабайт (1Тб) = $2^{43}$ бит;
\item 1 петабайт (1Пб) = $2^{53}$ бит;
\item 1 эксабайт (1Эб) = $2^{63}$ бит.
\end{itemize}

Теперь нам известно понятие информации, но необходимо еще конкретно знать сколько этой информации. Поэтому есть два важных определения:
\\
\\\textbf{Количество информации} - число, адекватно характеризующее разнообразие (структурированность, определённость,выбор состояний и т.д.) в оцениваемой системе. Количество информации часто оценивается в битах, причем такая оценка может выражаться и в долях бит (так как речь идет не об измерении или кодировании сообщений).
\\
\\\textbf{Мера информации} - численная оценка количества информации, которая обычно задана неотрицательной, определенной на множестве событий и являющейся аддитивной функцией (то есть, мера информации объединения событий (множеств) равна сумме мер каждого события). Заметим, что функция меры информации монотонна (при уменьшении или увеличении вероятности некоторого события количество иноформации в системе монотонно уменьшается или увеличивается). 
\\\textbf{Важно:} мера вероятности всегда находится в диапазоне от 0 до 1.
\par
Для измерения информации используются различные подходы и методы, например, с использованием меры информации по Р. Хартли и К. Шеннону.

\newpage
\subsection{Мера Хартли}

\begin{wrapfigure}{l}{2.7cm}
\includegraphics[width=2.7cm]{4_3_1}
\begin{center}
\footnotesize{Ральф Хартли}
\\\footnotesize{$1888 - 1970$}
\end{center}
\end{wrapfigure}

Пусть известны $N$ состояний системы $S$ ($N$  опытов с различными, равновозможными, последовательными состояниями системы). Если каждое состояние системы закодировать двоичными кодами, то минимальная длина $d$ полученного кода определяется из условия:

$$2^{d} \ge N \qquad \mbox{\emph{или}} \qquad  d \ge \log_{2}N$$
Значит, для однозначного описания системы требуется $\log_{2}N$ бит. В общем случае количество информации в системе $S$ равно:
$$H_{s} = \log_{k}N$$
Единицы измерения количества информации:
\begin{itemize}
  \item Бит ($k = 2$)
  \item Трит ($k = 3$)
  \item Дит (харт) ($k = 10$)
  \item Нит (нат) ($k = e$)
\end{itemize}
\begin{center}
\textbf{Примеры использования меры Хартли}
\end{center}
\emph{\textbf{Пример 1:}}
\\\emph{Задание:} мальчик загадывает число от 1 до 64. Какое количество вопросов типа "да-нет" понадобится, чтобы гарантированно угадать число?
\\\emph{Решение:}
\\$\bullet$ Первый вопрос: "Загаданное число меньше 32?". Ответ: "Да".
\\$\bullet$ Второй вопрос: "Загаданное число меньше 16?". Ответ: "Нет".
\\ \dots
\\$\bullet$ Шестой вопрос точно приведет к правильному ответу.
\\
\\Значит, в соответствии с мерой Хартли в загадке мальчика содержится $\log_{2}64 = 6$ бит информации ($N = 64$ так как возможно 64 вариантов загаданного числа).
\\Ответ: 6 бит.
\\
\\\emph{\textbf{Пример 2:}}
\\\emph{Задание:} мальчик держит за спиной шахматного ферзя и собирается поставить его на произвольную клетку пустой доски. Какое количество информации содержится в его действии?
\\\emph{Решение:} шахматная доска имеет размеры $8\times 8$ клеток. Ферзь может быть как белым, так и черным, поэтому количество равновероятных состояний будет равно $8\times 8 \times 2 = 128$. Получается, количество информации по мере Хартли равно $\log_{2}128 = 7$ бит.
\\Ответ: 7 бит.
\\Если во множестве $X = {x_1,x_2, ..., x_n}$ искать произвольный элемент, то для его нахождения (по Хартли) необходимо иметь не менее $\log_{a}n$ (единиц) информации. 
\\Уменьшение $Н$ говорит об уменьшении разнообразия состояний $N$ системы, увеличение $Н$ говорит об увеличении разнообразия состояний $N$ системы.
\\
\\Мера Хартли подходит лишь для идеальных, абстрактных систем, так как в реальных системах состояния системы неодинаково осуществимы (неравновероятны).

\subsection{Мера Шеннона }
\begin{wrapfigure}[12]{l}{2.5cm}
\includegraphics[width=2.5cm]{4_3_2}
\begin{center}
\footnotesize{Клод Шеннон}
\\\footnotesize{$1916 - 2001$}
\end{center}
\end{wrapfigure}

Если состояния системы не равновероятны, используют меру Шеннона. Мера Шеннона оценивает информацию отвлеченно от ее смысла:
$$I = - \sum^{N}_{i=1}p_{i}\times \log_{2}p_{i},$$ где:
\\$I$ - количество информации, выраженное в битах (в $\log_{k}p_{i}$ $k = 2$);
\\$N$ - число состояний системы;
\\$p_{i}$ - вероятность (относительная частота) перехода системы в $i$-е состояние (вероятность того, что система находится в состоянии $i$) \\Сумма всех $p_{i}$ должна быть равна единице.\\
\\Если все состояния рассматриваемой системы равновозможны, равновероятны, то есть $р_i = 1/n$ , то из \emph{формулы Шеннона} можно получить (как частный случай) \emph{формулу Хартли}:
$$I = \log_{2}n.$$
\\Обозначим величину:
$$f_i = -n\log_{2}p_i.$$
\newpage
Тогда из \emph{формулы К. Шеннона} следует, что количество информации I можно понимать как среднеарифметическое величин $f_i$ , то есть величину $f_i$ можно интерпретировать как \emph{информационное содержание символа алфавита} с индексом i и величиной $p_i$ вероятности появления этого символа в любом сообщении (слове), передающем информацию.\\
\\ В термодинамике известен так называемый коэффициент Больцмана $k = 1.38 * 10^{-16} (эрг.град)$ и выражение (\emph{формула Больцмана}) для энтропии или меры хаоса в термодинамической системе:
$$ S = -k \sum^{N}_{i=1}p_{i}\times \ln{p_{i}}$$
\\ Сравнивая выражения для I и S, можно заключить, что величину I можно понимать как энтропию из-за нехватки информации в системе (о системе).\\
\\Формулы энтропии и информации идентичны, но смысл разный. Энтропия априорная характеристика (до передачи), информация – апостериорная (после передачи).\\
\\Из этой формулы следуют важные выводы:
\begin{itemize}
\item увеличение меры Шеннона свидетельствует об уменьшении энтропии (увеличении порядка) системы;
\item уменьшение меры Шеннона свидетельствует об увеличении энтропии (увеличении беспорядка) системы.
\end{itemize}
Положительная сторона \emph{формулы Шеннона} – ее отвлеченность от смысла информации. Кроме того, в отличие от \emph{формулы Хартли}, она учитывает различность состояний, что делает ее пригодной для практических вычислений. Основная отрицательная сторона \emph{формулы Шеннона} – она не распознает различные состояния системы с одинаковой вероятностью.

\begin{center}
\textbf{Примеры использования меры Шеннона}
\end{center}
\emph{\textbf{Пример 1:}}
\\\emph{Задание:} девочка наугад вытаскивает из мешка мяч. Известно, что в мешке всего 8 мячей, из них: 4 красных, 2 синих, 1 зеленый и 1 белый. Какое количество информации содержится в этом событии?
\\\emph{Решение:}
\\$\bullet$ Вероятность вытащить красный мяч равна $^4/_8 = 0,5$
\\$\bullet$ Вероятность вытащить синий мяч равна $^2/_8 = 0,25$
\\$\bullet$ Вероятность вытащить зеленый мяч равна $^1/_8 = 0,125$
\\$\bullet$ Вероятность вытащить белый мяч равна $^1/_8 = 0,125$
\\Значит количество информации, выраженное в битах равно: $I = -(0,5\hm\times \log_{2}0,5\hm + 0,25\hm\times \log_{2}0,25 + 0,125\hm\times \log_{2}0,125\hm + 0,125\hm\times \log_{2}0,125) \hm = -(-0,5\times 1 - 0,25\times 2 - 0,125\times 3 - 0,125\times 3) = -(-0,5 - 0,5\hm - 0,375\hm - 0,375\hm) = 1,75$ бит.
\\Ответ: 1,75 бит

\section{Методы получения информации}
Методы получения информации можно разбить на три большие группы:
\begin{itemize}
\item \emph{Эмпирические};
\item \emph{Теоретические}; 
\item \emph{Эмпирико-теоретические}.
\end{itemize}
Кратко рассмотрим и охарактеризуем все три метода по отдельности. 
\subsection{Эмпирические методы}
\emph{Эмпирические методы или методы получения эмпирических данных.}
\begin{enumerate}
\item \emph{Наблюдение} -- сбор первичной информации об объекте, процессе, явлении.
\item \emph{Сравнение} -- обнаружение и соотнесение общего и различного.
\item \emph{Измерение} -- поиск с помощью измерительных приборов эмпирических фактов.
\item \emph{Эксперимент} -- преобразование, рассмотрение объекта, процесса, явления с целью выявления каких-то новых свойств.
\end{enumerate}
Кроме классических форм их реализации, в последнее время используются опрос, интервью, тестирование и другие.
\subsection{Теоретические методы}
\emph{Теоретические методы или методы построения различных теорий.}
\begin{enumerate}
\item \emph{Восхождение от абстрактного к конкретному} -- получение знаний о целом или о его частях на основе знаний об абстрактных проявлениях в сознании, в мышлении.
\item \emph{Идеализация} -- получение знаний о целом или его частях путем представления в мышлении целого или частей, не существующих в действительности.
\item \emph{Формализация} -- получение знаний о целом или его частях с помощью языков искусственного происхождения (формальное описание, представление).
\item \emph{Аксиоматизация} -- получение знаний о целом или его частях с помощью некоторых аксиом (не доказываемых в данной теории утверждений) и правил получения из них (и из ранее полученных утверждений) новых верных утверждений.
\item \emph{Виртуализация} -- получение знаний о целом или его частях с помощью искусственной среды, ситуации.
\end{enumerate}

\subsection{Эмпирико-теоретические методы}
\emph{Эмпирико-теоретические методы (смешанные) или методы построения теорий на основе полученных эмпирических данных об объекте, процессе, явлении.}
\begin{enumerate}
\item \emph{Абстрагирование} -- выделение наиболее важных для исследования свойств, сторон исследуемого объекта, процесса, явления и игнорирование несущественных и второстепенных.
\item \emph{Анализ} -- разъединение целого на части с целью выявления их связей.
\item \emph{Декомпозиция} -- разъединение целого на части с сохранением их связей с окружением.
\item \emph{Синтез} -- соединение частей в целое с целью выявления их взаимосвязей.
\item \emph{Композиция} -- соединение частей целого с сохранением их взаимосвязей с окружением.
\item \emph{Индукция} -- получение знания о целом по знаниям о частях.
\item \emph{Дедукция} -- получение знания о частях по знаниям о целом.
\item \emph{Эвристики, использование эвристических процедур} -- получение знания о целом по знаниям о частях и по наблюдениям, опыту, интуиции, предвидению.
\item \emph{Моделирование (простое моделирование)}, использование приборов -- получение знания о целом или о его частях с помощью модели или приборов.
\item \emph{Исторический метод} -- поиск знаний с использованием предыстории, реально существовавшей или же мыслимой.
\item \emph{Логический метод } -- поиск знаний путем воспроизведения частей, связей или элементов в мышлении.
\item \emph{Макетирование} -- получение информации по макету, представлению частей в упрощенном, но целостном виде.
\item \emph{Актуализация} -- получение информации с помощью перевода целого или его частей (а следовательно, и целого) из статического состояния в динамическое состояние.
\item \emph{Визуализация} -- получение информации с помощью наглядного или визуального представления состояний объекта, процесса, явления.
\end{enumerate}
Кроме указанных классических форм реализации теоретико-эмпирических методов часто используются и мониторинг (система наблюдений и анализа состояний), деловые игры и ситуации, экспертные оценки (экспертное оценивание), имитация (подражание) и другие формы.\\
\\\emph{\textbf{Пример :}}
\\Для построения модели планирования и управления производством в рамках страны, региона или крупной отрасли нужно решить следующие проблемы:
\begin{enumerate}
\item определить структурные связи, уровни управления и принятия решений, ресурсы; при этом чаще используются методы наблюдения, сравнения, измерения, эксперимента, анализа и синтеза, дедукции и индукции, эвристический, исторический и логический методы, макетирование и др.;
\item определить гипотезы, цели, возможные проблемы планирования; наиболее используемые методы – наблюдение, сравнение, эксперимент, абстрагирование, анализ, синтез, дедукция, индукция, эвристический, исторический, логический и др.;
\item конструирование эмпирических моделей; наиболее используемые методы – абстрагирование, анализ, синтез, индукция, дедукция, формализация, идеализация и др.;
\item поиск решения проблемы планирования и просчет различных вариантов, директив планирования, поиск оптимального решения; используемые чаще методы – измерение, сравнение, эксперимент, анализ, синтез, индукция, дедукция, актуализация, макетирование, визуализация, виртуализация и др.
\end{enumerate} % Повторение 01_theory
    % !TeX root = ../main.tex
\section{Сжатие данных}\label{sec:data-compression}

\hiddensection{О курсе}
\label{sec:introduction}



\subsection{Алгоритмы сжатия данных}\label{subsec:data-compression-algorithmss}


\subsubsection{Алгоритм Шеннона-Фано}
\label{subsubsec:shannon-fano-algo}

\begin{figure}[H]
\begin{wrapfigure}{l}{0.25\textwidth}
    \centering
    \vspace{-\intextsep}
    \includegraphics[width=0.25\textwidth]{person/fano-robert}
    \caption*{Роберт Фано\\1917 -- 2016}
\end{wrapfigure}

Алгоритм Шеннона-Фано --- один из первых алгоритмов сжатия.
Его сформулировали два ученых --- Клод Шеннон и Роберт Фано.
Алгоритм основан на частоте повторения.
Так, часто встречающийся символ кодируется кодом меньшей длины, а редко встречающийся --- кодом большей длины.
Коды, полученные при кодировании, префиксные, что позволяет декодировать любую последовательность.

\bigskip

Алгоритм Шеннона-Фано для некоторых последовательностей может сформировать неоптимальные коды.
\end{figure}

\bigskip
\paragraph{Описание алгоритма Шеннона-Фано}
\begin{enumerate}
    \item Символы входного (первичного) алфавита выписывают по убыванию вероятностей --- это корень будущего дерева;
    \item Строится дерево от корня к листьям. Находится середина, которая делит корень на два узла. Эти узлы (суммы вероятностей символов алфавита) примерно равны;
    \item Полученные узлы --- листья дерева. Левому узлу (с большей суммарной вероятностью) присваивается значение $1$, а правому --- $0$;
    \item Шаги 2--3 повторяются, пока в листьях дерева не останется один символ первичного алфавита.
    \item Символ входного (первичного) алфавита кодируется последовательностью нулей и единиц в соответствии с распределением их от корня к листьям (узлам).
\end{enumerate}

\example{1}
\task составить код Шеннона-Фано для последовательности $AAABCCCCCDEEEF$. Найти среднюю длину кодового слова.
\solution в последовательности $AAABCCCCCDEEEF$ алфавит состоит из 6 символов: A, B, C, D, E, F. Выпишем символы первичного алфавита по убыванию вероятностей:
\begin{table}[H]
\centering
\begin{tabular}{|r|c|c|c|c|c|c|}
\hline
\textbf{Символ} & C & A & E & B & D & F \\ \hline
\textbf{Вероятность} & $^5/_{14}$ & $^3/_{14}$ & $^3/_{14}$ & $^1/_{14}$ & $^1/_{14}$ & $^1/_{14}$ \\ \hline
\end{tabular}
\end{table}

\noindent Полученная последовательность $CAEDBF$ является корнем будущего дерева.

\noindent Построим дерево от корня к листьям:

\begin{table}[H]
    \centering
    \begin{tabular}{cccccc}
        \multicolumn{6}{c}{CAEBDF ($^5/_{14} + ^3/_{14} + ^3/_{14} + ^1/_{14} + ^1/_{14} + ^1/_{14}$)} \\
        % \multicolumn{6}{c}{CAEBDF ($\frac{5}{14} + \frac{3}{14} + \frac{3}{14} + \frac{1}{14} + \frac{1}{14} + \frac{1}{14}$)} \\
        \multicolumn{2}{c}{$\swarrow$} & \multicolumn{4}{c}{$\searrow$} \\
        \multicolumn{2}{c}{CA ($^5/_{14} + ^3/_{14}$)} & \multicolumn{4}{c}{EBDF ($^3/_{14} + ^1/_{14} + ^1/_{14} + ^1/_{14}$)} \\
        % \multicolumn{2}{c}{CA ($\frac{5}{14} + \frac{3}{14}$)} & \multicolumn{4}{c}{EBDF ($\frac{3}{14} + \frac{1}{14} + \frac{1}{14} + \frac{1}{14}$)} \\
        $\swarrow$ & $\searrow$ & \multicolumn{2}{c}{$\swarrow$} & \multicolumn{2}{c}{$\searrow$} \\
        C ($^5/_{14}$) & A ($^3/_{14}$) & \multicolumn{2}{c}{EB ($^3/_{14} + ^1/_{14}$)} & \multicolumn{2}{c}{DF ($^1/_{14} + ^1/_{14}$)} \\
         & & $\swarrow$ & $\searrow$ & $\swarrow$ & $\searrow$ \\
         & & E ($^3/_{14}$) & B ($^1/_{14}$) & D ($^1/_{14}$) & F($^1/_{14}$) \\
    \end{tabular}
\end{table}

\noindentПрисвоим левому символу (с большей вероятностью) значение $1$, а правому --- $0$:

\begin{table}[H]
    \centering
    \begin{tabular}{cccccc}
        \multicolumn{6}{c}{CAEBDF ($^5/_{14} + ^3/_{14} + ^3/_{14} + ^1/_{14} + ^1/_{14} + ^1/_{14}$)} \\
        \multicolumn{2}{c}{\textbf{[1]} $\swarrow$} & \multicolumn{4}{c}{$\searrow$ \textbf{[0]}} \\
        \multicolumn{2}{c}{CA ($^5/_{14} + ^3/_{14}$)} & \multicolumn{4}{c}{EBDF ($^3/_{14} + ^1/_{14} + ^1/_{14} + ^1/_{14}$)} \\
        \textbf{[1]} $\swarrow$ & $\searrow$ \textbf{[0]} & \multicolumn{2}{c}{\textbf{[1]} $\swarrow$} & \multicolumn{2}{c}{$\searrow$ \textbf{[0]}} \\
        C ($^5/_{14}$) & A ($^3/_{14}$) & \multicolumn{2}{c}{EB ($^3/_{14} + ^1/_{14}$)} & \multicolumn{2}{c}{DF ($^1/_{14} + ^1/_{14}$)} \\
         & & \textbf{[1]} $\swarrow$ & $\searrow$ \textbf{[0]} & \textbf{[1]} $\swarrow$ & $\searrow$ \textbf{[0]} \\
         & & E ($^3/_{14}$) & B ($^1/_{14}$) & D ($^1/_{14}$) & F($^1/_{14}$) \\
    \end{tabular}
\end{table}

\noindentПолучим следующую таблицу для кодировки:

\begin{table}[H]
\centering
\begin{tabular}{|r|c|c|c|c|c|c|}
\hline
\textbf{Символ} & C & A & E & B & D & F \\ \hline
% \textbf{Вероятность} & $^5/_{14}$ & $^3/_{14}$ & $^3/_{14}$ & $^1/_{14}$ & $^1/_{14}$ & $^1/_{14}$ \\ \hline
\textbf{Код} & 11 & 10 & 011 & 010 & 001 & 000 \\ \hline
\end{tabular}
\end{table}

\noindentИсходная последовательность AAABCCCCCDEEEF кодируется следующей: $10.10.10.010.11.11.11.11.11.001.011.011.011.000 - 34$ бита.

Средняя длина кодового слова: 
$$
2 \times \frac{5}{14} + 2 \times \frac{3}{14} + 3 \times \frac{3}{14} + 3 \times \frac{1}{14} + 3 \times \frac{1}{14} + 3 \times \frac{1}{14}  = \frac{34}{14} \approx 2,4
$$


\newline
\subsubsection{Код Хаффмана}
\label{subsubsec:huffman_code}

\begin{wrapfigure}{l}{0.25\textwidth}
    \centering
    \includegraphics[width=0.25\textwidth]{huffman_david}
    \caption*{Дэвид Хаффман\\1925 - 1999}
\end{wrapfigure}

Код (алгоритм) Хаффмана был разработан в 1952 году аспирантом Массачусетского технологического института Дэвидом Хаффманом при написании им курсовой работы.

Как и алгоритм Шеннона-Фано, основан на частоте повторения. Зная вероятности символов в сообщении, можно описать процедуру построения кодов переменной длины, состоящих из целого количества битов. Символам с большей вероятностью ставятся в соответствие более короткие коды. Коды Хаффмана обладают свойством префиксности, что позволяет однозначно их декодировать.

Однако, в отличие от кодов Шеннона-Фано, коды Хаффмана всегда являются оптимальными.

Сжатие данных по Хаффману применяется при сжатии фото- и видеоизображений (JPEG, стандарты сжатия MPEG), в архиваторах (PKZIP, LZH), в протоколах передачи данных MNP5 и MNP7.

\paragraph{Алгоритм Хаффмана для неоптимальных префиксных кодов}

\begin{enumerate}
  \item Символы входного алфавита образуют список свободных узлов. Каждый узел имеет вес, равный вероятности появления символа в сжимаемом тексте (исходной последовательности). Строится дерево от листьев (узлов) к корню:
  \item Выбираются два свободных узла дерева с наименьшими весами;
  \item Создается их родитель с весом, равным их суммарному весу;
  \item Родитель добавляется в список свободных узлов, а двое его детей удаляются из этого списка.
  \item Одной дуге, выходящей из родителя (узлу с большим весом), ставится в соответствие значение $1$, а другой (узлу с меньшим весом) значение $0$.
  \item Повторяем шаги 2-4, выбирая в качестве одного из свободных узлов родителя, до тех пор, пока в списке свободных узлов не останется только один свободный узел. Он и будет считаться корнем дерева.
  \item Символ входного (первичного) алфавита кодируется последовательностью нулей и единиц в соответствии с распределением их от корня дерева к узлам (листьям).
\end{enumerate}

\example{1}

\task

    % При обработке данных (хранение данных в памяти, передача по каналам связи) существует вероятность битовых ошибок. Они могут возникать из-за альфа частиц от примесей в чипе микросхемы или от нейтронов из фонового космического излучения. Частота единичных битовых ошибок на 1 гигабайт данных составляет от 1 раза в час до 1 раза в тысячелетие. По данным исследования Google получилось, что 1 раз в сутки.
\\Есть несколько способов обработки данных, полученных при передаче с ошибкой:
\begin{itemize}
  \item \emph{Использовать полученные данные без проверки на ошибки.} Для такого способа не нужно выполнять лишних действий. Например, при передаче текста или видеоизображения ошибка в одном блоке (слове, кадре) не исказит общего смысла.
  \item \emph{Обнаружить ошибку, выполнить запрос повторной передачи поврежденного блока.} В случае, если пользователь обнаружил ошибку, потребуется передать не блок, а целое сообщение заново, и при передаче файлов большого размера (видеоизображений, фотоизображений и музыкальных файлов высокого качества) это очень проблематично. Если ошибку обнаружила операционная система, то она обнаружила ее еще при передаче данных, и заново передается только поврежденный блок.
  \item \emph{Обнаружить ошибку и отбросить поврежденный блок.} Как и первый способ, он весьма оправдан при передаче текста или видеоизображений.
  \item \emph{Обнаружить и исправить ошибку.} Не тратится время на повторную передачу данных, но полученные данные корректны. Однако, при таком способе необходимо применять особые методы кодирования и наряду с информационными битами передавать служебные, которые позволяют исправить ошибку.
\end{itemize}
Последний способ характеризует помехоустойчивый код.
\textbf{Помехоустойчивые коды} - это коды, позволяющие обнаружить и (или) исправить ошибки в кодовых словах, которые возникают при передачи по каналам связи.\\
\\\emph{Помехоустойчивое кодирование} предполагает введение в передаваемое сообщение, наряду с информационными, так называемых проверочных разрядов, формируемых в устройствах защиты от ошибок (кодерах-на передающем конце, декодерах — на приемном). Избыточность позволяет отличить разрешенную и запрещенную (искаженную за счет ошибок) комбинации при приеме, иначе одна разрешенная комбинация переходила бы в другую.
\begin{center}
  \textbf{Классификация помехоустойчивых кодов}
\end{center}

\begin{itemize}
\item \textbf{Блочные} - фиксированные блоки информации длиной $k$ символов преобразуются в блоки длиной $n$ символов (независимо друг от друга). Например, при передаче файла объемом в 1 гигабайт он делится на равные блоки по 1 килобайту и каждый килобайт снабжается служебной информацией, которая позволяет понять - корректен данный блок или нет. И при обнаружении некорректного блока передается только он.
    \begin{itemize}
      \item \textbf{Неравномерные} - редко используемые символы кодируются большим количеством символов (имеют большую длину) (азбука Морзе)
      \item \textbf{Равномерные} - длина блока (символа) постоянна (таблица ASCII).
      \begin{itemize}
        \item \textbf{Неразделимые} - коды с постоянной плотностью единиц: информационные и проверочные разряды неразделимы (каждый блок данных на входе получает служебные биты, позволяющие обнаружить ошибки, которые далее не отделяются).
        \item \textbf{Разделимые} - можно отделить (выделить) служебные биты от информационных.
        \begin{itemize}
          \item \textbf{Систематические (линейные)} - циклические коды, биты четности/нечетности, код Хэмминга, код Рида-Соломона, код Боуза-Чоудхури-Хоквингема.
          \item \textbf{Несистематические} - коды с контрольным суммированием.
        \end{itemize}
      \end{itemize}
    \end{itemize}
\item \textbf{Непрерывные} - передаваемая информационная последовательность не разделяется на блоки. Кодирование осуществляется целого потока.
    \begin{itemize}
     \item \textbf{Сверточные} - корректирующие ошибки коды; работают с непрерывным потоком данных, кодируя их при помощи регистров сдвига с линейной обратной связью.
     \end{itemize}
\end{itemize}
\begin{center}
\emph{Помехоустойчивый код характеризуется:}
\end{center}
$i$ - числом информационных разрядов;
\\$r$ - числом проверочных разрядов;
\\$n$ - общим числом разрядов ($n = i + r$);
\\
\\\emph{\textbf{Коэффициент избыточности:} КИ} = $\frac{r}{n}$;

\section{Кодирование с помощью бита четности}
\textbf{Контрольная сумма (check sum)} - некоторое число, рассчитанное путем применения определенного алгоритма к набору данных и используемое для проверки целостности этого набора данных при их передаче или хранении.
\\
\\\textbf{Бит четности} - частный случай контрольной суммы, представляющий из себя 1 контрольный бит, используемый для проверки четности количества единичных битов в двоичном числе.
\\
\\Контроль некой двоичной последовательности (например, машинного слова) с помощью бита чётности также называют \textbf{контролем по паритету}. Контроль по паритету -- наиболее простой и наименее мощный метод контроля данных.
\\
\\\textbf{Метод расчета бита четности} - суммирование по модулю 2 всех бит числа.
\\
\\\textbf{Сумма по модулю 2} - исключающее "ИЛИ"(для двух операндов), логическое или битовое сложение, разность двух/трех множеств.
\\
\\Обозначение: $A\ mod2\ B = A \oplus B$
\\
\\
\begin{minipage}[l]{3.5cm}
\includegraphics[width=3cm]{6_1_1}
\center{$A \wedge B$}
\end{minipage}
\begin{minipage}[c]{3.5cm}
\includegraphics[width=3cm]{6_1_2}
\center{$A \vee B$}
\end{minipage}
\begin{minipage}[r]{3.5cm}
\includegraphics[width=3cm]{6_1_3}
\center{$A \oplus B$}
\end{minipage}
\\
\\
$A \oplus B = (\neg(A\wedge B))\wedge(A\vee B) = \neg((A\wedge B)\vee(\neg A\wedge \neg B))$
\\
\begin{center}
\textbf{Таблица истинности для $A \oplus B$}
\end{center}
\begin{minipage}[l]{4cm}
\begin{tabular}{|c|c|c|}
\hline
A & B & A $\oplus$ B \\
\hline
0 & 0 & 0 \\
0 & 1 & 1 \\
1 & 0 & 1 \\
1 & 1 & 0 \\
\hline
\end{tabular}
\end{minipage}
\begin{minipage}[l]{6cm}
\begin{tabular}{|c|c|c|c|}
\hline
A & B & C & A $\oplus$ B $\oplus$ C\\
\hline
0 & 0 & 0 & 0 \\
0 & 0 & 1 & 1 \\
0 & 1 & 0 & 1 \\
0 & 1 & 1 & 0 \\
1 & 0 & 0 & 1 \\
1 & 0 & 1 & 0 \\
1 & 1 & 0 & 0 \\
1 & 1 & 1 & 1 \\
\hline
\end{tabular}
\end{minipage}
\\
\\В случае двух переменных результат выполнения операции является истинным тогда и только тогда, когда лишь один из аргументов является истинным.
\\Для функции трех и более переменных результат выполнения операции будет истинным только тогда, когда количество аргументов, равных 1, - нечетное.
\begin{center}
  \textbf{Пример обнаружения ошибок}
\end{center}
Допустим, у нас есть один информационный бит $i = 1$. К нему идет один $r_1$ - бит четности, проверочный разряд №1.
\\$i = r_1$, $i \oplus r_1 = 0$.
\\При получении данных произошел сбой и данные некорректны. Нам известно, что сумма по модулю 2 информационного бита и бита четности равна 0.
\begin{table}[h]
\begin{tabular}{|c|c|c|c|c|}
\hline
i исх & $r_{1}$ исх & i рез & $r_{1}$ рез & i рез $\oplus$ $r_{1}$ рез \\
\hline
1 & 1 & 0 & 0 & 0 \\
1 & 1 & 0 & 1 & 1 \\
1 & 1 & 1 & 0 & 1 \\
1 & 1 & 1 & 1 & 0 \\
\hline
\end{tabular}
\end{table}
\\Соответственно, биты с пометкой "исх"\ - исходные, а "рез"\ - результирующие.
\\В первом случае видно, что несмотря на то, что оба бита пришли с ошибкой, сумма по модулю 2 равна нулю (верна), а значит компьютер не посчитает это как за ошибку.
\\Во втором и третьем случаях сумма по модулю 2 результирующих данных не верна, равна 1. Ошибка.
\\В последнем случае все верно. Биты корректны, сумма по модулю 2 верна. Ошибки нет.
\\
\\Способ обнаружения ошибок с помощью бита четности применяется в RAID-хранилищах (RAID - (англ. redundant array of independent disks - избыточный массив независимых дисков) технология виртуализации данных, которая объединяет несколько дисков в логический элемент для избыточности и повышения производительности).

\section{Код Хэмминга}
\begin{wrapfigure}[14]{l}{2.8cm}
\includegraphics[width=2.8cm]{6_2_1}
\begin{center}
\footnotesize{Ричард Уэсли Хэмминг}
\\\footnotesize{$1915 - 1998$}
\end{center}
\end{wrapfigure}
В 1945 году Хэмминг занимался программрованием одного из первых электронных цифровых компьютеров для расчета решения физических уравнений.  В период с 1946 по 1976 года Хэмминг работал в Bell Labs, где сотрудничал с Клодом Шенноном. Хэмминг часто работал в выходные дни, и все больше и больше раздражался, потому что часто был должен перезагружать свою программу из-за ненадежности перфокарт. На протяжении нескольких лет он проводил много времени над построением эффективных алгоритмов исправления ошибок. В 1950 году он опубликовал способ, который на сегодняшний день известен как \textbf{\emph{код Хэмминга}}. Коды Хэмминга — наиболее известные и, вероятно, первые из \emph{самоконтролирующихся} и \emph{самокорректирующихся} кодов.\\
Рассмотрим подробнее два этих вида кодов:
\begin{itemize}
\item \emph{Самоконтролирующиеся коды} -- коды, позволяющие автоматически обнаруживать ошибки при передаче данных. Для их построения достаточно приписать к каждому слову один добавочный (контрольный) двоичный разряд и выбрать цифру этого разряда так, чтобы общее количество единиц в изображении любого числа было, например, четным. Одиночная ошибка в каком-либо разряде передаваемого слова (в том числе, может быть, и в контрольном разряде) изменит четность общего количества единиц. Счетчики по модулю 2, подсчитывающие количество единиц, которые содержатся среди двоичных цифр числа, могут давать сигнал о наличии ошибок.\\
\\При этом невозможно узнать, в каком именно разряде произошла ошибка, и, следовательно, нет возможности исправить её. Остаются незамеченными также ошибки, возникающие одновременно в двух, в четырёх или вообще в четном количестве разрядов. Впрочем, двойные, а тем более четырёхкратные ошибки полагаются маловероятными.
\item \emph{Самокорректирующиеся коды} -- коды, в которых возможно автоматическое исправление ошибок. Для построения самокорректирующегося кода, рассчитанного на исправление одиночных ошибок, одного контрольного разряда недостаточно. Как видно из дальнейшего, количество контрольных разрядов k должно быть выбрано так, чтобы удовлетворялось неравенство $2^k \ge k+m+1$ или $k \ge log_2{k+m+1}$, где m -  количество основных двоичных разрядов кодового слова. Минимальные значения k при заданных значениях m, найденные в соответствии с этим неравенством, приведены в таблице.

\begin{table}[h]
\begin{center}
\begin{tabular}{|c|c|}
\hline
 Диапазон m & $k_{min}$  \\
\hline
1 & 2 \\
\hline
2-4 & 3\\
\hline
5-11 & 4\\
\hline
12-26 & 5\\
\hline
27-57 & 6 \\
\hline
\end{tabular}
\end{center}
\end{table}
\end{itemize}
В настоящее время наибольший интерес представляют двоичные блочные корректирующие коды. При использовании таких кодов информация передаётся в виде блоков одинаковой длины и каждый блок кодируется и декодируется независимо друг от друга. Почти во всех блочных кодах символы можно разделить на информационные и проверочные. Таким образом, все комбинации кодов разделяются на разрешенные (для которых соотношение информационных и проверочных символов возможно) и запрещенные.\\
\\\textbf{Код Хэмминга} - блочный равномерный разделимый самокорректирующийся код. Исправляет одиночные битовые ошибки, возникшие при передаче или хранении данных.
\\
\\На каждые $i$ информационных бит используется $r$ проверочных.
\\
\\Значение каждого контрольного бита зависит от значений информационных бит: контрольный бит с номером $N$ контролирует все последующие $N$ бит через каждые $N$ бит, начиная с позиции $N$.
\\\textbf{Синдром последовательности S} - набор контрольных сумм информационных и проверочных разрядов
\\
\\Рассмотрим таблицу:
\begin{table}[h]
\begin{tabular}{|c|c|c|c|c|c|c|c|c|c|c|c|c|c|c|c|}
\hline
& 1 & 2 & 3 & 4 & 5 & 6 & 7 & 8 & 9 & 10 & 11 & 12 & 13 & 14 & \\
\hline
$2^k$ & $r_{1}$ & $r_{2}$ & $i_{1}$ & $r_{3}$ & $i_{2}$ & $i_{3}$ & $i_{4}$ & $r_{4}$ & $i_{5}$ & $i_{6}$ & $i_{7}$ & $i_{8}$ & $i_{9}$ & $i_{10}$ & S\\
\hline
1 & \cellcolor{Gray1}{X} & & \cellcolor{Gray1}{X} & & \cellcolor{Gray1}{X} & & \cellcolor{Gray1}{X} & & \cellcolor{Gray1}{X} & & \cellcolor{Gray1}{X} & & \cellcolor{Gray1}{X} & & $s_{1}$\\
\hline
2 & & \cellcolor{Gray2}{X} & \cellcolor{Gray2}{X} & & & \cellcolor{Gray2}{X} & \cellcolor{Gray2}{X} & & & \cellcolor{Gray2}{X} & \cellcolor{Gray2}{X} & & & \cellcolor{Gray2}{X} & $s_{2}$ \\
\hline
4 & & & & \cellcolor{Gray3}{X} & \cellcolor{Gray3}{X} & \cellcolor{Gray3}{X} & \cellcolor{Gray3}{X} & & & & & \cellcolor{Gray3}{X} & \cellcolor{Gray3}{X} & \cellcolor{Gray3}{X} & $s_{3}$ \\
\hline
8 & & & & & & & & \cellcolor{Gray4}{X} & \cellcolor{Gray4}{X} & \cellcolor{Gray4}{X} & \cellcolor{Gray4}{X} & \cellcolor{Gray4}{X} & \cellcolor{Gray4}{X} & \cellcolor{Gray4}{X} & $s_{4}$ \\
\hline
\end{tabular}
\end{table}
\\Данная таблица представлена для кода из 14 бит, но ее можно расширить для нужного размера самостоятельно.
\\Сверху (в первой строчке) указан номер бита, а справа - синдром.
\\Знаком $X$ обозначены те биты, которые контролирует проверочный бит, с номером, который указан в левом столбце (степень двойки). Чтобы узнать какими битами контролируется бит с номером $N$ надо просто разложить $N$ по степеням двойки.
\\Например, видно, что проверочный бит $r_1$ контролирует информационные биты $i_1$, $i_2$, $i_4$, $i_5$, $i_7$, $i_9$. А 11 бит ($i_7$) контролируется битами 1 ($r_1$), 2 ($r_2$) и 8 ($r_4$).
\\Чтобы узнать значение проверочного бита необходимо сложить по модулю 2 все информационные биты, которые он контролирует.
\\В данном случае:
\\$r_1 = i_1\oplus i_2\oplus i_4\oplus i_5\oplus i_7\oplus i_9$;
\\$r_2 = i_1\oplus i_3\oplus i_4\oplus i_6\oplus i_7\oplus i_10$;
\\и так далее.
\\Чтобы узнать значение синдрома, необходимо сложить по модулю 2 все биты, которые содержит синдром.
\\В данном случае:
\\$s_1 = r_1\oplus i_1\oplus i_2\oplus i_4\oplus i_5\oplus i_7\oplus i_9$;
\\$s_2 = r_2\oplus i_1\oplus i_3\oplus i_4\oplus i_6\oplus i_7\oplus i_{10}$;
\\и так далее.
\\Синдром последовательности S определяется составляющими синдромами $s_1$, $s_2$ и так далее. То есть для кода, где $r = 3$, синдром будет иметь следующий формат: $S(s_1;s_2;s_3)$.
\\
\\Определение минимального количества контрольных разрядов: $$2^k \ge r + i + 1$$.
\\Классические коды Хэмминга с маркировкой (n, i): (7, 4); (15, 11); (31, 26).
\\
\\Разбрем код Хэмминга для $r = 3$.
\begin{center}
\textbf{Код Хэмминга для $r = 3$}
\end{center}
В данном случае, у нас 7 бит, из них 4 информационных и 3 проверочных. То есть, код имеет маркировку (7, 4).
\\Составим таблицу кода (7, 4):
\begin{table}[h]
\begin{tabular}{|c|c|c|c|c|c|c|c|c|}
\hline
& 1 & 2 & 3 & 4 & 5 & 6 & 7 & \\
\hline
$2^k$ & $r_{1}$ & $r_{2}$ & $i_{1}$ & $r_{3}$ & $i_{2}$ & $i_{3}$ & $i_{4}$ & S\\
\hline
1 & \cellcolor{Gray1}{X} & & \cellcolor{Gray1}{X} & & \cellcolor{Gray1}{X} & & \cellcolor{Gray1}{X} & $s_{1}$\\
\hline
2 & & \cellcolor{Gray2}{X} & \cellcolor{Gray2}{X} & & & \cellcolor{Gray2}{X} & \cellcolor{Gray2}{X} & $s_{2}$ \\
\hline
4 & & & & \cellcolor{Gray3}{X} & \cellcolor{Gray3}{X} & \cellcolor{Gray3}{X} & \cellcolor{Gray3}{X} & $s_{3}$ \\
\hline
\end{tabular}
\end{table}
\\Рассмотрим конкретно контроль информационных бит проверочными на кругах Эйлера:
\\
\begin{minipage}[l]{3.5cm}
\includegraphics[width=3.5cm]{6_2_2}
\end{minipage}
\begin{minipage}[r]{9cm}
Видно, что $r_1$ контролирует $i_1$, $i_2$ и $i_4$; $r_2$ контролирует $i_1$, $i_3$ и $i_4$; а $r_3$ контролирует $i_2$, $i_3$ и $i_4$.
\\$r_1 = i_1 \oplus i_2 \oplus i_4$
\\$r_2 = i_1 \oplus i_3 \oplus i_4$
\\$r_3 = i_2 \oplus i_3 \oplus i_4$
\end{minipage}
\\
\\$s_1 = r_1 \oplus i_1 \oplus i_2 \oplus i_4$
\\$s_2 = r_2 \oplus i_1 \oplus i_3 \oplus i_4$
\\$s_3 = r_3 \oplus i_2 \oplus i_3 \oplus i_4$
\\
\\Рассмотрим таблицу значений синдрома ($s_1;s_2;s_3$) и позицию ошибочного бита в сообщении:
\begin{table}[h]
\begin{tabular}{|c|c|c|}
\hline
Синдром ($s_1;s_2;s_3$) & Конфигурация ошибок & Ошибочный символ\\
\hline
000 & НЕТ & НЕТ \\
001 & 0001000 & $r_3$ \\
010 & 0100000 & $r_2$ \\
011 & 0000010 & $i_3$ \\
100 & 1000000 & $r_1$ \\
101 & 0000100 & $i_2$ \\
110 & 0010000 & $i_1$ \\
111 & 0000001 & $i_4$ \\
\hline
\end{tabular}
\end{table}\\
Возьмем, к примеру, 2 строку (см. таблицу на следующей странице) : синдром $S(0,0,1)$, это значит что $s_1 = 0$, $s_2 = 0$, $s_3 = 1$. По построенной таблице кода (7,4) находится, в каком бите ошибка - какой бит содержится только в 3 синдроме. Проще говоря, в каком столбце $X$ стоит только в 3 строчке (напротив $s_3$). По таблице видим, что такой бит - 4 (поэтому во втором столбце 0001000 - нули означают правильный бит, а единица - ошибочный. В данном случае ошибочный бит 4, поэтому он равен единице). Смотрим, какой именно бит занимает четвертое место - $r_3$.
\\Чтобы получить правильную последовательность, необходимо инвертировать ошибочный бит.
\\
\\\emph{\textbf{Пример 1:}}
\\\emph{Задание:} получена последовательность 1100100. Вычислить ошибочный бит, записать правильную последовательность.
\\\emph{Решение:} составим таблицу кода (7,4) с конкретной последовательностью.
\begin{table}[h]
\begin{tabular}{|c|c|c|c|c|c|c|c|c|}
\hline
& 1 & 2 & 3 & 4 & 5 & 6 & 7 & \\
\rowcolor{Gray1}
\hline
& 1 & 1 & 0 & 0 & 1 & 0 & 0 & \\
\hline
$2^k$ & $r_{1}$ & $r_{2}$ & $i_{1}$ & $r_{3}$ & $i_{2}$ & $i_{3}$ & $i_{4}$ & S\\
\hline
1 & \cellcolor{Gray1}{X} & & \cellcolor{Gray1}{X} & & \cellcolor{Gray1}{X} & & \cellcolor{Gray1}{X} & $s_{1}$\\
\hline
2 & & \cellcolor{Gray2}{X} & \cellcolor{Gray2}{X} & & & \cellcolor{Gray2}{X} & \cellcolor{Gray2}{X} & $s_{2}$ \\
\hline
4 & & & & \cellcolor{Gray3}{X} & \cellcolor{Gray3}{X} & \cellcolor{Gray3}{X} & \cellcolor{Gray3}{X} & $s_{3}$ \\
\hline
\end{tabular}
\end{table}
\\Рассчитаем значение контрольных бит результата:
\\$r_{1\mbox{ рез}} = i_1 \oplus i_2 \oplus i_4 = 0 \oplus 1 \oplus 0 = 1$
\\$r_{2\mbox{ рез}} = i_1 \oplus i_3 \oplus i_4 = 0 \oplus 0 \oplus 0 = 0$
\\$r_{3\mbox{ рез}} = i_2 \oplus i_3 \oplus i_4 = 1 \oplus 0 \oplus 0 = 1$
\\Рассчитаем синдромы:
\\$s_1 = r_1 \oplus i_1 \oplus i_2 \oplus i_4 = r_{1\mbox{ рез}} \oplus r_{1\mbox{ исх}} = 1 \oplus 1 = 0$
\\$s_2 = r_2 \oplus i_1 \oplus i_3 \oplus i_4 = r_{2\mbox{ рез}} \oplus r_{2\mbox{ исх}} = 0 \oplus 1 = 1$
\\$s_3 = r_3 \oplus i_2 \oplus i_3 \oplus i_4 = r_{3\mbox{ рез}} \oplus r_{3\mbox{ исх}} = 1 \oplus 0 = 1$
\\Полученный синдром: $S(0,1,1)$.
\\Смотрим по таблице, какой бит содержится в $s_2$ и $s_3$ одновременно. 6, то есть $i_3$.
\\Инвертируем ошибочный бит и получаем правильную последовательность.
\\Ответ: 6 бит ($i_3$), правильная последовательность: 1100110.
\begin{figure}
\includegraphics[width=\textwidth]{6_2_3}
\caption{Схема создания кода Хэмминга (7,4)}
\end{figure}
\begin{figure}
\includegraphics[width=\textwidth]{6_2_4}
\caption{Схема декодирования кода Хэмминга (7,4)}
\end{figure}
\newpage
Разбрем код Хэмминга для $r = 4$ аналогично тому, как мы сделали это выше.
\begin{center}
\textbf{Код Хэмминга для $r = 4$}
\end{center}
В данном случае, у нас 15 бит, из них 11 информационных и 4 проверочных. То есть, код имеет маркировку (15, 11).
\\Составим таблицу кода (15, 11):
\begin{table}[h]
\begin{tabular}{|c|c|c|c|c|c|c|c|c|c|c|c|c|c|c|c|c|}
\hline
& 1 & 2 & 3 & 4 & 5 & 6 & 7 & 8 & 9 & 10 & 11 & 12 & 13 & 14 & 15 & \\
\hline
$2^k$ & $r_{1}$ & $r_{2}$ & $i_{1}$ & $r_{3}$ & $i_{2}$ & $i_{3}$ & $i_{4}$ & $r_{4}$ & $i_{5}$ & $i_{6}$ & $i_{7}$ & $i_{8}$ & $i_{9}$ & $i_{10}$& $i_{11}$ &S\\
\hline
1 & \cellcolor{Gray1}{X} & & \cellcolor{Gray1}{X} & & \cellcolor{Gray1}{X} & & \cellcolor{Gray1}{X} & & \cellcolor{Gray1}{X} & & \cellcolor{Gray1}{X} & & \cellcolor{Gray1}{X} & & \cellcolor{Gray1}{X} & $s_{1}$\\
\hline
2 & & \cellcolor{Gray2}{X} & \cellcolor{Gray2}{X} & & & \cellcolor{Gray2}{X} & \cellcolor{Gray2}{X} & & & \cellcolor{Gray2}{X} & \cellcolor{Gray2}{X} & & & \cellcolor{Gray2}{X} & \cellcolor{Gray2}{X} & $s_{2}$ \\
\hline
4 & & & & \cellcolor{Gray3}{X} & \cellcolor{Gray3}{X} & \cellcolor{Gray3}{X} & \cellcolor{Gray3}{X} & & & & & \cellcolor{Gray3}{X} & \cellcolor{Gray3}{X} & \cellcolor{Gray3}{X} & \cellcolor{Gray3}{X} & $s_{3}$ \\
\hline
8 & & & & & & & & \cellcolor{Gray4}{X} & \cellcolor{Gray4}{X} & \cellcolor{Gray4}{X} & \cellcolor{Gray4}{X} & \cellcolor{Gray4}{X} & \cellcolor{Gray4}{X} & \cellcolor{Gray4}{X} & \cellcolor{Gray4}{X} & $s_{4}$\\
\hline
\end{tabular}
\end{table}
Получим:
\\$r_1 = i_1 \oplus i_2 \oplus i_4  \oplus i_5  \oplus i_7  \oplus i_9  \oplus i_{11}$
\\$r_2 = i_1 \oplus i_3 \oplus i_4  \oplus i_6  \oplus i_7  \oplus i_{10}  \oplus i_{11}$
\\$r_3 = i_2 \oplus i_3 \oplus i_4  \oplus i_8  \oplus i_9  \oplus i_{10}  \oplus i_{11}$
\\$r_4 = i_5 \oplus i_6 \oplus i_7  \oplus i_8  \oplus i_9  \oplus i_{10}  \oplus i_{11}$
\\
\\$s_1 = r_1 \oplus i_1 \oplus i_2 \oplus i_4  \oplus i_5  \oplus i_7  \oplus i_9  \oplus i_{11}$
\\$s_2 = r_2 \oplus i_1 \oplus i_3 \oplus i_4 \oplus i_6  \oplus i_7  \oplus i_{10}  \oplus i_{11}$
\\$s_3 = r_3 \oplus i_2 \oplus i_3 \oplus i_4 \oplus i_8  \oplus i_9  \oplus i_{10}  \oplus i_{11}$
\\$s_4 = r_4 \oplus i_5 \oplus i_6 \oplus i_7  \oplus i_8  \oplus i_9  \oplus i_{10}  \oplus i_{11}$
\\
\\Рассмотрим таблицу значений синдрома ($s_1;s_2;s_3;s_4$) и позицию ошибочного бита в сообщении: \\\\
Рассуждения будут аналогичны. Возьмем, например, 4ую строку (см. таблицу на следующей странице). Синдром S(0,0,1,1) означает, что $s_1 = 0$, $s_2 = 0$, $s_3 = 1$, $s_4 = 1$. По таблице кода (15,11) находится, в каком бите ошибка - какой бит содержится только в 4 синдроме. В данном случае, такой бит - 12. Напомним, что нули обозначают правильный бит, а единица - ошибочный. Двенадцатое место занимает бит $i_8$.
\begin{tabular}{|c|c|c|c|}
\hline
Синдром ($s_1;s_2;s_3;s_4$) & Конфигурация ошибок & Ошибочный символ\\
\hline
0000 & НЕТ & НЕТ \\
0001 & 000000010000000 & $r_4$ \\
0010 & 000100000000000 & $r_3$ \\
0011 & 000000000001000 & $i_8$ \\
0100 & 010000000000000 & $r_2$ \\
0101 & 000000000100000 & $i_6$ \\
0110 & 000001000000000 & $i_3$ \\
0111 & 000000000000010 & $i_{10}$ \\
1000 & 100000000000000 & $r_1$\\
1001 & 000000001000000 & $i_5$\\
1010 & 000010000000000& $i_2$\\
1011 & 000000000000100 & $i_9$\\
1100 & 001000000000000 & $i_1$\\
1101 & 000000000010000 & $i_7$\\
1110 & 000000100000000 & $i_4$\\
1111 & 000000000000001 & $i_{11}$\\
\hline
\end{tabular}

    % Как было рассмотрено ранее, информатика изучает знаковые(алфавитные) системы. \textbf{Алгебра} -- наиболее адекватный математический аппарат описания действий в них, поэтому алгебраический аппарат наилучшим образом подходит для описания информационных систем общей природы, отвлеченно от их предметной направленности. Информационные процессы хорошо формализуются с помощью различных алгебраических структур. \\
\\Кроме обычной алгебры существует специальная, основы которой были заложены английским математиком XIX века Дж. Булем. Эта алгебра занимается так называемым исчислением высказываний.Ее особенностью является применимость для описания работы так называемых дискретных устройств, к числу которых принадлежит целый класс устройств автоматики и вычислительной техники.
\section{Определение}
\begin{wrapfigure}[13]{l}{2.8cm}
\includegraphics[width=2.8cm]{7_1}
\begin{center}
\footnotesize{Джордж Буль}
\\\footnotesize{$1815 - 1864$}
\end{center}
\end{wrapfigure}
\textbf{Алгебра двоичной логики} -- раздел математики, в котором изучаются логические операции над высказываниями. Чаще всего предполагается, что высказывания могут быть только истинными или ложными, то есть используется так называемая бинарная или двоичная логика. Один из основателей алгебры логики - Джордж Буль.
\\
\\\textbf{Логическая (булева) переменная} – такая переменная, значения которой могут быть лишь "1" или "0".
\\\emph{В естественном языке:} "булева переменная" $=$ "высказывание".
\\
\textbf{\\Высказывание} -- утверждение, про которое можно однозначно сказать, истинно оно или ложно.
\\\emph{Обозначения:}
\begin{itemize}
  \item "истина" \ и "ложь"
  \item "true" \ и "false"
  \item "1" \ и "0"
\end{itemize}
Например, высказывание "Москва - столица России\"  -- истина, а "трава синего цвета\"  -- ложь.
\\
\textbf{Логическая (булева) функция $f(x, y, z, …)$} -- некоторая функциональная зависимость, в результате выполнения логических операций над логическими переменными $x, y, z,\dots$ получает значение 0 или 1.
\textbf{Таблица истинности} -- таблица всех значений некоторой \emph{логической функции}.
\begin{center}
  \emph{Основные операции}
\end{center}
$\bar{x}$ - отрицание или инверсия
\\$x \vee y$ - дизъюнкция или логическое сложение
\\$x \wedge y$ - конъюнкция или логическое умножение\\
\\Кроме указанных трех базовых операций можно с их помощью ввести еще следующие важные операции алгебры предикатов (можно их назвать небазовыми операциями):\\
$(x\to y) \equiv (\bar{x} \vee y)$ - импликация \\
$(x\leftrightarrow y) \equiv (x \wedge y \vee \bar{x} \wedge \bar{y})$ - эквиваленция\\
\\Операции импликации и эквиваленции хотя и часто используются, но не являются базовыми, ибо они определяемы через три введенные выше базовые операции. При выполнении логических операций в компьютере они сводятся к поразрядному сравнению битовых комбинаций. Эти операции достаточно быстро (аппаратно) выполняемы, так как сводятся к выяснению совпадения или несовпадения битов.\\
\\В логических формулах определено старшинство операций, например: скобки, отрицание, конъюнкция, дизъюнкция (остальные, небазовые операции пока не учитываем).\\
\\Всегда истинные формулы называют \textbf{тавтологиями}.\\
\\\emph{Логические функции} эквивалентны, если совпадают их таблицы истинности, то есть совпадают области определения и значения, а также сами значения функции при одних и тех же наборах переменных из числа всех допустимых значений. Если это совпадение происходит на части множества допустимых значений, то формулы называются эквивалентными лишь на этой части (на этом подмножестве).\\
\\Задача \textbf{упрощения логического выражения} состоит в преобразовании его к более простому (по числу переменных, операций или операндов) эквивалентному выражению. Наиболее простой вид получается при сведении функции к постоянной – 1 (истина) или 0 (ложь).


\section{Основные тождества}
\begin{enumerate}
  \item Аксиома двойного отрицания \\
     $$\bar{\bar{x}} = x$$
  \item Аксиома существования 1 и 0 \\
  \begin{minipage}{5cm}
     $$0 = \bar{1}$$
     $$1 = \bar{0}$$
  \end{minipage}
  \begin{minipage}{5cm}
     $$x \vee \bar{x} = 1$$
     $$x \wedge \bar{x} = 0$$
  \end{minipage}
  \item Закон идемпотенции \\
  \begin{minipage}{5cm}
     $$x \vee x = x$$
  \end{minipage}
  \begin{minipage}{5cm}
     $$x \wedge x = x$$
  \end{minipage}
  \item Закон коммутативности \\
  \begin{minipage}{5cm}
     $$x \vee y = y \vee x$$
  \end{minipage}
  \begin{minipage}{5cm}
     $$x \wedge y = y \wedge x$$
  \end{minipage}
  \item Закон поглощения \\
  \begin{minipage}{5cm}
     $$x \vee (x \wedge y) = x$$
  \end{minipage}
  \begin{minipage}{5cm}
     $$x \wedge (x \vee y) = x$$
  \end{minipage}
  \item Закон ассоциативности \\
  \begin{minipage}{5cm}
     $$x \vee (y \vee z) = (x \vee y) \vee z$$
  \end{minipage}
  \begin{minipage}{5cm}
     $$x \wedge (y \wedge z) = (x \wedge y) \wedge z$$
  \end{minipage}
  \item Закон дистрибутивности \\
  \begin{minipage}{5cm}
     $$x \vee (y \wedge z) = (x \vee y) \wedge (x \vee z)$$
  \end{minipage}
  \begin{minipage}{5cm}
     $$x \wedge (y \vee z) = (x \wedge y) \vee (x \wedge z)$$
  \end{minipage}
  \item Законы де Моргана \\
  \begin{minipage}{5cm}
     $$\bar{x \vee y} = \bar{x} \wedge \bar{y}$$
  \end{minipage}
  \begin{minipage}{5cm}
     $$\bar{x \wedge y} = \bar{x} \vee \bar{y}$$
  \end{minipage}
  \item Закон нейтральности \\
  \begin{minipage}{5cm}
     $$x \vee (y \wedge \bar{y}) = x$$
  \end{minipage}
  \begin{minipage}{5cm}
     $$x \wedge (y \vee \bar{y}) = x$$
  \end{minipage}
\end{enumerate}

\section{Таблица истинности}
Существует только 4 различные логические функции одной переменной $F(x)$. Любая другая функция (даже самая сложная вида $F(x) = \bar{x} \vee x \wedge \bar{x}$) будет иметь одну из следующих таблиц истинности:
\begin{table}[!h]
\begin{tabular}{|c||c|c|c|c|}
\hline
$x$ & $F_0(x)$ & $F_1(x)$ & $F_2(x)$ & $F_3(x)$ \\
\hline
\hline
0 & 0 & 1 & 0 & 1 \\
\hline
1 & 0 & 0 & 1 & 1 \\
\hline
\end{tabular}
\end{table}
\\Например $F(x) = \bar{x}$ будет иметь таблицу истинности $F_1(x)$, а $F(x) = 1$ - $F_3(x)$
\\
\\Обычно логическую функцию $F(x)$ обозначают как $FK,N$, где $K$ - количество операндов, а $N$ - число в десятичной системе счисления, которое при переводе в двоичную является таблицей истинности функции. Например, $F(x) = \bar{x}$ обозначается как $F1,2$, так как одна переменная и $2_{10} = 01_{2}$, что является таблицей истинности функции (смотреть снизу вверх).
\\
\\Для $k$ переменных будет существовать $N = 2^{2^{k}}$ функций.
\\Рассмотрим таблицу истинности:
\\
\\
\begin{minipage}{2cm}
\begin{center}
$L$ штук\\
(от 00..0\\ до 11..1 \\ сверху \\ вниз)
\end{center}
\end{minipage}
\begin{minipage}{0.3cm}
\quad \\
\quad \\
$\left\{
\begin{array}{c}
\\
\\
\\
\\
\end{array}
\right.$
\quad \\
\end{minipage}
\begin{minipage}[l]{9cm}
\begin{tabular}{c|c|c|c|c|c|c|c|c|}
\hhline{~--------}
& \multicolumn{4}{c|}{Значения $k$ штук} & \multicolumn{4}{c|}{\multirow{2}{*}{Значение булевых функций $F$}} \\
& \multicolumn{4}{c|}{булевых операндов} & \multicolumn{4}{c|}{}\\
\hhline{~--------}
& $x_1$ & $x_2$ & \dots & $x_k$ & $FK,0$ & $FK,1$ & \dots & $FK,N$ \\
\hhline{~--------}
 & 0 & 0 & \dots & 0 & 0 & 1 & \dots & 1\\
& 0 & 0 & \dots & 1 & 0 & 0 & \dots & 1\\
& \dots & \dots & \dots & \dots & \dots & \dots & \dots & \dots\\
\multirow{4}{*}{} & 1 & 1 & \dots & 1 & 0 & 0 & \dots & 1\\

\hhline{~--------}
\multicolumn{5}{c}{} & \multicolumn{4}{c}{$\underbrace{\qquad \qquad \qquad \qquad \qquad \qquad \qquad}_{\mbox{от N штук (от 00..0 }}$} \\
\multicolumn{5}{c}{} & \multicolumn{4}{c}{до 11..1 слева направо)}
\end{tabular}
\end{minipage}
\\
\\
\\
Так как всего $k$ переменных, то для них будет $L = 2^k$ строк в таблице истинности, а количество булевых функций будет равно $N = 2^L$. Отсюда:
\\
\begin{center}
\begin{tabular}{c c}
$L = 2^k$ & \multirow{2}{*}{$\Rightarrow N = 2^{2^{k}}$} \\
$N = 2^L$ & \\
\end{tabular}
\end{center}
Так, для 1 переменной будет 4 булевых функции, для 2 переменных - 16 и так далее.
\begin{table}[!h]
\begin{tabular}{|c|c|c|c|c|c|c|}
\hline
$x$ & 1 & 1 & 0 & 0 & \multirow{2}{*}{Обозначение} & \multirow{2}{*}{Название} \\
\hhline{-----~~}
$y$ & 1 & 0 & 1 & 0 & & \\
\hline
\multirow{2}{*}{F} & \multirow{2}{*}{0} & \multirow{2}{*}{0} & \multirow{2}{*}{0} & \multirow{2}{*}{0} & F2,0 = $FALSE$  & Противоречие,  \\
& & & & & & логичечский нуль \\
\hline
\multirow{3}{*}{F} & \multirow{3}{*}{0} & \multirow{3}{*}{0} & \multirow{3}{*}{0} & \multirow{3}{*}{1} & F2,1 = $x\downarrow y = x NOR y = $  & Cтрелка Пи\'рса, НЕ-ИЛИ, \\
& & & & & $= NOR(x,y) = x $НЕ-ИЛИ & антидизъюнкция \\
& & & & & $y = $НЕ-ИЛИ$(x,y)$ & \\
\hline
\multirow{2}{*}{F} & \multirow{2}{*}{0} & \multirow{2}{*}{0} & \multirow{2}{*}{1} & \multirow{2}{*}{0} & F2,2 = $x\leftarrow /y$  & Отрицание \\
& & & & & & обратной импликации \\
\hline
\multirow{1}{*}{F} & \multirow{1}{*}{0} & \multirow{1}{*}{0} & \multirow{1}{*}{1} & \multirow{1}{*}{1} & F2,3 = $\neg x $  &  Отрицание\\
\hline
\multirow{3}{*}{F} & \multirow{3}{*}{0} & \multirow{3}{*}{1} & \multirow{3}{*}{0} & \multirow{3}{*}{0} & F2,4 = $x\rightarrow /y $  & Материальная \\
& & & & & & обратная импликация \\
\hline
\multirow{1}{*}{F} & \multirow{1}{*}{0} & \multirow{1}{*}{1} & \multirow{1}{*}{0} & \multirow{1}{*}{1} & F2,5 = $ \neg y$  & Отрицание \\
\hline
\multirow{4}{*}{F} & \multirow{4}{*}{0} & \multirow{4}{*}{1} & \multirow{4}{*}{1} & \multirow{4}{*}{0} & F2,6 =  $x \oplus y$ = & Сложение по модулю 2, \\
& & & & & = $x XOR y = XOR(x,y)$ = & исключающее "или" \\
& & & & & = $x >< y = x <> y$ = & сумма Жегалкина, \\
& & & & & = $x NE y = NE(x,y)$ & не равно \\
\hline
& & & & & F2,7  = x | y = & Штрих Шеффера, \\
F & 0 & 1 & 1 & 1 & = $NAND(x,y)$ = $x NAND$ & НЕ-И, 2И-НЕ, \\
& & & & & $y$ = $x$ НЕ-И $y$ = НЕ-И$(x,y)$ & антиконъюнкция\\
\hline
& & & & & F2,8 = $x \wedge y = x \cdot y =$ &  \\
\multirow{2}{*}{F} & \multirow{2}{*}{1} & \multirow{2}{*}{0} & \multirow{2}{*}{0} & \multirow{2}{*}{0} &  = $xy = x \& y = x AND y$ = & Конъюнкция,\\
& & & & & = $AND(x,y) = x $И$ y =$ & 2И, минимум \\
& & & & & = И$(x,y) = min(x,y)$	& \\
\hline
& & & & & F2,9 = $(x \equiv y) = x ~ y $= & Эквивалентность \\
F & 1 & 0 & 0 & 1 & = $x \leftrightarrow y = x EQV y = $ & равенство\\
& & & & & = $EQV(x,y)$ & \\
\hline
\multirow{1}{*}{F} & \multirow{1}{*}{1} & \multirow{1}{*}{0} & \multirow{1}{*}{1} & \multirow{1}{*}{0} & F2,10 = $y $  & Проекция, повторение\\
\hline
\multirow{2}{*}{F} & \multirow{2}{*}{1} & \multirow{2}{*}{0} & \multirow{2}{*}{1} & \multirow{2}{*}{1} & F2,11 = $x \to y = x  \supset y$ = & Импликация \\
& & & & &  = $x \le y = x LE y = LE(x,y)$ & следование  \\
\hline
\multirow{1}{*}{F} & \multirow{1}{*}{1} & \multirow{1}{*}{1} & \multirow{1}{*}{0} & \multirow{1}{*}{0} & F2,12 = $x $  & Проекция, повторение\\
\hline
\multirow{1}{*}{F} & \multirow{1}{*}{1} & \multirow{1}{*}{1} & \multirow{1}{*}{0} & \multirow{1}{*}{1} & F2,13 = $x \leftarrow y $  & Обратная импликация\\
\hline
& & & & & F2,14 = $x \vee y = x + y =$ &  \\
\multirow{2}{*}{F} & \multirow{2}{*}{1} & \multirow{2}{*}{1} & \multirow{2}{*}{1} & \multirow{2}{*}{0} & =$ x OR y = OR(x,y) = $ & Дизъюнкция,\\
& & & & & = $x$ ИЛИ $y$ = ИЛИ$(x,y)$ = & 2ИЛИ, максимум \\
& & & & & $ = max(x,y)$	& \\
\hline
\multirow{1}{*}{F} & \multirow{1}{*}{1} & \multirow{1}{*}{1} & \multirow{1}{*}{1} & \multirow{1}{*}{1} & F2,15 = $TRUE $  & Тавтология\\
\hline
\end{tabular}
\caption{Таблица значений и названий булевых функций от двух переменных}
\end{table}
\section{Обозначение на электрической схеме булевых функций}
Для аппаратной реализации булевых функций и их корректного обозначения на электрической схеме принято использовать логические элементы.
\\
\\\textbf{Логический элемент} – это простейшее устройство ЭВМ, выполняющее одну определенную логическую операцию над входными сигналами согласно правилам алгебры логики. То есть, одну из функций, рассмотренных ранее.
\begin{center}
  \textbf{Современные стандарты для условных графических обозначений (УГО) логических элементов}
\end{center}
\begin{itemize}
  \item ANSI (англ.\emph{American national standards institute} - американский национальный институт стандартов).
  \item MIL/IEC (англ.\emph{MILitary} – военный, {International Electrotechnical Commission} - международная электротехническая комиссия).
  \item DIN (нем. \emph{Deutsches Institut f{\"u}r Normung e. V.} — немецкий институт по стандартизации).
  \item ГОСТ 2.743-91, Единая система конструкторской документации. Обозначения условные графические в схемах. Элементы цифровой техники. 
\end{itemize}
\begin{table}[!h]
\centering
\begin{tabular}{|c|c|c|c|}
\hline
Название & IEC & ANSI \\
\hline
\multirow{3}{*}{НЕ, $\bar{A}$ (Инвертор)} & \multirow{3}{*}{\includegraphics[width=2cm]{7_2(1)}} & \multirow{3}{*}{\includegraphics[width=2cm]{7_2(2)}} \\
& & \\
& & \\
\hline
\multirow{3}{*}{И, $A \wedge B$} & \multirow{3}{*}{\includegraphics[width=2cm]{7_3(1)}} & \multirow{3}{*}{\includegraphics[width=2cm]{7_3(2)}} \\
& & \\
& & \\
\hline
\multirow{3}{*}{ИЛИ, $A \vee B$} & \multirow{3}{*}{\includegraphics[width=2cm]{7_5(1)}} & \multirow{3}{*}{\includegraphics[width=2cm]{7_5(2)}} \\
& & \\
& & \\
\hline
\multirow{3}{*}{Сумма по модулю 2, $A \oplus B$} & \multirow{3}{*}{\includegraphics[width=2cm]{7_4(1)}} & \multirow{3}{*}{\includegraphics[width=2cm]{7_4(2)}} \\
& & \\
& & \\
\hline
\end{tabular}
\caption{Обозначение некоторых булевых функций на эл. схеме}
\end{table}
%\begin{minipage}{\textwidth}
Это основные обозначения. Они могут совмещаться. Например, так будет выглядеть И-НЕ:
\\\includegraphics[width=2cm]{7_6}
%\end{minipage}
\\
\\\textbf{Цена функции по Квайну} – суммарное число входов логических элементов в составе схемы.
\\\textbf{Минимизация функции} – сокращение цены функции, с помощью преобразования её к более простому эквивалентному выражению.
Наиболее простой вид получается при сведении функции к постоянной - 1 (истина) или 0 (ложь).
\section{Логический базис}
\textbf{Логический базис} - набор булевых функций, позволяющий реализовать любую другую булеву функцию.
\\Три наиболее востребованных логических базиса: И, ИЛИ, НЕ; ИЛИ-НЕ; И-НЕ.
\begin{figure}[!h]
\centering
\caption{Пример реализации функций И, ИЛИ, НЕ в базисе И-НЕ}
\includegraphics[width=\textwidth]{7_7}
\end{figure}
\section{Формы записи математических выражений}
Форма записи по-другому называется нотацией.
\\"Арность операции" означает количество операндов, участвующих в операции.
\\Например: $\sqrt{A}$ (унарная), $A\times B$ (бинарная).
\\
\begin{center}
 \textbf{Виды нотаций}
\end{center}
\begin{description}
  \item[1489 г.] - инфиксная запись $A + B$
  \item[1920 г.] - префиксная (польская) запись $+AB$
  \item[1957 г.] - постфиксная (обратная польская) запись $AB+$
\end{description}
\textbf{Стек} - абстрактный тип данных, представляющий собой список элементов, организованных по принципу LIFO (англ. \emph{last in - first out}, "последним пришел - первым вышел"). Чаще всего принцип работы стека сравнивают со стопкой тарелок: чтобы взять вторую сверху, нужно снять верхнюю.
\subsection{Префиксная нотация}
\textbf{\emph{Пример:}}
\\Инфиксная нотация: $(A + B + C) - E^{D\times F\times G}$
\\Префиксная нотация: $-++ABC^{\wedge}E\times\times DFG$
\\
\\Рассмотрим, как же происходит запись в префиксной нотации:
\begin{enumerate}
 \item Исходное выражение: $(A + B + C) - E^{D\times F\times G}$. Расставляем порядок выполнения операций, согласно математическим правилам.
\item Последним выполняется вычитание, а именно из $(A + B + C)$ вычитается $E^{D\times F\times G}$. \textbf{Ставим знак в начало.} Получается $-(A + B + C)(E^{D\times F\times G})$.
\item Рассмотрим $(A + B + C)$, расставим порядок действий. Сначала происходит сложение $A$ и $B$, а потом $A+B$ и $C$. Начнем с последнего действия, получается: $+(A+B)C$. В $(A+B)$ происходит сложение $A$ и $B$, получаем $(+AB)$. Совместим все вместе: $(+(+AB)C)$.
\item Рассмотрим $(E^{D\times F\times G})$, происходит возведение в степень. А именно $E$ возводится в степень $D\times F\times G$. Получается $^{\wedge}E(D\times F\times G)$.
\item Рассмотрим $D\times F\times G$, расставляем порядок действий. Сначала происходит умножение $D$ и $F$, а потом $D\times F$ и $G$. Начнем с последнего действия, получается: $\times(D \times F)G$. В $(D \times F)$ происходит умножение $D$ и $F$, получаем $(\times DF)$. Совместим все вместе: $(\times(\times DF)G)$.
\item Совмещаем все вместе. Получается: $-(+(+AB)C)(^{\wedge}E(\times(\times DF)G)$. Убираем скобки. Получили $-++ABC^{\wedge}E\times\times DFG$.
\end{enumerate}
Существует популярная Lisp-разновидность префиксной нотации (Lisp - семейство языков программирования). И в ней запись будет выглядеть так: $(-(+ABC)(^{\wedge}E(\times DFG)))$
\begin{center}
  \textbf{Особенности префиксной нотации}
\end{center}
\begin{itemize}
  \item Не требуется скобок, если арность фиксирована.
  \item Запись выражения получается короче, чем инфиксная.
  \item Не требуется знать приоритет операций.
  \item Легко декодировать выражение с помощью стека.
  \item Малоприменима на практике (кроме Lisp).
\end{itemize}
\subsection{Постфиксная нотация}
\textbf{\emph{Пример:}}
\\Инфиксная нотация: $(A + B + C) - E^{D\times F\times G}$
\\Постфиксная нотация: $CAB++EGDF\times\times ^{\wedge}-$
\\
\\Рассмотрим, как же происходит запись в постфиксной нотации:
\begin{enumerate}
\item Исходное выражение: $(A + B + C) - E^{D\times F\times G}$. Расставляем порядок выполнения операций, согласно математическим правилам.
\item Последним выполняется вычитание, а именно из $(A + B + C)$ вычитается $E^{D\times F\times G}$. \textbf{Ставим знак в конец.} Получается $(A + B + C)(E^{D\times F\times G})-$.
\item Рассмотрим $(A + B + C)$, расставим порядок действий. Сначала происходит сложение $A$ и $B$, а потом $A+B$ и $C$. Начнем с последнего действия, получается: $С(A+B)+$. В $(A+B)$ происходит сложение $A$ и $B$, получаем $(AB+)$. Совместим все вместе: $(C(AB+)+)$.
\item Рассмотрим $(E^{D\times F\times G})$, происходит возведение в степень. А именно $E$ возводится в степень $D\times F\times G$. Получается $E(D\times F\times G)^{\wedge}$.
\item Рассмотрим $D\times F\times G$, расставляем порядок действий. Сначала происходит умножение $D$ и $F$, а потом $D\times F$ и $G$. Начнем с последнего действия, получается: $G(D \times F)\times$. В $(D \times F)$ происходит умножение $D$ и $F$, получаем $(DF\times)$. Совместим все вместе: $(G(DF\times )\times)$.
\item Совмещаем все вместе. Получается: $(C(AB+)+)(E(G(DF\times)\times)^{\wedge})-$. Убираем скобки. Получили $CAB++EGDF\times\times ^{\wedge}-$.
\end{enumerate}
\begin{center}
  \textbf{Особенности постфиксной нотации}
\end{center}
\begin{itemize}
  \item Не требуется скобок, если арность фиксирована.
  \item Запись выражения получается короче, чем инфиксная.
  \item Не требуется знать приоритет операций.
  \item Легко декодировать выражение с помощью стека.
  \item Успешно применяется в компиляторах, в небольшом количестве языков программирования и некоторых ЭВМ (калькуляторы "Электроника" и HP).
\end{itemize}
\begin{center}
  \textbf{Алгоритм вычисления для постфиксной нотации}
\end{center}
\begin{enumerate}
  \item Обработка входного символа
  \begin{itemize}
    \item Если на вход подан операнд, он помещается на вершину стека.
    \item Если на вход подан знак операции, то соответствующая операция выполняется над требуемым количеством значений, извлеченных из стека, взятых в порядке добавления. Результат выполненной операции кладется на вершину стека.
  \end{itemize}
  \item Если входной набор символов обработан не полностью, перейти к шагу 1.
  \item После полной обработки входного набора символов результат вычисления выражения лежит на вершине стека.
\end{enumerate}
\begin{figure}[!h]
\centering
\includegraphics[width=\textwidth]{7_8}
\caption{Вычисления выражения $ab+ac\times +$ (в инфиксной нотации $a\hm+b\hm+a\times c$)}
\end{figure}
Рассмотрим вычисление постфиксной нотации в стеке. Стек не имеет адресных операций. В нем есть две операции с переменными: $push$ (положить) и $pop$ (забрать). Если необходимо положить значение в стек, то он кладется с помощью команды $push$ на вершину стека. При этом, все другие значения сдвигаются вниз. Арифметические операции ($add$ - сложить, $mul$ - умножить) выполняются с переменными (или переменной - зависит от арности операции), которые лежат на вершине стека.
    % \section{Офисное ПО}
Компания Microsoft открыла формат doc (стандарт по которому он создается) только в 2000 г. До этого разработчикам приходилось методом обратного инженеринга вручную раскрывать этот стандарт. Это характеризует формат doc не в лучшую сторону - не все имели доступ (Microsoft Word - платное ПО) и формат проверен малым количеством людей. До сих пор существует много ошибок, которые исправлены только в docx.
\\Форматы odt и docx используют XML. Технология XML открыта - любой может посмотреть, найти ошибки, предложить исправление. Чтобы убедиться, что odt и docx используют XML, есть простой способ: создайте docx или odt документ, смените расширение на zip и распакуйте. Можно увидеть, что в основном все файлы имеют расширение xml.
\\Нормальная криптография появилась только в docx. В формате doc ничего нельзя было шифровать. Только сторонними приложениями.
\\
\begin{center}
Стоимость продуктов Microsoft Office
\end{center}
\begin{itemize}
  \item Для дома и учебы (Word, Excel, PowerPoint, OneNote) $\approx$ 3000р.
  \item Для дома и бизнеса (Word, Excel, PowerPoint, OneNote, Outlook) $\approx$ 10000р.
  \item Профессиональный (Word, Excel, PowerPoint, OneNote, Outlook, Access, Publisher) $\approx$ 20000р.
\end{itemize}
LibreOffice, OpenOffice, Calligra Suite = 0р.
\\
\\В 2010 году был принят ГОСТ Р ИСО/МЭК 26300-2010, обязывающий госучреждения перейти на бесплатный формат документов (Open Document Format - ODF). Но это вовсе не означает, что будет использоваться LibreOffice и OpenOffice. В последних версиях Microsoft Office есть поддержка этого формата.
\begin{figure}
\includegraphics[width=\textwidth]{8_1}
\caption{ГОСТ Р ИСО/МЭК 26300-2010}
\end{figure}
\begin{center}
\emph{ ODF}
\end{center}

\begin{itemize}
  \item Распоряжение Правительства Российской Федерации от 17 декабря 2010 г. №2299-р "О плане перехода федеральных органов исполнительной власти и федеральных бюджетных учреждений на использование свободного программного обеспечения (2011 - 2015 годы)"
  \item OpenDocument это единственный формат электронной документации, который реализован несколькими производителями ПО и утвержден ISO в качестве международного стандарта.
  \item Абсолютно любой производитель ПО может использовать формат OpenDocument при разработке своего собственного редактора
электронных документов.
  \item Продолжительный цикл существования формата OpenDocument обеспечен стабильностью спецификации.
  \item Открытость формата OpenDocument способствует росту его популярности среди производителей ПО, в числе которых есть производители бесплатных и открытых продуктов.
  \item Открытость формата OpenDocument позволяет пользователю быть свободным при выборе программного обеспечения, не зависеть от конкретного поставщика ПО и его маркетинговой политики.
  \item Исчезает угроза утери информации из-за изменения закрытых форматов
\end{itemize}
Недостатки ODF:
\begin{itemize}
  \item Спецификация  стандарта не определяет некоторые  элементы офисного программного обеспечения. По этой причине каждый разработчик может реализовывать эти элементы по-своему, что может привести к несовместимости соответствующих файлов.
  \item Новые версии стандарта выходят спустя большие промежутки времени. По этой причине недостатки текущей версии стандарта находят отражения в большем числе его реализаций.
  \item Некоторые специалисты считают, что Microsoft, с целью вернуть монопольное положение на рынке, может создать свою реализацию формата OpenDocument с закрытым кодом,  которая  станет неоспоримым лидером среди реализаций OpenDocument. Далее, по мнению этих специалистов,  на эту реализацию перейдёт основная масса пользователей и государственных органов – после этого Microsoft внесёт изменения в очередную версию продукта, благодаря которым файлы, созданные в этой программе перестанут быть совместимыми с форматом OpenDocument.
\end{itemize}
Работа по стандартизации OpenDocument включает в себя:
\begin{itemize}
  \item \textbf{OpenDocument 1.0} (второе издание) OASIS (англ. Organization for the Advancement of Structured Information Standards) соответствует опубликованному стандарту ISO/IEC 26300:2006. Содержание ISO/IEC 26300 и OASIS OpenDocument v1.0 2-е издание идентичны. Оно включает в себя изменения, внесенные редакционным решением JTC1 ( англ. Joint Technical Committee 1) и доступно в ODF, HTML и PDF форматах.
  \item \textbf{OpenDocument 1.1} включает дополнительные возможности для решения проблем доступности. Он был утвержден в качестве стандарта OASIS на 2007-02-01 после голосования, опубликованного 2007-01-16. Публичное заявление было сделано 2007-02-13. Эта версия не была изначально представлена ISO / IEC, потому что это считается незначительным обновлением только ODF 1.0, и OASIS работали уже на ODF 1.2 в тот момент, когда ODF 1.1 была утверждена. Однако позднее было представлено ISO / IEC (по состоянию на март 2011 года, он был в "стадии запроса'' как проект поправки 1- ISO/IEC 26300:2006/DAM 1) и опубликовано в марте 2012 года, как ISO / IEC 26300: 2006 / Amd 1: 2012 - Open Document Format for Office Applications (OpenDocument) v1.1.
  \item  \textbf{OpenDocument 1.2} был утвержден в качестве спецификации OASIS на 2011-03-17 и в качестве стандарта OASIS на 2011-09-29. Он включает в себя дополнительные функции доступности, RDF-метаданные, электронную таблицу спецификаций формул, основанную на OpenFormula, поддержку цифровых подписей и некоторые особенности, предложенные общественностью. В октябре 2011 года ожидалось, что технический комитет OASIS ODF "скоро начнет процесс предоставления ODF 1.2 в ISO / IEC JTC 1 ". В мае 2012 года преставители ISO / IEC JTC 1 / SC 34 / WG 6 сообщили, что после некоторой задержки, процесс подготовки ODF 1.2 для представления JTC 1 для PAS транспозиции ведется в настоящее время.
\end{itemize}
\subsection{Наиболее популярные офисные пакеты}

\begin{table}[!h]
 \begin{tabular}{|l|c|c|c|}
\hline
Название  &   & Примерная  &  \\
офисного  & Особенности &  стоимость на  & Исходный\\
пакета & & 2017 год, руб. &  код  \\
\hline
Google Docs, & Узкая ориентация & &\\
Яндекс.Диск, & на публичные &  бесплатно & закрытый \\
Облако Mail.ru  & облачные решения & & \\
\hline
 &  Имеет наиболее богатый &  &  \\
 Microsoft Office &  функционал, захватил  &  5000-35000&  закрытый\\
 & > 90\% desktop-установок  & & \\
\hline
LibreOffice,& Слабая поддержка & & \\
OpenOffice, & одновременного &  бесплатно & открытый \\
Calligra Suite & редактирования & & \\
\hline
 & Узкая ориентация на & & \\
iWork & на технику фирмы Apple &  бесплатно & закрытый \\
\hline
 & Интерфейс идентичен & 5000 & \\
WPS Office & Microsoft Office &  (0 с рекламой) & закрытый \\
\hline
WordPerfect & Узкая ориентация на рынок&  & \\
Office &  персональных компьютеров & 5000-25000 & закрытый \\

\hline
OnlyOffice & Приоритетная ориентация  & & \\
Feng Office & на частные и публичные &  бесплатно & открытый \\
  & офисные решения & & \\
\hline
 \end{tabular}
\caption{Наиболее популярные офисные пакеты}
\end{table}
В таблице рассмотрены не все 30 разновидностей офисных пакетов, а наиболее популярные. Популярность оценена  с помощью  сайта Trends Google, где отслеживается частота запросов пользователей со всего мира.
В таблице указаны офисные пакеты по убыванию популярности.
\subsection{Сравнение возможностей OO и MS Office}
\begin{table}[!h]
 \begin{tabular}{|l|c|c|}
 \hline
 Свойства & Open Office Calc & Microsoft Excel \\
 \hline
 Размерность & 1 024 $\times$ 1 048 576  & 16 384 $\times$ 1 048 576 \\
 \hline
 Кол-во цветов & 104 & 16 777 216 \\
  \hline
 Работа с & от 1 января 0001 г. & от 1 января 1900 г. \\
 датами & до 31 декабря 9999г. & до 31 декабря 9999г. \\
 \hline
 Поддержка  & met, pbm, pgm, ppm,  & cdr, emz, mix, pcz, \\
 графических & psd, ras, sbm, sgg, &  wmz, wpg, fpx,  \\
 форматов & svg, xpm, xbm &  drw \\
 \hline
 & Работа с MySQL. Макросы & Продвинутые сводные  \\
 Прочее & на разных языках & таблицы и условное \\
 & (Python, JavaScript) & форматирование  \\
  \hline
 \end{tabular}

\end{table}
\begin{center}
  \emph{Open Office Writer}
\end{center}
\begin{itemize}
  \item Частые обновления
  \item Независимые стили страниц в одном документе
  \item Автоматическое создание указателя формул
  \item Продвинутая навигация (по ссылкам, разделам, примечаниям, изображениям)
  \item Проверка орфографии любого количества языков в одном документе
  \item Вложенные, скрытые, защищенные паролем индивидуально оформленные разделы
  \item Перекрестные вычисления между разными таблицами в одном документе
  \item Несколько оглавлений в одном документе.
  \item Автоматизированное перемещение элементов оглавления и списков с подпунктами.
\end{itemize}
\begin{center}
  \emph{Microsoft Word}
\end{center}
\begin{itemize}
  \item Встроенные средства для продвинутой проверки грамматики русского языка
  \item Распознавание голоса и рукописного ввода
  \item Подробная справочная система с примерами
  \item Широкая распространенность
\end{itemize}
Если сравнивать \emph{производительность} OpenOffice Writer и Microsoft Word, то Writer уступает приблизительно в два раза.
\\Если сравнивать \emph{безопасность} OpenOffice и Microsoft Office, то Microsoft намного надежнее, чем OpenOffice (MS Office 2010: 17 крахов и 0 потенциальных уязвимостей, В OpenOffice 3.2.1: 163 краха и 18 потенциальных уязвимостей). Одна из причин - разработкой свободного ПО занимаются любители.
\subsection{Концепция стилей и шаблонов}
\begin{itemize}
  \item 1-я ошибка - форматирование вручную без стилей.
  \item 2-я ошибка - создание оформления вместо создания структуры.
  \item При подготовке документа главное то, чем текст является. А как он выглядит - вторично.
  \item Забыть про "размер шрифта 14pt"\, "гарнитура Times New Roman"\, "расположение по центру" и так далее.
  \item Помнить только стили: "Заголовок"\, "Заголовок $n$-ого уровня"\, "основной текст", и так далее.
  \item Создание нового документа начинается с продумывания структуры документа и создания системы стилей.
  \item Как будет выглядеть конечный документ, (шрифты, гарнитура, и так далее) решается, когда документ уже готов, путем изменения соответствующего стиля.
\end{itemize}
\subsection{Панграммы}
\textbf{Панграмма} (греч. "\emph{все буквы}") или разнобуквица - текст, использующий все или почти все буквы алфавита.
Используется для:
\begin{itemize}
  \item Демонстрация шрифтов.
  \item Проверки передачи текста по линиям связи.
  \item Тестирование печатающих устройств.
\end{itemize}
\begin{description}
  \item[Microsoft] Съешь [же] ещё этих мягких французских булок, да выпей чаю.
  \item[KDE] Широкая электрификация южных губерний даст мощный толчок подъёму сельского хозяйства.
  \item[Gnome] В чащах юга жил бы цитрус? Да, но фальшивый экземпляр!
\end{description}
\subsection{Автозаполнение}
\textbf{Lorem ipsum} - название классического текста-"рыбы".
\\\textbf{"Рыба"} - слово из жаргона дизайнеров, обозначает условный, зачастую бессмысленный текст, вставляемый в макет страницы.
\\
\\Lorem ipsum представляет собой искаженный отрывок из философского трактата Цицерона "О пределах добра и зла", написанного в 45 году до нашей эры на латинском языке. Впервые этот текст был применен для набора шрифтовых образцов неизвестным печатником в XVI веке.
\\
$$=rand(m, n)$$
Где:
\\$m$ – количество абзацев;
\\$n$ – количество предложений в каждом абзаце;
\\Так же
$$=lorem(m, n)$$
\subsection{Табличный процессор}
Табличный процессор офисного ПО обладает многими очень полезными функциями, которые заметно упрощают работу с электронными таблицами:
\begin{itemize}
  \item Запрет на ввод некорректных значений в ячейку.
  \item Условное форматирование.
  \item Фильтры для заполненных таблиц.
  \item Расчет доверительного интервала.
  \item Подбор параметра (решение уравнений, имеющих только единственное решение).
\end{itemize}
\section{Вспомогательное ПО для программирования}
\begin{itemize}
  \item Автоматическое создание документации для программы (doxygen).
  \item Контроль версий (SVN, Git, Mercurial).
  \item Управления жизненным циклом найденных ошибок (bug tracking system).
  \item Автоматизированное тестирование кода и функциональности.
\end{itemize}
\subsection{Автоматизированное создание документации}
Самая известная система для автоматизации создания документации программного обеспечения на С/С++ - это \emph{doxygen}. Используется в KDE, IBM, AbiWord, Adobe, DC++, Qt, \dots
\\При работе с Doxygen размечается код (в комментариях появляется собственный синтаксис, в результате чего комментарии преобразовываются в документацию).
\\
\\Такие программы значительно облегчают работу: не нужно отдельно создавать документацию. Более того, в комментариях все подробно описано и любой программист сможет разобраться в программе.
\subsection{Системы управления (контроля) версиями}
\begin{itemize}
  \item \textbf{Клиент-серверные (централизованные)}: CVS, Subversion, Microsoft SourceSafe, Perforce, VSS
  \item \textbf{Распределенные:} Mercurial, git
  \end{itemize}
\emph{Принцип работы}: пометка версий, которые отдаются пользователю, выкладываются на сайт (release версий) и версий для разработчиков, в которую возможно внести изменения (которые пользователю не отдаются). Это необходимо для того, чтобы редактировать новые версии (вносить изменения, тестировать), и, при необходимости, была возможность откатиться на старую версию. Программа
\\Преимущества Git над SVN: удобная работа с большим количеством веток, хранение всей истории изменения файлов проекта. Система управления хранит все предыдущие версии.
\subsection{Жизненный цикл обнаруженной ошибки}
\begin{description}
  \item[Тестировщик] находит ошибки;
  \item[Менеджер проекта] назначает того, кто исправит ошибку;
  \item[Программист] исправляет или объясняет, почему нельзя исправить (дубль; нет смысла исправлять; нельзя воспроизвести);
  \item[Тестировщик] проверяет, была ли исправлена ошибка.
\end{description}
\subsection{Тестирование программного обеспечения}
Самые известные СУБД ошибок: JIRA, Redmine, Bugzilla, TrackGear.
\begin{center}
  Описание ошибки
\end{center}
\begin{itemize}
  \item кто сообщил об ошибке;
  \item дата и время обнаружения;
  \item серьезность ошибки;
  \item перечень шагов воспроизведения ошибки;
  \item текущий статус ошибки.
\end{itemize}
\textbf{Автоматизированное тестирование программного обеспечения} - часть процесса тестирования на этапе контроля качества в процессе разработки программного обеспечения.
\\Оно использует программные средства для выполнения тестов и проверки результатов выполнения, что помогает сократить время тестирования и упростить его процесс.
\begin{center}
\textbf{Наиболее известный инструментарий для тестирования:}
\end{center}
\begin{itemize}
  \item JUnit — тестирование приложений для Java
  \item NUnit — порт JUnit под .NET
  \item xUnit — тестирование приложений для .NET
  \item TestNG — тестирование приложений для Java
  \item Selenium — тестирование приложений HTML
  \item WatiN — тестирование веб-приложений
  \item TOSCA Testsuite — тестирование приложений HTML, .NET, Java, SAP
  \item UniTESK — тестирование приложений на Java, Си.
\end{itemize}
\section{Лицензии}
\textbf{Лицензии на программное обеспечение} – это правовой инструмент, определяющий использование и распространение программного
обеспечения, защищенного авторским правом.
\begin{description}
  \item[Проприетарное ПО] $\Rightarrow$ закрытый исходный код. Может быть платным и бесплатным.
  \item[Свободное ПО] $\Rightarrow$ открытый исходный код. Может быть платным и бесплатным.
  \item[Коммерческое ПО ] $\Rightarrow$ платное. Может иметь как закрытый, так и открытый исходный код.
  \item[Бесплатное ПО] $\Rightarrow$ бесплатное. Может иметь как закрытый, так и открытый исходный код.
\end{description}
\begin{center}
Разновидности лицензий на свободное ПО
\end{center}
\begin{itemize}
  \item\textbf{Пермиссивные лицензии (BSD)}: можно менять и закрывать.
  \item\textbf{Копилефт (GPL)}: можно менять, нельзя закрывать.
\end{itemize}

\begin{wrapfigure}[12]{l}{3cm}
\includegraphics[width=3cm]{8_2}
\begin{center}
\footnotesize{Ричард Мэттью Столлман}
\\\footnotesize{$\mbox{род.} 1953$}
\end{center}
\end{wrapfigure}
Основоположником движения за открытый код является Ричард Мэттью Столлман. Он и создал первую лицензию GNU (General Public License).
\\Всего около 70 лицензий на владение свободным ПО (одобренных на opensource.org). Самые популярные: Apache License, BSD license, GPL, LGPL, MIT license, MPL.
\subsection{Базовые права, предоставляемые свободным ПО}
Все они предоставляют 4 базовых права:
\begin{itemize}
  \item Право на запуск программы в любых целях (только если она не нанесет вред своим действием или бездействием).
  \item Право на изучение исходного и бинарного кода программы
  \item Право на платное или бесплатное распространение программы.
  \item Право на развитие программы.
\end{itemize}
Если программист передает пользователю свою программу , но не прилагает лицензию, то действует "право свободного пользования":
\begin{itemize}
  \item Можно установить программу на 1 компьютер.
  \item Можно запускать программу на 1 компьютере.
  \item Нельзя копировать программу на другие компьютеры.
  \item Нельзя модифицировать программу.
  \item Данная лицензия действует 5 лет (п.4 ст. 1235 ГК РФ).
\end{itemize}
\subsection{Особенности различных свободных лицензий}
\begin{center}
  \textbf{GNU GPL}
\end{center}
\begin{itemize}
  \item Запрещено включать исходные тексты в закрытое ПО, запрещено менять тип лицензии (copyleft $\textcopyleft$).
  \item Запрещено динамическое связывание GNU GPL-библиотек с не-GNUGPL библиотеками (dll).
\end{itemize}
\begin{center}
  \textbf{GNU LGPL}
\end{center}
\begin{itemize}
  \item Допускается динамическое связывание с закрытыми библиотеками.
  \item Запрещено использование кода в другом ПО.
\end{itemize}
\begin{center}
  \textbf{MPL (Mozilla public license)}
\end{center}
\begin{itemize}
  \item Можно использовать исходные тексты в закрытом ПО, но
лишь частично и с гарантией доступа к изменениям.
\end{itemize}
\begin{center}
  \textbf{BSD License}
\end{center}
\begin{itemize}
  \item Можно использовать исходные коды в закрытом ПО без ограничений.
\end{itemize}
\subsection{Ответственность за пиратское ПО}
\begin{description}
  \item[Административная ответственность за пиратское ПО] Статья 7.12 КоАП РФ: нарушение авторских прав при ущербе на сумму до 100 000 рублей:
  \begin{itemize}
    \item штраф до 2 000 (физическое лицо)
    \item штраф до 20 000 (должностное лицо)
    \item штраф до 40 000 (юридическое лицо)
  \end{itemize}
  \item[Уголовная ответственность за пиратское ПО] Статья 146.2 УК РФ: незаконное использование объектов авторского права (в т.ч. приобретение, хранение) при ущербе на сумму от 100 000 рублей:
\begin{itemize}
  \item штраф до 200 000 р.
  \item исправительные работы вплоть до 2 лет
  \item – арест вплоть до 2 лет
\end{itemize}
 Статья 146.3 УК РФ: Незаконное использование объектов авторского права (в т.ч. приобретение, хранение) при ущербе на сумму от 1 000 000 рублей:
  \begin{itemize}
    \item штраф до 500 000 р.
    \item арест вплоть до 6 лет
  \end{itemize}
  \item[Уголовная ответственность за плагиат ПО] Статья 146.1 УК РФ: присвоение авторства, если это причинило крупный ущерб автору:
 \begin{itemize}
   \item штраф до 200 000 р.
   \item исправительные работы вплоть до 1 года
   \item арест вплоть до 6 месяцев
 \end{itemize}
  \item[Гражданская ответственность за нарушение лицензии ПО]  Статья 1301 ГК РФ: нарушение авторских, интеллектуальных и исключительных прав:
  \begin{itemize}
    \item штраф до 5 000 000 руб. в пользу обладателя ПО
    \\\textbf{либо}
    \item двукратное возмещение убытков обладателю ПО
  \end{itemize}
\end{description}
\section{Visual Basic for Applications}
\textbf{Visual Basic for Applications} (VBA, Visual Basic для приложений) — немного упрощенная реализация языка программирования Visual Basic, встроенная в линейку продуктов Microsoft Office (включая версии для Mac OS), а также во многие другие программные пакеты, такие как AutoCAD, SolidWorks, CorelDRAW, WordPerfect и ESRI ArcGIS. VBA покрывает и расширяет функциональность ранее использовавшихся специализированных макро-языков, таких как WordBasic.
\\
\\\textbf{Макрос} – программа, написанная на внутреннем для текстового процессора языке программирования, которую можно выполнять по желанию пользователя. Эту программу можно изменять, как и на другом языке программирования. Это позволяет автоматизировать повторяющиеся или сложные действия пользователя, которые отсутствуют в стандартном функционале текстового процессора. Таким образом можно расширять стандартный функционал.
\\
\\Использование макросов предполагает знание встроенного в текстовый процессор языка программирования. Однако современные текстовые процессоры позволяют пользователям, не знающим язык макросов, записывать свои действия для дальнейшего их преобразования в макрос. Действия пользователя автоматически описываются с помощью встроенного языка программирования.
\\
\\Автоматической записи макроса часто бавает недостаточно, поэтому советуем ознакомиться с основами языка программирования макросов в линейке продуктов Microsoft Office Visual Basic for Applications.
\subsection{Имя переменной}
\begin{itemize}
  \item Начинается с буквы латинского алфавита.
  \item Не может содержать пробелы, точки символы операций (+, -, *, /, \#, \$, \%, \&, !, <, >, = и так далее).
  \item Не может превышать 254 символов в длину.
  \item Должно быть уникальным в своей области действия.
  \item Не может дублировать зарезервированные слова.
  \item Не различает регистр букв: MyNumber = mYnUmBeR.
\end{itemize}
\subsection{Типы данных}
\begin{minipage}{\textwidth}
\centering
\begin{tabular}{|l|c|c|c|}
\hline
Тип данных & Резервируемая & Минимальное & Максимальное \\
& память, байт & значение & значение \\
\hline
Byte & 1 & 0 & 255 \\
Boolean & 2 & False & True \\
Integer & 2 & -32768 & 32767 \\
Long & 4 & -2147483648 & 2147483647 \\
Date & 8 & 1 января 100 г. & 31 декабря 9999 г. \\
String & Длина строки & 1 & 65400 \\
Variant & \multirow{2}{*}{16} & & \\
(число) & & & \\
Variant & 22 байта + & & 2147483647 \\
(символ) & длина строки & & символов \\
\hline
\end{tabular}
\end{minipage}
\begin{center}
 \textbf{Объявление переменных}
\end{center}
\begin{enumerate}
  \item \textbf{Неявное}:
  \\sum = 100
  \\В данном случае присваивается тип Variant. Это обобщенный тип, переменная нетипизирована.
  \item \textbf{Явное}:
  \\Dim sum As Integer
  \\\emph{Преимущества}:
  \begin{itemize}
    \item Программа быстрее работает.
    \item Программе требуется меньше памяти.
    \item Легче обнаружить некоторые ошибки.
   \item Не возникает проблем со сложными типами (например, как отличить дату от текста).
  \end{itemize}
  \emph{Недостатки}:
  \begin{itemize}
    \item Приходится думать.
    \item Требуется использовать больше переменных.
  \end{itemize}
\end{enumerate}
\section{\TeX}
При взаимодействии человека с компьютером возникает вопрос: каким же образом представлять документы на экране и затем в памяти компьютера и в печатном виде. На этот счет существует 2 парадигмы:
\begin{itemize}
\item \textbf{WYSIWYG} (англ. \emph{What You See Is What You Get} - "что видишь, то и получишь") - свойство прикладных программ или веб-интерфейсов, в которых содержание отображается в процессе редактирования и выглядит максимально близко похожим на конечную продукцию, которая может быть печатным документом, веб-страницей или презентацией. В настоящее время для подобных программ также широко используется понятие "визуальный редактор". (пример: Microsoft Word)
\item \textbf{WYSIWYM} (англ. \emph{What You See Is What You Mean} - "что видишь, есть то, что имеешь в виду") - парадигма редактирования документов, возникшая как альтернатива более распространенной парадигме WYSIWYG. В WYSIWYM редакторе пользователь задает только логическую структуру документа и собственно контент. Оформление документа, его итоговый внешний вид возложено на отдельное ПО, либо, во всяком случае, вынесено в отдельный блок. Таким образом достигается полная независимость содержания документа
от его формы. (пример: \TeX)
\end{itemize}
\textbf{\TeX} - система компьютерной верстки, разработанная американским профессором информатики Дональдом Кнутом в целях создания компьютерной типографии. В нее входят средства для секционирования документов, для работы с перекрестными ссылками. Номер версии \TeX приближается к $\pi$, номер редактора формул - к числу $e$.
\subsection{Сравнение \LaTeX с WYSIWYG редакторами}
\LaTeX (произносится латех) — наиболее популярный набор макрорасширений (или макропакет) системы компьютерной вёрстки \TeX, который облегчает набор сложных документов. Разработан Лэсли Лэмпортом в 1984 году и назван в его честь.
\begin{table}
\begin{tabular}{|l|l|l|}
\hline
Критерий & \LaTeX & MS Word, \\
 & & LibreOffice Writer \\
\hline
 Работа& Линейное текстовое & Переключение между \\
с формулами &  представление формул & линейным и математич.  \\
 &  &  видом формулы \\
\hline
 Дизайнерские& Текстовый векторный & Встроенный векторный \\
 задачи& редактор, сложно&  WYSIWYG редактор, \\
&  управлять располо-& наглядное управление \\
&жением графики & структурой доукумента \\
\hline
Порог вхождения& Высокий& Низкий \\
\hline
Написание& Является мировым& Большинство  \\
научных статей& стандартом& российских журналов \\
\hline
Стоимость&Бесплатно&LibreOfice бесплатно \\
\hline
Рецензирование&Нет, можно&Продвинутые встроенные \\
текстов& комментировать &возможности \\
\hline
Экспорт в &\multicolumn{2}{l|}{Полноценной бесплатной утилиты для } \\
 другие форматы& \multicolumn{2}{l|}{экспорта из docx в tex и обратно не существует} \\
\hline
Требования к& Очень низкие:& \\
аппарат. обеспечению& достаточно консоли & Высокие\\
\hline
Автогенерация&Удобно генерировать& Требуется знание\\
документов&отчеты изнутри&  сложной структуры \\
 &работающих программ&  docx (odt) \\
\hline
Коллективная& &Поддерживается MSW\\
работа с файлами&Разрабатывается&  по умолчанию\\
\hline
Количество & Мало&Много \\
квалифицированных&(ученые, пользователи с& (подавляющее \\
пользователей&технич. образованием)& большинство людей) \\
\hline
Кросс-платфор-& & Большинство \\
менность ПО&Любая ОС & популярных ОС  \\
 для редактирования & &  с наличием GUI \\
\hline
Кросс-платфор-&Обратная &Проблемы при \\
менность&совместимость & обновлении версий \\
формата файла &хорошо обеспечена & \\
\hline
Проверка пунктуации&Отстутвует&Доступна по\\
и грамматики&(утилита hunspell)&умолчанию \\
\hline
Автоматизация&Отсутствует (можно& Встроенная поддержка  \\
повтор. действий &использовать скрипты) & макросов \\
\hline
Работа с & Нет ограничений&Компьютер  \\
большими файлами& & "подвисает" \\
\hline
Кодировка& На выходе получается& Проблемы из-за \\
 &PDF файл& несоответствия кодировок \\
\hline
\end{tabular}
\caption{Сравнение \LaTeX с WYSIWYG редакторами}
\end{table}
\section{Вебинары}
\textbf{Вебинар (онлайн-семинар)} - разновидность веб-конференции, проведение онлайн-встреч или презентаций через Интернет. Во время веб-конференции каждый из участников находится у своего компьютера, а связь между ними поддерживается через Интернет посредством загружаемого приложения, установленного на компьютере каждого участника, или через веб-приложение.
\\1988 г. – появление первых IRC (англ. Internet Relay Chat).
\\Середина 1990-х – появление и распространение IM (Instant Messaging).
\\1998 г. – регистрация торгового знака "Webinar" Эриком Р. Корбом (Eric R. Korb).
\begin{center}
  Существуют следующие приложения для вебинаров
\end{center}
\begin{itemize}
  \item GoToMeeting
    \begin{itemize}
    \item Создана в 2004 году компанией Citrix Online.
    \item Поддерживаемые ОС: Macintosh, Microsoft Windows
    \item http://www.gotomeeting.com/
  \end{itemize}
  \item StartMeeting
    \begin{itemize}
    \item Создана в 2011 году как start-up.
    \item Поддерживаемые ОС: Microsoft Windows
    \item http://www.startmeeting.com/
  \end{itemize}
  \item Team Viewer
    \begin{itemize}
    \item Создана в 2005 году.
    \item Поддерживаемые ОС: Windows, Mac OS, Linux, Android, Apple iOS, Windows Phone
    \item http://www.teamviewer.com
  \end{itemize}
\end{itemize}

    % Люди пытались сделать компьютер достаточно давно.
\\Первым (после антикитерского механизма) был \textbf{механизм для суммирования и умножения}. Изобрел его Вильгельм Шиккард (\emph{1592 - 1635}) в 1623 году.
\\Машина содержала суммирующее и множительное устройства, а также механизм для записи промежуточных результатов. Первый блок - шестиразрядная суммирующая машина - представлял собой соединение зубчатых передач. На каждой оси имелись шестерня с десятью зубцами и вспомогательное однозубое колесо - палец. Палец служил для того, чтобы передавать единицу в следующий разряд (поворачивать шестеренку на десятую часть полного оборота, после того как шестеренка предыдущего разряда сделает такой оборот). При вычитании шестеренки следовало вращать в обратную сторону. Контроль хода вычислений можно было вести при помощи специальных окошек, где появлялись цифры. Для перемножения использовалось устройство, чью главную часть составляли шесть осей с "навернутыми" на них таблицами умножения.
\\Второй арифметической машиной был \textbf{механизм для суммирования и вычитания "Паскалина"}.
%"Паскал\'ина" } - прописать ударение!!!!
Изобрел его Блез Паскаль (\emph{1623—1662}) в 1642 году.
\\Машина Паскаля представляла собой механическое устройство в виде ящичка с многочисленными связанными одна с другой шестеренками. Складываемые числа вводились в машину при помощи соответствующего поворота наборных колесиков. На каждое из этих колесиков, соответствовавших одному десятичному разряду числа, были нанесены деления от 0 до 9. При вводе числа, колесики прокручивались до соответствующей цифры. Совершив полный оборот, избыток над цифрой 9 колесико переносило на соседний разряд, сдвигая соседнее колесо на 1 позицию.
\\Следующим был \textbf{арифмометр Лейбница}, умеющий выполнять операции сложения, вычитания, деления и умножения. Изобрел ее Готфрид Вильгельм Лейбниц (\emph{1646 - 1716}) в 1673 году.
\\Сложение чисел выполнялось при помощи связанных друг с другом колес, так же как на "Паскалине". Добавленная в конструкцию движущаяся часть и специальная рукоятка, позволявшая крутить ступенчатое колесо (в последующих вариантах машины - цилиндры), позволяли ускорить повторяющиеся операции сложения, при помощи которых выполнялось деление и перемножение чисел. Необходимое число повторных сложений выполнялось автоматически.
\\Прообразом современного компьютера стала \textbf{идея создания универсальной аналитической вычислительной машины}, которую выдвинул Чарльз Бэббидж (\emph{1791 - 1871}) в 1823 году.
\\В 1822 году Бэббидж построил \textbf{малую разностную машину}. Ее работа была основана на методе конечных разностей. Малая машина была полностью механической и состояла из множества шестеренок и рычагов. В ней использовалась десятичная система счисления. Она оперировала 18-разрядными числами с точностью до восьмого знака после запятой и обеспечивала скорость вычислений 12 членов последовательности в 1 минуту. Малая разностная машина могла считать значения многочленов 7-й степени.
\\И в том же 1822 году Бэббидж задумался о создании \textbf{большой разностной машины} предназначенной для автоматизации вычислений путем аппроксимации функций многочленами и вычисления конечных разностей. Возможность приближенного представления в многочленах логарифмов и тригонометрических функций позволяло бы рассматривать эту машину как довольно универсальный вычислительный прибор. В 1823 году он приступил к проектированию, однако, в 1842 году государство отказалось финансировать проект и машина так и не была достроена. Но для развития вычислительной техники имело значение другое: идея создания аналитической вычислительной машины. В единую логическую схему Бэббидж увязал арифметическое устройство (названное им "мельницей"), регистры памяти, объединенные в единое целое ("склад"), и устройство ввода-вывода, реализованное с помощью перфокарт трех типов. Перфокарты операций переключали машину между режимами сложения, вычитания, деления и умножения. Перфокарты переменных управляли передачей данных из памяти в арифметическое устройство и обратно. Числовые перфокарты могли быть использованы как для ввода данных в машину, так и для сохранения результатов вычислений, если памяти было недостаточно.
\\Следующим этапом в развитии вычислительной техники стал \textbf{электромеханический перфокарточный табулятор для переписи населения}, который изобрел Герман Холлерит (\emph{1860 - 1929}) в 1880 году.
\\Табуляторы предназначены для автоматической обработки (суммирования и категоризации) числовой и буквенной информации, записанной на перфокартах, с выдачей результатов на бумажную ленту или специальные бланки. Умножение и деление выполнялись методом последовательного многократного сложения и вычитания. Работа табулятора производилась в соответствии с набираемой на коммутационной панели программой.
\\В 1911 году Алексей Николаевич Крылов (\emph{1863 - 1945}) изобрел \textbf{аналоговый решатель дифференциальных уравнений}. Он интегрировал обыкновенные дифференциальные уравнения.
\\В 1919 году Николай Николаевич Павловский (\emph{1884 - 1937}) сконструировал \textbf{аналоговую вычислительную машину (АВМ)}. Она была создана для реализации метода исследования природных явлений при помощи аналого-математического моделирования, который разработал Павловский в том же 1919 году.
\\\\Наличие заданного набора исполняемых команд и программ было характерной чертой первых компьютерных систем. Сегодня подобный дизайн применяют с целью упрощения конструкции вычислительного устройства. Так, настольные калькуляторы, в принципе, являются устройствами с фиксированным набором выполняемых программ. Их можно использовать дляматематических расчётов, но невозможно применить для обработки текста и компьютерных игр, для просмотра графическихизображений или видео. Изменение встроенной программы для такого рода устройств требует практически полной их переделки, и в большинстве случаев невозможно. Впрочем, перепрограммирование ранних компьютерных систем всё-таки выполнялось, однако требовало огромного объёма ручной работы по подготовке новой документации, перекоммутации и перестройки блоков и устройств и т. п.
\\\\Всё изменила идея хранения компьютерных программ в общей памяти. Ко времени её появления использование архитектур, основанных на наборах исполняемых инструкций, и представление вычислительного процесса как процесса выполнения инструкций, записанных в программе, чрезвычайно увеличило гибкость вычислительных систем в плане обработки данных. Один и тот же подход к рассмотрению данных и инструкций сделал лёгкой задачу изменения самих программ.

\section{ЭВМ Джона фон Неймана}
\begin{minipage}[l]{3cm}
\includegraphics[width=3cm]{9_1}
\begin{center}
\footnotesize{Джон фон Нейман}
\\\footnotesize{$1903 - 1957$}
\end{center}
\end{minipage}
\hfill
\begin{minipage}[r]{7.5cm}
В 1930-х годах началась разработка архитектуры ЭВМ для военно-морской артиллерии по заказу правительства США. В разработке участвовали Гарвардский университет и Принстонский университет (в том числе и Джон фон Нейман). На рисунке \ref{tag:EVM_von_Neumann} приведена схема ЭВМ, предложенная фон Нейманом.
\end{minipage}
\\
\begin{figure}[h]
\includegraphics[width=\textwidth]{9_2}
\caption{Структурная схема ЭВМ фон Неймана}
\label{tag:EVM_von_Neumann}
\end{figure}
\subsection{Узлы ЭВМ фон Неймана}
\begin{itemize}
  \item \textbf{Процессор} - исполнитель машинных инструкций (кода программ), главная часть аппаратного обеспечения ЭВМ. В состав процессора входят:
      \begin{itemize}
        \item устройство управления выборкой команд из памяти и их выполнением;
        \item арифметико-логическое устройство, производящее операции над данными;
        \item регистры, осуществляющие временное хранение данных и состояний процессора;
        \item схемы для управления и связи с подсистемами памяти и ввода-вывода.
      \end{itemize}
  \item \textbf{Устройства ввода} обеспечивают считывание данных с носителей информации и ее представление в форме электрических сигналов, воспринимаемых другими устройствами ЭВМ (процессором или памятью) (мышь, клавиатура, сканер).
  \item \textbf{Устройства вывода} представляют результаты обработки данных в ЭВМ в форме, удобной для визуального восприятия человеком (монитор, принтер) или хранения (DVD-привод, стример). При необходимости они обеспечивают запоминание результатов на носителях, с которых эти результаты могут быть снова введены в ЭВМ для дальнейшей обработки (перфоленты, магнитная лента, магнитный диск и т. п.), или передачу результатов на исполнительные органы управляемого объекта (например, робота). 
  \item \textbf{Устройства ввода-вывода} используются как для хранения данных, так и для их считывания (жесткий диск, флешка).
\end{itemize}
\subsection{Принципы работы архитектуры фон Неймана}
Бёркс, Голдстайн и фон Нейман в 1946 г. в книге "Предварительное рассмотрение логического конструирования электронного вычислительного устройства"  описали принципы:
\begin{itemize}
  \item \textbf{Принцип двоичного кодирования} - вся информация, поступающая в ЭВМ, кодируется с помощью двоичных сигналов.
  \item \textbf{Принцип однородности памяти} - программы и данные хранятся в одной и той же памяти. Поэтому ЭВМ не различает, что хранится в данной ячейке памяти - число, текст или команда. Над командами можно выполнять такие же действия, как и над данными.
  \item \textbf{Принцип адресуемости памяти} - структурно основная память состоит из пронумерованных ячеек, процессору в произвольный момент времени доступна любая ячейка.
  \item \textbf{Принцип жесткости архитектуры} - неизменяемость в процессе работы топологии, архитектуры, списка команд.
  \newpage
  \item \textbf{Принцип программного управления}:
  \begin{enumerate}
    \item В начале процессору сообщается адрес первой команды программы (который заносится в специальный \textbf{регистр команд}), после этого программа управляет сама собой.
    \item После выполнения команды процессор увеличивает адрес, хранимый в регистре команд, на длину только что выполненной команды, чтобы получить адрес следующей команды. Так можно выполнить цепочку команд из \textbf{последовательно} расположенных ячеек памяти.
    \item Существуют специальные \textbf{команды переходов}, которые сразу содержат в себе адрес следующей команды. После выполнения таких команд указанный адрес просто заносится в регистр команд. Так можно выполнить цепочку команд из \textbf{непоследовательно} расположенных ячеек памяти.
  \end{enumerate}
\end{itemize}
\section{Классификация архитектур ЭВМ}
\textbf{Архитектура ЭВМ} - концептуальная структура вычислительной машины, определяющая проведение обработки информации и включающая методы преобразования информации в данные и принципы взаимодействия технических средств и программного обеспечения. \textbf{Архитектурой ЭВМ} определяется, как именно в этой ЭВМ происходит обработка и преобразование данных с учетом конкретных принципов взаимодействия технических средств и программного обеспечения.
\\В 30-х годах правительство США поручило Гарвардскому и Принстонскому университетам разработать архитектуру ЭВМ для военно-морской артиллерии. Победила разработка Принстонского университета (более известная как архитектура фон Неймана, названная так по имени разработчика, первым предоставившего отчет об архитектуре), так как она была проще в реализации. Гарвардская архитектура использовалась советским учёным А. И. Китовым.
\\В настоящее время наибольшее распространение в ЭВМ получили эти 2 типа архитектуры: \emph{принстонская (неймановская)} и \emph{гарвардская}. Обе они выделяют 2 основных узла ЭВМ: центральный процессор и память компьютера. Различие заключается в структуре памяти:
\begin{itemize}
  \item \textbf{Принстонская архитектура}: программы и данные хранятся в одном массиве памяти (микросхеме) и передаются в процессор по одному каналу связи (шине). Проще реализовать (сконструировать), гибкость модификации программ.
  \item \textbf{Гарвардская архитектура}: предусматривает раздельные хранилища и потоки передачи (шины) для команд и данных. Возможность одновременной работы с данными и командами.
\end{itemize}
В более подробное описание, определяющее конкретную архитектуру, также входят: структурная схема ЭВМ, средства и способы доступа к элементам этой структурной схемы, организация и разрядность интерфейсов ЭВМ, набор и доступность регистров, организация памяти и способы её адресации, набор и формат машинных команд процессора, способы представления и форматы данных, правила обработки прерываний. \\
\\По перечисленным признакам и их сочетаниям среди архитектур выделяют:
\begin{enumerate}
  \item \textbf{По разрядности интерфейсов и машинных слов}: 8-, 16-, 32-, 64-, 128-разрядные.
  \item \textbf{По особенностям набора команд и регистров}:
  \begin{itemize}
    \item CISC – Complete Instruction Set Computer
    \item RISC – Restricted (Reduced) Instruction Set Computer
    \item CRISP – Complex-Reduced-Instruction-Set Processor
    \item VLIW – Very Long Instruction Word
  \end{itemize}
  \item \textbf{По количеству вычислителей}: однопроцессорные, многопроцессорные, одноядерные, многоядерные.
  \item \textbf{Многопроцессорные по принципу взаимодействия с памятью}: симметричные многопроцессорные (SMP), масcивно-параллельные (MPP), распределенные.
\end{enumerate}
Существуют следующие архитектуры систем команд (рисунок \ref{commands}):
\begin{itemize}
  \item \textbf{Регистровая архитектура} - внутри процессора существует специальная память (регистры) для хранения промежуточных результатов.
  \item \textbf{Аккумуляторная архитектура} - существует один регистр (аккумулятор), в котором хранится результат.
  \item \textbf{Стековая архитектура (MISC } - Minimal Instruction Set Computer) - команды не имеют операндов.
\end{itemize}
Типичные операции (сложение и умножение) требуют от любого вычислительного устройства нескольких действий: выборку двух операндов, выбор инструкции и её выполнение, и, наконец, сохранение результата. Идея, предложенная и  реализованная Эйкеном, заключалась в физическом разделении линий передачи команд и данных. В первом компьютере Эйкена «Марк I» для хранения инструкций использовалась перфорированная лента, а для работы с данными — электромеханические регистры. Это позволяло одновременно пересылать и обрабатывать команды и данные, благодаря чему значительно повышалось общее быстродействие.
\\Соответствующая схема реализации доступа к памяти имеет один очевидный недостаток — высокую стоимость. При разделении каналов передачи команд и данных накристалле процессора последний должен иметь почти в два раза больше выводов (так как шины адреса и данных составляют основную часть выводов микропроцессора). Способом решения этой проблемы стала идея использовать общую шину данных и шину адреса для всех внешних данных, а внутри процессора использовать шину данных, шину команд и две шины адреса. Такую концепцию стали называть \textbf { \emph{модифицированной Гарвардской архитектурой.}}
\\Такой подход применяется в современных сигнальных процессорах. Еще дальше по пути уменьшения стоимости пошли при создании однокристалльных ЭВМ - \textbf{микроконтроллеров}. В них одна шина команд и данных применяется и внутри кристалла.
\\Разделение шин в \emph{модифицированной Гарвардской структуре} осуществляется при помощи раздельных управляющих сигналов: чтения, записи или выбора области памяти.
\\\\В процессоре ограниченное количество команд, как и в языках программирования. И каждую операцию он разбивает на более мелкие и простые микрокоманды. Архитектуры CISC и RISC определяют количество этих микрокоманд. Например, есть всего 10 команд, а все остальные операции можно составить из этих команд - это архитектура RISC. Или на каждую возможную операцию необходима своя команда - это будет архитектура CISC. Рассмотрим подробнее.
\subsection{Архитектура CISC}
\textbf{CISC} - complex instruction set computer - компьютер с полным набором команд.
\begin{itemize}
  \item много команд;
  \item мало регистров общего назначения (памяти для хранения операндов арифметико-логических инструкций, а также адресов или отдельных компонентов адресов ячеек памяти) (до 32);
  \item разнообразие способов адресации (прямая, косвенная);
  \item много форматов команд различной разрядности;
  \item обработка совмещается с обращением к памяти;
  \item плавающая длина команд.
\end{itemize}
Доля сложных дополнительных команд CISC в общем объеме программ не превышает 10-20\%
. Емкость микропрограммной памяти для поддержании сложных команд может увеличиваться на 60\%
.
\subsection{Архитектура RISC}
\textbf{RISC} - reduced instruction set computer - компьютер с сокращенным набором команд.
\begin{itemize}
  \item мало команд (только наиболее часто используемые);
  \item много регистров общего назначения (РОН) (сотни);
  \item есть только две команды обращения к памяти (все остальные команды могут работать только с РОНами);
  \item мало форматов команд и способов указания адресов операндов;
  \item фиксированная длина команд (упрощает обработку).
\end{itemize}
\subsection{Преимущества RISC над CISC}
\begin{enumerate}
  \item Возможность повышения тактовой частоты и упрощения кристалла с высвобождением площади под кэш.
  \item Снижение энергопотребления процессора засчет уменьшения числа транзисторов.
  \item Доля сложных дополнительных команд CISC в общем объеме программ не превышает 10-20\%, а остальные операции похожи на соответствующие аналоги RISC-архитектуры.
  \item Возможность упреждающего выполнения команд.
\end{enumerate}
\textbf{Результат}: большинство современных процессоров являются либо чистыми RISC, либо "CISC-поверх-RISC".
\begin{figure}[h]
\centering
\includegraphics[width=10cm]{9_3}
\caption{Классификация систем команд}
\label{commands}
\end{figure}
\section{Принципы построения ЭВМ}
Перечислим и обощим принципы построения ЭВМ в целом, сформулированные так же фон Нейманом:
\begin{itemize}
  \item наличие единого вычислительного устройства, включающего процессор, средства передачи информации и память;
  \item линейная структура адресации памяти, состоящей из слов фиксированной длины;
  \item двоичная система исчисления;
  \item централизованное последовательное управление, устройство управления выборкой команд из памяти и их выполнением; ;
  \item хранимая программа;
  \item низкий уровень машинного языка;
  \item наличие команд условной и безусловной передачи управления;
  \item АЛУ с представлением чисел в форме с плавающей точкой и производящее операции над данными;;
  \item регистры, осуществляющие временное хранение данных и состояний процессора.
\end{itemize}
\section{Команды процессора}
\begin{figure}[h]
\centering
\includegraphics[width=\textwidth]{9_4}
\caption{Форматы команд процессора: а) многоадресная, б) адресная, в) безадресная}
\end{figure}
\begin{center}
  \textbf{Этапы выполнения команд процессором}
\end{center}
Полный цикл выполнения команды может включать в себя следующие этапы:
\begin{itemize}
  \item Вычисление адреса команды (ВАК) - счетчик команд указывает на некоторую ячейку, содержащую команду;
  \item Выборка команды (ВК) - обращение к ячейке памяти по адресу, указанному в счетчике команд, считывание команды;
  \item Декодирование команды (ДК) - процессор распознает, что за команда (сложение, вычитание, переход и так далее);
  \item Вычисление адреса операнда (ВАО) - если команда адресная, то производится вычисление адреса операнда, который содержит команда;
  \item Выборка операнда (ВО) - обращение к ячейке памяти по адресу, указанному в команде, считывание операнда;
  \item Выполнение заданной операции (ВЗО);
  \item Запись результата (ЗР).
\end{itemize}
\begin{center}
  \textbf{Конвейерная обработка команд}
\end{center}
При выполнении почти каждой команды, необходимо осуществить 4-7 действий. При выполнении команд последовательно (как это делали старые процессоры) следующая команда будет ждать, пока полностью осуществится первая. Понятно, что можно выполнять команды по принципу конвейера, что существенно увеличит скорость работы.
\begin{figure}[h]
\centering
Пример 5-уровневого конвейера в реальном RISC-процессоре:
\includegraphics[width=\textwidth]{9_5}
\end{figure}
\begin{description}
  \item[IF] - ВАК+ВК (вычисление адреса команды и выборка команды);
  \item[ID] - ДК+ВАО+ВО (декодирование команды, вычисление адреса операнда, выборка операнда);
  \item[EX] - ВЗО (выполнение заданной операции);
  \item[MEM] - ЗР (запись результата);
  \item[WB] - ЗР (запись результата);
\end{description}
Видно, что пока первая команда находится на этапе WB, задействован один узел - записи результата. Тем временем другой узел - выполнения заданной операции, может не простаивать, а выполнять третью команду. И таким образом, все отдельные независящие друг от друга узлы, работают одновременно, выполняя этапы разных команд, чем значительно увеличивают производительность.
    % \section{Устройство памяти}
\begin{enumerate}
 \item Память состоит из адресуемых ячеек (размером 1 - 128 бит). Ячейки в основном организованы по 8 бит и свой собственный адрес имеет только каждый восьмой бит.
 \item Ячейки состоят из запоминающих электрических элементов. Это может быть конденсатор, транзистор, попупроводниковый материал и т.п., умеющий хранить два состояния (заряжен или разряжен - конденсатор, находится в проводящем или непроводящем состоянии - транзистор, имеет высокое или низкое удельное сопротивление - попупроводниковый материал). Есть так же элементы, умеющие хранить три состояния, но эти три состояния сложнее считывать.
 \item Электрический элемент может находиться в одном из двух устойчивых состояний (для хранения 1 бита):
 \begin{itemize}
   \item конденсатор заряжен/разряжен;
   \item транзистор в проводящем/непроводящем состоянии;
   \item полупроводниковый материал имеет высокое/низкое сопротивление.
 \end{itemize}
\end{enumerate}
Одно из таких физических состояний создает высокий уровень выходного напряжения элемента памяти, а другое - низкий. В элементах памяти ряда микроЭВМ это электрические напряжения порядка 4В и 0В соответственно, причем первое обычно принимается за двоичную единицу, а второе - за двоичный нуль (возможно и обратное кодирование.)
\begin{figure}[h]
\includegraphics[width=\textwidth]{10_1}
\end{figure}
\\На рисунке показан выходной сигнал элемента памяти (например, одного разряда регистра) при изменении его состояний (при переключениях) под воздействием некоторого входного сигнала. Хотя переход от 0 к 1 и от 1 к 0 происходит не мгновенно, однако в определенные моменты времени этот сигнал достигает значений, которые воспринимаются элементами ЭВМ как 0 или 1. Если сигнал попадает в незаштрихованную область (между 0,5В и 2,5В) то сигнал не распознается.
\\
\\\textbf{Запоминающим элементом} называется элемент, который способен принимать и хранить код двоичной цифры (исходные значения некоторых величин, промежуточные значения обработки и окончательные результаты вычислений). Элементы памяти могут запоминать и сохранять исходные значения некоторых величин, промежуточные значения обработки и окончательные результаты вычислений. Только запоминающие элементы в схемах ЭВМ позволяют проводить обработку информации с учетом ее развития.
\\\textbf{Триггер} - элементарный цифровой автомат, обладающий способностью длительно находиться в одном из двух устойчивых состояний и чередовать их под воздействием внешних сигналов. Состояние $0$ на выходе $Q$ соответствует выключенному состоянию, а $Q = 1$ - включенному. Триггеры осуществляют запоминание информации и остаются в заданном состоянии после прекращения действия переключающих сигналов. Они широко применяются широко применяются при цифровой обработке информации.
\\По способу организации логических связей, определяющие особенности функционирования, различают триггеры:
\begin{itemize}
  \item \textbf{RS-триггер} - меняет свое состояние в зависимости от того, на какой из входов была подана единица. При подаче сигнала на вход $S$ (\emph{Set}) на выходе устанавливается единица. При подаче сигнала на вход $R$ (\emph{Reset}) сигнал на выходе пропадает.
  \item  \textbf{D-триггер} (\emph{Delay} или \emph{Data}) - запоминает (задерживает) состояние входа на один такт. При кратковременной подаче сигнала на $C$ (\emph{Clock}) (обычно, с тактового генератора), запоминает сигнал на входе $D$ (\emph{Data}) и выдает его на выход до следующей итерации.
  \item \textbf{Т-триггер} (\emph{Toggle}) - при подаче сигнала на вход, меняет сигнал на выходе на противоположный.
  \item \textbf{JK-триггер} - аналогичен $RS$-триггеру ($J$ (\emph{Jump}) = $Set$, $K$ (\emph{Kill}) = $Reset$), с одним лишь исключением: при подаче единицы на оба входа, состояние выхода изменяется на противоположное.
\end{itemize}
Из них JK триггер называется универсальным, так как из него можно получить все остальные виды триггеров.
\begin{figure}[!h]
\begin{minipage}{5cm}
\includegraphics[width=5cm]{10_2}
\caption{Асинхронный RS-триггер на элементах 2И-НЕ (IEC)}
\end{minipage}
\hfill
\begin{minipage}{5cm}
\includegraphics[width=4.7cm]{10_3}
\caption{Асинхронный RS-триггер на элементах 2ИЛИ-НЕ (IEC)}
\label{tag:RS_NOR}
\end{minipage}
\end{figure}

\textbf{Регистры} - это узлы ЭВМ, служащие для хранения информации в виде машинных слов или его частей, а так же для выполнения над словами некоторых логических преобразований.
Регистры способны выполнять следующие операции:
\begin{itemize}
  \item установка регистра в состояние 0 или 1 (на всех выходах);
  \item прием и хранение в регистре n разрядного слова;
  \item сдвиг хранимого в регистре двоичного кода слова вправо или влево на заданное значение разрядов;
  \item преобразование кода хранимого слова в последовательный, и наоборот, при приеме или при выдачи двоичных данных;
  \item поразрядные логические операции.
\end{itemize}
\textbf{Счетчики} - узлы ЭВМ, которые осуществляют счет и хранение кода числа подсчитанных сигналов.Они представляют собой цифровые автоматы Мура, в которых новое состояние счетчика определяется его предыдущим состоянием и состоянием логической переменной на входе.
\\Внутреннее состояние счетчиков характеризуется коэффициентом пересчета К, определяющим число его устойчивых состояний. Основными параметрами являются разрешающая способность (минимальное время между двумя сигналами, которые надежно фиксируются) или максимальное быстродействие и информационная емкость.
\\\textbf{Дешифратор (избирательная схема)} - это узел ЭВМ, в котором каждой комбинации входных сигналов соответствует наличие сигнала на одной вполне определенной шине на выходе (комбинационное устройство). Дешифраторы широко используются для преобразования двоичных кодов в управляющие сигналы для различных устройств ЭВМ.
\\\textbf{Шифратор (кодер)} - это узел ЭВМ, преобразующий унитарный код в некоторый позиционный код. Если выходной код является двоичным позиционным, то шифратор называется двоичным. С помощью шифраторов возможно преобразование цифр десятичных чисел в двоичное представление с использованием любого другого двоично-десятичного кода.
\\\textbf{Преобразователи кодов} - это узлы ЭВМ, предназначенные для кодирования чисел. В число преобразователей кодов входят: двоично-десятичные преобразователи, преобразователи цифровой индикации, преобразователи прямого кода двоичных чисел в обратный или дополнительный код и т. д.
\\\textbf{Мультиплексоры} - это узлы, преобразующие параллельные цифровые коды в последовательные. В этом устройстве выход соединяется с одним из входов в зависимости от значения адресных входов. Мультиплексоры широко используются для синтеза комбинационных устройств, так как это способствует значительному уменьшению числа используемых микросхем.
\\\textbf{Демультиплексоры} - это узлы, преобразующие информацию из последовательной формы в параллельную. Информационный вход D подключается к одному из выходов Qi определяемый адресными сигналами A0 и A1.
\begin{figure}[!h]
\begin{minipage}[c]{6cm}
\center{\includegraphics[width=3.5cm]{10_4}}
\caption{Мультиплексор}
\end{minipage}
\hfill
\begin{minipage}[c]{6cm}
\center{\includegraphics[width=3.5cm]{10_5}}
\caption{Демультиплексор}
\end{minipage}
\end{figure}

\textbf{Сумматор} - это узел, в котором выполняется арифметическая операция суммирования цифровых кодов двух двоичных чисел.
\\Используя одноразрядные сумматоры можно построить многоразрядные сумматоры.
\\\textbf{Шина} - набор коммуникационных линий, каждая из которых способная передавать сигналы, представляющие двоичные цифры 0 и 1. 
\\Электрическая цепь, соединяющая регистр с другим регистром или иным устройством ЭВМ, называется \textbf{шиной (bus)}. \emph{Шина} состоит из параллельных проводов, каждый из которых предназначен для передачи соответствующего бита регистра. Два 8-битовых регистра соединяются между собой шиной из восьми проводов. Про такую шину говорят, что ее ширина равна 8. В действительности шина обычно содержит несколько дополнительных проводов, используемых для передачи сигналов синхронизации и управления, однако подобный анализ структуры шин нас пока не интересует.
\section{Характеристики систем памяти}
\begin{enumerate}
  \item\textbf{ Место расположения:}
  \begin{itemize}
    \item \emph{Процессорная}, то есть на общем кристалле с центральным процессором (ЦП) (регистры, кэш-память 1-го уровня);
    \item \emph{Внутренняя}, то есть на системной плате (основная память (ОП), кэш-память 2-го и последующего уровней);
    \item \emph{Внешняя} (медленные запоминающие устройства (ЗУ) большой ёмкости).
  \end{itemize}
  \item \textbf{Емкость ЗУ} - число бит/байт, которое можно хранить на ЗУ.
  \item \textbf{Единица пересылки.} Для ОП единица пересылки определяется шириной шины данных, то есть количество бит, передаваемых по линиям шины параллельно. Обычно равна длине слова.
  \item \textbf{Метод доступа к данным:}
  \begin{itemize}
    \item \emph{Последовательный доступ} - ЗУ ориентировано на хранение информации в виде последовательности блоков, называемых записями. Для доступа к нужному элементу необходимо прочитать все предшествующие блоки (ЗУ на магнитной ленте);
    \item \emph{Прямой доступ} - каждая запись имеет уникальный адрес, отражающий ее физическое размещение на носителе информации. Обращение определяется как адресный доступ к началу записи плюс последующий последовательный доступ к определённой информации внутри записи (магнитные диски);
    \item \emph{Произвольный доступ} - каждая ячейка памяти имеет уникальный физический адрес. Обращение к любой ячейке занимает одно и то же время и может водиться в произвольной очередности (ОП);
    \item \emph{Ассоциативный доступ} - позволяет выполнять поиск ячеек, содержащих такую информацию, в которой значение отдельных бит совпадает с состоянием одноименных битов в заданном образце. Сравнение осуществляется параллельно для всех ячеек памяти, независимо от ее емкости (кэш-память).
  \end{itemize}
  \item \textbf{Быстродействие} - один из важнейших показателей:
  \begin{itemize}
    \item \emph{Время доступа ($\mbox{Т}_{\mbox{д}}$)} - интервал времени от момента поступления адреса до момента, когда данные заносятся в память или становятся доступными.
    \item \emph{Длительность цикла памяти или период обращения ($\mbox{Т}_{\mbox{ц}}$)} - понятие применяется к памяти с произвольным доступом, для которой оно означает минимальное время между двумя последовательными обращениями к памяти. Период обращения включает в себя время доступа плюс некоторое дополнительное время.
    \item \emph{Скорость передачи} - скорость, с которой данные могут передаваться в память или из нее.
  \end{itemize}
  \item \textbf{Физический тип}
  \begin{itemize}
    \item Полупроводниковая память;
    \item Память с магнитным носителем информации (используемая в магнитных лентах и дисках);
    \item Память с оптическим носителем (оптические диски);
  \end{itemize}
  \item \textbf{Физические особенности} (например, энергозависимость).
  \item \textbf{Стоимость} - стоимость хранения одного бита информации.
\end{enumerate}
\section{Иерархия памяти}
\begin{figure}[!h]
\includegraphics[width=9cm]{10_6}
\end{figure}
Чем меньше время доступа, тем выше стоимость хранения бита. Чем больше емкость, тем ниже стоимость хранения бита, но больше время доступа.

\section{Физическое устройство памяти}
\subsection{Кэш-память}
\begin{wrapfigure}[7]{l}{1.7cm}
\includegraphics[width=1.7cm]{10_7_(1)}
\end{wrapfigure}
Рассмотрим биполярный \emph{n-p-n} транзистор: К - коллектор, Б - база, Э - эмиттер. На коллектор подано напряжение. Если на базу подать напряжение - транзистор откроется и ток с коллектора пойдет на эмиттер.
\\Небольшая особенность - напряжение на базе должно быть выше, чем на коллекторе (на сколько - зависит от конкретного транзистора, обычно немного. Например, 5В на коллекторе, 6В на базе).
\\
\\Теперь рассмотрим следующую схему:
\\
\begin{minipage}[l]{5cm}
\includegraphics[width=5cm]{10_7}
\end{minipage}
\begin{minipage}[l]{7cm}
$V_{CC}$ - линия питания устройства (например 5В).
\\$GND$ - линия 0В.
\\Если оба транзистора закрыты (на базу не подается напряжение, $V_1$ и $V_2$ равны 0), то ток уходит напрямую с $V_{CC}$ на $V_{OUT}$. В результате получается логическая единица.
\\Если подать напряжение хотя бы на один транзистор ($V_1$ или $V_2$), то ток с $V_{CC}$ будет уходить через транзистор в $GND$ и в $V_{OUT}$ не пойдет. В результате получается логический нуль.
\end{minipage}
\\
\\Так как напряжение на базе должно быть выше, чем на коллекторе (в данном случае, на линии питания $V_{CC}$), а повышенное взять неоткуда, то имеющееся напряжение на $V_{CC}$ занижается с помощью резистора и получается 5В на базе и чуть меньше на коллекторе.
\\
\\Получается, что данная схема реализует логическую функцию ИЛИ-НЕ (ANSI) (также известную как стрелка Пирса) (где $A$ и $B$ соответственно $V_1$ и $V_2$):

\begin{minipage}[l]{4cm}
\includegraphics[width=2.8cm]{10_8}
\end{minipage}
\\
\\Таблица истинности для ИЛИ-НЕ:
\begin{table}[!h]
\begin{tabular}{|c|c|c|}
\hline
A & B & X \\
\hline
 0 & 0 & 1 \\
 0 & 1 & 0 \\
 1 & 0 & 0 \\
 1 & 1 & 0 \\
\hline
\end{tabular}
\end{table}
\\Объединим два элемента ИЛИ-НЕ обратной связью.
\begin{wrapfigure}[10]{l}{3.8cm}
\includegraphics[width=3.8cm]{10_9}
\end{wrapfigure}
Выход одного элемента ИЛИ-НЕ поступает на вход другого. Получилась самая простая память для хранения 1 бита, использующая 4 транзистора.
\\В данном случае, изображен асинхронный RS-триггер (изображен на рисунке \ref{tag:RS_NOR}). Также, память может состоять и из других триггеров.
\\При подаче единицы на вход $S$ выходное состояние становится равным логической единице. А при подаче единицы на вход $R$ выходное состояние становится равным логическому нулю. Если на оба входа $R$ и $S$ одновременно поданы логические единицы, оба выхода переходят в состояние логического нуля, которое является неустойчивым и переходит в одно из устойчивых состояний при снятии управляющего сигнала с одного из входов, иначе говоря, в ячейку может записаться любое значение.
\subsection{Оперативная память}
\begin{wrapfigure}[10]{l}{4cm}
\includegraphics[width=4cm]{10_10}
\end{wrapfigure}
В отличие от кэш-памяти, оперативная память устроена намного проще. Она устроена из 1 конденсатора и 1 транзистора, что дешево и занимает мало места. Один конденсатор легче воспринимать как память (конденсатор разряжен - 0, заряжен - 1). Транзистор нужен с одной целью - чтобы не разряжать постоянно конденсатор. Конденсатор необходимо периодически подзаряжать, а заряжается и разряжается он медленно. Переключить транзистор быстрее чем, зарядить или разрядить конденсатор.
\\Если на базу подано напряжение, транзистор открыт, мы можем считать значение - 1. Если напряжения нет, значение не считывается, записывается 0.
\section{Локальность памяти}
\subsection{Пространственная локальность памяти}
С очень высокой вероятностью адрес очередной команды программы либо следует непосредственно за адресом, по которому была считана текущая команда, либо расположен вблизи него. Такое расположение адресов называется \textbf{пространственной локальностью программы}.
\\Обрабатываемые данные, как правило, структурированы, и такие структуры обычно хранятся в последовательных ячейках памяти. Такая особенность программ называется \textbf{пространственной локальностью данных}.
\subsection{Временн\'ая локальность памяти}
Кроме того, программы содержат множество небольших циклов и подпрограмм. Это означает, что небольшие наборы команд могут многократно
повторяться в течение некоторого интервала времени, то есть имеет место \textbf{временная локальность}.
\\Все три вида локальности объединяет понятие \textbf{локальность по обращению}. Принцип локальности часто облекают в численную форму и представляют в виде так называемого правила "90/10": 90 \%
 времени работы программы связано с доступом к 10\%
  адресного пространства этой программы.
  \subsection{Применение локальности памяти}
 
 Рассмотренные принципы локальности не являются просто любопытным наблюдением. Их использовуют для устранения проблемы узкого места архитектур фон Неймана — шины взаимодействия между процессором и памятью.
 \\
 \\ Память, как правило, работает на меньшей частоте и с меньшей скоростью чем процессор, но программа достаточно часто обращается к памяти. Это приводит к тому, что скорость работы программы и скорость работы компьютера определяется не скоростью работы процессора, а скоростью работы медленной оперативной памяти.
\\Чтобы устранить эту проблему обычно используют кэш, и эффект от него достаточно ощутим: сильный выигрыш в производительности.
Но почему же тогда вместо медленной оперативной памяти не использовать быструю кэш-память? Рассмотрим пример.
\\
\\Так как мы хотим полностью заменить оперативную памать на кэш, то стоимость компьютера увеличится в сто или даже в тысячу раз (оперативной памяти обычно устанавливают гигабайты, а кэш-память измеряется всего лишь мегабайтами). 
\\Чтобы посчитать к какому эффекту это приведёт, будем использовать принцип Парето (\emph{`20\% усилий дают 80\% результата, а остальные 80\% усилий — лишь 20\% результата`}), в соответствии с которым, из-за локальности обращений, 80\% таких обращений попадают в кэш. То есть, при первом обращении большой объем данных приходится копировать из оперативной памяти, что достаточно медленно. Зато далее, в процессе работы программы, с очень высокой вероятностью (в нашем случае – 80\%) при записи/чтении очередной порции данных из памяти мы можем взять её в готовом виде из кэш-памяти. Будем считать, что кэш-память работает в 10 раз быстрее оперативной памяти.
\\Посчитаем, сколько времени понадобится, чтобы выполнить N операций записи или чтения. 
\\Оказывается, что в компьютере, где используется медленная оперативная память, то есть до гипотетической модернизации, время выполнения программы будет равно 2,8N (условно измеряемое в наносекундах).
После замены оперативной памяти на кэш-память, все обращения будут происходить со скоростью кэша, следовательно, общее время сократится до N.
\\
\\
Увидев эти цифры можно сделать любопытный вывод. Мы потратили деньги для того, чтобы ускорить работу памяти в 10 раз, но при этом 10-кратное ускорение памяти привело лишь к трехкратному увеличению производительности. 
В этом и состоит эффект кэширования: вовсе не обязательно устанавливать в компьютере дорогостоящую быструю память, можно обойтись несколькими уровнями кэша, каждый из которых чуть быстрее (а желательно – на порядок быстрее) предыдущего. Это позволит очень эффективно бороться с узким местом принстонской архитектуры.

\section{Порядок хранения байт в памяти}
Существует несколько способов хранения байт в памяти:
\begin{itemize}
  \item \textbf{От старшего к младшему} (англ. \emph{big-endian}): $A_n,\dots,A_0$ запись начинается со старшего и заканчивается младшим. Этот порядок является стандартным для протоколов TCP/IP, он используется в заголовках пакетов данных и во многих протоколах более высокого уровня, разработанных для использования поверх TCP/IP. Поэтому, порядок байтов от старшего к младшему часто называют сетевым порядком байтов.
  \item \textbf{От младшего к старшему} (англ. \emph{little-endian}): $A_0,\dots,A_n$ запись начинается с младшего и заканчивается старшим. Этот порядок записи принят в памяти персональных компьютеров с x86-процессорами, в связи с чем иногда его называют интеловский порядок байт (по названию фирмы-создателя архитектуры x86).
  \item \textbf{Переключаемый порядок} (англ. \emph{bi-endian}). Многие процессоры могут работать и в порядке от младшего к старшему, и в обратном. Обычно порядок байтов выбирается программно во время инициализации операционной системы, но может быть выбран и аппаратно перемычками на материнской плате. В этом случае правильнее говорить о порядке байтов операционной системы.
  \item \textbf{Смешанный порядок} (англ. \emph{middle-endian}) иногда используется при работе с числами, длина которых превышает машинное слово. Число представляется последовательностью машинных слов, которые записываются в формате, естественном для данной архитектуры, но сами слова следуют в обратном порядке.
\end{itemize}
\begin{figure}[h]
\includegraphics[width=11cm]{10_11}
\caption{Сравнение порядков от младшего к старшему и от старшего к младшему}
\end{figure}
    % \section{Многоуровневая модель OSI (Open Systems Interconnection)}
Процесс передачи данных по компьютерной сети очень сложен, поэтому специалисты International Standards Organization решили разделить его на семь логических независимых уровней. Специалист на одном уровне может работать независимо от специалиста на другом уровне, не мешая друг другу.
\begin{table}[!h]
\begin{tabular}{|c|c|c|}
\hline
№ & Название уровня (layer) & Основная функция \\
\hline
\multirow{2}{*}{7} & \multirow{2}{*}{прикладной (application)} & взаимодействие программы пользователя \\
& & с сетевой подсистемой ОС (API) \\
\hline
\multirow{2}{*}{6} & уровень представления & \multirow{2}{*}{шифрование, сжатие, выбор кодировки}\\
 & (presentation) & \\
\hline
5 & сеансовый (session)  & установление соединения\\
\hline
4 & транспортный (transport)  & надежность доставки, реакция на потери\\
\hline
\multirow{3}{*}{3} & \multirow{3}{*}{сетевой (network)} & маршрутизация, объединение  \\
& & разнородных локальных сетей, \\
& & адресация в глобальной сети (IP) \\
\hline
\multirow{3}{*}{2} & \multirow{3}{*}{канальный (data link)} & связь между узлами одной локальной \\
& & сети, адресация в локальной сети \\
& & (МАС-адрес)\\
\hline
\multirow{2}{*}{1} & \multirow{2}{*}{физический (physical)} & физические характеристики каналов  \\
& & связи и передаваемых сигналов \\
\hline
\end{tabular}
\end{table}
\subsection{Прикладной уровень}
\emph{Субъекты взаимодействия:} пользовательская программа на передающем/принимающем компьютере; ОС.
\\\emph{Объекты взаимодействия:} Пользовательские данные, представленные в "родном" понятном виде для приемной и передающей программы.
\\\emph{Основные функции:} Вызов специальных функций ОС для работы с сетью (API). Программист не обязан знать о внутреннем устройстве
сети, для него передача данных по сети не отличается от сохранения в файл (просто надо вызвать нужную функцию API ОС).
\\\textbf{API} (интерфейс программирования приложений, интерфейс прикладного программирования) (англ. \emph{application programming interface}) — набор готовых классов, процедур, функций, структур и констант, предоставляемых приложением (библиотекой, сервисом) для использования во внешних программных продуктах. Используется программистами при написании всевозможных приложений.
\subsection{Уровень представления}
\emph{Субъекты взаимодействия:} специальное ПО для шифрования, сжатия, кодирования; ОС.
\\\emph{Объекты взаимодействия:} Закодированные пользовательские данные (пользовательская программа уже не может работать с такими данными без декодирования).
\\\emph{Основные функции:} Шифрование, сжатие, выбор кодировки, выбор способа представления порядка байт (little-endian, big-endian).
Каждый этап может выполняться несколько раз разными субъектами.
\subsection{Сеансовый уровень}
\emph{Субъекты взаимодействия:} ОС на компьютере-передатчике; ОС на компьютере-приемнике.
\\\emph{Объекты взаимодействия:} Служебные данные о об установке соединения: логины, пароли, сертификаты, цифровые подписи, пустые пакеты для проверки отсутствия обрывов связи, служебные пакеты с командами типа "запрос соединения", "подтверждение соединения", "разрыв соединения" (т.е. никакие пользовательские данные на этом уровне не передаются).
\\\emph{Основные функции:}
\begin{itemize}
\item Установление соединения (с возможной аутентификацией абонентов).
\item Отслеживание состояния соединения (возможное автопереподключение при обнаружении ошибок).
\item Реагирование на долгую неактивность сеанса связи (например, автоотсоединение по таймауту).
\item Принудительный разрыв соединения при окончании передачи (попутно освобождаются ресурсы ОС, которые хранят информацию о состоянии сеанса).
\end{itemize}
\subsection{Транспортный уровень}
\emph{Субъекты взаимодействия:} ОС; драйвер сетевой карты.
\\\emph{Объекты взаимодействия:} Пользовательские данные, снабженные служебными заголовками для обнаружения проблем передачи (контрольная сумма, порядковые номера фрагментов), служебные пакеты-подтверждения.
\\\emph{Основные функции:}
\begin{itemize}
  \item Отслеживание проблемных пакетов: искаженных, потерянных, пришедших в неверном порядке или дубликатов.
  \item Реакция на обнаружение проблемных пакетов (запрос повторной передачи или игнорирование, сбор целых пакетов из пришедших в разном порядке фрагментов).
  \item Реализация механизма повторной передачи (передается весь файл целиком или только проблемные части).
\end{itemize}
\subsection{Сетевой уровень}
\emph{Субъекты взаимодействия:} ОС; драйвер сетевой карты.
\\\emph{Объекты взаимодействия:} Данные, нарезанные на фрагменты, которые можно передавать в конкретной локальной сети (например, в
проводных сетях Fast Ethernet предельный размер фрагмента $\approx$ 1500 байт, а в сетях Wi-Fi он равен $\approx$ 8000 байт). Каждый фрагмент снабжается глобальным адресом (например, IP-адресом), который понятен в любой локальной сети, но при этом уникален для всей
глобальной сети.
\\\emph{Основные функции:} Маршрутизация в большой сети; обеспечение возможности объединить несколько разнородных локальных сетей в одну сеть.
\subsection{Канальный уровень}
\emph{Субъекты взаимодействия:} драйвер сетевой карты; модуль сетевой карты, который генерирует физические сигналы (ток, радиоволна, пучок света).
\\\emph{Объекты взаимодействия:} Набор битов, полностью готовых к передаче от одного компьютера локальной сети к другому (без выхода в глобальную сеть). Помимо данных пользователя, в этот набор включают адреса приёмника и передатчика внутри локальной сети (например, MAC-адреса).
\\\emph{Основные функции:}
\begin{itemize}
  \item Проверка доступности (свободности) канала связи, если он общий для нескольких абонентов. Например, в Wi-Fi-канал является общим для нескольких устройств в радиусе действия базовой станции, поэтому он не всегда доступен для передачи и каждому устройству приходится ждать своей очереди.
  \item Передача данных и адресация осуществляются только внутри локальной сети (MAC-адрес имеет смысл только в пределах локальной сети, так как он не передаётся в глобальную сеть).
\end{itemize}
\subsection{Физический уровень}
\emph{Субъекты взаимодействия:} модуль сетевой карты, который генерирует физические сигналы (ток, радиоволна, пучок света); проводник сигнала (медный кабель, оптоволокно, радиоэфир).
\\\emph{Объекты взаимодействия:} Физические сигналы (ток, пучок света, радиоволна).
\\\emph{Основные функции:} Выбор носителя сигнала (ток, свет, радиоволна). Выбор свойств проводника сигнала (материал: медь,
оптоволокно; диаметр сечения, сопротивление, предельно допустимая длина). Выбор способа представления цифровых данных в виде физического сигнала (кодирование, модуляция).
\subsubsection{Кодирование}
0 и 1 можно представить в виде разного напряжения электрического тока. Самый интуитивно-понятный способ называется $NRZ$. Однако существует много других способов, устраняющих недостатки $NRZ$ (например, проблему вырождения переменного сигнала в постоянный ток, если передаются много единиц подряд).
\\
\begin{minipage}{\textwidth}
\includegraphics[width=10cm]{11_1}
\begin{center}
Различные системы кодирования данных
\end{center}
\end{minipage}
\\
\subsubsection{Модуляция}
Если сетевая карта умеет генерировать физический сигнал в виде синусоиды, то управляя амплитудой/частотой/фазой этой синусоиды, можно кодировать 0 и 1.
\\
\begin{minipage}{\textwidth}
\includegraphics[width=10cm]{11_2}
\begin{center}
а) информационный сигнал, б) амплитудная модуляция (AM), в) частотная модуляция (FM), г) фазовая модуляция (PM)
\end{center}
\end{minipage}
\subsection{Адекватность OSI-модели}
Не существует ни одной сетевой технологии, в которой бы была идеально реализована вся OSI-модель с четким разделением уровней.
Модель OSI далека от реальности, ее назначение - быть идеальной абстракцией.
\begin{table}[!h]
\begin{tabular}{r|l}
Реальность & OSI-уровни \\
\hline
Skype & 7,6,5 \\
\hline
FTP & 7,3 \\
\hline
TCP & 7,5,4,3 \\
\hline
IP & 3,4 \\
\hline
Wi-Fi & 1,2 \\
\hline
Fast-Ethernet & 1,2 \\
\end{tabular}
\end{table}
\section{Отличие TCP от UDP}
\begin{table}[h]
\begin{tabular}{|l|c|c|}
\hline
Свойство & TCP & UDP \\
\hline
Установка соединения & $\checkmark$ & $\times$ \\
Разрыв соединения &  $\checkmark$ & $\times$ \\
Подтверждение доставки &  $\checkmark$ & $\times$ \\
Проверка контрольной суммы  & $\checkmark$ & $\checkmark$ \\
Обнаружение искаженных пакетов & $\checkmark$ & $\checkmark$ \\
Обнаружение потерянных пакетов & $\checkmark$ & $\times$ \\
Повторная передача потерянных/искаженных & $\checkmark$ & $\times$ \\
\hline
\end{tabular}
\end{table}
\textbf{TCP} применяют, если необходимо удостовериться, что все данные дошли корректно, получив об этом подтверждение и организовав повторную передачу поврежденных данных (пример: передача почты).
\\\textbf{UDP} применяют либо если канал связи абсолютно надежен, либо если нет смысла повторно передавать потерянные/искаженные пакеты (пример: видео-звонок), но при этом хочется сэкономить на передаче ненужных служебных данных, используемых в ТСР.
\section{Сетевые устройства}

\begin{minipage}{\textwidth}
\includegraphics[width=10cm]{11_3}
\end{minipage}

\begin{center}
  \textbf{Сравнение коммутатора и маршрутизатора}
\end{center}

\begin{table}[!h]
 \begin{tabular}{|c|c|c|}
 \hline
 \multirow{2}{*}{Свойство} & Коммутатор & Маршрутизатор \\
 & (switch) & (router) \\
 \hline
 & & Много (ровно по \\
 Наличие MAC-адреса & Нет & по одному на каждый \\
 & & порт/антенну) \\
 \hline
 & & Много (минимум по \\
 Наличие IP-адреса & Нет & по одному на каждый \\
 & &  порт/антенну) \\
 \hline
 Уровни OSI-модели & 1,2 & 1,2,3 \\
 \hline
 Умение выбирать & Нет (так как в локаль-  & \\
 маршруты & ной сети всегда только  & Да\\
 & один маршрут & \\
 \hline
 & Обмен данными между & Обмен данными  \\
 Назначение & компьютерами внутри & между несколькими  \\
 & локальной сети & локальными сетями \\
 \hline
  \end{tabular}
\end{table}
\textbf{Примечание:} существуют гибридные устройства, совмещающие в себе коммутатор и маршрутизатор (они используются у большинства пользователей домашнего интернета, однако в корпоративных сетях применяются реже).

    
    % \section{Основы теории информации}
\label{sec:theory}

\subsection{Терминология информатики}

Начать изучение информатики невозможно, не разобравшись в точном значении термина «информатика». Однако, до сих пор в мировой научной общественности не сложилось четкого понимания этого термина. Рассмотрим одно из популярных определений:

\textbf{Информатика} --- дисциплина, изучающая свойства и структуру информации, закономерности ее создания, преобразования, накопления, передачи и использования. За рубежом сложилась чуть более узкая трактовка термина информатика. Там под этим понимают пересечение сразу трех областей науки – это информационные технологии, теория информации и computer science. Всё обозначенное выше подходит под определение самого курса "Информатика".

Изучая некоторую науку важно представлять основные даты, вехи её развития:
\begin{itemize}
\item 1956(57) – появление термина <<информатика>> (\textit{нем.} Informatik, Штейнбух).
\item 1968 – первое упоминание в СССР (информология, Харкевич).
\item 197Х – информатика стала отдельной наукой.
\item 4 декабря – день российской информатики.
\end{itemize}
\subsection{Терминология теории информации}

Рассмотрим некоторые терминологические тонкости. В обыденном языке, слова <<информация>> и <<данные>> считаются синонимами. Они, как правило, употребляются взаимозаменяемо. И так обстоит дело в информатике и в целом, в компьютерных науках.

Понятие \textit{``информация''} имеет различные трактовки в различных предметных областях. Например, \textit{информация} может пониматься как:
\begin{itemize}[noitemsep]
    \item сигналы для управления, приспособления рассматриваемой системы (в кибернетике);
    \item мера хаоса в рассматриваемой системе (в физике);
    \item вероятность выбора в рассматриваемой системе (в теории вероятностей);
    \item мера разнообразия в рассматриваемой системе (в биологии) и др.
\end{itemize}
Но мы остановимся на понятиях, близких к информатике.

\begin{description}
    \item [Информация] --- это некоторая упорядоченная последовательность сообщений, отражающих, передающих и увеличивающих наши знания.
    \item [Информация] --- это сведения об окружающем мире (объекте, процессе, явлении, событии), которые являются объектом преобразования (включая хранение, передачу и т.д.) и используются для выработки поведения, для принятия решения, для управления или для обучения.
    \item [Информация] --- это новые сведения, подлежащие передаче, хранению и обработке.
\end{description}

Рассмотрим это фундаментальное понятие информатики на основе понятия \textit{``алфавит''} (``алфавитный'', формальный подход). Дадим формальное определение \textit{алфавита}.

\begin{description}
    \item [Алфавит] --- конечное множество различных знаков (букв), символов, для которых определена операция \emph{конкатенации} (присоединения символа к символу или цепочке символов); с ее помощью по определенным правилам соединения символов и слов можно получать слова (цепочки знаков) и словосочетания (цепочки \textit{слов}) в этом \textit{алфавите} (над этим \textit{алфавитом}).
    \item [Знак (буква)] --- любой элемент алфавита (элемент $x$ алфавита $X$, где $x \in X$). Понятие знака неразрывно связано с тем, что им обозначается (``со смыслом''), они вместе могут рассматриваться как пара элементов ($x$, $y$), где $x$ – сам знак, а $y$ – обозначаемое этим знаком.
\end{description}

\example{1}
\noindent
Примеры \emph{алфавитов:} множество из десяти цифр, множество из знаков русского языка, точка и тире в азбуке Морзе и др. В \emph{алфавите} цифр знак 5 связан с понятием ``быть в количестве пяти элементов''.

\textbf{Слово} в алфавите (или над алфавитом) - конечная последовательность знаков (букв) алфавита.

\textbf{Длина} |p| некоторого слова $p$ в алфавите (над алфавитом) - число составляющих его букв.

\textbf{Словарь (словарный запас)} - множество различных слов в алфавите (над алфавитом).
В отличие от конечного \emph{алфавита}, словарный запас может быть и бесконечным.
\emph{Слова} над некоторым заданным \emph{алфавитом} и определяют так называемые \emph{сообщения}.

\example{2}
\noindent
\emph{Слова} над \emph{алфавитом} кириллицы --- ``Информатика'',``инто'', ``ииии'', ``и''.

\noindent
\emph{Слова} над \emph{алфавитом} десятичных цифр и знаков арифметических операций --- ``1256'', ``23+78'', ``35–6+89'', ``4''.

\noindent
\emph{Слова} над \emph{алфавитом} азбуки Морзе --- ``.'', ``. . –'', ``– – –''.

В \emph{алфавите} должен быть определен порядок следования \emph{букв} (порядок типа ``предыдущий элемент --- последующий элемент''), то есть любой \emph{алфавит} имеет упорядоченный вид $X = {x_1, x_2, \ldots, x_n}$ .

Таким образом, \emph{алфавит} должен позволять решать задачу лексикографического (алфавитного) упорядочивания, или задачу расположения \emph{слов} над этим \emph{алфавитом}, в соответствии с порядком, определенным в \emph{алфавите} (то есть по символам \emph{алфавита}).

\subsection{Признаки классификации информации}

Рассмотрим две классификации информации. Первая из них --- классификация по форме \emph{сообщений} --- определенного вида сигналов, символов:
\begin{itemize}[noitemsep]
  \item отношение к источнику или приемнику (входная, выходная и внутренняя);
  \item отношение к конечному результату (исходная, промежуточная и результирующая);
  \item актуальность;
  \item адекватность;
  \item доступность (открытая, закрытая);
  \item понятность;
  \item полнота (достаточная, недостаточная, избыточная);
  \item достоверность;
  \item массовость;
  \item изменчивость (постоянная, переменная, смешанная);
  \item объективность;
  \item точность;
  \item стадия использования (первичная, вторичная);
  \item ценность.
\end{itemize}

Вторая классификация --- по форме преставления информации, способам ее кодирования и хранения:
\begin{itemize}[noitemsep]
  \item графическая;
  \item звуковая;
  \item текстовая;
  \item числовая;
  \item видеоинформация.
\end{itemize}

\subsection{Измерение количества информации}

Любые сообщения измеряются в \emph{байтах, килобайтах, мегабайтах, гигабайтах, терабайтах, петабайтах} и \emph{эксабайтах}, а кодируются, например, в компьютере, с помощью \emph{алфавита} из нулей и единиц, записываются и реализуются в ЭВМ в \emph{битах}.

Приведем основные соотношения между единицами измерения \emph{сообщений}:
\begin{itemize}[noitemsep]
    \item 1 бит (\textbf{bi}nary digi\textbf{t} - двоичное число) = 0 или 1;
    \item 1 байт = 8 бит;
    \item 1 килобайт (1 Кб) = $2^{13}$ бит;
    \item 1 мегабайт (1 Мб) = $2^{23}$ бит;
    \item 1 гигабайт (1 Гб) = $2^{33}$ бит;
    \item 1 терабайт (1 Тб) = $2^{43}$ бит;
    \item 1 петабайт (1 Пб) = $2^{53}$ бит;
    \item 1 эксабайт (1 Эб) = $2^{63}$ бит.
\end{itemize}

Теперь нам известно понятие информации, но необходимо еще конкретно знать сколько этой информации. Поэтому есть два важных определения:
\begin{description}
    \item [Количество информации] --- число, адекватно характеризующее разнообразие (структурированность, определённость,выбор состояний и т.д.) в оцениваемой системе. Количество информации часто оценивается в битах, причем такая оценка может выражаться и в долях бит (так как речь идет не об измерении или кодировании сообщений).
    \item [Мера информации] --- численная оценка количества информации, которая обычно задана неотрицательной, определенной на множестве событий и являющейся аддитивной функцией (то есть, мера информации объединения событий (множеств) равна сумме мер каждого события). Заметим, что функция меры информации монотонна (при уменьшении или увеличении вероятности некоторого события количество иноформации в системе монотонно уменьшается или увеличивается). 
    \\\textbf{Важно:} мера вероятности всегда находится в диапазоне от 0 до 1
\end{description}

Для измерения информации используются различные подходы и методы, например, с использованием меры информации по Р. Хартли и К. Шеннону.

\newpage
\subsubsection{Мера Хартли}

\begin{wrapfigure}{l}{0.21\textwidth}
    \centering
    \includegraphics[width=0.2\textwidth]{hartley}
    \caption*{Ральф Хартли\\1888 -- 1970}
\end{wrapfigure}

Пусть известны $N$ состояний системы $S$ ($N$  опытов с различными, равновозможными, последовательными состояниями системы). Если каждое состояние системы закодировать двоичными кодами, то минимальная длина $d$ полученного кода определяется из условия:

$$
2^{d} \ge N \qquad \mbox{\emph{или}} \qquad  d \ge \log_{2}N
$$

Значит, для однозначного описания системы требуется $\log_{2}N$ бит. В общем случае количество информации в системе $S$ равно:
$$
H_{s} = \log_{k}N
$$

Единицы измерения количества информации:
\begin{itemize}[noitemsep]
  \item Бит ($k = 2$)
  \item Трит ($k = 3$)
  \item Дит (харт) ($k = 10$)
  \item Нит (нат) ($k = e$)
\end{itemize}

\paragraph{Примеры использования меры Хартли}

\example{1}

\task мальчик загадывает число от 1 до 64. Какое количество вопросов типа "да-нет" понадобится, чтобы гарантированно угадать число?

\solution

\begin{itemize}[noitemsep]
    \item Первый вопрос: "Загаданное число меньше 32?". Ответ: "Да".
    \item Второй вопрос: "Загаданное число меньше 16?". Ответ: "Нет".
    \item [] \dots
    \item Шестой вопрос точно приведет к правильному ответу.
\end{itemize}

\noindentЗначит, в соответствии с мерой Хартли в загадке мальчика содержится $\log_{2}64 = 6$ бит информации ($N = 64$ так как возможно 64 вариантов загаданного числа).

\answer 6 бит.


\example{2}

\task Мальчик держит за спиной шахматного ферзя и собирается поставить его на произвольную клетку пустой доски. Какое количество информации содержится в его действии?

\solution Шахматная доска имеет размеры $8\times 8$ клеток.
Ферзь может быть как белым, так и черным, поэтому количество равновероятных состояний будет равно $8 \times 8 \times 2 = 128$.
Получается, количество информации по мере Хартли равно $\log_{2}128 = 7$ бит.

\answer 7 бит.

\bigskip

Если во множестве $X = {x_1,x_2, ..., x_n}$ искать произвольный элемент, то для его нахождения (по Хартли) необходимо иметь не менее $\log_{a}n$ (единиц) информации. 

Уменьшение $H$ говорит об уменьшении разнообразия состояний $N$ системы, а увеличение $H$ говорит об увеличении разнообразия состояний $N$ системы.

Мера Хартли подходит лишь для идеальных, абстрактных систем, так как в реальных системах состояния системы неодинаково осуществимы (неравновероятны).
\subsubsection{Мера Шеннона}

\begin{wrapfigure}{l}{0.21\textwidth}
    \centering
    \includegraphics[width=0.2\textwidth]{shannon}
    \caption*{Клод Шеннон\\1916 -- 2001}
\end{wrapfigure}

Если состояния системы не равновероятны, используют меру Шеннона. Мера Шеннона оценивает информацию отвлеченно от ее смысла:
$$I = - \sum^{N}_{i=1}p_{i}\times \log_{2}p_{i},$$ где:
\begin{description}[noitemsep]
    \item [$I$] -- количество информации, выраженное в битах (в $\log_{k}p_{i}$ $k = 2$);
    \item [$N$] --- число состояний системы;
    \item [$p_{i}$] --- вероятность (относительная частота) перехода системы в $i$-е состояние (вероятность того, что система находится в состоянии $i$)
\end{description}

Сумма всех $p_{i}$ должна быть равна единице.

Если все состояния рассматриваемой системы равновозможны, равновероятны, то есть $p_i = 1/n$, то из \emph{формулы Шеннона} можно получить (как частный случай) \emph{формулу Хартли}:
$$I = \log_{2}n.$$

Обозначим величину:
$$f_i = -n\log_{2}p_i.$$

Тогда из \emph{формулы К. Шеннона} следует, что количество информации I можно понимать как среднеарифметическое величин $f_i$ , то есть величину $f_i$ можно интерпретировать как \emph{информационное содержание символа алфавита} с индексом i и величиной $p_i$ вероятности появления этого символа в любом сообщении (слове), передающем информацию.

В термодинамике известен так называемый коэффициент Больцмана $k = 1.38 * 10^{-16} \mbox{(эрг.град)}$ и выражение (\emph{формула Больцмана}) для энтропии или меры хаоса в термодинамической системе:
$$
S = -k \sum^{N}_{i=1}p_{i}\times \ln{p_{i}}
$$

Сравнивая выражения для I и S, можно заключить, что величину I можно понимать как энтропию из-за нехватки информации в системе (о системе).

Формулы энтропии и информации идентичны, но смысл разный. Энтропия априорная характеристика (до передачи), информация – апостериорная (после передачи).

Из этой формулы следуют важные выводы:
\begin{itemize}
    \item увеличение меры Шеннона свидетельствует об уменьшении энтропии (увеличении порядка) системы;
    \item уменьшение меры Шеннона свидетельствует об увеличении энтропии (увеличении беспорядка) системы.
\end{itemize}

Положительная сторона \emph{формулы Шеннона} --- ее отвлеченность от смысла информации. Кроме того, в отличие от \emph{формулы Хартли}, она учитывает различность состояний, что делает ее пригодной для практических вычислений. Основная отрицательная сторона \emph{формулы Шеннона} – она не распознает различные состояния системы с одинаковой вероятностью.

\paragraph{Примеры использования меры Шеннона}

\example{1}

\task девочка наугад вытаскивает из мешка мяч. Известно, что в мешке всего 8 мячей, из них: 4 красных, 2 синих, 1 зеленый и 1 белый. Какое количество информации содержится в этом событии?

\solution
\begin{itemize}
    \item Вероятность вытащить красный мяч равна $\displaystyle\frac{4}{8} = 0,5$
    \item Вероятность вытащить синий мяч равна $\displaystyle\frac{4}{8} = 0,25$
    \item Вероятность вытащить зеленый мяч равна $\displaystyle\frac{1}{8} = 0,125$
    \item Вероятность вытащить белый мяч равна $\displaystyle\frac{1}{8} = 0,125$
\end{itemize}

\noindentЗначит количество информации, выраженное в битах равно:
\begin{flalign*}
I = {} & -( \frac{1}{2} \times \log_{2}\frac{4}{8}  + \frac{1}{4} \times \log_{2}\frac{1}{8} + \frac{1}{8} \times \log_{2}\frac{1}{8} + \frac{1}{8} \times \log_{2}\frac{1}{8})  \\
         & = -(-0,5 \times 1 - 0,25 \times 2 - 0,125 \times 3 - 0,125 \times 3) \\
         & =  -(-0,5 - 0,5 - 0,375 - 0,375) \\
         & = 1,7 \mbox{ бит}
\end{flalign*}
% $I = -(0,5 \times \log_{2}0,5  + 0,25 \times \log_{2}0,25 + 0,125 \times \log_{2}0,125 + 0,125 \times \log_{2}0,125) = -(-0,5\times 1 - 0,25\times 2 - 0,125\times 3 - 0,125\times 3) = -(-0,5 - 0,5\hm - 0,375\hm - 0,375\hm) = 1,75$ бит.

\answer 1,75 бит
\subsection{Методы получения информации}

Методы получения информации можно разбить на три большие группы:
\begin{itemize}
    \item \emph{Эмпирические};
    \item \emph{Теоретические}; 
    \item \emph{Эмпирико-теоретические}.
\end{itemize}

Кратко рассмотрим и охарактеризуем все три метода по отдельности. 

\subsubsection{Эмпирические методы}
Эмпирические методы или методы получения эмпирических данных.
\begin{description}
    \item [Наблюдение] --- сбор первичной информации об объекте, процессе, явлении.
    \item [Сравнение] --- обнаружение и соотнесение общего и различного.
    \item [Измерение] --- поиск с помощью измерительных приборов эмпирических фактов.
    \item [Эксперимент] --- преобразование, рассмотрение объекта, процесса, явления с целью выявления каких-то новых свойств.
\end{description}

\noindent
Кроме классических форм их реализации, в последнее время используются опрос, интервью, тестирование и другие.

\subsubsection{Теоретические методы}
Теоретические методы или методы построения различных теорий.
\begin{description}
    \item [Восхождение от абстрактного к конкретному] --- получение знаний о целом или о его частях на основе знаний об абстрактных проявлениях в сознании, в мышлении.
    \item [Идеализация] --- получение знаний о целом или его частях путем представления в мышлении целого или частей, не существующих в действительности.
    \item [Формализация] --- получение знаний о целом или его частях с помощью языков искусственного происхождения (формальное описание, представление).
    \item [Аксиоматизация] --- получение знаний о целом или его частях с помощью некоторых аксиом (не доказываемых в данной теории утверждений) и правил получения из них (и из ранее полученных утверждений) новых верных утверждений.
    \item [Виртуализация] --- получение знаний о целом или его частях с помощью искусственной среды, ситуации.
\end{description}

\subsubsection{Эмпирико-теоретические методы}
tЭмпирико-теоретические методы (смешанные) или методы построения теорий на основе полученных эмпирических данных об объекте, процессе, явлении.

\begin{itemize}
    \item \textbf{Абстрагирование} --- выделение наиболее важных для исследования свойств, сторон исследуемого объекта, процесса, явления и игнорирование несущественных и второстепенных.
    \item \textbf{Анализ} --- разъединение целого на части с целью выявления их связей.
    \item \textbf{Декомпозиция} --- разъединение целого на части с сохранением их связей с окружением.
    \item \textbf{Синтез} --- соединение частей в целое с целью выявления их взаимосвязей.
    \item \textbf{Композиция} --- соединение частей целого с сохранением их взаимосвязей с окружением.
    \item \textbf{Индукция} --- получение знания о целом по знаниям о частях.
    \item \textbf{Дедукция} --- получение знания о частях по знаниям о целом.
    \item \textbf{Эвристики, использование эвристических процедур} --- получение знания о целом по знаниям о частях и по наблюдениям, опыту, интуиции, предвидению.
    \item \textbf{Моделирование (простое моделирование)}, использование приборов -- получение знания о целом или о его частях с помощью модели или приборов.
    \item \textbf{Исторический метод} --- поиск знаний с использованием предыстории, реально существовавшей или же мыслимой.
    \item \textbf{Логический метод} --- поиск знаний путем воспроизведения частей, связей или элементов в мышлении.
    \item \textbf{Макетирование} --- получение информации по макету, представлению частей в упрощенном, но целостном виде.
    \item \textbf{Актуализация} --- получение информации с помощью перевода целого или его частей (а следовательно, и целого) из статического состояния в динамическое состояние.
    \item \textbf{Визуализация} --- получение информации с помощью наглядного или визуального представления состояний объекта, процесса, явления.
\end{itemize}

\noindent
Кроме указанных классических форм реализации теоретико-эмпирических методов часто используются и мониторинг (система наблюдений и анализа состояний), деловые игры и ситуации, экспертные оценки (экспертное оценивание), имитация (подражание) и другие формы.

\example{1}

\noindent
Для построения модели планирования и управления производством в рамках страны, региона или крупной отрасли нужно решить следующие проблемы:
\begin{enumerate}
    \item Определить структурные связи, уровни управления и принятия решений, ресурсы; при этом чаще используются методы наблюдения, сравнения, измерения, эксперимента, анализа и синтеза, дедукции и индукции, эвристический, исторический и логический методы, макетирование и др.;
    \item Определить гипотезы, цели, возможные проблемы планирования; наиболее используемые методы --- наблюдение, сравнение, эксперимент, абстрагирование, анализ, синтез, дедукция, индукция, эвристический, исторический, логический и др.;
    \item Конструирование эмпирических моделей; наиболее используемые методы --- абстрагирование, анализ, синтез, индукция, дедукция, формализация, идеализация и др.;
    \item Поиск решения проблемы планирования и просчет различных вариантов, директив планирования, поиск оптимального решения; используемые чаще методы – измерение, сравнение, эксперимент, анализ, синтез, индукция, дедукция, актуализация, макетирование, визуализация, виртуализация и др.
\end{enumerate}

    % \include{section02_0}
    % \include{section03_0}
    % \include{section04_theor_q}
    
    % Labs from old book
    % \include{section05_0_lab1}
    % \include{section06_0_lab2}
    % \include{section07_0_lab3}
    % \include{section08_0_lab4}
    % \include{section09_0_lab5}
    % \include{section10_0_lab6}
    \include{conclusion}
    \addbibliography{Список использованных и рекомендованных источников}
    \include{lastPage}
    \include{backPage} %задняя страница обложки, не идёт в счёт страниц

\end{document}
