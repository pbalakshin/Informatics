\section{Сжатие данных}
\label{sec:data_compression}

\textit{\textbf{Сжатие данных} --- это процесс, обеспечивающий уменьшение объема данных путем сокращения их избыточности.}

\rightline{К. Шеннон}

\bigskip

Клода Шеннона принято считать основоположником науки о сжатии информации. Его теорема об оптимальном кодировании показывает, к чему нужно стремиться при кодировании информации и насколько та или иная информация при этом сожмется. Кроме того, им были проведены опыты по эмпирической оценке избыточности английского текста. Шенон предлагал людям угадывать следующую букву и оценивал вероятность правильного угадывания. На основе ряда опытов он пришел к выводу, что количество информации в английском тексте колеблется в пределах 0,6 – 1,3 бита на символ. Несмотря на то, что результаты исследований Шеннона были по-настоящему востребованы лишь десятилетия спустя, трудно переоценить их значение.

\bigskip

\emph{Сжатие данных} --- это частный случай \emph{кодирования} данных и важнейший аспект передачи данных, что дает возможность более оперативно передавать данные. 

\bigskip

\emph{Цель сжатия} --- уменьшение количества бит, необходимых для хранения или передачи заданной информации, что дает возможность передавать сообщения более быстро и хранить более экономно и оперативно (последнее означает, что операция извлечения данной информации с устройства ее хранения будет проходить быстрее, что возможно, если скорость распаковки данных выше скорости считывания данных с носителя информации).

\bigskip

Сжатие позволяет, например, записать больше информации на дискету, ``увеличить'' размер жесткого диска, ускорить работу с модемом и т.д. При работе с компьютерами широко используются программы-архиваторы данных формата ZIP, GZ, ARJ и других. Методы сжатия информации были разработаны как математическая теория, которая долгое время (до первой половины 80-х годов), мало использовалась в компьютерах на практике.

\bigskip

\noindentВведем ряд определений, которые будут использоваться далее в изложении материала.
\begin{description}
    \item [Алгоритм сжатия данных (алгоритм архивации)] --- это алгоритм, который устраняет избыточность записи данных.
    \item [Символ] --- наименьшая единица данных, рассматриваемая как единое целое при кодировании/декодировании.
    \item [Алфавит] --- множество всех возможных символов.
    При сжатии англоязычных текстов обычно используют множество из 128 ASCII кодов.
    При сжатии изображений множество значений пиксела может содержать 2, 16, 256 или другое количество элементов.
\end{description}

\bigskip

Рассмотрим на примере. Возьмем обычную книгу и будем считать ее содержимое как за исходные данные.
За символ мы можем взять как букву (и тогда алфавитом будет являться обычный алфавит русского языка), так и целое слово (тогда алфавит будет состоять из всех уникальных слов, встречающихся в этой книге).
Символ - не обязательно один знак.
Символом могут являться слова и даже целые предложения.

\begin{description}
    \item [Токен] --- единица данных, записываемая в сжатый поток некоторым алгоритмом сжатия.
    Токен состоит из нескольких полей фиксированной или переменной длины.
    \item [Фраза] --- фрагмент данных, помещаемый в словарь для дальнейшего использования в сжатии.
    \item [Код] --- правило соответствия набора знаков одного множества $X$ знакам другого множества $Y$.
    \item [Кодирование] --- процесс преобразования символов алфавита $X$ в символы алфавита $Y$.
    \item [Декодирование] --- процесс, обратный кодированию, при котором осуществляется восстановление данных.
    \item[Кодовый символ] --- наименьшая единица данных, подлежащая сжатию. Обычно символ --- это 1 байт, но он может быть битом, тритом {0,1,2}, или чем-либо еще.
    \item[Кодовое слово] --- это последовательность кодовых символов из алфавита кода.
    \item [Средняя длина кодового слова] --- это величина, которая вычисляется как взвешенная вероятностями сумма длин всех кодовых слов. \\ То есть:
    $$ L = \sum_{i=1}^{N}p_{i} \times l_{i} $$
    где:
    \begin{description}[noitemsep]
        \item [$N$] --- количество кодовых слов в алфавите;
        \item [$p_{i}$] --- вероятность появления кодового слова в последовательности;
        \item [$l_{i}$] --- количество символов в кодовом слове (длина кодового слова).
    \end{description}
    Сумма всех $p_{i}$ должна быть равна единице.
\end{description}

Если все кодовые слова имеют одинаковую длину, то код называется \textbf{равномерным} (фиксированной длины).
Если встречаются слова разной длины, то --- \textbf{неравномерным} (переменной длины)

Классический пример равномерного кода --- таблица символов ASCII.
Любой символ из этой таблицы будет закодирован одним байтом (один символ всегда кодируется двумя цифрами в шестнадцатеричной системе счисления).
Пример неравномерного кода --- азбука Морзе.

\paragraph{Характеристики кодирования}

\begin{equation*}
\mbox{Коэффициент сжатия} = \frac{\mbox{Размер входного потока}}{\mbox{Размер выходного потока}}
\end{equation*}

\noindentЗначения больше 1 обозначают сжатие, а значения меньше 1 --- расширение.

\begin{equation*}
\mbox{Отношение сжатия} = \frac{\mbox{Размер выходного потока}}{\mbox{Размер входного потока}}    
\end{equation*}

\noindentЗначение $0,6$ означает, что данные занимают $60\%$ от первоначального объема.
Значения больше 1 означают, что выходной поток больше входного (отрицательное сжатие, или расширение).

\bigskip

\paragraph{Сжатие данных можно разделить на два основных типа:}

\begin{description}
    \item [Сжатие без потерь (\emph{полностью обратимое})] --- это метод сжатия данных, при котором ранее закодированная порция данных восстанавливается после их распаковки полностью без внесения изменений.
    Для каждого типа данных, как правило, существуют свои оптимальные алгоритмы сжатия без потерь.
    \item [Сжатие с потерями (\emph{частично обратимое})] --- это метод сжатия данных, при котором для обеспечения максимальной степени сжатия исходного массива данных часть содержащихся в нем данных отбрасывается.
    Для текстовых, числовых и табличных данных использование программ, реализующих подобные методы сжатия, является неприемлемыми.
    В основном такие алгоритмы применяются для сжатия аудио- и видеоданных, статических изображений.
\end{description}

\paragraph{Существуют два основных способа проведения сжатия:}

\begin{description}
    \item [Статические методы] --- методы сжатия, присваивающие коды переменной длины символам входного потока, причем более короткие коды присваиваются символам или группам символам, имеющим большую вероятность появления во входном потоке.
    Лучшие статистические методы применяют кодирование Хаффмана.
    \item [Словарное сжатие] --- это методы сжатия, хранящие фрагменты данных в "словаре" (некоторая структура данных).
    Если строка новых данных, поступающих на вход, идентична какому-либо фрагменту, уже находящемуся в словаре, в выходной поток помещается указатель на этот фрагмент. 
\end{description}

\paragraph{Основные понятия кодирования:}

\subparagraph{Префиксный код} --- это код, в котором никакое кодовое слово не является префиксом любого другого кодового слова. Эти коды имеют переменную длину.

\subparagraph{Оптимальный префиксный код} --- это префиксный код, имеющий минимальную среднюю длину.

\subsection{Алгоритмы сжатия данных}
\label{subsec:data_compression_algorithms}


\subsubsection{Алгоритм Шеннона-Фано}
\label{subsubsec:shannon-fano-algo}

\begin{figure}[H]
\begin{wrapfigure}{l}{0.25\textwidth}
    \centering
    \vspace{-\intextsep}
    \includegraphics[width=0.25\textwidth]{person/fano-robert}
    \caption*{Роберт Фано\\1917 -- 2016}
\end{wrapfigure}

Алгоритм Шеннона-Фано --- один из первых алгоритмов сжатия.
Его сформулировали два ученых --- Клод Шеннон и Роберт Фано.
Алгоритм основан на частоте повторения.
Так, часто встречающийся символ кодируется кодом меньшей длины, а редко встречающийся --- кодом большей длины.
Коды, полученные при кодировании, префиксные, что позволяет декодировать любую последовательность.

\bigskip

Алгоритм Шеннона-Фано для некоторых последовательностей может сформировать неоптимальные коды.
\end{figure}

\bigskip
\paragraph{Описание алгоритма Шеннона-Фано}
\begin{enumerate}
    \item Символы входного (первичного) алфавита выписывают по убыванию вероятностей --- это корень будущего дерева;
    \item Строится дерево от корня к листьям. Находится середина, которая делит корень на два узла. Эти узлы (суммы вероятностей символов алфавита) примерно равны;
    \item Полученные узлы --- листья дерева. Левому узлу (с большей суммарной вероятностью) присваивается значение $1$, а правому --- $0$;
    \item Шаги 2--3 повторяются, пока в листьях дерева не останется один символ первичного алфавита.
    \item Символ входного (первичного) алфавита кодируется последовательностью нулей и единиц в соответствии с распределением их от корня к листьям (узлам).
\end{enumerate}

\example{1}
\task составить код Шеннона-Фано для последовательности $AAABCCCCCDEEEF$. Найти среднюю длину кодового слова.
\solution в последовательности $AAABCCCCCDEEEF$ алфавит состоит из 6 символов: A, B, C, D, E, F. Выпишем символы первичного алфавита по убыванию вероятностей:
\begin{table}[H]
\centering
\begin{tabular}{|r|c|c|c|c|c|c|}
\hline
\textbf{Символ} & C & A & E & B & D & F \\ \hline
\textbf{Вероятность} & $^5/_{14}$ & $^3/_{14}$ & $^3/_{14}$ & $^1/_{14}$ & $^1/_{14}$ & $^1/_{14}$ \\ \hline
\end{tabular}
\end{table}

\noindent Полученная последовательность $CAEDBF$ является корнем будущего дерева.

\noindent Построим дерево от корня к листьям:

\begin{table}[H]
    \centering
    \begin{tabular}{cccccc}
        \multicolumn{6}{c}{CAEBDF ($^5/_{14} + ^3/_{14} + ^3/_{14} + ^1/_{14} + ^1/_{14} + ^1/_{14}$)} \\
        % \multicolumn{6}{c}{CAEBDF ($\frac{5}{14} + \frac{3}{14} + \frac{3}{14} + \frac{1}{14} + \frac{1}{14} + \frac{1}{14}$)} \\
        \multicolumn{2}{c}{$\swarrow$} & \multicolumn{4}{c}{$\searrow$} \\
        \multicolumn{2}{c}{CA ($^5/_{14} + ^3/_{14}$)} & \multicolumn{4}{c}{EBDF ($^3/_{14} + ^1/_{14} + ^1/_{14} + ^1/_{14}$)} \\
        % \multicolumn{2}{c}{CA ($\frac{5}{14} + \frac{3}{14}$)} & \multicolumn{4}{c}{EBDF ($\frac{3}{14} + \frac{1}{14} + \frac{1}{14} + \frac{1}{14}$)} \\
        $\swarrow$ & $\searrow$ & \multicolumn{2}{c}{$\swarrow$} & \multicolumn{2}{c}{$\searrow$} \\
        C ($^5/_{14}$) & A ($^3/_{14}$) & \multicolumn{2}{c}{EB ($^3/_{14} + ^1/_{14}$)} & \multicolumn{2}{c}{DF ($^1/_{14} + ^1/_{14}$)} \\
         & & $\swarrow$ & $\searrow$ & $\swarrow$ & $\searrow$ \\
         & & E ($^3/_{14}$) & B ($^1/_{14}$) & D ($^1/_{14}$) & F($^1/_{14}$) \\
    \end{tabular}
\end{table}

\noindentПрисвоим левому символу (с большей вероятностью) значение $1$, а правому --- $0$:

\begin{table}[H]
    \centering
    \begin{tabular}{cccccc}
        \multicolumn{6}{c}{CAEBDF ($^5/_{14} + ^3/_{14} + ^3/_{14} + ^1/_{14} + ^1/_{14} + ^1/_{14}$)} \\
        \multicolumn{2}{c}{\textbf{[1]} $\swarrow$} & \multicolumn{4}{c}{$\searrow$ \textbf{[0]}} \\
        \multicolumn{2}{c}{CA ($^5/_{14} + ^3/_{14}$)} & \multicolumn{4}{c}{EBDF ($^3/_{14} + ^1/_{14} + ^1/_{14} + ^1/_{14}$)} \\
        \textbf{[1]} $\swarrow$ & $\searrow$ \textbf{[0]} & \multicolumn{2}{c}{\textbf{[1]} $\swarrow$} & \multicolumn{2}{c}{$\searrow$ \textbf{[0]}} \\
        C ($^5/_{14}$) & A ($^3/_{14}$) & \multicolumn{2}{c}{EB ($^3/_{14} + ^1/_{14}$)} & \multicolumn{2}{c}{DF ($^1/_{14} + ^1/_{14}$)} \\
         & & \textbf{[1]} $\swarrow$ & $\searrow$ \textbf{[0]} & \textbf{[1]} $\swarrow$ & $\searrow$ \textbf{[0]} \\
         & & E ($^3/_{14}$) & B ($^1/_{14}$) & D ($^1/_{14}$) & F($^1/_{14}$) \\
    \end{tabular}
\end{table}

\noindentПолучим следующую таблицу для кодировки:

\begin{table}[H]
\centering
\begin{tabular}{|r|c|c|c|c|c|c|}
\hline
\textbf{Символ} & C & A & E & B & D & F \\ \hline
% \textbf{Вероятность} & $^5/_{14}$ & $^3/_{14}$ & $^3/_{14}$ & $^1/_{14}$ & $^1/_{14}$ & $^1/_{14}$ \\ \hline
\textbf{Код} & 11 & 10 & 011 & 010 & 001 & 000 \\ \hline
\end{tabular}
\end{table}

\noindentИсходная последовательность AAABCCCCCDEEEF кодируется следующей: $10.10.10.010.11.11.11.11.11.001.011.011.011.000 - 34$ бита.

Средняя длина кодового слова: 
$$
2 \times \frac{5}{14} + 2 \times \frac{3}{14} + 3 \times \frac{3}{14} + 3 \times \frac{1}{14} + 3 \times \frac{1}{14} + 3 \times \frac{1}{14}  = \frac{34}{14} \approx 2,4
$$


\newline
\subsubsection{Код Хаффмана}
\label{subsubsec:huffman_code}

\begin{wrapfigure}{l}{0.25\textwidth}
    \centering
    \includegraphics[width=0.25\textwidth]{huffman_david}
    \caption*{Дэвид Хаффман\\1925 - 1999}
\end{wrapfigure}

Код (алгоритм) Хаффмана был разработан в 1952 году аспирантом Массачусетского технологического института Дэвидом Хаффманом при написании им курсовой работы.

Как и алгоритм Шеннона-Фано, основан на частоте повторения. Зная вероятности символов в сообщении, можно описать процедуру построения кодов переменной длины, состоящих из целого количества битов. Символам с большей вероятностью ставятся в соответствие более короткие коды. Коды Хаффмана обладают свойством префиксности, что позволяет однозначно их декодировать.

Однако, в отличие от кодов Шеннона-Фано, коды Хаффмана всегда являются оптимальными.

Сжатие данных по Хаффману применяется при сжатии фото- и видеоизображений (JPEG, стандарты сжатия MPEG), в архиваторах (PKZIP, LZH), в протоколах передачи данных MNP5 и MNP7.

\paragraph{Алгоритм Хаффмана для неоптимальных префиксных кодов}

\begin{enumerate}
  \item Символы входного алфавита образуют список свободных узлов. Каждый узел имеет вес, равный вероятности появления символа в сжимаемом тексте (исходной последовательности). Строится дерево от листьев (узлов) к корню:
  \item Выбираются два свободных узла дерева с наименьшими весами;
  \item Создается их родитель с весом, равным их суммарному весу;
  \item Родитель добавляется в список свободных узлов, а двое его детей удаляются из этого списка.
  \item Одной дуге, выходящей из родителя (узлу с большим весом), ставится в соответствие значение $1$, а другой (узлу с меньшим весом) значение $0$.
  \item Повторяем шаги 2-4, выбирая в качестве одного из свободных узлов родителя, до тех пор, пока в списке свободных узлов не останется только один свободный узел. Он и будет считаться корнем дерева.
  \item Символ входного (первичного) алфавита кодируется последовательностью нулей и единиц в соответствии с распределением их от корня дерева к узлам (листьям).
\end{enumerate}

\example{1}

\task