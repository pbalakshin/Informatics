\subsection{Оптимальная система счисления}
\label{subsec:optimal_notation_system}

Давайте представим, что Вы по несчастливой (или счастливой) случайности попали на необитаемый остров.
Обеспокоенный количеством дней, которые Вам суждено провести на острове в ожидании спасателей, Вы решаете вести счет дней с помощью камней.
Но вот незадача --- камней на острове нашлось всего 60 штук (небогатый на камни остров оказался).
И для того, чтобы вести учет дней как можно продуктивнее (учесть как можно больше дней) необходимо выбрать систему счисления, плотность записи числа которой максимальна при данных обстоятельствах.

Существует зависимость плотности записи информации от основания системы счисления.
Если взять $N$ камней, а за основание принять число $X$, то получится $^N/_X$ разрядов, которыми можно закодировать $X^{^N/_X}$ чисел.

То есть с помощью 60 камней мы можем закодировать: $2^{30}$, $3^{20}$, $4^{15}$, $5^{12}$, $6^{10}$, $10^{6}$, $12^{2}$, $15^{4}$, $20^{3}$, $30^{2}$ или $60^1$ чисел.
Все зависит от того, какую систему счисления мы выберем. Возведя все числа в степени, мы увидим, что самое большое из них это $3^{20} = 3486784401$.

Удельная натуральнологарифмическая плотность записи числа зависит от основания системы счисления $x$ и выражается функцией $y = \dfrac{\ln{x}}{x}$.
Эта функция имеет максимум при $x = e = 2.718281828\ldots$

\begin{center}
\begin{tikzpicture}
\begin{axis}[
    xlabel=$x$, ylabel=$y$,
    domain=0:20,
    xmin=-1, xmax=20,
    ymin=0.1, ymax=0.5,
    extra x ticks={2.71},
    extra x tick labels={$e$},
]
\addplot[red, smooth] {ln(x)/x};
\draw[dashed] (axis cs:2.71,0) -- (axis cs:2.71,0.368);
\end{axis}
\end{tikzpicture}
\end{center}


\begin{figure}[H]
\centering
\includegraphics[width=6cm]{1_2_3(1)}
\end{figure}

Таким образом, самая оптимальная система счисления имеет основание равное $e = 2.718281828\ldots$, то есть нецелочисленное.
Она обладает наибольшей плотностью записи информации.

Возвращаясь к нашему пребыванию на острове, мы не можем взять дробное основание для системы счисления. Поэтому мы берем самое близкое целое к $e$ - это 3.