\subsubsection{Алгоритм Шеннона-Фано}
\label{subsubsec:shannon_fano_algorithm}

\begin{figure}[!h]
\noindent
\begin{minipage}[t]{.25\linewidth}
\vspace{0pt}
\includegraphics[width=\textwidth]{fano_robert}
\caption*{Роберт Фано\\1917 -- 2016}
\end{minipage}
\hfill
\noindent
\begin{minipage}[t]{.7\linewidth}
Алгоритм Шеннона-Фано --- один из первых алгоритмов сжатия.
Его сформулировали два ученых --- Клод Шеннон и Роберт Фано.
Алгоритм основан на частоте повторения.
Так, часто встречающийся символ кодируется кодом меньшей длины, а редко встречающийся --- кодом большей длины.
Коды, полученные при кодировании, префиксные, что позволяет декодировать любую последовательность.

\bigskip

Алгоритм Шеннона-Фано для некоторых последовательностей может сформировать неоптимальные коды.
\end{minipage}
\end{figure}

\paragraph{Описание алгоритма Шеннона-Фано}

\begin{enumerate}
    \item Символы входного (первичного) алфавита выписывают по убыванию вероятностей - это корень будущего дерева;
    \item Строится дерево от корня к листьям. Находится середина, которая делит корень на два узла. Эти узлы (суммы вероятностей символов алфавита) примерно равны;
    \item Полученные узлы - листья дерева. Левому узлу (с большей суммарной вероятностью) присваивается значение $1$, а правому - $0$;
    \item Шаги 2-3 повторяются, пока в листьях дерева не останется один символ первичного алфавита.
    \item Символ входного (первичного) алфавита кодируется последовательностью нулей и единиц в соответствии с распределением их от корня к листьям (узлам).
\end{enumerate}

\example{1}

\task составить код Шеннона-Фано для последовательности $AAABCCCCCDEEEF$. Найти среднюю длину кодового слова.

\solution в последовательности $AAABCCCCCDEEEF$ алфавит состоит из 6 символов: A, B, C, D, E, F. Выпишем символы первичного алфавита по убыванию вероятностей:
\begin{itemize}[noitemsep]
    \item Вероятность символа $C$ - $^5/_{14}$;
    \item Вероятность символа $A$ - $^3/_{14}$;
    \item Вероятность символа $E$ - $^3/_{14}$;
    \item Вероятность символа $B$ - $^1/_{14}$;
    \item Вероятность символа $D$ - $^1/_{14}$;
    \item Вероятность символа $F$ - $^1/_{14}$;
\end{itemize}

\noindent Полученная последовательность $CAEDBF$ является корнем будущего дерева.

\noindent Построим дерево от корня к листьям:

% \begin{table}[H]
% \centering
% \begin{tabular}{c c c c c c}
% \multicolumn{6}{c}{CAEBDF $ (^5/_{14} + ^3/_{14} + ^3/_{14} + ^1/_{14} + ^1/_{14} + ^1/_{14})$} \\
% \multicolumn{2}{c}{$\downarrow$} & \multicolumn{4}{c}{$\downarrow$} \\
% \multicolumn{2}{c}{CA ($^5/_{14} + ^3/_{14}$)} & \multicolumn{4}{c}{EBDF ($^3/_{14} + ^1/_{14} + ^1/_{14} + ^1/_{14}$)} \\
% $\downarrow$ & $\downarrow$ & \multicolumn{2}{c}{$\downarrow$} & \multicolumn{2}{c}{$\downarrow$} \\
% C ($^5/_{14}$) & A ($^3/_{14}$) & \multicolumn{2}{c}{EB ($^3/_{14} + ^1/_{14}$)} & \multicolumn{2}{c}{DF ($^1/_{14} + ^1/_{14}$)} \\
%  & & $\downarrow$ & $\downarrow$ & $\downarrow$ & $\downarrow$ \\
%  & & E ($^3/_{14}$) & B ($^1/_{14}$) & D ($^1/_{14}$) & F($^1/_{14}$) \\
% \end{tabular}
% \end{table}

\begin{figure}[H]
    \centering
    \includegraphics[width=\textwidth]{shannon_1}
\end{figure}

\noindentПрисвоим левому символу (с большей вероятностью) значение $1$, а правому - $0$:

\begin{figure}[H]
    \centering
    \includegraphics[width=\textwidth]{shannon_2}
\end{figure}

\noindentПолучим следующую таблицу для кодировки:
\begin{table}[H]
\begin{tabular}{|c|c|c|}
\hline
Символ & Вероятность & Код \\
\hline
C & $^5/_{14}$ & 11 \\
A & $^3/_{14}$ & 10 \\
E & $^3/_{14}$ & 011 \\
B & $^1/_{14}$ & 010 \\
D & $^1/_{14}$ & 001 \\
F & $^1/_{14}$ & 000 \\
\hline
\end{tabular}
\end{table}

\noindentИсходная последовательность AAABCCCCCDEEEF кодируется следующей: $10.10.10.010.11.11.11.11.11.001.011.011.011.000 - 34$ бита.
Средняя длина кодового слова: 
$$
2 \times \frac{5}{14} + 2 \times \frac{3}{14} + 3 \times \frac{3}{14} + 3 \times \frac{1}{14} + 3 \times \frac{1}{14} + 3 \times \frac{1}{14}  = \frac{34}{14} \approx 2,4
$$