% !TeX root = ../main.tex
\section{Единицы измерения объема данных}\label{sec:data-measurement-units}

В области цифровой и вычислительной техники двоичная система счисления (основанная на степени двойки) получила широкое распространение. В следствие этого стали употреблять двоичные приставки:

\noindent
$\qty{1}{\KB} = \qty[exponent-base=2]{e10}{\B}$ --- $1$ килобайт равен $2^{10}$ байт \\
$\qty{1}{\MB} = \qty[exponent-base=2]{e20}{\B}$ --- $1$ мегабайт равен $2^{20}$ байт

Однако, такая система противоречит СИ, которая использует десятичные приставки (основанные на степени десяти): \\
$\qty{1}{\kilo\relax} = \num{e3}, \qty{1}{\mega\relax} = \num{e6}$

Поначалу это противоречие не было существенной проблемой.
Число $2^{10} = 1024$ достаточно близко к тысяче и при объемах памяти, измерявшихся килобайтами, ошибка была всего в 2,4\%.
Но по мере развития технологий, разница между \enquote{двоичным} и \enquote{десятичным} гигабайтом была в 7\%.

Тогда IEEE, Институт инженеров электротехники и электроники (англ. Institute of Electrical and Electronics Engineers), утвердил стандарт IEEE 1541-2002.

\paragraph{Рекомендации стандарта IEEE 1541-2002}\label{par:ieee-1541-2002}
\begin{itemize}
    \item Бит (bit) (символ '\unit{\bit}') --- двоичный знак;
    \item Байт (byte) (символ '\unit{\byte}') --- равен 8 битам (\qty{1}{\byte} = \qty{8}{\bit});
    \item Киби (kibi) (символ '\unit{\kibi\relax}') --- $2^{10} = 1024$;
    \item Меби (mebi) (символ '\unit{\mebi\relax}') --- $2^{20} = 1048576$;
    \item Гиби (gibi) (символ '\unit{\gibi\relax}') --- $2^{30} = 1073741824$;
    \item Теби (tebi) (символ '\unit{\tebi\relax}') --- $2^{40} = 1099511627776$;
\end{itemize}

Сегодня в кибибайтах, мебибайтах и т.д. измеряется память --- оперативная память, жесткие диски, flash-накопители.
Однако, скорость передачи данных измеряется в килобайтах и мегабайтах в секунду (например 512 kbps (kilobits per second) --- 512 килобит в секунду).
Операционные системы считают по-разному.
*nix системы (unix, linux) используют \enquote{двоичные} приставки, в то время как Windows использует приставки СИ.
