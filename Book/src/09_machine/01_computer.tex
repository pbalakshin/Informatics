% !TeX root = ../../main.tex
\subsection{ЭВМ Джона фон Неймана}
\label{subsec:von-neumann-evm}

\noindent
\begin{minipage}[t]{\textwidth}
\begin{wrapfigure}{l}{0.25\textwidth}
    \centering
    \vspace{-\intextsep}
    \includegraphics[width=0.25\textwidth]{person/neumann_john}
    \caption*{Джон фон Нейман \\ 1903 -- 1957}
\end{wrapfigure}

В 1930-х годах началась разработка архитектуры ЭВМ для военно-морской артиллерии по заказу правительства США. В разработке участвовали Гарвардский университет и Принстонский университет (в том числе и Джон фон Нейман). На рисунке \ref{fig:von-neumann-evm} приведена схема ЭВМ, предложенная фон Нейманом.
\end{minipage}
\vspace{3\baselineskip}

\begin{figure}[H]
    \includegraphics[width=\textwidth]{9.1-von-neumann}
    \caption{Структурная схема ЭВМ фон Неймана2}
    \label{fig:von-neumann-evm}
\end{figure}


\subsection{Узлы ЭВМ фон Неймана}
\begin{itemize}
    \item \textbf{Процессор} --- исполнитель машинных инструкций (кода программ), главная часть аппаратного обеспечения ЭВМ. \\
        В состав процессора входят:
        \begin{itemize}[noitemsep]
            \item устройство управления выборкой команд из памяти и их выполнением;
            \item арифметико-логическое устройство, производящее операции над данными;
            \item регистры, осуществляющие временное хранение данных и состояний процессора;
            \item схемы для управления и связи с подсистемами памяти и ввода-вывода.
        \end{itemize}
    \item \textbf{Устройства ввода} обеспечивают считывание данных с носителей информации и ее представление в форме электрических сигналов, воспринимаемых другими устройствами ЭВМ (процессором или памятью) (мышь, клавиатура, сканер).
    \item \textbf{Устройства вывода} представляют результаты обработки данных в ЭВМ в форме, удобной для визуального восприятия человеком (монитор, принтер) или хранения (DVD-привод, стример). При необходимости они обеспечивают запоминание результатов на носителях, с которых эти результаты могут быть снова введены в ЭВМ для дальнейшей обработки (перфоленты, магнитная лента, магнитный диск и т. п.), или передачу результатов на исполнительные органы управляемого объекта (например, робота).
    \item \textbf{Устройства ввода-вывода} используются как для хранения данных, так и для их считывания (жесткий диск, флешка).
\end{itemize}

\subsubsection{Принципы работы архитектуры фон Неймана}
Бёркс, Голдстайн и фон Нейман в 1946 г. в книге ``Предварительное рассмотрение логического конструирования электронного вычислительного устройства''  описали принципы:
\begin{itemize}
    \item \textbf{Принцип двоичного кодирования} --- вся информация, поступающая в ЭВМ, кодируется с помощью двоичных сигналов.
    \item \textbf{Принцип однородности памяти} --- программы и данные хранятся в одной и той же памяти. Поэтому ЭВМ не различает, что хранится в данной ячейке памяти - число, текст или команда. Над командами можно выполнять такие же действия, как и над данными.
    \item \textbf{Принцип адресуемости памяти} --- структурно основная память состоит из пронумерованных ячеек, процессору в произвольный момент времени доступна любая ячейка.
    \item \textbf{Принцип жесткости архитектуры} --- неизменяемость в процессе работы топологии, архитектуры, списка команд.
    %\newpage
    \item \textbf{Принцип программного управления}:
        \begin{enumerate}
            \item В начале процессору сообщается адрес первой команды программы (который заносится в специальный \textbf{регистр команд}), после этого программа управляет сама собой.
            \item После выполнения команды процессор увеличивает адрес, хранимый в регистре команд, на длину только что выполненной команды, чтобы получить адрес следующей команды. Так можно выполнить цепочку команд из \textbf{последовательно} расположенных ячеек памяти.
            \item Существуют специальные \textbf{команды переходов}, которые сразу содержат в себе адрес следующей команды. После выполнения таких команд указанный адрес просто заносится в регистр команд. Так можно выполнить цепочку команд из \textbf{непоследовательно} расположенных ячеек памяти.
        \end{enumerate}
\end{itemize}
