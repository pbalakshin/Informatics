Люди пытались сделать компьютер достаточно давно.
Первым (после антикитерского механизма) был \textbf{механизм для суммирования и умножения}. Изобрел его Вильгельм Шиккард (\emph{1592 -- 1635}) в 1623 году.

Машина содержала суммирующее и множительное устройства, а также механизм для записи промежуточных результатов.
Первый блок --- шестиразрядная суммирующая машина --- представлял собой соединение зубчатых передач.
На каждой оси имелись шестерня с десятью зубцами и вспомогательное однозубое колесо --- палец.
Палец служил для того, чтобы передавать единицу в следующий разряд (поворачивать шестеренку на десятую часть полного оборота, после того как шестеренка предыдущего разряда сделает такой оборот).
При вычитании шестеренки следовало вращать в обратную сторону.
Контроль хода вычислений можно было вести при помощи специальных окошек, где появлялись цифры.
Для перемножения использовалось устройство, чью главную часть составляли шесть осей с ``навернутыми'' на них таблицами умножения.

\bigskip
Второй арифметической машиной был \textbf{механизм для суммирования и вычитания ``Паскал\'{и}на''}. \\
Изобрел его Блез Паскаль (\emph{1623--1662}) в 1642 году.
Машина Паскаля представляла собой механическое устройство в виде ящичка с многочисленными связанными одна с другой шестеренками.
Складываемые числа вводились в машину при помощи соответствующего поворота наборных колесиков.
На каждое из этих колесиков, соответствовавших одному десятичному разряду числа, были нанесены деления от 0 до 9.
При вводе числа, колесики прокручивались до соответствующей цифры.
Совершив полный оборот, избыток над цифрой 9 колесико переносило на соседний разряд, сдвигая соседнее колесо на 1 позицию.

\bigskip
Следующим был \textbf{арифмометр Лейбница}, умеющий выполнять операции сложения, вычитания, деления и умножения. Изобрел ее Готфрид Вильгельм Лейбниц (\emph{1646 -- 1716}) в 1673 году.

Сложение чисел выполнялось при помощи связанных друг с другом колес, так же как на ``Паскалине''.
Добавленная в конструкцию движущаяся часть и специальная рукоятка, позволявшая крутить ступенчатое колесо (в последующих вариантах машины --- цилиндры), позволяли ускорить повторяющиеся операции сложения, при помощи которых выполнялось деление и перемножение чисел.
Необходимое число повторных сложений выполнялось автоматически.

Прообразом современного компьютера стала \textbf{идея создания универсальной аналитической вычислительной машины}, которую выдвинул Чарльз Бэббидж (\emph{1791 --- 1871}) в 1823 году.

\bigskip
В 1822 году Бэббидж построил \textbf{малую разностную машину}. Ее работа была основана на методе конечных разностей. Малая машина была полностью механической и состояла из множества шестеренок и рычагов. В ней использовалась десятичная система счисления. Она оперировала 18-разрядными числами с точностью до восьмого знака после запятой и обеспечивала скорость вычислений 12 членов последовательности в 1 минуту. Малая разностная машина могла считать значения многочленов 7-й степени.

\bigskip
И в том же 1822 году Бэббидж задумался о создании \textbf{большой разностной машины} предназначенной для автоматизации вычислений путем аппроксимации функций многочленами и вычисления конечных разностей.
Возможность приближенного представления в многочленах логарифмов и тригонометрических функций позволяло бы рассматривать эту машину как довольно универсальный вычислительный прибор.
В 1823 году он приступил к проектированию, однако, в 1842 году государство отказалось финансировать проект и машина так и не была достроена. 
Но для развития вычислительной техники имело значение другое: идея создания аналитической вычислительной машины.
В единую логическую схему Бэббидж увязал арифметическое устройство (названное им ``мельницей''), регистры памяти, объединенные в единое целое ("склад''), и устройство ввода-вывода, реализованное с помощью перфокарт трех типов.
Перфокарты операций переключали машину между режимами сложения, вычитания, деления и умножения.
Перфокарты переменных управляли передачей данных из памяти в арифметическое устройство и обратно.
Числовые перфокарты могли быть использованы как для ввода данных в машину, так и для сохранения результатов вычислений, если памяти было недостаточно.

\bigskip
Следующим этапом в развитии вычислительной техники стал \textbf{электромеханический перфокарточный табулятор для переписи населения}, который изобрел Герман Холлерит (\emph{1860 -- 1929}) в 1880 году.

Табуляторы предназначены для автоматической обработки (суммирования и категоризации) числовой и буквенной информации, записанной на перфокартах, с выдачей результатов на бумажную ленту или специальные бланки.
Умножение и деление выполнялись методом последовательного многократного сложения и вычитания
Работа табулятора производилась в соответствии с набираемой на коммутационной панели программой.

\bigskip
В 1911 году Алексей Николаевич Крылов (\emph{1863 -- 1945}) изобрел \textbf{аналоговый решатель дифференциальных уравнений}. 
Он интегрировал обыкновенные дифференциальные уравнения.

\bigskip
В 1919 году Николай Николаевич Павловский (\emph{1884 -- 1937}) сконструировал \textbf{аналоговую вычислительную машину (АВМ)}.
Она была создана для реализации метода исследования природных явлений при помощи аналого-математического моделирования, который разработал Павловский в том же 1919 году.

Наличие заданного набора исполняемых команд и программ было характерной чертой первых компьютерных систем.
Сегодня подобный дизайн применяют с целью упрощения конструкции вычислительного устройства.
Так, настольные калькуляторы, в принципе, являются устройствами с фиксированным набором выполняемых программ.
Их можно использовать для математических расчётов, но невозможно применить для обработки текста и компьютерных игр, для просмотра графических изображений или видео.
Изменение встроенной программы для такого рода устройств требует практически полной их переделки, и в большинстве случаев невозможно.
Впрочем, перепрограммирование ранних компьютерных систем всё-таки выполнялось, однако требовало огромного объёма ручной работы по подготовке новой документации, перекоммутации и перестройки блоков и устройств и т. п.

Всё изменила идея хранения компьютерных программ в общей памяти.
Ко времени её появления использование архитектур, основанных на наборах исполняемых инструкций, и представление вычислительного процесса как процесса выполнения инструкций, записанных в программе, чрезвычайно увеличило гибкость вычислительных систем в плане обработки данных.
Один и тот же подход к рассмотрению данных и инструкций сделал лёгкой задачу изменения самих программ.
