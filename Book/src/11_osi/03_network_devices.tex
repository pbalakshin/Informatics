% !TeX root = ../../main.tex
\subsection{Сетевые устройства}

\begin{minipage}{\textwidth}
\centering
\includegraphics[width=\textwidth]{src/11_osi/11_3.jpg}
\end{minipage}

\paragraph{Сравнение коммутатора и маршрутизатора}

\begin{table}[H]
    \centering
    \caption{Сравнение коммутатора и маршрутизатора}
    \label{tab:switch-vs-router}
    \begin{tabularx}{\linewidth}{|C|C|C|}
        \hline
        \thead{Свойство}         & \thead{Коммутатор \\ (switch)} & \thead{Маршрутизатор \\ (router)} \\ \hline
        Наличие MAC-адреса       & Нет & Много (ровно по одному на каждый порт/антенну) \\ \hline
        Наличие IP-адреса        & Нет & Много (минимум по одному на каждый порт/антенну) \\ \hline
        Уровни OSI-модели        & 1,2 & 1,2,3 \\ \hline
        Умение выбирать маршруты & Нет (так как в локальной сети всегда только один маршрут) & Да \\ \hline
        Назначение               & Обмен данными между компьютерами внутри локальной сети    & Обмен данными между несколькими локальными сетями \\ \hline
    \end{tabularx}
\end{table}

\textbf{Примечание:} существуют гибридные устройства, совмещающие в себе коммутатор и маршрутизатор (они используются у большинства пользователей домашнего интернета, однако в корпоративных сетях применяются реже).
