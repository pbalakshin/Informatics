% !TeX root = ../../main.tex
% TODO: remove?
\begin{table}[!h]
\centering
\small % Уменьшаем шрифт, чтобы таблица поместилась на узкой странице A5
\begin{tabular}{|c|c|c|c|c|p{4.5cm}|p{4.5cm}|} % Используем p-колонки для переноса текста
\hline
$x$ & 1 & 1 & 0 & 0 & \multirow{2}{*}{Обозначение} & \multirow{2}{*}{Название} \\
$y$ & 1 & 0 & 1 & 0 & & \\
\hline
\multirow{2}{*}{F} & \multirow{2}{*}{0} & \multirow{2}{*}{0} & \multirow{2}{*}{0} & \multirow{2}{*}{0} & F2,0 = $FALSE$ & Противоречие, \\
& & & & & & логический нуль \\
\hline
\multirow{3}{*}{F} & \multirow{3}{*}{0} & \multirow{3}{*}{0} & \multirow{3}{*}{0} & \multirow{3}{*}{1} & F2,1 = $x\downarrow y = x \text{ NOR } y = $ & Cтрелка Пирса, НЕ-ИЛИ, \\
& & & & & $= \text{NOR}(x,y) = x $НЕ-ИЛИ & антидизъюнкция \\
& & & & & $y = $НЕ-ИЛИ$(x,y)$ & \\
\hline
\multirow{2}{*}{F} & \multirow{2}{*}{0} & \multirow{2}{*}{0} & \multirow{2}{*}{1} & \multirow{2}{*}{0} & F2,2 = $x\leftarrow /y$ & Отрицание \\
& & & & & & обратной импликации \\
\hline
F & 0 & 0 & 1 & 1 & F2,3 = $\neg x $ & Отрицание \\
\hline
\multirow{2}{*}{F} & \multirow{2}{*}{0} & \multirow{2}{*}{1} & \multirow{2}{*}{0} & \multirow{2}{*}{0} & F2,4 = $x\rightarrow /y $ & Материальная \\
& & & & & & обратная импликация \\
\hline
F & 0 & 1 & 0 & 1 & F2,5 = $ \neg y$ & Отрицание \\
\hline
\multirow{4}{*}{F} & \multirow{4}{*}{0} & \multirow{4}{*}{1} & \multirow{4}{*}{1} & \multirow{4}{*}{0} & F2,6 = $x \oplus y = $ & Сложение по модулю 2, \\
& & & & & $= x \text{ XOR } y = \text{XOR}(x,y) = $ & исключающее ``или'' \\
& & & & & $= x >< y = x <> y = $ & сумма Жегалкина, \\
& & & & & $= x \text{ NE } y = \text{NE}(x,y)$ & не равно \\
\hline
\multirow{3}{*}{F} & \multirow{3}{*}{0} & \multirow{3}{*}{1} & \multirow{3}{*}{1} & \multirow{3}{*}{1} & F2,7 = $x | y = $ & Штрих Шеффера, \\
& & & & & $= \text{NAND}(x,y) = x \text{ NAND } y = $ & НЕ-И, 2И-НЕ, \\
& & & & & $= x $ НЕ-И $y = $ НЕ-И$(x,y)$ & антиконъюнкция \\
\hline
\multirow{4}{*}{F} & \multirow{4}{*}{1} & \multirow{4}{*}{0} & \multirow{4}{*}{0} & \multirow{4}{*}{0} & F2,8 = $x \wedge y = x \cdot y = $ & Конъюнкция, \\
& & & & & $= xy = x \& y = x \text{ AND } y = $ & 2И, минимум \\
& & & & & $= \text{AND}(x,y) = x $И$ y = $ & \\
& & & & & $= \text{И}(x,y) = \min(x,y)$ & \\
\hline
\multirow{2}{*}{F} & \multirow{2}{*}{1} & \multirow{2}{*}{0} & \multirow{2}{*}{0} & \multirow{2}{*}{1} & F2,9 = $(x \equiv y) = x \sim y = $ & Эквивалентность \\
& & & & & $= x \leftrightarrow y = x \text{ EQV } y = \text{EQV}(x,y)$ & равенство \\
\hline
F & 1 & 0 & 1 & 0 & F2,10 = $y $ & Проекция, повторение \\
\hline
\multirow{2}{*}{F} & \multirow{2}{*}{1} & \multirow{2}{*}{0} & \multirow{2}{*}{1} & \multirow{2}{*}{1} & F2,11 = $x \to y = x \supset y = $ & Импликация \\
& & & & & $= x \le y = x \text{ LE } y = \text{LE}(x,y)$ & следование \\
\hline
F & 1 & 1 & 0 & 0 & F2,12 = $x $ & Проекция, повторение \\
\hline
F & 1 & 1 & 0 & 1 & F2,13 = $x \leftarrow y $ & Обратная импликация \\
\hline
\multirow{4}{*}{F} & \multirow{4}{*}{1} & \multirow{4}{*}{1} & \multirow{4}{*}{1} & \multirow{4}{*}{0} & F2,14 = $x \vee y = x + y = $ & Дизъюнкция, \\
& & & & & $= x \text{ OR } y = \text{OR}(x,y) = $ & 2ИЛИ, максимум \\
& & & & & $= x$ ИЛИ $y = \text{ИЛИ}(x,y) = $ & \\
& & & & & $= \max(x,y)$ & \\
\hline
F & 1 & 1 & 1 & 1 & F2,15 = $TRUE $ & Тавтология \\
\hline
\end{tabular}
\caption{Таблица значений и названий булевых функций от двух переменных}
\end{table}
