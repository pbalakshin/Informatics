\subsection{Определение} % TODO: Может определениЯ?
\label{subsec:definitions}

\begin{wrapfigure}{l}{0.25\textwidth}
    \centering
    \vspace{-\intextsep}
    \includegraphics[width=0.25\textwidth]{person/boole_george}
    \caption*{Джордж Буль \\ 1815 -- 1864}
\end{wrapfigure}

\textbf{Алгебра двоичной логики} --- раздел математики, в котором изучаются логические операции над высказываниями. Чаще всего предполагается, что высказывания могут быть только истинными или ложными, то есть используется так называемая бинарная или двоичная логика. Один из основателей алгебры логики --- Джордж Буль.

\textbf{Логическая (булева) переменная} --- такая переменная, значения которой могут быть лишь ``1'' или ``0''.
\emph{В естественном языке:} $\text{``булева переменная''} = \text{``высказывание''}$.

\textbf{Высказывание} --- утверждение, про которое можно однозначно сказать, истинно оно или ложно.

\emph{Обозначения:}
\begin{itemize}
  \item ``истина'' и ``ложь''
  \item ``true'' и ``false''
  \item ``1'' и ``0''
\end{itemize}

Например, высказывание ``Москва --- столица России'' --- истина, а ``трава синего цвета'' --- ложь.

\bigskip
\textbf{Логическая (булева) функция $f(x, y, z, …)$} --- некоторая функциональная зависимость, в результате выполнения логических операций над логическими переменными $x, y, z,\dots$ получает значение 0 или 1.

\textbf{Таблица истинности} --- таблица всех значений некоторой \emph{логической функции}.


\paragraph{Основные операции}

\noindent
{
\setlength{\tabcolsep}{2pt}
\begin{tabular}{r c l}
    $ \overline{x} $ & --- & отрицание или инверсия; \\
    $ x \vee y $     & --- & дизъюнкция или логическое сложение; \\
    $ x \wedge y $   & --- & конъюнкция или логическое умножение; \\
\end{tabular}
}

\bigskip
Кроме указанных трех базовых операций можно с их помощью ввести еще следующие важные операции алгебры предикатов (можно их назвать небазовыми операциями):

\noindent
\begin{description}[noitemsep]
    \item [$(x\to y) \equiv (\bar{x} \vee y)$] --- импликация;
    \item [$(x\leftrightarrow y) \equiv (x \wedge y \vee \bar{x} \wedge \bar{y})$] --- эквиваленция;
\end{description}

Операции импликации и эквиваленции хотя и часто используются, но не являются базовыми, ибо они определяемы через три введенные выше базовые операции.
При выполнении логических операций в компьютере они сводятся к поразрядному сравнению битовых комбинаций.
Эти операции достаточно быстро (аппаратно) выполняемы, так как сводятся к выяснению совпадения или несовпадения битов.

В логических формулах определено старшинство операций, например: скобки, отрицание, конъюнкция, дизъюнкция (остальные, небазовые операции пока не учитываем).

\noindent
Всегда истинные формулы называют \textbf{тавтологиями}.

\textbf{Логические функции} эквивалентны, если совпадают их таблицы истинности, то есть совпадают области определения и значения, а также сами значения функции при одних и тех же наборах переменных из числа всех допустимых значений. Если это совпадение происходит на части множества допустимых значений, то формулы называются эквивалентными лишь на этой части (на этом подмножестве).

Задача \textbf{упрощения логического выражения} состоит в преобразовании его к более простому (по числу переменных, операций или операндов) эквивалентному выражению. Наиболее простой вид получается при сведении функции к постоянной – 1 (истина) или 0 (ложь).
