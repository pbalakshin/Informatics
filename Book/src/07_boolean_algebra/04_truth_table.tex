\subsection{Таблица истинности}
\label{subsec:truth-table}

Существует только 4 различные логические функции одной переменной $F(x)$. 
Любая другая функция (даже самая сложная вида $F(x) = \overline{x} \vee x \wedge \overline{x}$) будет иметь одну из следующих таблиц истинности:
\begin{table}[H]
    \centering
    \renewcommand{\arraystretch}{1}
    \begin{tabular}{|c||c|c|c|c|}
        \hhline{-||----}
        $x$ & $F_0(x)$ & $F_1(x)$ & $F_2(x)$ & $F_3(x)$ \\
        \hhline{=::====}
        0 & 0 & 1 & 0 & 1 \\
        \hhline{-||----}
        1 & 0 & 0 & 1 & 1 \\
        \hhline{-||----}
    \end{tabular}
\end{table}

\noindent
Например, $F(x) = \overline{x}$ будет иметь таблицу истинности $F_1(x)$, а $F(x) = 1$ - $F_3(x)$


Обычно логическую функцию $F(x)$ обозначают как $FK,N$, где $K$ - количество операндов, а $N$ - число в десятичной системе счисления, которое при переводе в двоичную является таблицей истинности функции. Например, $F(x) = \overline{x}$ обозначается как $F1,2$, так как одна переменная и $2_{10} = 01_{2}$, что является таблицей истинности функции (смотреть снизу вверх).

\bigskip
Для $k$ переменных будет существовать $N = 2^{2^{k}}$ функций. Рассмотрим таблицу истинности:


\begin{minipage}{2cm}
\begin{center}
$L$ штук\\
(от 00..0\\ до 11..1 \\ сверху \\ вниз)
\end{center}
\end{minipage}
\begin{minipage}{0.3cm}
\quad \\
\quad \\
$\left\{
\begin{array}{c}
\\
\\
\\
\\
\end{array}
\right.$
\quad \\
\end{minipage}
\begin{minipage}[l]{9cm}
\begin{tabular}{c|c|c|c|c|c|c|c|c|}
%\hhline{~--------}
& \multicolumn{4}{c|}{\makecell[c]{Значения $k$ штук \\ булевых операндов}} 
& \multicolumn{4}{c|}{Значение булевых функций $F$} \\
%\hhline{~--------}
& $x_1$ & $x_2$ & \dots & $x_k$ & $FK,0$ & $FK,1$ & \dots & $FK,N$ \\
%\hhline{~--------}
 & 0 & 0 & \dots & 0 & 0 & 1 & \dots & 1\\
& 0 & 0 & \dots & 1 & 0 & 0 & \dots & 1\\
& \dots & \dots & \dots & \dots & \dots & \dots & \dots & \dots\\
\multirow{4}{*}{} & 1 & 1 & \dots & 1 & 0 & 0 & \dots & 1\\
%\hhline{~--------}
\multicolumn{5}{c}{} & \multicolumn{4}{c}{$\underbrace{\qquad \qquad \qquad \qquad \qquad \qquad \qquad}_{\mbox{от N штук (от 00..0 }}$} \\
\multicolumn{5}{c}{} & \multicolumn{4}{c}{до 11..1 слева направо)}
\end{tabular}
\end{minipage}

\bigskip
Так как всего $k$ переменных, то для них будет $L = 2^k$ строк в таблице истинности, а количество булевых функций будет равно $N = 2^L$. Отсюда:

\[
\left.
\begin{array}{l} L = 2^k \\ N = 2^L \end{array}
\right\}
\Rightarrow N = 2^{2^{k}}
\]

% \begin{center}
% \begin{tabular}{c c}
% $L = 2^k$ & \multirow{2}{*}{$\Rightarrow N = 2^{2^{k}}$} \\
% $N = 2^L$ & \\
% \end{tabular}
% \end{center}

Так, для 1 переменной будет 4 булевых функции, для 2 переменных - 16 и так далее.

\begin{nirtable}{tab:2-variable-func-names}{Таблица значений и названий булевых функций от двух переменных}{|c|c|c|c|c|l|l|}{7}{
    \thead{$x$} & \thead{1} & \thead{1} & \thead{0} & \thead{0} & \multirow{2}{*}{\thead{Обозначение}} & \multirow{2}{*}{\thead{Название}}\\
    \thead{$y$} & \thead{1} & \thead{0} & \thead{1} & \thead{0} & &
}
    $F$ & 0 & 0 & 0 & 0 & \makecell[l]{$F_{2,0} = FALSE$} & \makecell[l]{Противоречие, \\ логический нуль} \\ \hline
    $F$ & 0 & 0 & 0 & 1 & \makecell[l]{$F_{2,1} = x \downarrow y =$ \\ $x \text{ NOR } y = \text{NOR}(x,y) =$ \\ $x \text{ НЕ-ИЛИ } y = \text{НЕ-ИЛИ}(x,y)$} & \makecell[l]{Cтрелка Пирса, \\ НЕ-ИЛИ, \\ антидизъюнкция} \\ \hline

    $F$ & 0 & 0 & 1 & 0 & \makecell[l]{$F_{2,2} = x\leftarrow /y $} & \makecell[l]{Отрицание \\ обратной импликации} \\ \hline
    $F$ & 0 & 0 & 1 & 1 & \makecell[l]{$F_{2,3} = \neg x = \overline{x} $} & \makecell[l]{Отрицание} \\ \hline
    $F$ & 0 & 1 & 0 & 0 & \makecell[l]{$F_{2,4} = $} & \makecell[l]{Противоречие, \\ логический нуль} \\ \hline
    $F$ & 0 & 1 & 0 & 1 & \makecell[l]{$F_{2,5} = $} & \makecell[l]{Противоречие, \\ логический нуль} \\ \hline
    $F$ & 0 & 1 & 1 & 0 & \makecell[l]{$F_{2,6} = $} & \makecell[l]{Противоречие, \\ логический нуль} \\ \hline
    $F$ & 0 & 1 & 1 & 1 & \makecell[l]{$F_{2,7} = $} & \makecell[l]{Противоречие, \\ логический нуль} \\ \hline
    $F$ & 1 & 0 & 0 & 0 & \makecell[l]{$F_{2,8} = $} & \makecell[l]{Противоречие, \\ логический нуль} \\ \hline
    $F$ & 1 & 0 & 0 & 1 & \makecell[l]{$F_{2,9} = $} & \makecell[l]{Противоречие, \\ логический нуль} \\ \hline
    $F$ & 1 & 0 & 1 & 0 & \makecell[l]{$F_{2,10} = $} & \makecell[l]{Противоречие, \\ логический нуль} \\ \hline
    $F$ & 1 & 0 & 1 & 1 & \makecell[l]{$F_{2,11} = $} & \makecell[l]{Противоречие, \\ логический нуль} \\ \hline
    $F$ & 1 & 1 & 0 & 0 & \makecell[l]{$F_{2,12} = $} & \makecell[l]{Противоречие, \\ логический нуль} \\ \hline
    $F$ & 1 & 1 & 0 & 1 & \makecell[l]{$F_{2,13} = $} & \makecell[l]{Противоречие, \\ логический нуль} \\ \hline
    $F$ & 1 & 1 & 1 & 0 & \makecell[l]{$F_{2,14} = $} & \makecell[l]{Противоречие, \\ логический нуль} \\ \hline
    $F$ & 1 & 1 & 1 & 1 & \makecell[l]{$F_{2,15} = $} & \makecell[l]{Противоречие, \\ логический нуль} \\ \hline

\end{nirtable}



\begin{table}[H]
\centering
\footnotesize
\begin{tabular}{|c|c|c|c|c|p{5cm}|p{5cm}|} 
\hline
$x$ & 1 & 1 & 0 & 0 & \multirow{2}{*}{Обозначение} & \multirow{2}{*}{Название} \\
$y$ & 1 & 0 & 1 & 0 & & \\
\hline
\multirow{2}{*}{F} & \multirow{2}{*}{0} & \multirow{2}{*}{0} & \multirow{2}{*}{0} & \multirow{2}{*}{0} & F2,0 = $FALSE$ & Противоречие, \\
& & & & & & логический нуль \\
\hline
\multirow{3}{*}{F} & \multirow{3}{*}{0} & \multirow{3}{*}{0} & \multirow{3}{*}{0} & \multirow{3}{*}{1} & F2,1 = $x\downarrow y = x \text{ NOR } y = $ & Cтрелка Пирса, НЕ-ИЛИ, \\
& & & & & $= \text{NOR}(x,y) = x $НЕ-ИЛИ & антидизъюнкция \\
& & & & & $y = $НЕ-ИЛИ$(x,y)$ & \\
\hline
\multirow{2}{*}{F} & \multirow{2}{*}{0} & \multirow{2}{*}{0} & \multirow{2}{*}{1} & \multirow{2}{*}{0} & F2,2 = $x\leftarrow /y$ & Отрицание \\
& & & & & & обратной импликации \\ 
\hline
F & 0 & 0 & 1 & 1 & F2,3 = $\neg x $ & Отрицание \\
\hline
\end{tabular}
\end{table}


\begin{table}[H]
\centering
\footnotesize
\begin{tabular}{|c|c|c|c|c|p{5cm}|p{5cm}|}
\hline
\multirow{2}{*}{F} & \multirow{2}{*}{0} & \multirow{2}{*}{1} & \multirow{2}{*}{0} & \multirow{2}{*}{0} & F2,4 = $x\rightarrow /y $ & Материальная \\
& & & & & & обратная импликация \\
\hline
F & 0 & 1 & 0 & 1 & F2,5 = $ \neg y$ & Отрицание \\
\hline
\multirow{4}{*}{F} & \multirow{4}{*}{0} & \multirow{4}{*}{1} & \multirow{4}{*}{1} & \multirow{4}{*}{0} & F2,6 = $x \oplus y = $ & Сложение по модулю 2, \\
& & & & & $= x \text{ XOR } y = \text{XOR}(x,y) = $ & исключающее "или" \\
& & & & & $= x >< y = x <> y = $ & сумма Жегалкина, \\
& & & & & $= x \text{ NE } y = \text{NE}(x,y)$ & не равно \\
\hline
\multirow{3}{*}{F} & \multirow{3}{*}{0} & \multirow{3}{*}{1} & \multirow{3}{*}{1} & \multirow{3}{*}{1} & F2,7 = $x | y = $ & Штрих Шеффера, \\
& & & & & $= \text{NAND}(x,y) = x \text{ NAND } y = $ & НЕ-И, 2И-НЕ, \\
& & & & & $= x $ НЕ-И $y = $ НЕ-И$(x,y)$ & антиконъюнкция \\
\hline
\multirow{4}{*}{F} & \multirow{4}{*}{1} & \multirow{4}{*}{0} & \multirow{4}{*}{0} & \multirow{4}{*}{0} & F2,8 = $x \wedge y = x \cdot y = $ & Конъюнкция, \\
& & & & & $= xy = x \& y = x \text{ AND } y = $ & 2И, минимум \\
& & & & & $= \text{AND}(x,y) = x $И$ y = $ & \\
& & & & & $= \text{И}(x,y) = \min(x,y)$ & \\
\hline
\multirow{2}{*}{F} & \multirow{2}{*}{1} & \multirow{2}{*}{0} & \multirow{2}{*}{0} & \multirow{2}{*}{1} & F2,9 = $(x \equiv y) = x \sim y = $ & Эквивалентность \\
& & & & & $= x \leftrightarrow y = x \text{ EQV } y = \text{EQV}(x,y)$ & равенство \\
\hline
F & 1 & 0 & 1 & 0 & F2,10 = $y $ & Проекция, повторение \\
\hline
\multirow{2}{*}{F} & \multirow{2}{*}{1} & \multirow{2}{*}{0} & \multirow{2}{*}{1} & \multirow{2}{*}{1} & F2,11 = $x \to y = x \supset y = $ & Импликация \\
& & & & & $= x \le y = x \text{ LE } y = \text{LE}(x,y)$ & следование \\
\hline
F & 1 & 1 & 0 & 0 & F2,12 = $x $ & Проекция, повторение \\
\hline
F & 1 & 1 & 0 & 1 & F2,13 = $x \leftarrow y $ & Обратная импликация \\
\hline
\multirow{4}{*}{F} & \multirow{4}{*}{1} & \multirow{4}{*}{1} & \multirow{4}{*}{1} & \multirow{4}{*}{0} & F2,14 = $x \vee y = x + y = $ & Дизъюнкция, \\
& & & & & $= x \text{ OR } y = \text{OR}(x,y) = $ & 2ИЛИ, максимум \\
& & & & & $= x$ ИЛИ $y = \text{ИЛИ}(x,y) = $ & \\
& & & & & $= \max(x,y)$ & \\
\hline
F & 1 & 1 & 1 & 1 & F2,15 = $TRUE $ & Тавтология \\
\hline
\end{tabular}
\caption{Таблица значений и названий булевых функций от двух переменных}
\end{table}

