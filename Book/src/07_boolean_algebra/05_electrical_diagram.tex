Для аппаратной реализации булевых функций и их корректного обозначения на электрической схеме принято использовать логические элементы.

\bigskip
\textbf{Логический элемент} --- это простейшее устройство ЭВМ, выполняющее одну определенную логическую операцию над входными сигналами согласно правилам алгебры логики. То есть, одну из функций, рассмотренных ранее.

\paragraph{Современные стандарты для условных графических обозначений (УГО) логических элементов}
\begin{itemize}
  \item ANSI (англ. \emph{American national standards institute} --- американский национальный институт стандартов).
  \item MIL/IEC (англ. \emph{MILitary} --- военный, {International Electrotechnical Commission} - международная электротехническая комиссия).
  \item DIN (нем. \emph{Deutsches Institut f{\"u}r Normung e. V.} --- немецкий институт по стандартизации).
  \item ГОСТ 2.743-91, Единая система конструкторской документации. Обозначения условные графические в схемах. Элементы цифровой техники. 
\end{itemize}

{
\newcommand{\gate}[2]{
    \begin{minipage}{.2\textwidth}
        \includegraphics[width=\linewidth]{logic_gates/#1/#2}
    \end{minipage} 
}

\begin{table}[H]
    \label{tab:logic_gates}
    \caption{Обозначение некоторых булевых функций на эл. схеме}
    \centering
    \begin{tabular}{|c|c|c|}
        \hline
        \thead{Название}                & \thead{IEC}       & \thead{ANSI}      \\ \hline
        НЕ, $\bar{A}$ (Инвертор)        & \gate{iec}{not}   & \gate{ansi}{not}  \\ \hline 
        И, $A \wedge B$                 & \gate{iec}{and}   & \gate{ansi}{and}  \\ \hline
        ИЛИ, $A \vee B$                 & \gate{iec}{or}    & \gate{ansi}{or}   \\ \hline
        Сумма по модулю 2, $A \oplus B$ & \gate{iec}{xor}   & \gate{ansi}{xor}  \\ \hline
    \end{tabular}
\end{table}

\noindent
Это основные обозначения. Они могут совмещаться. Например, так будет выглядеть И-НЕ:

\gate{iec}{nand}
}

\textbf{Цена функции по Квайну} – суммарное число входов логических элементов в составе схемы.

\textbf{Минимизация функции} – сокращение цены функции, с помощью преобразования её к более простому эквивалентному выражению.
Наиболее простой вид получается при сведении функции к постоянной - 1 (истина) или 0 (ложь).