% !TeX root = ../../main.tex
\subsection{Основные тождества}
\label{subsec:basic-identities}


% Как nirtable только с возможностью расширить на всю страницу (сделать environment как nirtable не вышло)
{%\small
    \renewcommand{\arraystretch}{1.5}
    \begin{xltabular}{\textwidth}{|c|X|}
        \caption{Основные тождества}\label{tab:basic-indentities}\\\hline

        \textbf{Название} & \textbf{Тождества} \\\hline
        \endfirsthead%

        \multicolumn{2}{r}{\small Продолжение таблицы~\thetable}\\\hline
        \textbf{Название} & \textbf{Тождества} \\\hline
        \endhead%

        \makecell*[r]{Аксиома\\двойного отрицания} &
        $\overline{\overline{x}} = x$ \\\hline

        \makecell*[r]{Аксиома\\существования 1 и 0} &
        \makecell*[X]{
            $0 = \overline{1}$; $x \vee \overline{x} = 1$ \\
            $1 = \overline{0}$; $x \wedge \overline{x} = 0$
        } \\\hline

        \makecell*[r]{Закон\\идемпотенции} &
        \makecell*[X]{
            $x \vee x = x$ \\
            $x \wedge x = x$
        } \\\hline

        \makecell*[r]{Закон\\коммутативности} &
        \makecell*[X]{
            $x \vee y = y \vee x$ \\
            $x \wedge y = y \wedge x$
        } \\\hline

        \makecell*[r]{Закон\\поглощения} &
        \makecell*[X]{
            $x \vee (x \wedge y) = x$ \\
            $x \wedge (x \vee y) = x$
        } \\\hline

        \makecell*[r]{Закон\\ассоциативности} &
        \makecell*[X]{
            $x \vee (y \vee z) = (x \vee y) \vee z$ \\
            $x \wedge (y \wedge z) = (x \wedge y) \wedge z$
        } \\\hline

        \makecell*[r]{Закон\\дистрибутивности} &
        \makecell*[X]{
            $x \vee (y \wedge z) = (x \vee y) \wedge (x \vee z)$ \\
            $x \wedge (y \vee z) = (x \wedge y) \vee (x \wedge z)$
        } \\\hline

        \makecell*[r]{Законы\\де Моргана} &
        \makecell*[X]{
            $\overline{x \vee y} = \overline{x} \wedge \overline{y}$ \\
            $\overline{x \wedge y} = \overline{x} \vee \overline{y}$
        } \\\hline

        \makecell*[r]{Закон\\нейтральности} &
        \makecell*[X]{
            $x \vee (y \wedge \overline{y}) = x$ \\
            $x \wedge (y \vee \overline{y}) = x$
        } \\\hline
    \end{xltabular}
}

\todo{Выбрать}

\begin{enumerate}
    \item Аксиома двойного отрицания
          \begin{center}
              % \vspace*{-\baselineskip}
              \begin{tabular}{c}
                  $\overline{\overline{x}} = x$ \\
              \end{tabular}
          \end{center}

    \item Аксиома существования 1 и 0
          \begin{center}
              \begin{tabular}{cc}
                  $0 = \overline{1}$ & $x \vee \overline{x} = 1$ \\
                  $1 = \overline{0}$ & $x \wedge \overline{x} = 0$ \\
              \end{tabular}
          \end{center}

    \item Закон идемпотенции
          \begin{center}
              \begin{tabular}{cc}
                  $x \vee x = x$ & $x \wedge x = x$ \\
              \end{tabular}
          \end{center}

    \item Закон коммутативности
          \begin{center}
              \begin{tabular}{cc}
                  $x \vee y = y \vee x$ & $x \wedge y = y \wedge x$ \\
              \end{tabular}
          \end{center}

    \item Закон поглощения
          \begin{center}
              \begin{tabular}{cc}
                  $x \vee (x \wedge y) = x$ & $x \wedge (x \vee y) = x$ \\
              \end{tabular}
          \end{center}

    \item Закон ассоциативности
          \begin{center}
              % \vspace*{-\baselineskip}
              \begin{tabular}{cc}
                  $x \vee (y \vee z) = (x \vee y) \vee z$ & $x \wedge (y \wedge z) = (x \wedge y) \wedge z$ \\
              \end{tabular}
          \end{center}

    \item Закон дистрибутивности
          \begin{center}
              \begin{tabular}{cc}
                  $x \vee (y \wedge z) = (x \vee y) \wedge (x \vee z)$ & $x \wedge (y \vee z) = (x \wedge y) \vee (x \wedge z)$ \\
              \end{tabular}
          \end{center}

    \item Законы де Моргана
          \begin{center}
              \begin{tabular}{cc}
                  $\overline{x \vee y} = \overline{x} \wedge \overline{y}$ & $\overline{x \wedge y} = \overline{x} \vee \overline{y}$ \\
              \end{tabular}
          \end{center}

    \item Закон нейтральности
          \begin{center}
              \begin{tabular}{cc}
                  $x \vee (y \wedge \overline{y}) = x$ & $x \wedge (y \vee \overline{y}) = x$ \\
              \end{tabular}
          \end{center}
\end{enumerate}
