\newpage
\subsubsection{Мера Хартли}

\begin{wrapfigure}{l}{0.21\textwidth}
    \centering
    \includegraphics[width=0.2\textwidth]{hartley}
    \caption*{Ральф Хартли\\1888 -- 1970}
\end{wrapfigure}

Пусть известны $N$ состояний системы $S$ ($N$  опытов с различными, равновозможными, последовательными состояниями системы). Если каждое состояние системы закодировать двоичными кодами, то минимальная длина $d$ полученного кода определяется из условия:

$$
2^{d} \ge N \qquad \mbox{\emph{или}} \qquad  d \ge \log_{2}N
$$

Значит, для однозначного описания системы требуется $\log_{2}N$ бит. В общем случае количество информации в системе $S$ равно:
$$
H_{s} = \log_{k}N
$$

Единицы измерения количества информации:
\begin{itemize}[noitemsep]
  \item Бит ($k = 2$)
  \item Трит ($k = 3$)
  \item Дит (харт) ($k = 10$)
  \item Нит (нат) ($k = e$)
\end{itemize}

\paragraph{Примеры использования меры Хартли}

\example{1}

\task мальчик загадывает число от 1 до 64. Какое количество вопросов типа "да-нет" понадобится, чтобы гарантированно угадать число?

\solution

\begin{itemize}[noitemsep]
    \item Первый вопрос: "Загаданное число меньше 32?". Ответ: "Да".
    \item Второй вопрос: "Загаданное число меньше 16?". Ответ: "Нет".
    \item [] \dots
    \item Шестой вопрос точно приведет к правильному ответу.
\end{itemize}

\noindentЗначит, в соответствии с мерой Хартли в загадке мальчика содержится $\log_{2}64 = 6$ бит информации ($N = 64$ так как возможно 64 вариантов загаданного числа).

\answer 6 бит.


\example{2}

\task Мальчик держит за спиной шахматного ферзя и собирается поставить его на произвольную клетку пустой доски. Какое количество информации содержится в его действии?

\solution Шахматная доска имеет размеры $8\times 8$ клеток.
Ферзь может быть как белым, так и черным, поэтому количество равновероятных состояний будет равно $8 \times 8 \times 2 = 128$.
Получается, количество информации по мере Хартли равно $\log_{2}128 = 7$ бит.

\answer 7 бит.

\bigskip

Если во множестве $X = {x_1,x_2, ..., x_n}$ искать произвольный элемент, то для его нахождения (по Хартли) необходимо иметь не менее $\log_{a}n$ (единиц) информации. 

Уменьшение $H$ говорит об уменьшении разнообразия состояний $N$ системы, а увеличение $H$ говорит об увеличении разнообразия состояний $N$ системы.

Мера Хартли подходит лишь для идеальных, абстрактных систем, так как в реальных системах состояния системы неодинаково осуществимы (неравновероятны).