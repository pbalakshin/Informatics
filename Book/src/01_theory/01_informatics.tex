\subsection{Терминология информатики}

Начать изучение информатики невозможно, не разобравшись в точном значении термина «информатика». Однако, до сих пор в мировой научной общественности не сложилось четкого понимания этого термина. Рассмотрим одно из популярных определений:

\textbf{Информатика} --- дисциплина, изучающая свойства и структуру информации, закономерности ее создания, преобразования, накопления, передачи и использования. За рубежом сложилась чуть более узкая трактовка термина информатика. Там под этим понимают пересечение сразу трех областей науки – это информационные технологии, теория информации и computer science. Всё обозначенное выше подходит под определение самого курса "Информатика".

Изучая некоторую науку важно представлять основные даты, вехи её развития:
\begin{itemize}
\item 1956(57) – появление термина <<информатика>> (\textit{нем.} Informatik, Штейнбух).
\item 1968 – первое упоминание в СССР (информология, Харкевич).
\item 197Х – информатика стала отдельной наукой.
\item 4 декабря – день российской информатики.
\end{itemize}