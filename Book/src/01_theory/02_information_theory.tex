\subsection{Терминология теории информации}

Рассмотрим некоторые терминологические тонкости. В обыденном языке, слова <<информация>> и <<данные>> считаются синонимами. Они, как правило, употребляются взаимозаменяемо. И так обстоит дело в информатике и в целом, в компьютерных науках.

Понятие \textit{``информация''} имеет различные трактовки в различных предметных областях. Например, \textit{информация} может пониматься как:
\begin{itemize}[noitemsep]
    \item сигналы для управления, приспособления рассматриваемой системы (в кибернетике);
    \item мера хаоса в рассматриваемой системе (в физике);
    \item вероятность выбора в рассматриваемой системе (в теории вероятностей);
    \item мера разнообразия в рассматриваемой системе (в биологии) и др.
\end{itemize}
Но мы остановимся на понятиях, близких к информатике.

\begin{description}
    \item [Информация] --- это некоторая упорядоченная последовательность сообщений, отражающих, передающих и увеличивающих наши знания.
    \item [Информация] --- это сведения об окружающем мире (объекте, процессе, явлении, событии), которые являются объектом преобразования (включая хранение, передачу и т.д.) и используются для выработки поведения, для принятия решения, для управления или для обучения.
    \item [Информация] --- это новые сведения, подлежащие передаче, хранению и обработке.
\end{description}

Рассмотрим это фундаментальное понятие информатики на основе понятия \textit{``алфавит''} (``алфавитный'', формальный подход). Дадим формальное определение \textit{алфавита}.

\begin{description}
    \item [Алфавит] --- конечное множество различных знаков (букв), символов, для которых определена операция \emph{конкатенации} (присоединения символа к символу или цепочке символов); с ее помощью по определенным правилам соединения символов и слов можно получать слова (цепочки знаков) и словосочетания (цепочки \textit{слов}) в этом \textit{алфавите} (над этим \textit{алфавитом}).
    \item [Знак (буква)] --- любой элемент алфавита (элемент $x$ алфавита $X$, где $x \in X$). Понятие знака неразрывно связано с тем, что им обозначается (``со смыслом''), они вместе могут рассматриваться как пара элементов ($x$, $y$), где $x$ – сам знак, а $y$ – обозначаемое этим знаком.
\end{description}

\example{1}
\noindent
Примеры \emph{алфавитов:} множество из десяти цифр, множество из знаков русского языка, точка и тире в азбуке Морзе и др. В \emph{алфавите} цифр знак 5 связан с понятием ``быть в количестве пяти элементов''.

\textbf{Слово} в алфавите (или над алфавитом) - конечная последовательность знаков (букв) алфавита.

\textbf{Длина} |p| некоторого слова $p$ в алфавите (над алфавитом) - число составляющих его букв.

\textbf{Словарь (словарный запас)} - множество различных слов в алфавите (над алфавитом).
В отличие от конечного \emph{алфавита}, словарный запас может быть и бесконечным.
\emph{Слова} над некоторым заданным \emph{алфавитом} и определяют так называемые \emph{сообщения}.

\example{2}
\noindent
\emph{Слова} над \emph{алфавитом} кириллицы --- ``Информатика'',``инто'', ``ииии'', ``и''.

\noindent
\emph{Слова} над \emph{алфавитом} десятичных цифр и знаков арифметических операций --- ``1256'', ``23+78'', ``35–6+89'', ``4''.

\noindent
\emph{Слова} над \emph{алфавитом} азбуки Морзе --- ``.'', ``. . –'', ``– – –''.

В \emph{алфавите} должен быть определен порядок следования \emph{букв} (порядок типа ``предыдущий элемент --- последующий элемент''), то есть любой \emph{алфавит} имеет упорядоченный вид $X = {x_1, x_2, \ldots, x_n}$ .

Таким образом, \emph{алфавит} должен позволять решать задачу лексикографического (алфавитного) упорядочивания, или задачу расположения \emph{слов} над этим \emph{алфавитом}, в соответствии с порядком, определенным в \emph{алфавите} (то есть по символам \emph{алфавита}).
