\subsection{Признаки классификации информации}

Рассмотрим две классификации информации. Первая из них --- классификация по форме \emph{сообщений} --- определенного вида сигналов, символов:
\begin{itemize}[noitemsep]
  \item отношение к источнику или приемнику (входная, выходная и внутренняя);
  \item отношение к конечному результату (исходная, промежуточная и результирующая);
  \item актуальность;
  \item адекватность;
  \item доступность (открытая, закрытая);
  \item понятность;
  \item полнота (достаточная, недостаточная, избыточная);
  \item достоверность;
  \item массовость;
  \item изменчивость (постоянная, переменная, смешанная);
  \item объективность;
  \item точность;
  \item стадия использования (первичная, вторичная);
  \item ценность.
\end{itemize}

Вторая классификация --- по форме преставления информации, способам ее кодирования и хранения:
\begin{itemize}[noitemsep]
  \item графическая;
  \item звуковая;
  \item текстовая;
  \item числовая;
  \item видеоинформация.
\end{itemize}
