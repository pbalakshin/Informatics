\subsection{Методы получения информации}

Методы получения информации можно разбить на три большие группы:
\begin{itemize}
    \item \emph{Эмпирические};
    \item \emph{Теоретические}; 
    \item \emph{Эмпирико-теоретические}.
\end{itemize}

Кратко рассмотрим и охарактеризуем все три метода по отдельности. 

\subsubsection{Эмпирические методы}
Эмпирические методы или методы получения эмпирических данных.
\begin{description}
    \item [Наблюдение] --- сбор первичной информации об объекте, процессе, явлении.
    \item [Сравнение] --- обнаружение и соотнесение общего и различного.
    \item [Измерение] --- поиск с помощью измерительных приборов эмпирических фактов.
    \item [Эксперимент] --- преобразование, рассмотрение объекта, процесса, явления с целью выявления каких-то новых свойств.
\end{description}

\noindent
Кроме классических форм их реализации, в последнее время используются опрос, интервью, тестирование и другие.

\subsubsection{Теоретические методы}
Теоретические методы или методы построения различных теорий.
\begin{description}
    \item [Восхождение от абстрактного к конкретному] --- получение знаний о целом или о его частях на основе знаний об абстрактных проявлениях в сознании, в мышлении.
    \item [Идеализация] --- получение знаний о целом или его частях путем представления в мышлении целого или частей, не существующих в действительности.
    \item [Формализация] --- получение знаний о целом или его частях с помощью языков искусственного происхождения (формальное описание, представление).
    \item [Аксиоматизация] --- получение знаний о целом или его частях с помощью некоторых аксиом (не доказываемых в данной теории утверждений) и правил получения из них (и из ранее полученных утверждений) новых верных утверждений.
    \item [Виртуализация] --- получение знаний о целом или его частях с помощью искусственной среды, ситуации.
\end{description}

\subsubsection{Эмпирико-теоретические методы}
tЭмпирико-теоретические методы (смешанные) или методы построения теорий на основе полученных эмпирических данных об объекте, процессе, явлении.

\begin{itemize}
    \item \textbf{Абстрагирование} --- выделение наиболее важных для исследования свойств, сторон исследуемого объекта, процесса, явления и игнорирование несущественных и второстепенных.
    \item \textbf{Анализ} --- разъединение целого на части с целью выявления их связей.
    \item \textbf{Декомпозиция} --- разъединение целого на части с сохранением их связей с окружением.
    \item \textbf{Синтез} --- соединение частей в целое с целью выявления их взаимосвязей.
    \item \textbf{Композиция} --- соединение частей целого с сохранением их взаимосвязей с окружением.
    \item \textbf{Индукция} --- получение знания о целом по знаниям о частях.
    \item \textbf{Дедукция} --- получение знания о частях по знаниям о целом.
    \item \textbf{Эвристики, использование эвристических процедур} --- получение знания о целом по знаниям о частях и по наблюдениям, опыту, интуиции, предвидению.
    \item \textbf{Моделирование (простое моделирование)}, использование приборов -- получение знания о целом или о его частях с помощью модели или приборов.
    \item \textbf{Исторический метод} --- поиск знаний с использованием предыстории, реально существовавшей или же мыслимой.
    \item \textbf{Логический метод} --- поиск знаний путем воспроизведения частей, связей или элементов в мышлении.
    \item \textbf{Макетирование} --- получение информации по макету, представлению частей в упрощенном, но целостном виде.
    \item \textbf{Актуализация} --- получение информации с помощью перевода целого или его частей (а следовательно, и целого) из статического состояния в динамическое состояние.
    \item \textbf{Визуализация} --- получение информации с помощью наглядного или визуального представления состояний объекта, процесса, явления.
\end{itemize}

\noindent
Кроме указанных классических форм реализации теоретико-эмпирических методов часто используются и мониторинг (система наблюдений и анализа состояний), деловые игры и ситуации, экспертные оценки (экспертное оценивание), имитация (подражание) и другие формы.

\example{1}

\noindent
Для построения модели планирования и управления производством в рамках страны, региона или крупной отрасли нужно решить следующие проблемы:
\begin{enumerate}
    \item Определить структурные связи, уровни управления и принятия решений, ресурсы; при этом чаще используются методы наблюдения, сравнения, измерения, эксперимента, анализа и синтеза, дедукции и индукции, эвристический, исторический и логический методы, макетирование и др.;
    \item Определить гипотезы, цели, возможные проблемы планирования; наиболее используемые методы --- наблюдение, сравнение, эксперимент, абстрагирование, анализ, синтез, дедукция, индукция, эвристический, исторический, логический и др.;
    \item Конструирование эмпирических моделей; наиболее используемые методы --- абстрагирование, анализ, синтез, индукция, дедукция, формализация, идеализация и др.;
    \item Поиск решения проблемы планирования и просчет различных вариантов, директив планирования, поиск оптимального решения; используемые чаще методы – измерение, сравнение, эксперимент, анализ, синтез, индукция, дедукция, актуализация, макетирование, визуализация, виртуализация и др.
\end{enumerate}