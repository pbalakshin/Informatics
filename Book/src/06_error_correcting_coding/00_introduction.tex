% !TeX root = ../../main.tex
При обработке данных (хранение данных в памяти, передача по каналам связи) существует вероятность битовых ошибок. Они могут возникать из-за альфа частиц от примесей в чипе микросхемы или от нейтронов из фонового космического излучения. Частота единичных битовых ошибок на 1 гигабайт данных составляет от 1 раза в час до 1 раза в тысячелетие. По данным исследования Google получилось, что 1 раз в сутки.

Есть несколько способов обработки данных, полученных при передаче с ошибкой:

\begin{itemize}
    \item \emph{Использовать полученные данные без проверки на ошибки.} Для такого способа не нужно выполнять лишних действий. Например, при передаче текста или видеоизображения ошибка в одном блоке (слове, кадре) не исказит общего смысла.
    \item \emph{Обнаружить ошибку, выполнить запрос повторной передачи по\-вре\-ждё\-нно\-го блока.} В случае, если пользователь обнаружил ошибку, потребуется передать не блок, а целое сообщение заново, и при передаче файлов большого размера (видеоизображений, фотоизображений и музыкальных файлов высокого качества) это очень проблематично. Если ошибку обнаружила операционная система, то она обнаружила ее еще при передаче данных, и заново передается только поврежденный блок.
    \item \emph{Обнаружить ошибку и отбросить поврежденный блок.} Как и первый способ, он весьма оправдан при передаче текста или видеоизображений.
    \item \emph{Обнаружить и исправить ошибку.} Не тратится время на повторную передачу данных, но полученные данные корректны. Однако, при таком способе необходимо применять особые методы кодирования и наряду с информационными битами передавать служебные, которые позволяют исправить ошибку.
\end{itemize}

\noindentПоследний способ характеризует помехоустойчивый код.
\textbf {Помехоустойчивые коды} --- это коды, позволяющие обнаружить и (или) исправить ошибки в кодовых словах, которые возникают при передачи по каналам связи.

\bigskip
\noindent\emph{Помехоустойчивое кодирование} предполагает введение в передаваемое сообщение, наряду с информационными, так называемых проверочных разрядов, формируемых в устройствах защиты от ошибок (кодерах-на передающем конце, декодерах — на приемном). Избыточность позволяет отличить разрешенную и запрещенную (искаженную за счет ошибок) комбинации при приеме, иначе одна разрешенная комбинация переходила бы в другую.

\paragraph{Классификация помехоустойчивых кодов}

% FIXME: Выглядит сложно. 4 уровня вложенности
\begin{itemize}
    \item \textbf{Блочные} --- фиксированные блоки информации длиной $k$ символов преобразуются в блоки длиной $n$ символов (независимо друг от друга). Например, при передаче файла объемом в 1 гигабайт он делится на равные блоки по 1 килобайту и каждый килобайт снабжается служебной информацией, которая позволяет понять - корректен данный блок или нет. И при обнаружении некорректного блока передается только он.
          \begin{itemize}
              \item \textbf{Неравномерные} --- редко используемые символы кодируются большим количеством символов (имеют большую длину) (азбука Морзе)
              \item \textbf{Равномерные} --- длина блока (символа) постоянна (таблица ASCII).
                    \begin{itemize}
                        \item \textbf{Неразделимые} --- коды с постоянной плотностью единиц: информационные и проверочные разряды неразделимы (каждый блок данных на входе получает служебные биты, позволяющие обнаружить ошибки, которые далее не отделяются).
                        \item \textbf{Разделимые} --- можно отделить (выделить) служебные биты от информационных.
                              \begin{itemize}
                                  \item \textbf{Систематические (линейные)} --- циклические коды, биты четности/нечетности, код Хэмминга, код Рида-Соломона, код Боуза-Чоудхури-Хоквингема.
                                  \item \textbf{Несистематические} --- коды с контрольным суммированием.
                              \end{itemize}
                    \end{itemize}
          \end{itemize}
    \item \textbf{Непрерывные} --- передаваемая информационная последовательность не разделяется на блоки. Кодирование осуществляется целого потока.
          \begin{itemize}
              \item \textbf{Сверточные} --- корректирующие ошибки коды; работают с непрерывным потоком данных, кодируя их при помощи регистров сдвига с линейной обратной связью.
          \end{itemize}
\end{itemize}

\bigskip\noindent
\emph{Помехоустойчивый код характеризуется:}
\begin{description}[noitemsep]
    \item [$i$] --- числом информационных разрядов;
    \item [$r$] --- числом проверочных разрядов;
    \item [$n$] --- общим числом разрядов ($n = i + r$);
\end{description}

\noindent\emph{\textbf{Коэффициент избыточности:}} $\text{КИ}=\displaystyle\frac{r}{n}$;
