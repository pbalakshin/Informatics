\subsection{Кодирование с помощью бита четности}
\label{subsec:bit-parity}

\textbf{Контрольная сумма (check sum)} --- некоторое число, рассчитанное путем применения определенного алгоритма к набору данных и используемое для проверки целостности этого набора данных при их передаче или хранении.

\bigskip\textbf{Бит четности} --- частный случай контрольной суммы, представляющий из себя 1 контрольный бит, используемый для проверки четности количества единичных битов в двоичном числе.

Контроль некой двоичной последовательности (например, машинного слова) с помощью бита чётности также называют \textbf{контролем по паритету}. Контроль по паритету -- наиболее простой и наименее мощный метод контроля данных.

\bigskip\textbf{Метод расчета бита четности} --- суммирование по модулю 2 всех бит числа.

\bigskip\textbf{Сумма по модулю 2} --- исключающее ``ИЛИ'' (для двух операндов), логическое или битовое сложение, разность двух/трех множеств.

\medskip\noindent
\emph{Обозначение:} $A\ mod2\ B = A \oplus B$

\bigskip

\begin{table}[H]
    \centering
    \begin{tabular}{ccc}
        \includegraphics[width=.25\linewidth]{vienna/and} & 
        \includegraphics[width=.25\linewidth]{vienna/or} & 
        \includegraphics[width=.25\linewidth]{vienna/xor} \\
         
         $A \wedge B$ & $A \vee B$ & $A \oplus B$ \\
         \multicolumn{3}{c}{$A \oplus B = (\overline{A \wedge B}) \wedge (A \vee B) = \overline{(A \wedge B) \vee (\overline{A} \wedge \overline{B})}$} \\
    \end{tabular}
\end{table}

\paragraph*{Таблица истинности для $A \oplus B$}

\noindent
\begin{minipage}[t]{.45\linewidth}
    \vspace*{0mm}
    \centering
    \begin{tabular}{|c|c|c|}
    \hline
    A & B & A $\oplus$ B \\ \hline
    0 & 0 & 0 \\
    0 & 1 & 1 \\
    1 & 0 & 1 \\
    1 & 1 & 0 \\ \hline
    \end{tabular}
\end{minipage}
\hfill
\begin{minipage}[t]{.45\linewidth}
    \vspace*{0mm}
    \centering
    \begin{tabular}{|c|c|c|c|}
    \hline
    A & B & C & A $\oplus$ B $\oplus$ C\\ \hline
    0 & 0 & 0 & 0 \\
    0 & 0 & 1 & 1 \\
    0 & 1 & 0 & 1 \\
    0 & 1 & 1 & 0 \\
    1 & 0 & 0 & 1 \\
    1 & 0 & 1 & 0 \\
    1 & 1 & 0 & 0 \\
    1 & 1 & 1 & 1 \\     \hline
    \end{tabular}
\end{minipage}

\bigskip\noindent
В случае двух переменных результат выполнения операции является истинным тогда и только тогда, когда лишь один из аргументов является истинным.

\noindent Для функции трех и более переменных результат выполнения операции будет истинным только тогда, когда количество аргументов, равных 1, --- нечетное.

\paragraph{Пример обнаружения ошибок}

Допустим, у нас есть один информационный бит $i = 1$. К нему идет один $r_1$ - бит четности, проверочный разряд №1.

\noindent $i = r_1$, $i \oplus r_1 = 0$.

\noindent При получении данных произошел сбой и данные некорректны. Нам известно, что сумма по модулю 2 информационного бита и бита четности равна 0.

\begin{table}[H]
    \centering 
    \begin{tabular}{|c|c|c|c|c|}
    \hline
    i исх & $r_{1}$ исх & i рез & $r_{1}$ рез & i рез $\oplus$ $r_{1}$ рез \\ \hline
    1 & 1 & 0 & 0 & 0 \\
    1 & 1 & 0 & 1 & 1 \\
    1 & 1 & 1 & 0 & 1 \\
    1 & 1 & 1 & 1 & 0 \\ \hline
    \end{tabular}
\end{table}

\noindent Соответственно, биты с пометкой ``исх'' --- исходные, а ``рез'' --- результирующие.

\noindent В первом случае видно, что несмотря на то, что оба бита пришли с ошибкой, сумма по модулю 2 равна нулю (верна), а значит компьютер не посчитает это как за ошибку.

\noindent Во втором и третьем случаях сумма по модулю 2 результирующих данных не верна, равна 1. Ошибка.

\noindent В последнем случае все верно. Биты корректны, сумма по модулю 2 верна. Ошибки нет.

\bigskip\noindent Способ обнаружения ошибок с помощью бита четности применяется в RAID-хранилищах \footnote{RAID --- (англ. redundant array of independent disks -- избыточный массив независимых дисков) технология виртуализации данных, которая объединяет несколько дисков в логический элемент для избыточности и повышения производительности}.