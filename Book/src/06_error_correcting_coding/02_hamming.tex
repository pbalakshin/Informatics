% !TeX root = ../../main.tex
\newpage
\subsection{Код Хэмминга}
\label{subsec:hamming}


\begin{wrapfigure}{l}{0.25\textwidth}
    \centering
    \vspace{-\intextsep}
    \includegraphics[width=0.25\textwidth]{person/hamming_richard}
    \caption*{Р.У. Хэмминг \\ 1915 -- 1998}
\end{wrapfigure}

В 1945 году Ричард Уэсли Хэмминг занимался программрованием одного из первых электронных цифровых компьютеров для расчета решения физических уравнений.  В период с 1946 по 1976 года Хэмминг работал в Bell Labs, где сотрудничал с Клодом Шенноном. Хэмминг часто работал в выходные дни, и все больше и больше раздражался, потому что часто был должен перезагружать свою программу из-за ненадежности перфокарт. На протяжении нескольких лет он проводил много времени над построением эффективных алгоритмов исправления ошибок. В 1950 году он опубликовал способ, который на сегодняшний день известен как \textbf{\emph{код Хэмминга}}. Коды Хэмминга — наиболее известные и, вероятно, первые из \emph{самоконтролирующихся} и \emph{самокорректирующихся} кодов.

\bigskip\noindent
Рассмотрим подробнее два этих вида кодов:
\begin{itemize}
    \item \textbf{Самоконтролирующиеся коды} --- коды, позволяющие автоматически обнаруживать ошибки при передаче данных. Для их построения достаточно приписать к каждому слову один добавочный (контрольный) двоичный разряд и выбрать цифру этого разряда так, чтобы общее количество единиц в изображении любого числа было, например, четным. Одиночная ошибка в каком-либо разряде передаваемого слова (в том числе, может быть, и в контрольном разряде) изменит четность общего количества единиц. Счетчики по модулю 2, подсчитывающие количество единиц, которые содержатся среди двоичных цифр числа, могут давать сигнал о наличии ошибок.

    При этом невозможно узнать, в каком именно разряде произошла ошибка, и, следовательно, нет возможности исправить её. Остаются незамеченными также ошибки, возникающие одновременно в двух, в четырёх или вообще в четном количестве разрядов. Впрочем, двойные, а тем более четырёхкратные ошибки полагаются маловероятными.

    \item \textbf{Самокорректирующиеся коды} --- коды, в которых возможно автоматическое исправление ошибок. Для построения самокорректирующегося кода, рассчитанного на исправление одиночных ошибок, одного контрольного разряда недостаточно. Как видно из дальнейшего, количество контрольных разрядов k должно быть выбрано так, чтобы удовлетворялось неравенство $2^k \ge k+m+1$ или $k \ge log_2{k+m+1}$, где $m$ --- количество основных двоичных разрядов кодового слова. Минимальные значения k при заданных значениях m, найденные в соответствии с этим неравенством, приведены в таблице.

    \begin{table}[H]
        \centering
        \begin{tabularx}{0.8\textwidth}{|r|C|C|C|C|C|}
            \hline
            \thead{Диапазон m} & 1 & 2 -- 4 & 5 -- 11 & 12 -- 26 & 27 -- 57 \\ \hline
            \thead{$k_{min}$}  & 2 & 3 & 4 & 5 & 6 \\ \hline
        \end{tabularx}
    \end{table}

    % \begin{table}[H]
    %     \centering
    %     \begin{tabular}{|c|c|}
    %         \hline
    %         Диапазон m & $k_{min}$ \\ \hline
    %         1       & 2 \\ \hline
    %         2 -- 4    & 3\\ \hline
    %         5 -- 11   & 4\\ \hline
    %         12 -- 26  & 5\\ \hline
    %         27 -- 57  & 6 \\ \hline
    %     \end{tabular}
    % \end{table}
\end{itemize}

В настоящее время наибольший интерес представляют двоичные блочные корректирующие коды. При использовании таких кодов информация передаётся в виде блоков одинаковой длины и каждый блок кодируется и декодируется независимо друг от друга. Почти во всех блочных кодах символы можно разделить на информационные и проверочные. Таким образом, все комбинации кодов разделяются на разрешенные (для которых соотношение информационных и проверочных символов возможно) и запрещенные.

\bigskip
\textbf{Код Хэмминга} --- блочный равномерный разделимый самокорректирующийся код. Исправляет одиночные битовые ошибки, возникшие при передаче или хранении данных.

\noindent
На каждые $i$ информационных бит используется $r$ проверочных.

\noindent
Значение каждого контрольного бита зависит от значений информационных бит: контрольный бит с номером $N$ контролирует все последующие $N$ бит через каждые $N$ бит, начиная с позиции $N$.

\bigskip
\textbf{Синдром последовательности S} --- набор контрольных сумм информационных и проверочных разрядов

Рассмотрим таблицу:
\begin{table}[H]
    % TODO: Надо ли?
    % \caption{Caption}
    % \label{tab:my_label}
    \centering
    \begin{tabularx}{\linewidth}{|C|C|C|C|C|C|C|C|C|C|C|C|C|C|C|C|}
        \hline
        & 1 & 2 & 3 & 4 & 5 & 6 & 7 & 8 & 9 & 10 & 11 & 12 & 13 & 14 & \\ \hline
        $2^k$ & $r_{1}$ & $r_{2}$ & $i_{1}$ & $r_{3}$ & $i_{2}$ & $i_{3}$ & $i_{4}$ & $r_{4}$ & $i_{5}$ & $i_{6}$ & $i_{7}$ & $i_{8}$ & $i_{9}$ & $i_{10}$ & S \\ \hline
        1 & \cellcolor{iogreen}{X} & & \cellcolor{iogreen}{X} & & \cellcolor{iogreen}{X} & & \cellcolor{iogreen}{X} & & \cellcolor{iogreen}{X} & & {\cellcolor{iogreen}X} & & \cellcolor{iogreen}{X} & & $s_{1}$\\\hline
        2 & & \cellcolor{ioyellow}{X} & \cellcolor{ioyellow}{X} & & & \cellcolor{ioyellow}{X} & \cellcolor{ioyellow}{X} & & & \cellcolor{ioyellow}{X} & \cellcolor{ioyellow}{X} & & & \cellcolor{ioyellow}{X} & $s_{2}$ \\ \hline
        4 & & & & \cellcolor{ioorange}{X} & \cellcolor{ioorange}{X} & \cellcolor{ioorange}{X} & \cellcolor{ioorange}{X} & & & & & \cellcolor{ioorange}{X} & \cellcolor{ioorange}{X} & \cellcolor{ioorange}{X} & $s_{3}$ \\  \hline
        8 & & & & & & & & \cellcolor{iored}{X} & \cellcolor{iored}{X} & \cellcolor{iored}{X} & \cellcolor{iored}{X} & \cellcolor{iored}{X} & \cellcolor{iored}{X} & \cellcolor{iored}{X} & $s_{4}$ \\ \hline
    \end{tabularx}

\end{table}

Данная таблица представлена для кода из 14 бит, но ее можно расширить для нужного размера самостоятельно. \\
Сверху (в первой строчке) указан номер бита, а справа - синдром. \\
Знаком $X$ обозначены те биты, которые контролирует проверочный бит, с номером, который указан в левом столбце (степень двойки). Чтобы узнать какими битами контролируется бит с номером $N$ надо просто разложить $N$ по степеням двойки.

Например, видно, что проверочный бит $r_1$ контролирует информационные биты $i_1$, $i_2$, $i_4$, $i_5$, $i_7$, $i_9$.
А 11 бит ($i_7$) контролируется битами 1 ($r_1$), 2 ($r_2$) и 8 ($r_4$).

Чтобы узнать значение проверочного бита необходимо сложить по модулю 2 все информационные биты, которые он контролирует. \\
В данном случае:

\noindent
\begin{equation*}
    \begin{aligned}
    r_1 &= i_1\oplus i_2\oplus i_4\oplus i_5\oplus i_7\oplus i_9; \\
    r_2 &= i_1\oplus i_3\oplus i_4\oplus i_6\oplus i_7\oplus i_{10};
    \end{aligned}
\end{equation*}
и так далее.

Чтобы узнать значение синдрома, необходимо сложить по модулю 2 все биты, которые содержит синдром.\\
В данном случае:

\noindent
\begin{equation*}
    \begin{aligned}
    s_1 &= r_1\oplus i_1\oplus i_2\oplus i_4\oplus i_5\oplus i_7\oplus i_9; \\
    s_2 &= r_2\oplus i_1\oplus i_3\oplus i_4\oplus i_6\oplus i_7\oplus i_{10};
    \end{aligned}
\end{equation*}
и так далее.

Синдром последовательности S определяется составляющими синдромами $s_1$, $s_2$ и так далее.
То есть для кода, где $r = 3$, синдром будет иметь следующий формат: $S(s_1;s_2;s_3)$.

Определение минимального количества контрольных разрядов: $$2^k \ge r + i + 1$$

\noindent
Классические коды Хэмминга с маркировкой (n, i): (7, 4); (15, 11); (31, 26).

\paragraph{Код Хэмминга для r = 3}

Разбрем код Хэмминга для $r = 3$.
В данном случае, у нас 7 бит, из них 4 информационных и 3 проверочных.
То есть, код имеет маркировку (7, 4).

\noindent
Составим таблицу кода (7, 4):



\begin{table}[H]
    \centering
    \begin{tabular}{|c|c|c|c|c|c|c|c|c|}
    \hline
    & 1 & 2 & 3 & 4 & 5 & 6 & 7 & \\ \hline
    $2^k$ & $r_{1}$ & $r_{2}$ & $i_{1}$ & $r_{3}$ & $i_{2}$ & $i_{3}$ & $i_{4}$ & S \\ \hline
    1 & \cellcolor{iogreen}{X} & & \cellcolor{iogreen}{X} & & \cellcolor{iogreen}{X} & & \cellcolor{iogreen}{X} & $s_{1}$ \\ \hline
    2 & & \cellcolor{ioyellow}{X} & \cellcolor{ioyellow}{X} & & & \cellcolor{ioyellow}{X} & \cellcolor{ioyellow}{X} & $s_{2}$ \\ \hline
    4 & & & & \cellcolor{ioorange}{X} & \cellcolor{ioorange}{X} & \cellcolor{ioorange}{X} & \cellcolor{ioorange}{X} & $s_{3}$ \\ \hline
    \end{tabular}
\end{table}

\bigskip\noindent
Рассмотрим конкретно контроль информационных бит проверочными на кругах Эйлера:

\begin{figure}[H]

\begin{wrapfigure}{l}{.3\textwidth}
    \centering
    \vspace{-\intextsep}
    \includegraphics[width=.25\textwidth]{6.2-euler_r3}
\end{wrapfigure}

\noindent
Видно, что $r_1$ контролирует $i_1$, $i_2$ и $i_4$; $r_2$ контролирует $i_1$, $i_3$ и $i_4$; а $r_3$ контролирует $i_2$, $i_3$ и $i_4$.

\noindent
\begin{equation*}
    \begin{aligned}
    r_1 &= i_1 \oplus i_2 \oplus i_4; \\
    r_2 &= i_1 \oplus i_3 \oplus i_4; \\
    r_3 &= i_2 \oplus i_3 \oplus i_4; \\
    s_1 &= r_1 \oplus i_1 \oplus i_2 \oplus i_4; \\
    s_2 &= r_2 \oplus i_1 \oplus i_3 \oplus i_4; \\
    s_3 &= r_3 \oplus i_2 \oplus i_3 \oplus i_4;
    \end{aligned}
\end{equation*}
\end{figure}

Рассмотрим таблицу значений синдрома ($s_1;s_2;s_3$) и позицию ошибочного бита в сообщении:

\begin{table}[H]
    \centering
    \begin{tabular}{|c|c|c|}
        \hline
        \thead{Синдром ($s_1;s_2;s_3$)} & \thead{Конфигурация ошибок} & \thead{Ошибочный символ} \\ \hline
        000 & НЕТ & НЕТ \\
        001 & 0001000 & $r_3$ \\
        010 & 0100000 & $r_2$ \\
        011 & 0000010 & $i_3$ \\
        100 & 1000000 & $r_1$ \\
        101 & 0000100 & $i_2$ \\
        110 & 0010000 & $i_1$ \\
        111 & 0000001 & $i_4$ \\
        \hline
    \end{tabular}
\end{table}

Возьмем, к примеру, 2 строку (см. таблицу на следующей странице) : синдром $S(0,0,1)$, это значит что $s_1 = 0$, $s_2 = 0$, $s_3 = 1$. По построенной таблице кода (7,4) находится, в каком бите ошибка - какой бит содержится только в 3 синдроме. Проще говоря, в каком столбце $X$ стоит только в 3 строчке (напротив $s_3$). По таблице видим, что такой бит - 4 (поэтому во втором столбце 0001000 - нули означают правильный бит, а единица - ошибочный. В данном случае ошибочный бит 4, поэтому он равен единице). Смотрим, какой именно бит занимает четвертое место - $r_3$.

Чтобы получить правильную последовательность, необходимо инвертировать ошибочный бит.

\example{1}
\task получена последовательность 1100100. Вычислить ошибочный бит, записать правильную последовательность.
\solution составим таблицу кода (7,4) с конкретной последовательностью.
\begin{table}[H]
    \centering
    \begin{tabular}{|c|c|c|c|c|c|c|c|c|}
        \hline
        & 1 & 2 & 3 & 4 & 5 & 6 & 7 & \\ \hline
        & 1 & 1 & 0 & 0 & 1 & 0 & 0 & \\ \hline
        $2^k$ & $r_{1}$ & $r_{2}$ & $i_{1}$ & $r_{3}$ & $i_{2}$ & $i_{3}$ & $i_{4}$ & S \\ \hline
        1 & \cellcolor{iogreen}{X} & & \cellcolor{iogreen}{X} & & \cellcolor{iogreen}{X} & & \cellcolor{iogreen}{X} & $s_{1}$ \\ \hline
        2 & & \cellcolor{ioyellow}{X} & \cellcolor{ioyellow}{X} & & & \cellcolor{ioyellow}{X} & \cellcolor{ioyellow}{X} & $s_{2}$ \\ \hline
        4 & & & & \cellcolor{ioorange}{X} & \cellcolor{ioorange}{X} & \cellcolor{ioorange}{X} & \cellcolor{ioorange}{X} & $s_{3}$ \\ \hline
\end{tabular}
\end{table}


\noindent
Рассчитаем значение контрольных бит результата:

\noindent
\begin{equation*}
    \begin{aligned}
        r_{1\text{ рез}} &= i_1 \oplus i_2 \oplus i_4 &= 0 \oplus 1 \oplus 0 &= 1; \\
        r_{2\text{ рез}} &= i_1 \oplus i_3 \oplus i_4 &= 0 \oplus 0 \oplus 0 &= 0; \\
        r_{3\text{ рез}} &= i_2 \oplus i_3 \oplus i_4 &= 1 \oplus 0 \oplus 0 &= 1; \\
    \end{aligned}
\end{equation*}

\noindent
Рассчитаем синдромы:

\noindent
\begin{equation*}
    \begin{aligned}
        s_1 &= r_1 \oplus i_1 \oplus i_2 \oplus i_4 &= r_{1\text{ рез}} \oplus r_{1\text{ исх}} &= 1 \oplus 1 &= 0; \\
        s_2 &= r_2 \oplus i_1 \oplus i_3 \oplus i_4 &= r_{2\text{ рез}} \oplus r_{2\text{ исх}} &= 0 \oplus 1 &= 1; \\
        s_3 &= r_3 \oplus i_2 \oplus i_3 \oplus i_4 &= r_{3\text{ рез}} \oplus r_{3\text{ исх}} &= 1 \oplus 0 &= 1; \\
    \end{aligned}
\end{equation*}

\noindent
Полученный синдром: $S(0,1,1)$. \\
Смотрим по таблице, какой бит содержится в $s_2$ и $s_3$ одновременно. 6, то есть $i_3$.
Инвертируем ошибочный бит и получаем правильную последовательность.

\answer 6 бит ($i_3$), правильная последовательность: 1100110.

\begin{figure}[H]
     \includegraphics[width=\linewidth]{6.2-hamming-encoder}
     \caption{Схема создания кода Хэмминга (7,4)}
     \label{fig:hamming-encoding}
\end{figure}

\begin{figure}[H]
     \includegraphics[width=\linewidth]{6.2-hamming-decoder}
     \caption{Схема декодирования кода Хэмминга (7,4)}
     \label{fig:hamming-decoding}
\end{figure}

\paragraph{Код Хэмминга для r = 4}

Разбрем код Хэмминга для $r = 4$ аналогично тому, как мы сделали это выше.
В данном случае, у нас 15 бит, из них 11 информационных и 4 проверочных. То есть, код имеет маркировку (15, 11). \\
Составим таблицу кода (15, 11):

\begin{table}[H]
    \centering
    \begin{tabularx}{\linewidth}{|C|C|C|C|C|C|C|C|C|C|C|C|C|C|C|C|C|}
    \hline
    & 1 & 2 & 3 & 4 & 5 & 6 & 7 & 8 & 9 & 10 & 11 & 12 & 13 & 14 & 15 & \\ \hline
    $2^k$ & $r_{1}$ & $r_{2}$ & $i_{1}$ & $r_{3}$ & $i_{2}$ & $i_{3}$ & $i_{4}$ & $r_{4}$ & $i_{5}$ & $i_{6}$ & $i_{7}$ & $i_{8}$ & $i_{9}$ & $i_{10}$& $i_{11}$ & S \\ \hline
    1 & \cellcolor{iogreen}{X} & & \cellcolor{iogreen}{X} & & \cellcolor{iogreen}{X} & & \cellcolor{iogreen}{X} & & \cellcolor{iogreen}{X} & & \cellcolor{iogreen}{X} & & \cellcolor{iogreen}{X} & & \cellcolor{iogreen}{X} & $s_{1}$ \\ \hline
    2 & & \cellcolor{ioyellow}{X} & \cellcolor{ioyellow}{X} & & & \cellcolor{ioyellow}{X} & \cellcolor{ioyellow}{X} & & & \cellcolor{ioyellow}{X} & \cellcolor{ioyellow}{X} & & & \cellcolor{ioyellow}{X} & \cellcolor{ioyellow}{X} & $s_{2}$ \\ \hline
    4 & & & & \cellcolor{ioorange}{X} & \cellcolor{ioorange}{X} & \cellcolor{ioorange}{X} & \cellcolor{ioorange}{X} & & & & & \cellcolor{ioorange}{X} & \cellcolor{ioorange}{X} & \cellcolor{ioorange}{X} & \cellcolor{ioorange}{X} & $s_{3}$ \\ \hline
    8 & & & & & & & & \cellcolor{iored}{X} & \cellcolor{iored}{X} & \cellcolor{iored}{X} & \cellcolor{iored}{X} & \cellcolor{iored}{X} & \cellcolor{iored}{X} & \cellcolor{iored}{X} & \cellcolor{iored}{X} & $s_{4}$\\ \hline
    \end{tabularx}
\end{table}

\noindent
Получим:

\noindent
\begin{equation*}
    \begin{aligned}
    r_1 &= i_1 \oplus i_2 \oplus i_4  \oplus i_5 \oplus i_7 \oplus i_{9}  \oplus i_{11}; \\
    r_2 &= i_1 \oplus i_3 \oplus i_4  \oplus i_6 \oplus i_7 \oplus i_{10} \oplus i_{11}; \\
    r_3 &= i_2 \oplus i_3 \oplus i_4  \oplus i_8 \oplus i_9 \oplus i_{10} \oplus i_{11}; \\
    r_4 &= i_5 \oplus i_6 \oplus i_7  \oplus i_8 \oplus i_9 \oplus i_{10} \oplus i_{11};
    \end{aligned}
\end{equation*}

\noindent
\begin{equation*}
    \begin{aligned}
    s_1 &= r_1 \oplus i_1 \oplus i_2 \oplus i_4  \oplus i_5 \oplus i_7 \oplus i_{9}  \oplus i_{11}; \\
    s_2 &= r_2 \oplus i_1 \oplus i_3 \oplus i_4  \oplus i_6 \oplus i_7 \oplus i_{10} \oplus i_{11}; \\
    s_3 &= r_3 \oplus i_2 \oplus i_3 \oplus i_4  \oplus i_8 \oplus i_9 \oplus i_{10} \oplus i_{11}; \\
    s_4 &= r_4 \oplus i_5 \oplus i_6 \oplus i_7  \oplus i_8 \oplus i_9 \oplus i_{10} \oplus i_{11};
    \end{aligned}
\end{equation*}

Рассмотрим таблицу значений синдрома ($s_1;s_2;s_3;s_4$) и позицию ошибочного бита в сообщении:

\bigskip\noindent
Рассуждения будут аналогичны. Возьмем, например, 4ую строку (см. таблицу на следующей странице). % TODO: Сделать ссылку
Синдром S(0,0,1,1) означает, что $s_1 = 0$, $s_2 = 0$, $s_3 = 1$, $s_4 = 1$.
По таблице кода (15,11) находится, в каком бите ошибка - какой бит содержится только в 4 синдроме.
В данном случае, такой бит --- 12.
Напомним, что нули обозначают правильный бит, а единица --- ошибочный.
Двенадцатое место занимает бит $i_8$.

\noindent
\begin{table}[H]
    % \caption{Caption}
    % \label{tab:my_label}
    \centering
    \begin{tabular}{|c|c|c|c|}
        \hline
        \thead{Синдром ($s_1;s_2;s_3$)} & \thead{Конфигурация ошибок} & \thead{Ошибочный символ} \\ \hline
        0000 & НЕТ & НЕТ \\
        0001 & 000000010000000 & $r_4$ \\
        0010 & 000100000000000 & $r_3$ \\
        0011 & 000000000001000 & $i_8$ \\
        0100 & 010000000000000 & $r_2$ \\
        0101 & 000000000100000 & $i_6$ \\
        0110 & 000001000000000 & $i_3$ \\
        0111 & 000000000000010 & $i_{10}$ \\
        1000 & 100000000000000 & $r_1$ \\
        1001 & 000000001000000 & $i_5$ \\
        1010 & 000010000000000& $i_2$ \\
        1011 & 000000000000100 & $i_9$ \\
        1100 & 001000000000000 & $i_1$ \\
        1101 & 000000000010000 & $i_7$ \\
        1110 & 000000100000000 & $i_4$ \\
        1111 & 000000000000001 & $i_{11}$ \\
        \hline
    \end{tabular}
\end{table}
