% !TeX root = ../../main.tex
\subsection{Visual Basic for Applications}\label{subsec:vba}

\textbf{Visual Basic for Applications} (VBA, Visual Basic для приложений) --- немного упрощенная реализация языка программирования Visual Basic, встроенная в линейку продуктов Microsoft Office (включая версии для Mac OS), а также во многие другие программные пакеты, такие как AutoCAD, SolidWorks, CorelDRAW, WordPerfect и ESRI ArcGIS. VBA покрывает и расширяет функциональность ранее использовавшихся специализированных макро-языков, таких как WordBasic.


\textbf{Макрос} --- программа, написанная на внутреннем для текстового процессора языке программирования, которую можно выполнять по желанию пользователя. Эту программу можно изменять, как и на другом языке программирования. Это позволяет автоматизировать повторяющиеся или сложные действия пользователя, которые отсутствуют в стандартном функционале текстового процессора. Таким образом можно расширять стандартный функционал.

\medskip\noindent
Использование макросов предполагает знание встроенного в текстовый процессор языка программирования. Однако современные текстовые процессоры позволяют пользователям, не знающим язык макросов, записывать свои действия для дальнейшего их преобразования в макрос. Действия пользователя автоматически описываются с помощью встроенного языка программирования.

\medskip\noindent
Автоматической записи макроса часто бавает недостаточно, поэтому советуем ознакомиться с основами языка программирования макросов в линейке продуктов Microsoft Office Visual Basic for Applications.

\subsubsection{Имя переменной}\label{subsubsec:vba-variable-name}
\begin{itemize}[noitemsep]
    \item Начинается с буквы латинского алфавита.
    \item Не может содержать пробелы, точки символы операций (+, -, *, /, \#, \$, \%, \&, !, <, >, = и так далее).
    \item Не может превышать 254 символов в длину.
    \item Должно быть уникальным в своей области действия.
    \item Не может дублировать зарезервированные слова.
    \item Не различает регистр букв: MyNumber = mYnUmBeR.
\end{itemize}

\subsubsection{Типы данных}\label{subsubsec:vba-data-types}

\begin{table}[H]
    \centering
    \caption{Типы данных в VBA}\label{tab:vba-data-types}
    \begin{tabular}{|c|c|c|c|}
        \hline
        \thead{Тип данных} & \thead{Резервируемая \\ память, байт} & \thead{Минимальное \\ значение} & \thead{Максимальное \\ значение} \\ \hline
        \code{Byte} & 1 & \code{0} & \code{255}\\
        \code{Boolean} & 2 & \code{False} & \code{True} \\
        \code{Integer} & 2 & \code{-32768} & \code{32767} \\
        \code{Long} & 4 & \code{-2147483648} & \code{2147483647} \\
        \code{Date} & 8 & 1 января 100 г. & 31 декабря 9999 г. \\
        \code{String} & Длина строки & \code{1} & \code{65400} \\
        \code{Variant} (число) & 16 &  & \\
        \code{Variant}(символ) & \makecell{22 байта + \\ длина строки} &  & \makecell{2147483647 \\ символов}\\
        \hline
    \end{tabular}
\end{table}

\paragraph{Объявление переменных}\label{par:vba-variable-declaration}

\noindent
\textbf{Неявно: }
\begin{minted}{vb.net}
    sum = 100
\end{minted}
В данном случае присваивается тип Variant. Это обобщенный тип, переменная нетипизирована.

\bigskip\noindent
\textbf{Явно:}
\begin{minted}{vb.net}
    Dim sum As Integer
\end{minted}
\emph{Преимущества}:
\begin{itemize}[noitemsep]
    \item Программа быстрее работает.
    \item Программе требуется меньше памяти.
    \item Легче обнаружить некоторые ошибки.
    \item Не возникает проблем со сложными типами (например, как отличить дату от текста).
\end{itemize}
\emph{Недостатки}:
\begin{itemize}[noitemsep]
    \item Приходится думать.
    \item Требуется использовать больше переменных.
\end{itemize}
