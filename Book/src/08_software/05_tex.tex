% !TeX root = ../../main.tex
\subsection{\TeX}
При взаимодействии человека с компьютером возникает вопрос: каким же образом представлять документы на экране и затем в памяти компьютера и в печатном виде. На этот счет существует 2 парадигмы:
\begin{itemize}
    \item \textbf{WYSIWYG}~\footnote{WYSIWYG (от англ.``What You See Is What You Get'') --- ``что видишь, то и получишь''} --- свойство прикладных программ или веб-интерфейсов, в которых содержание отображается в процессе редактирования и выглядит максимально близко похожим на конечную продукцию, которая может быть печатным документом, веб-страницей или презентацией. В настоящее время для подобных программ также широко используется понятие ``визуальный редактор''. (пример: Microsoft Word)
    \item \textbf{WYSIWYM}~\footnote{WYSIWYM (от англ. ``What You See Is What You Mean'') --- ``что видишь, есть то, что имеешь в виду''} --- парадигма редактирования документов, возникшая как альтернатива более распространенной парадигме WYSIWYG. В WYSIWYM редакторе пользователь задает только логическую структуру документа и собственно контент. Оформление документа, его итоговый внешний вид возложено на отдельное ПО, либо, во всяком случае, вынесено в отдельный блок. Таким образом достигается полная независимость содержания документа от его формы. (пример: \TeX)
\end{itemize}

\textbf{\TeX} --- система компьютерной верстки, разработанная американским профессором информатики Дональдом Кнутом в целях создания компьютерной типографии. В нее входят средства для секционирования документов, для работы с перекрестными ссылками. Номер версии \TeX\ приближается к $\pi$, номер редактора формул - к числу $e$.

\subsubsection{Сравнение \LaTeX\ с WYSIWYG редакторами}
\LaTeX (произносится латех) — наиболее популярный набор макрорасширений (или макропакет) системы компьютерной вёрстки \TeX, который облегчает набор сложных документов. Разработан Лэсли Лэмпортом в 1984 году и назван в его честь.

% Как nirtable только с возможностью расширить на всю страницу (сделать environment как nirtable не вышло)
\begin{xltabular}{\textwidth}{|>{\hsize=0.8\hsize\small}C|>{\hsize=1.1\hsize\small}C|>{\hsize=1.1\hsize\small}C|}
    \caption{Сравнение \LaTeX\ с WYSIWYG редакторами\label{tab:latex-vs-word}}
    \\\hline
    \textbf{Критерий} & \textbf{\LaTeX} & \textbf{MS Word, LibreOffice Writer} \\\hline \endfirsthead
    \multicolumn{3}{r}{\normalsize{Продолжение таблицы~\thetable}} \\\hline
    \textbf{Критерий} & \textbf{\LaTeX} & \textbf{MS Word, LibreOffice Writer} \\\hline \endhead

    Работа с формулами & Линейное текстовое представление формул & Переключение между линейным и математическим видом формулы \\
    \hline
    Дизайнерские задачи & Текстовый векторный редактор, сложно управлять расположением графики & Встроенный векторный WYSIWYG редактор, наглядное управление структурой документа \\
    \hline
    Порог вхождения & Высокий & Низкий \\
    \hline
    Написание научных статей & Является мировым стандартом & Большинство российских журналов \\
    \hline
    Стоимость & Бесплатно & LibreOffice бесплатно \\
    \hline
    Рецензирование текстов & Нет, можно комментировать & Продвинутые встроенные возможности \\
    \hline
    Экспорт в другие форматы & \multicolumn{2}{>{\hsize=2.2\hsize\small}C|}{Полноценной бесплатной утилиты для экспорта из docx в tex и обратно не существует} \\
    \hline
    Требования к аппарат. обеспечению & Очень низкие: достаточно консоли & Высокие \\
    \hline
    Автогенерация документов & Удобно генерировать отчеты изнутри работающих программ & Требуется знание сложной структуры docx (odt) \\
    \hline
    Коллективная работа с файлами & Разрабатывается & Поддерживается MSW по умолчанию \\
    \hline
    Количество квалифицированных пользователей & Мало (ученые, пользователи с техническим образованием) & Много (подавляющее большинство людей) \\
    \hline
    Кросс-платформенность ПО для редактирования & Любая ОС & Большинство популярных ОС с наличием GUI \\
    \hline
    Кросс-платформенность формата файла & Обратная совместимость хорошо обеспечена & Проблемы при обновлении версий \\
    \hline
    Проверка пунктуации и грамматики & Отсутствует (утилита hunspell) & Доступна по умолчанию \\
    \hline
    Автоматизация повтор. действий & Отсутствует (можно использовать скрипты) & Встроенная поддержка макросов
    \\\hline
    Работа с большими файлами & Нет ограничений & Компьютер ``подвисает''
    \\\hline
    Кодировка & На выходе получается PDF файл & Проблемы из-за несоответствия кодировок
    \\\hline
\end{xltabular}


% \begin{xnirtable}{\linewidth}{|c|C|C|}{3}{Сравнение \LaTeX\ с WYSIWYG редакторами}{tab:latex-vs-word}{
%      \textbf{Критерий} & \textbf{\LaTeX} & \textbf{MS Word, LibreOffice Writer}
% }
%     Работа с формулами & Линейное текстовое представление формул & Переключение между линейным и математическим видом формулы \\
%     \hline
%     Дизайнерские задачи & Текстовый векторный редактор, сложно управлять расположением графики & Встроенный векторный WYSIWYG редактор, наглядное управление структурой документа \\
%     \hline
%     Порог вхождения & Высокий & Низкий \\
%     \hline
%     Написание научных статей & Является мировым стандартом & Большинство российских журналов \\
%     \hline
%     Стоимость & Бесплатно & LibreOffice бесплатно \\
%     \hline
%     Рецензирование текстов & Нет, можно комментировать & Продвинутые встроенные возможности \\
%     \hline
%     Экспорт в другие форматы & \multicolumn{2}{p{0.7\textwidth}|}{Полноценной бесплатной утилиты для экспорта из docx в tex и обратно не существует} \\ % Общая ширина 11.5 см
%     \hline
%     Требования к аппарат. обеспечению & Очень низкие: достаточно консоли & Высокие \\
%     \hline
%     Автогенерация документов & Удобно генерировать отчеты изнутри работающих программ & Требуется знание сложной структуры docx (odt) \\
%     \hline
%     Коллективная работа с файлами & Разрабатывается & Поддерживается MSW по умолчанию \\
%     \hline
%     Количество квалифицированных пользователей & Мало (ученые, пользователи с техническим образованием) & Много (подавляющее большинство людей) \\
%     \hline
%     Кросс-платформенность ПО для редактирования & Любая ОС & Большинство популярных ОС с наличием GUI \\
%     \hline
%     Кросс-платформенность формата файла & Обратная совместимость хорошо обеспечена & Проблемы при обновлении версий \\
%     \hline
%     Проверка пунктуации и грамматики & Отсутствует (утилита hunspell) & Доступна по умолчанию \\
%     \hline
%     Автоматизация повтор. действий & Отсутствует (можно использовать скрипты) & Встроенная поддержка макросов \\
%     \hline
%     Работа с большими файлами & Нет ограничений & Компьютер ``подвисает'' \\
%     \hline
%     Кодировка & На выходе получается PDF файл & Проблемы из-за несоответствия кодировок \\
%     \hline
% \end{xnirtable}

% 

\begin{table}[H]
    \caption{Сравнение \LaTeX с WYSIWYG редакторами}
    \centering
    \small
    \begin{tabular}{|p{5cm}|p{5.5cm}|p{5.5cm}|} % Увеличиваем ширину колонок
    \hline
    Критерий & \LaTeX & MS Word, LibreOffice Writer \\
    \hline
    Работа с формулами & Линейное текстовое представление формул & Переключение между линейным и математическим видом формулы \\
    \hline
    Дизайнерские задачи & Текстовый векторный редактор, сложно управлять расположением графики & Встроенный векторный WYSIWYG редактор, наглядное управление структурой документа \\
    \hline
    Порог вхождения & Высокий & Низкий \\
    \hline
    Написание научных статей & Является мировым стандартом & Большинство российских журналов \\
    \hline
    Стоимость & Бесплатно & LibreOffice бесплатно \\
    \hline
    Рецензирование текстов & Нет, можно комментировать & Продвинутые встроенные возможности \\
    \hline
    Экспорт в другие форматы & \multicolumn{2}{|p{11.5cm}|}{Полноценной бесплатной утилиты для экспорта из docx в tex и обратно не существует} \\ % Общая ширина 11.5 см
    \hline
    Требования к аппарат. обеспечению & Очень низкие: достаточно консоли & Высокие \\
    \hline
    Автогенерация документов & Удобно генерировать отчеты изнутри работающих программ & Требуется знание сложной структуры docx (odt) \\
    \hline
    Коллективная работа с файлами & Разрабатывается & Поддерживается MSW по умолчанию \\
    \hline
    Количество квалифицированных пользователей & Мало (ученые, пользователи с техническим образованием) & Много (подавляющее большинство людей) \\
    \hline
    Кросс-платформенность ПО для редактирования & Любая ОС & Большинство популярных ОС с наличием GUI \\
    \hline
    Кросс-платформенность формата файла & Обратная совместимость хорошо обеспечена & Проблемы при обновлении версий \\
    \hline
    Проверка пунктуации и грамматики & Отсутствует (утилита hunspell) & Доступна по умолчанию \\
    \hline
    Автоматизация повтор. действий & Отсутствует (можно использовать скрипты) & Встроенная поддержка макросов \\
    \hline
    Работа с большими файлами & Нет ограничений & Компьютер "подвисает" \\
    \hline
    Кодировка & На выходе получается PDF файл & Проблемы из-за несоответствия кодировок \\
    \hline
    \end{tabular}
\end{table}

