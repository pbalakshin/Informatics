
% !TeX root = ../../main.tex
\subsection{Лицензии}\label{subsec:licenses}

\textbf{Лицензии на программное обеспечение} --- это правовой инструмент, определяющий использование и распространение программного
обеспечения, защищенного авторским правом.
\begin{description}
    \item[Проприетарное ПО] $\Rightarrow$ закрытый исходный код. \\ Может быть платным и бесплатным.
    \item[Свободное ПО] $\Rightarrow$ открытый исходный код. \\ Может быть платным и бесплатным.
    \item[Коммерческое ПО ] $\Rightarrow$ платное. \\ Может иметь как закрытый, так и открытый исходный код.
    \item[Бесплатное ПО] $\Rightarrow$ бесплатное. \\ Может иметь как закрытый, так и открытый исходный код.
\end{description}


\paragraph{Разновидности лицензий на свободное ПО}\label{par:free-software-licenses}
\begin{itemize}
    \item \textbf{Пермиссивные лицензии (BSD)}: можно менять и закрывать.
    \item \textbf{Копилефт (GPL)}: можно менять, нельзя закрывать.
\end{itemize}

\noindent
\begin{minipage}[t]{\textwidth}
    \begin{wrapfigure}{l}{0.25\textwidth}
        \centering
        \vspace{-\intextsep}
        \includegraphics[width=0.25\textwidth]{person/stallman_richard}
        \caption*{Ричард Мэттью Столлман \\ род. 1953}
    \end{wrapfigure}

    Основоположником движения за открытый код является Ричард Мэттью Столлман. Он и создал первую лицензию GNU (General Public License). \\
    Всего около 70 лицензий на владение свободным ПО (одобренных на opensource.org). Самые популярные: Apache License, BSD license, GPL, LGPL, MIT license, MPL.
\end{minipage}
\\[3\baselineskip]

\subsubsection{Базовые права, предоставляемые свободным ПО}\label{subsubsec:free-software-basic-rights}

\noindent
Все они предоставляют 4 базовых права:

\begin{itemize}[noitemsep]
    \item Право на запуск программы в любых целях (только если она не нанесет вред своим действием или бездействием).
    \item Право на изучение исходного и бинарного кода программы
    \item Право на платное или бесплатное распространение программы.
    \item Право на развитие программы.
\end{itemize}

\noindent
Если программист передает пользователю свою программу, но не прилагает лицензию, то действует \enquote{право свободного пользования}:

\begin{itemize}[noitemsep]
    \item Можно установить программу на 1 компьютер.
    \item Можно запускать программу на 1 компьютере.
    \item Нельзя копировать программу на другие компьютеры.
    \item Нельзя модифицировать программу.
    \item Данная лицензия действует 5 лет (п.4 ст. 1235 ГК РФ).
\end{itemize}

\subsubsection{Особенности различных свободных лицензий}\label{subsubsec:free-software-licenses-features}

\paragraph{GNU GPL}\label{par:gnu-gpl-license}
\begin{itemize}[noitemsep]
    \item Запрещено включать исходные тексты в закрытое ПО, запрещено менять тип лицензии (copyleft \textcopyleft).
    \item Запрещено динамическое связывание GNU GPL-библиотек с не-GNUGPL библиотеками (dll).
\end{itemize}

\paragraph{GNU LGPL}\label{par:gnu-lgpl-license}
\begin{itemize}[noitemsep]
    \item Допускается динамическое связывание с закрытыми библиотеками.
    \item Запрещено использование кода в другом ПО.
\end{itemize}

\paragraph{MPL (Mozilla public license)}\label{par:mpl-license}
\begin{itemize}[noitemsep]
    \item Можно использовать исходные тексты в закрытом ПО, но лишь частично и с гарантией доступа к изменениям.
\end{itemize}

\paragraph{BSD License}\label{par:bsd-license}
\begin{itemize}[noitemsep]
    \item Можно использовать исходные коды в закрытом ПО без ограничений.
\end{itemize}

\subsubsection{Ответственность за пиратское ПО}\label{subsubsec:pirated-software-responsibility}

\todo{Выбрать}
\todo{Как будто тут ссылки на КОАП РФ и УК РФ?}

\paragraph{Административная ответственность за пиратское ПО}\label{par:administrative-responsibility-for-pirated-software}
\noindent
Статья 7.12~КоАП~РФ: нарушение авторских прав при ущербе на сумму до 100~000 рублей:
\begin{itemize}[noitemsep]
    \item штраф до 2~000 (физическое лицо)
    \item штраф до 20~000 (должностное лицо)
    \item штраф до 40~000 (юридическое лицо)
\end{itemize}

\paragraph{Уголовная ответственность за пиратское ПО}\label{par:criminal-responsibility-for-pirated-software}
\noindent
Статья 146.2~УК~РФ: незаконное использование объектов авторского права (в т.ч. приобретение, хранение) при ущербе на сумму от 100~000 рублей:
\begin{itemize}[noitemsep]
    \item штраф до 200~000 р.
    \item исправительные работы вплоть до 2 лет
    \item – арест вплоть до 2 лет
\end{itemize}

\noindent
Статья 146.3~УК~РФ: Незаконное использование объектов авторского права (в т.ч. приобретение, хранение) при ущербе на сумму от 1~000~000 рублей:
\begin{itemize}[noitemsep]
    \item штраф до 500~000 р.
    \item арест вплоть до 6 лет
\end{itemize}

\paragraph{Уголовная ответственность за плагиат ПО}\label{par:criminal-responsibility-for-software-plagiarism}
\noindent
Статья 146.1~УК~РФ: присвоение авторства, если это причинило крупный ущерб автору:
\begin{itemize}[noitemsep]
    \item штраф до 200~000 р.
    \item исправительные работы вплоть до 1 года
    \item арест вплоть до 6 месяцев
\end{itemize}

\paragraph{Гражданская ответственность за нарушение лицензии ПО}\label{par:civil-responsibility-for-software-license-violation}
\noindent
Статья 1301~ГК~РФ: нарушение авторских, интеллектуальных и исключительных прав:
\begin{itemize}[noitemsep]
    \item штраф до 5~000~000 руб. в пользу обладателя ПО \\
          \textbf{либо}
    \item двукратное возмещение убытков обладателю ПО
\end{itemize}


\todo{ИЛИ}

\begin{description}
    \item[Административная ответственность за пиратское ПО] Статья 7.12 КоАП РФ: нарушение авторских прав при ущербе на сумму до 100 000 рублей: \\
          \begin{itemize}[noitemsep]
              \item штраф до 2 000 (физическое лицо)
              \item штраф до 20 000 (должностное лицо)
              \item штраф до 40 000 (юридическое лицо)
          \end{itemize}

    \item[Уголовная ответственность за пиратское ПО] Статья 146.2 УК РФ: незаконное использование объектов авторского права (в т.ч. приобретение, хранение) при ущербе на сумму от 100 000 рублей:
          \begin{itemize}[noitemsep]
              \item штраф до 200 000 р.
              \item исправительные работы вплоть до 2 лет
              \item – арест вплоть до 2 лет
          \end{itemize}
          Статья 146.3 УК РФ: Незаконное использование объектов авторского права (в т.ч. приобретение, хранение) при ущербе на сумму от 1 000 000 рублей:
          \begin{itemize}[noitemsep]
              \item штраф до 500 000 р.
              \item арест вплоть до 6 лет
          \end{itemize}

    \item[Уголовная ответственность за плагиат ПО] Статья 146.1 УК РФ: присвоение авторства, если это причинило крупный ущерб автору:
          \begin{itemize}[noitemsep]
              \item штраф до 200 000 р.
              \item исправительные работы вплоть до 1 года
              \item арест вплоть до 6 месяцев
          \end{itemize}

    \item[Гражданская ответственность за нарушение лицензии ПО]  Статья 1301 ГК РФ: нарушение авторских, интеллектуальных и исключительных прав:
          \begin{itemize}[noitemsep]
              \item штраф до 5 000 000 руб. в пользу обладателя ПО
                    \textbf{либо}
              \item двукратное возмещение убытков обладателю ПО
          \end{itemize}
\end{description}
