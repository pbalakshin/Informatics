\subsection{Офисное ПО}
\label{subsec:office-software}

Компания Microsoft открыла формат doc (стандарт по которому он создается) только в 2000~г.
До этого разработчикам приходилось методом обратного инженеринга вручную раскрывать этот стандарт.
Это характеризует формат doc не в лучшую сторону --- не все имели доступ (Microsoft Word --- платное ПО) и формат проверен малым количеством людей.
До сих пор существует много ошибок, которые исправлены только в docx.

Форматы odt и docx используют XML.
Технология XML открыта --- любой может посмотреть, найти ошибки, предложить исправление.
Чтобы убедиться, что odt и docx используют XML, есть простой способ: создайте docx или odt документ, смените расширение на zip и распакуйте.
Можно увидеть, что в основном все файлы имеют расширение xml.

Нормальная криптография появилась только в docx.
В формате doc ничего нельзя было шифровать.
Только сторонними приложениями.

\paragraph*{Стоимость продуктов Microsoft Office}
\begin{itemize}
  \item Для дома и учебы (Word, Excel, PowerPoint, OneNote) $\approx$ 3000р.
  \item Для дома и бизнеса (Word, Excel, PowerPoint, OneNote, Outlook) $\approx$ 10000р.
  \item Профессиональный (Word, Excel, PowerPoint, OneNote, Outlook, Access, Publisher) $\approx$ 20000р.
\end{itemize}

% TODO: Может быть добавить тоже какой-то параграф?
\noindent
LibreOffice, OpenOffice, Calligra Suite = 0р.

\bigskip
В 2010 году был принят ГОСТ Р ИСО/МЭК 26300-2010, обязывающий госучреждения перейти на бесплатный формат документов (Open Document Format --- ODF).
Но это вовсе не означает, что будет использоваться LibreOffice и OpenOffice.
В последних версиях Microsoft Office есть поддержка этого формата.

% TODO: Хочется убрать красную обводку
\begin{figure}[H]
    \centering
    \includegraphics[width=15cm]{src/08_software/8_1.png}
    \caption{ГОСТ Р ИСО/МЭК 26300-2010}
    \label{fig:gost-iso-26300-2010}
\end{figure}

\paragraph{ODF}
\begin{itemize}
    \item Распоряжение Правительства Российской Федерации от 17 декабря 2010 г. №2299-р ``О плане перехода федеральных органов исполнительной власти и федеральных бюджетных учреждений на использование свободного программного обеспечения (2011 - 2015 годы)''
    \item OpenDocument это единственный формат электронной документации, который реализован несколькими производителями ПО и утвержден ISO в качестве международного стандарта.
    \item Абсолютно любой производитель ПО может использовать формат OpenDocument при разработке своего собственного редактора
    электронных документов.
    \item Продолжительный цикл существования формата OpenDocument обеспечен стабильностью спецификации.
    \item Открытость формата OpenDocument способствует росту его популярности среди производителей ПО, в числе которых есть производители бесплатных и открытых продуктов.
    \item Открытость формата OpenDocument позволяет пользователю быть свободным при выборе программного обеспечения, не зависеть от конкретного поставщика ПО и его маркетинговой политики.
    \item Исчезает угроза утери информации из-за изменения закрытых форматов
\end{itemize}

\noindent
Недостатки ODF:
\begin{itemize}
    \item   Спецификация стандарта не определяет некоторые элементы офисного программного обеспечения.
            По этой причине каждый разработчик может реализовывать эти элементы по-своему, что может привести к несовместимости соответствующих файлов.
    \item   Новые версии стандарта выходят спустя большие промежутки времени.
            По этой причине недостатки текущей версии стандарта находят отражения в большем числе его реализаций.
    \item   Некоторые специалисты считают, что Microsoft, с целью вернуть монопольное положение на рынке, может создать свою реализацию формата OpenDocument с закрытым кодом, которая станет неоспоримым лидером среди реализаций OpenDocument.
            Далее, по мнению этих специалистов, на эту реализацию перейдёт основная масса пользователей и государственных органов --- после этого Microsoft внесёт изменения в очередную версию продукта, благодаря которым файлы, созданные в этой программе перестанут быть совместимыми с форматом OpenDocument.
\end{itemize}

\noindent
Работа по стандартизации OpenDocument включает в себя:
\begin{itemize}
    \item \textbf{OpenDocument 1.0 (второе издание)} OASIS (англ. Organization for the Advancement of Structured Information Standards) соответствует опубликованному стандарту ISO/IEC 26300:2006. Содержание ISO/IEC 26300 и OASIS OpenDocument v1.0 2-е издание идентичны. Оно включает в себя изменения, внесенные редакционным решением JTC1 ( англ. Joint Technical Committee 1) и доступно в ODF, HTML и PDF форматах.
    \item \textbf{OpenDocument 1.1} включает дополнительные возможности для решения проблем доступности. Он был утвержден в качестве стандарта OASIS на 2007-02-01 после голосования, опубликованного 2007-01-16. Публичное заявление было сделано 2007-02-13. Эта версия не была изначально представлена ISO / IEC, потому что это считается незначительным обновлением только ODF 1.0, и OASIS работали уже на ODF 1.2 в тот момент, когда ODF 1.1 была утверждена. Однако позднее было представлено ISO / IEC (по состоянию на март 2011 года, он был в "стадии запроса'' как проект поправки 1- ISO/IEC 26300:2006/DAM 1) и опубликовано в марте 2012 года, как ISO / IEC 26300: 2006 / Amd 1: 2012 - Open Document Format for Office Applications (OpenDocument) v1.1.
    \item  \textbf{OpenDocument 1.2} был утвержден в качестве спецификации OASIS на 2011-03-17 и в качестве стандарта OASIS на 2011-09-29. Он включает в себя дополнительные функции доступности, RDF-метаданные, электронную таблицу спецификаций формул, основанную на OpenFormula, поддержку цифровых подписей и некоторые особенности, предложенные общественностью. В октябре 2011 года ожидалось, что технический комитет OASIS ODF "скоро начнет процесс предоставления ODF 1.2 в ISO / IEC JTC 1 ". В мае 2012 года преставители ISO / IEC JTC 1 / SC 34 / WG 6 сообщили, что после некоторой задержки, процесс подготовки ODF 1.2 для представления JTC 1 для PAS транспозиции ведется в настоящее время.
\end{itemize}

\subsubsection{Наиболее популярные офисные пакеты}
\label{subsubsec:popular-office-suites}

% TODO: Актуализировать данные стоимости?
\begin{table}[H]
    \centering
    \resizebox{\textwidth}{!}{
    \begin{tabular}{|c|c|c|c|}
    \hline
        \thead{Название \\ офисного пакета} & \thead{Особенности} & \thead{Примерная \\ стоимость \\ на 2017~год} & \thead{Исходный \\ код} \\ \hline
        \makecell*{Google Docs, \\ Яндекс.Диск, \\ Облако Mail.ru} & \makecell*{Узкая ориентация \\ на публичные \\ облачные решения} & Бесплатно & Закрытый \\ \hline
         Microsoft Office & \makecell*{Имеет наиболее богатый \\  функционал, захватил \\ > 90\% desktop-установок } & 5000 -- 35000 & Закрытый \\ \hline
        \makecell*{LibreOffice, \\ OpenOffice, \\ Calligra Suite} & \makecell*{Слабая поддержка \\ одновременного редактирования} & Бесплатно & Открытый \\ \hline
        iWork & \makecell*{Узкая ориентация \\ на технику фирмы Apple } & Бесплатно & Закрытый \\ \hline
        WPS Office & \makecell*{Интерфейс индентичен \\ Microsoft Office} & \makecell*{5000 \\ (0 с рекламой)} & Закрытый \\ \hline
        \makecell*{WordPerfect \\ Office} & \makecell*{Узкая ориентация на рынок \\ персональных компьютеров} & 5000 -- 25000 & Закрытый \\ \hline
        \makecell*{OnlyOffice, \\ Feng Office} & \makecell*{Приоритетная ориентация \\ на частные и публичные \\ офисные решения} & Бесплатно & Открытый \\ \hline
    \end{tabular}}
    \caption{Наиболее популярные офисные пакеты}
    \label{tab:popular-office-suites}
\end{table}

В таблице~\ref{tab:popular-office-suites} рассмотрены не все 30 разновидностей офисных пакетов, а наиболее популярные. 
Популярность оценена с помощью сайта Trends Google, где отслеживается частота запросов пользователей со всего мира.
В таблице~\ref{tab:popular-office-suites} указаны офисные пакеты по убыванию популярности.

% TODO: rename to open office
\subsubsection{Сравнение возможностей OO и MS Office}
\label{subsubec:openoffice-msoffice}

\begin{table}[H]
    \centering
    \begin{tabular}{|c|c|c|}
        \hline
        \thead{Свойства} & \thead{Open Office Calc} & \thead{Microsoft Excel} \\ \hline
        Размерность & 1 024 $\times$ 1 048 576  & 16 384 $\times$ 1 048 576 \\ \hline
        Кол-во цветов & 104 & 16 777 216 \\ \hline
        \makecell*{Работа \\ с датами} & \makecell*{от 1 января 0001~г. \\ до 31 декабря 9999~г.} & \makecell*{от 1 января 1900~г. \\ до 31 декабря 9999~г.} \\ \hline
        
        \makecell*{Поддержка \\ графических \\ форматов} & 
        \makecell*{met, pbm, pgm, ppm, \\ psd, ras, sbm, sgg, \\ svg, xpm, xbm} & 
        \makecell*{cdr, emz, mix, pcz, \\ wmz, wpg, fpx, drw} \\ \hline

        Прочее & 
        \makecell*{Работа с MySQL. \\ Макросы на разных языках \\ (Python, JavaScript)} & 
        \makecell*{Продвинутые сводные \\ таблицы и условное \\ форматирование} \\ \hline 
    \end{tabular}
    \caption{Сравнение возможностей ``Open Office'' и ``Microsoft Office''}
    \label{tab:oo-calc-vs-ms-excel}
\end{table}


\paragraph{Open Office Writer}
\begin{itemize}[noitemsep]
  \item Частые обновления
  \item Независимые стили страниц в одном документе
  \item Автоматическое создание указателя формул
  \item Продвинутая навигация (по ссылкам, разделам, примечаниям, изображениям)
  \item Проверка орфографии любого количества языков в одном документе
  \item Вложенные, скрытые, защищенные паролем индивидуально оформленные разделы
  \item Перекрестные вычисления между разными таблицами в одном документе
  \item Несколько оглавлений в одном документе.
  \item Автоматизированное перемещение элементов оглавления и списков с подпунктами.
\end{itemize}

\paragraph{Microsoft Word}
\begin{itemize}[noitemsep]
  \item Встроенные средства для продвинутой проверки грамматики русского языка
  \item Распознавание голоса и рукописного ввода
  \item Подробная справочная система с примерами
  \item Широкая распространенность
\end{itemize}

Если сравнивать \emph{производительность} OpenOffice Writer и Microsoft Word, то Writer уступает приблизительно в два раза. \\
Если сравнивать \emph{безопасность} OpenOffice и Microsoft Office, то Microsoft намного надежнее, чем OpenOffice (MS Office 2010: 17 крахов и 0 потенциальных уязвимостей, В OpenOffice 3.2.1: 163 краха и 18 потенциальных уязвимостей). Одна из причин - разработкой свободного ПО занимаются любители.

\subsubsection{Концепция стилей и шаблонов}
\label{subsubsec:template-concept}

\begin{itemize}[noitemsep]
  \item 1-я ошибка --- форматирование вручную без стилей.
  \item 2-я ошибка --- создание оформления вместо создания структуры.
  \item При подготовке документа главное то, чем текст является. А как он выглядит - вторично.
  \item Забыть про ``размер шрифта 14pt'', ``гарнитура Times New Roman'', ``расположение по центру'' и так далее.
  \item Помнить только стили: ``Заголовок'', ``Заголовок $n$-ого уровня'', ``основной текст'', и так далее.
  \item Создание нового документа начинается с продумывания структуры документа и создания системы стилей.
  \item Как будет выглядеть конечный документ, (шрифты, гарнитура, и так далее) решается, когда документ уже готов, путем изменения соответствующего стиля.
\end{itemize}

% TODO: Для чего эта секция?
\subsubsection{Панграммы}
\label{subsubsec:pangrams}

\textbf{Панграмма} (греч. \emph{``все буквы''}) или разнобуквица --- текст, использующий все или почти все буквы алфавита. \\
Используется для:

\begin{itemize}[noitemsep]
  \item Демонстрация шрифтов.
  \item Проверки передачи текста по линиям связи.
  \item Тестирование печатающих устройств.
\end{itemize}

\begin{description}
  \item [Microsoft] Съешь [же] ещё этих мягких французских булок, да выпей чаю.
  \item [KDE] Широкая электрификация южных губерний даст мощный толчок подъёму сельского хозяйства.
  \item [Gnome] В чащах юга жил бы цитрус? Да, но фальшивый экземпляр!
\end{description}

\subsubsection{Автозаполнение}
\label{subsubsec:autocomplete}

\textbf{Lorem ipsum} --- название классического текста-"рыбы". \\
\textbf{"Рыба"} --- слово из жаргона дизайнеров, обозначает условный, зачастую бессмысленный текст, вставляемый в макет страницы.

Lorem ipsum представляет собой искаженный отрывок из философского трактата Цицерона "О пределах добра и зла", написанного в 45 году до нашей эры на латинском языке. 
Впервые этот текст был применен для набора шрифтовых образцов неизвестным печатником в XVI веке.

% TODO: Не понимаю к чму тут формула
$$=rand(m, n)$$
Где:
\\$m$ – количество абзацев;
\\$n$ – количество предложений в каждом абзаце;
\\Так же
$$=lorem(m, n)$$


\subsubsection{Табличный процессор}
\label{subsubsec:spreadsheets}

Табличный процессор офисного ПО обладает многими очень полезными функциями, которые заметно упрощают работу с электронными таблицами:
\begin{itemize}[noitemsep]
  \item Запрет на ввод некорректных значений в ячейку.
  \item Условное форматирование.
  \item Фильтры для заполненных таблиц.
  \item Расчет доверительного интервала.
  \item Подбор параметра (решение уравнений, имеющих только единственное решение).
\end{itemize}
