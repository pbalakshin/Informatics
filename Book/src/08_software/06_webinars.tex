% !TeX root = ../../main.tex
\subsection{Вебинары}
\textbf{Вебинар (онлайн-семинар)} --- разновидность веб-конференции, проведение онлайн-встреч или презентаций через Интернет. Во время веб-конференции каждый из участников находится у своего компьютера, а связь между ними поддерживается через Интернет посредством загружаемого приложения, установленного на компьютере каждого участника, или через веб-приложение.

\par\medskip\noindent
1988~г. --- появление первых IRC (англ. Internet Relay Chat) \\
Середина 1990-х --- появление и распространение IM (Instant Messaging). \\
1998~г.--- регистрация торгового знака ``Webinar'' Эриком Р. Корбом (Eric R. Korb).

Существуют следующие приложения для вебинаров:
\begin{itemize}[noitemsep]
  \item GoToMeeting
    \begin{itemize}[noitemsep]
    \item Создана в 2004 году компанией Citrix Online.
    \item Поддерживаемые ОС: Macintosh, Microsoft Windows
    \item \url{http://www.gotomeeting.com}
  \end{itemize}
  \item StartMeeting
    \begin{itemize}[noitemsep]
    \item Создана в 2011 году как start-up.
    \item Поддерживаемые ОС: Microsoft Windows
    \item \url{http://www.startmeeting.com}
  \end{itemize}
  \item Team Viewer
    \begin{itemize}[noitemsep]
    \item Создана в 2005 году.
    \item Поддерживаемые ОС: Windows, Mac OS, Linux, Android, Apple iOS, Windows Phone
    \item \url{http://www.teamviewer.com}
  \end{itemize}
\end{itemize}
