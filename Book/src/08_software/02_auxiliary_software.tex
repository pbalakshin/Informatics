\subsection{Вспомогательное ПО для программирования}
\label{subsec:auxiliary-software-for-programming}

\begin{itemize}[noitemsep]
  \item Автоматическое создание документации для программы (doxygen).
  \item Контроль версий (SVN, Git, Mercurial).
  \item Управления жизненным циклом найденных ошибок (bug tracking system).
  \item Автоматизированное тестирование кода и функциональности.
\end{itemize}

\subsubsection{Автоматизированное создание документации}
Самая известная система для автоматизации создания документации программного обеспечения на С/С++ --- это \emph{doxygen}. 
Используется в KDE, IBM, AbiWord, Adobe, DC++, Qt, \dots \\
При работе с Doxygen размечается код (в комментариях появляется собственный синтаксис, в результате чего комментарии преобразовываются в документацию).

Такие программы значительно облегчают работу: не нужно отдельно создавать документацию. Более того, в комментариях все подробно описано и любой программист сможет разобраться в программе.

\subsubsection{Системы управления (контроля) версиями}
\label{subsubsec:git}

\begin{itemize}[noitemsep]
    \item \textbf{Клиент-серверные (централизованные):} CVS, Subversion, Microsoft SourceSafe, Perforce, VSS
    \item \textbf{Распределенные:} Mercurial, git
\end{itemize}
  
\subparagraph{Принцип работы:} пометка версий, которые отдаются пользователю, выкладываются на сайт (release версий) и версий для разработчиков, в которую возможно внести изменения (кторые пользователю не отдаются). Это необходимо для того, чтобы редактировать новые версии (вносить изменения, тестировать), и, при необходимости, была возможность откатиться на старую версию программы

\subparagraph{Преимущества Git над SVN:} удобная работа с большим количеством веток, хранение всей истории изменения файлов проекта. Система управления хранит все предыдущие версии.

\subsubsection{Жизненный цикл обнаруженной ошибки}
\begin{itemize}[noitemsep]
  \item \textbf{Тестировщик} находит ошибки;
  \item \textbf{Менеджер проекта} назначает того, кто исправит ошибку;
  \item \textbf{Программист} исправляет или объясняет, почему нельзя исправить (дубль; нет смысла исправлять; нельзя воспроизвести);
  \item \textbf{Тестировщик} проверяет, была ли исправлена ошибка.
\end{itemize}

\subsubsection{Тестирование программного обеспечения}
Самые известные СУБД ошибок: JIRA, Redmine, Bugzilla, TrackGear.

\paragraph{Описание ошибки}
\begin{itemize}[noitemsep]
  \item кто сообщил об ошибке;
  \item дата и время обнаружения;
  \item серьезность ошибки;
  \item перечень шагов воспроизведения ошибки;
  \item текущий статус ошибки.
\end{itemize}

\textbf{Автоматизированное тестирование программного обеспечения} --- часть процесса тестирования на этапе контроля качества в процессе разработки программного обеспечения.\\
Оно использует программные средства для выполнения тестов и проверки результатов выполнения, что помогает сократить время тестирования и упростить его процесс.

\paragraph{Наиболее известный инструментарий для тестирования:}
\begin{itemize}[noitemsep]
  \item JUnit --- тестирование приложений для Java
  \item NUnit --- порт JUnit под .NET
  \item xUnit --- тестирование приложений для .NET
  \item TestNG --- тестирование приложений для Java
  \item Selenium --- тестирование приложений HTML
  \item WatiN --- тестирование веб-приложений
  \item TOSCA Testsuite --- тестирование приложений HTML, .NET, Java, SAP
  \item UniTESK --- тестирование приложений на Java, Си.
\end{itemize}
