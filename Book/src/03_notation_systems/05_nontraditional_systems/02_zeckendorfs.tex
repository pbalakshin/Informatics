% !TeX root = ../../../main.tex
\subsubsection{Система счисления Цекендорфа}\label{subsubsec:zeckendorfs-notation-system}
Любое натуральное число можно представить в виде

\[
    X = \sum^{n}_{k = 1} d_{k} \times F_{k} \quad \text{где  } d_{k} \in \{0,1\},\text{ а } F_{k} \text{ --- числа Фибоначчи}
\]
Каждое число Фибоначчи есть сумма двух предыдущих чисел: \\
$F_{k} = \{1, 1, 2, 3, 5, 8, 13, 21, \ldots\}$.

В записи чисел в системе счисления Цекендорфа первая единица из ряда чисел Фибоначчи \textbf{не используется} (т.к. первая единица это $F_{0}$).

Запись числа в системе счисления Цекендорфа будет иметь вид
\[
    X_{\text{ц}} = d_{n}d_{n-1} \ldots d_{1}
\]

В записи чисел в системе счисления Цекендорфа \textbf{не допускается использование двух единиц подряд}.

\paragraph{Перевод из системы счисления Цекендорфа в десятичную}\label{par:zeckendorfs-to-decimal}

Алгоритм перевода из системы счисления Цекендорфа в десятичную очень похож на алгоритм перевода из системы счисления с основанием $N$ в десятичную (раздел~\ref{subsubsec:notation-conversion-n-to-10}):

\[
    X_{10} = d_{n} \times F_{n} + d_{n-1}\times F_{n-1} + d_{n-2} \times F_{n-2} + \ldots + d_{2} \times F_{2} + d_{1} \times F_{1},
\]

\noindentгде:
\begin{description}
    \item [$X_{10}$] --- искомое число в десятичной системе счисления;
    \item [$d_{i}$] --- натуральные числа меньше или равные $i$;
    \item [$n$] --- количество разрядов исходного числа.
\end{description}

\example{1}

\task{} перевести число $100101_{\text{ц}}$ в десятичную систему счисления
\solution{} $100101_{\text{ц}} = 1 \times 13 + 0 \times 8 + 0 \times 5 + 1 \times 3 + 0 \times 2 + 1 \times 1 = \allowbreak 13 + 3 + 1 = \allowbreak 17_{10}$
\answer{} $100101_{\text{ц}} = 17_{10}$

\paragraph{Перевод из десятичной системы счисления в систему счисления Цекендорфа}\label{par:decimal-to-zeckendorfs}

Для перевода воспользуемся все той же формулой:
\[
    X_{10} = d_{n} \times F_{n} + d_{n-1} \times F_{n-1} + d_{n-2} \times F_{n-2} + \ldots + d_{2} \times F_{2} + d_{1} \times F_{1}
\]

\begin{enumerate}
    \item Находим число $F_{k}$ в ряду чисел Фибоначчи, которое больше $X_{10}$, но ближе всего к нему. Тогда $n = k - 1$, где $n$ - количество разрядов искомого числа в системе Цекендорфа.
    \item Записываем с уже полученным $n$. \[X_{10} = d_{n} \times F_{n} + d_{n-1} \times F_{n-1} + d_{n-2} \times F_{n-2} + ... + d_{2}\times F_{2} + d_{1} \times F_{1}\]
    \item Начиная с $d_{n}$ с помощью ума и смекалки начинаем подбирать коэффициенты, помня, что $d_{1} \in \{0,1\}$ и что две единицы не могут стоять рядом.
\end{enumerate}

\example{2}

\task{} перевести число $19_{10}$ в систему счисления Цекендорфа

\solution{} $19_{10} < 21 (21 = F_{7})$, значит количество разрядов равно $7 - 1 = 6$. \\
Запишем формулу для $n = 6$: \\
$d_{6}\times 13 + d_{5}\times 8 + d_{4}\times 5 + d_{3}\times 3 + d_{2}\times 2 + d_{1}\times 1$ \\
Подберем коэффициенты: \\
$19_{10} = 1 \times 13 + 0 \times 8 + 1 \times 5 + 0 \times 3 + 0 \times 2 + 1\times 1 = \allowbreak 101001_{\text{ц}}
$

\answer{} $19_{10} = 101001_{\text{ц}}$

\textbf{Применение системы счисления Цекендорфа:} кодирование данных с маркером завершения \enquote{11} , у некоторых народов в сельском хозяйстве --- минимизация необходимого числа зерен.
