% !TeX root = ../../../main.tex
\subsubsection{Система счисления Бергмана}\label{subsubsec:bergmans}

Любое действительное неотрицательное число можно представить в виде
\[
    X = \sum^{\infty}_{k = -\infty} d_{k} \times z^{k} \quad \text{где } d_{k} \in \{0,1\}, \text{ а } z = \frac{1+\sqrt{5}}{2}
\]
Число $z$ - число золотой пропорции.

Запись числа в системе счисления Бергмана будет иметь вид: \[
    X_{\text{Б}} = d_{n}d_{n-1} \ldots d_{2}d_{1}d_{0},d_{-1}d_{-2} \ldots d_{-m}
\]

Чтобы исключить неоднозначность, используется запись с наибольшим количеством разрядов.

\paragraph{Перевод из системы счисления Бергмана в десятичную}
Алгоритм перевода из системы счисления Бергмана в десятичную очень похож на алгоритм перевода из системы счисления с основанием $N$ в десятичную (\cref{subsubsec:notation-conversion-n-to-10}).
\[
    X_{10} = d_{n}\times z^{n} + d_{n-1}\times z^{n-1} + \ldots + d_{1} \times z^{1} + d_{0} \times z^{0} + d_{-1}\times z^{-1}\ + \ldots + d_{-m}\times z^{-m}
\]

\noindentГде:
\begin{description}[noitemsep]
    \item [$X_{10}$] --- искомое число в десятичной системе счисления;
    \item [$d_{i}$] --- число, принимающее значение 0 или 1;
    \item [$n$] --- количество разрядов целой части
    \item [$m$] ---  количество разрядов целой части
\end{description}

\example{1}

\medskip
\task{} Перевести число $100,01_{\text{Б}}$ в десятичную систему счисления
\solution{} Пусть $\displaystyle X = \frac{1+\sqrt{5}}{2}$, тогда

\begin{flalign*}
    100,01_{\text{Б}} & = 1 \times X^2 + 0 \times X^1 + 0 \times X^0 + 0 \times X^{-1} + 1 \times X^{-2}
        = X^2 + X^{-2} && \\
        & = (\frac{1+\sqrt{5}}{2})^{2} + (\frac{1+\sqrt{5}}{2})^{-2}
        = \frac{6 + 2\sqrt{5}}{4} + \frac{4}{6 + 2\sqrt{5}} && \\
        & = \frac{9 + 3\sqrt{5}}{3 + \sqrt{5}}
        = \frac{3(3 + \sqrt{5})}{3 + \sqrt{5}}
        = 3_{10} &&
\end{flalign*}

\answer{} $100,01_{\text{Б}} = 3_{10}$

\paragraph{Перевод чисел из десятичной системы в систему счисления Бергмана происходит методом подбора}
\textbf{Применение системы счисления Бергмана:} запись иррациональных чисел конечным числом цифр, контроль арифметических операций, коррекция ошибок, самосинхронизация кодовых последовательностей при передаче по каналу связи.
