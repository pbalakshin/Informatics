\subsubsection{Симметричная система счисления}
\label{subsubsec:symmetic_system}

Это системы счисления с отрицательными числами.
Центром симметрии является 0, поэтому основанием для таких систем счисления могут быть только нечетные числа.
Главная особенность таких систем, как и нега-позиционных, в том, что не нужно никаких специальных знаков для обозначения отрицательных чисел.

\bigskip
\textbf{Перевод чисел из симметричных систем счисления и полностью описан в разделе~\ref{subsubsec:notation_conversion_n_to_10} Обратный перевод происходит методом подбора.}

\bigskip
\emph{Примеры в симметричной пятеричной системе счисления:}
Если в обычной пятеричной системе используются цифры \{0, 1, 2, 3, 4\}, то в симметричной пятеричной системе используются \{-2, -1, 0, 1, 2\}.
\begin{flalign*}
10\overline{2}\overline{1}2_{5\mbox{С}} & = 1 \times 5^4 + 0 \times 5^3 + (-2) \times 5^2 + (-1) \times 5^1 + 2 \times 5^0  \\
                                        & = 1\times 625 + 0\times 125 - 2\times 25 - 1\times 5 + 2\times 1 \\
                                        & = 625 - 50 - 5 + 2 \\
                                        & = 572_{10}
\end{flalign*}

\begin{flalign*}
\overline{1}021\overline{2}_{5\mbox{С}} & = (-1) \times 5^4 + 0\times 5^3 + 2\times 5^2 + 1\times 5^1 + (-2)\times 5^0 \\
                                        & = - 1\times 625 + 0\times 125 + 2\times 25 + 1 \times 5 - 2\times 1 \\
                                        & = - 625 + 50 + 5 - 2 \\
                                        & = -572_{10}
\end{flalign*}

Стоит обратить внимание, что цифры с чертой сверху --- отрицательные.
