% !TeX root = ../../../main.tex
\subsubsection{Факториальная система счисления}
\label{subsubsec:factorial_system}

Любое натуральное число можно представить в виде

\[
    X = \sum^{n}_{k = 1} d_{k}\times k! \quad \text{где  } 0 \leqslant d_{k} \leqslant k
\]

В основании факториальной системы счисления используется факториал.
Запись числа в факториальной системе счисления будет иметь вид:
\[
    X_{\text{ф}} = d_{n} d_{n-1} \ldots d_{1}
\]

\paragraph{Перевод из факториальной системы счисления в десятичную}

Алгоритм перевода из факториальной системы счисления в десятичную очень похож на алгоритм перевода из системы счисления с основанием $N$ в десятичную (раздел~\ref{subsubsec:notation_conversion_n_to_10}).

\[
    X_{10} = d_{n} \times n! + d_{n-1} \times (n-1)! + d_{n-2} \times (n-2)! + \ldots + d_{2}\times 2! + d_{1}\times 1!
\]

\noindent
Где:
\begin{description}[noitemsep]
    \item [$X_{10}$] --- искомое число в десятичной системе счисления;
    \item [$d_{i}$] --- натуральные числа меньше или равные $i$;
    \item [$n$] --- количество разрядов исходного числа
\end{description}

\example{1}
\task{} перевести число $221_{\text{ф}}$ в десятичную систему счисления
\solution{}
\begin{flalign*}
221_{\text{ф}} & = 2 \times 3! + 2\times 2! + 1\times 1! \\
               & = 2\times 6 + 2\times 2 + 1\times 1 \\
               & = 12 + 4 + 1 \\
               & = 17_{10}
\end{flalign*}

\answer{}  $221_{\text{ф}} = 17_{10}$

\paragraph{Перевод из десятичной системы счисления в факториальную}

Для перевода воспользуемся все той же формулой:

\[
    X_{10} = d_{n} \times n! + d_{n-1} \times (n-1)! + d_{n-2} \times (n-2)! + \ldots + d_{2}\times 2! + d_{1}\times 1!
\]

\noindent
Стоит обратить внимание, что $0 \leqslant d_{1}\leqslant 1$; $0 \leqslant d_{2}\leqslant 2$ и так далее.

\begin{enumerate}
    \item Находим факториал $k!$, значение которого больше $X_{10}$, но ближе всего к нему. Тогда $n = k - 1$, где $n$ - количество разрядов искомого числа в факториальной системе.
    \item Записываем \[ X_{10} = d_{n}\times n! + d_{n-1}\times (n-1)! + d_{n-2}\times (n-2)! + ... + d_{2}\times 2! + d_{1}\times 1!\] с уже полученным $n$.
    \item Начиная с $d_{n}$ с помощью ума и смекалки начинаем подбирать коэффициенты, помня, что $d_{1} \in \{0,1\}$, $d_{2} \in \{0,1,2\}$ и т.д.
\end{enumerate}

\example{2}
\task{} перевести число $54_{10}$ в факториальную систему счисления
\solution{} $54_{10} < 5! (5! = 120)$, значит количество разрядов равно $5 - 1 = 4$.

Запишем формулу для $n = 4$: $54_{10} = d_{4}\times 4! + d_{3}\times 3! + d_{2} \times 2! + d_{1}\times 1!$

Подберем коэффициенты: $54_{10} = 2\times 4! + 1\times 3! + 0\times 2! + 0 \times 1!$

\answer{}  $54_{10} = 2100_{\text{ф}}$

\noindent
\textbf{Применение факториальной системы счисления:} декодирование и кодирование перестановок.

\example{3}

\noindent
\begin{minipage}[t]{.75\linewidth}
    \flushleft%
    \task{} Имеется $n = 5$ чисел (1, 2, 3, 4, 5), нужно найти все их перестановки. Известно, что существует $n! = 5! = 120$ таких перестановок. Найти перестановку, если известен ее номер $k = 52$.

    \solution{} Переведем $k$ в факториальную систему: $52_{10} = 2 \times 4! + 0 \times 3! + 2\times 2! + 0 \times 1! = 2020_{\text{ф}}$
    Дополним результат до $n - 1$ разрядов (при необходимости), расставим символы по местам:

    \bigskip
    \begin{tabular}{rll}
        1. & Справа от $5$ есть $2$ меньшие цифры & \code{( - - 5 - - )}; \\
        2. & Справа от $4$ есть $0$ меньших цифр  & \code{( - - 5 - 4 )}; \\
        3. & Справа от $3$ есть $2$ меньшие цифры & \code{( 3 - 5 - 4 )}; \\
        4. & Справа от $2$ есть $0$ меньших цифр  & \code{( 3 - 5 2 4 )}; \\
    \end{tabular}

    \bigskip
    \begin{tabular}{rl|ccccc|}
        \cline{3-7}
        1. & Справа от $5$ есть $2$ меньшие цифры: & - & - & 5 & - & - \\ \cline{3-7}
        2. & Справа от $4$ есть $0$ меньших цифр:  & - & - & 5 & - & 4 \\ \cline{3-7}
        3. & Справа от $3$ есть $2$ меньшие цифры: & 3 & - & 5 & - & 4 \\ \cline{3-7}
        4. & Справа от $2$ есть $0$ меньших цифр:  & 3 & - & 5 & 2 & 4 \\ \cline{3-7}
        \cline{3-7}
    \end{tabular}

    \bigskip
    \begin{enumerate}[noitemsep]
        \item Справа от $5$ есть $2$ меньшие цифры \ $(- - 5 - -)$;
        \item Справа от $4$ есть $0$ меньших цифр \ \ $(-\ - 5\ -\ 4)$;
        \item Справа от $3$ есть $2$ меньшие цифры \  $(3\ -\ 5\ - 4)$;
        \item Справа от $2$ есть $0$ меньших цифр \ \ $(3\ -\ 5\ \ 2\ \ 4)$;
    \end{enumerate}

    \answer{} (3 1 5 2 4)
\end{minipage}
\hfill
\noindent
\begin{minipage}[t]{.2\linewidth}
    \vspace{0pt}
    \flushright%
    \begin{tabular}{|c|c|}
        \hline
        0   & 12345 \\
        1   & ????? \\
        ... & ..... \\
        52  & ????? \\
        ... & ..... \\
        119 & 54321 \\
        \hline
    \end{tabular}
\end{minipage}
