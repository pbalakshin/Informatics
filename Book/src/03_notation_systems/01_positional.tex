\subsection{Позиционная система счисления}
\label{subsec:positional_notation_systems}

Рассмотрим формулу записи числа в позиционной системе счисления:

\noindent
$$X_{(q)} = x_{n-1} \times q^{n-1} + x_{n-2} \times q^{n-2} + \ldots + x_{1} \times q^{1} + x_{0}\times q^{0} + x_{-1} \times q^{-1} + \ldots + x_{-m} \times q^{-m}$$

Или
\[
X_{(q)} = \sum_{i=-m}^{n-1} x_{i} \times q^{i}
\]

\noindent
Где:

\begin{description}[noitemsep]
    \item [$X_{(q)}$] --- запись числа в системе счисления с основанием $q$
    \item [$x_{i}$] --- натуральные числа меньше $q$, то есть цифры
    \item [$n$] --- число разрядов целой части
    \item [$m$] --- число разрядов дробной части
    \item [$q$] --- показатель системы счисления
\end{description}

\noindentСамо число $X_{(q)}$ имеет следующий вид: 
$X_{(q)} = x_{n-1}x_{n-2} \ldots x_{1}x_{0}x_{-1} \ldots x_{1-m}x_{-m}$


\noindent
Рассмотрим данную формулу на примере:

$$123,45_{10} = 1\times 10^{2} + 2\times 10^{1} + 3\times 10^{0} + 4\times 10^{-1} + 5\times 10^{-2}$$

\noindent
Мы разложили число $123,45$ по этой формуле. В данном случае $q$ = 10, $n = 3$, $m = 2$, $X_{(q)} = 123,45$, а $x_{3-1} = 1$ ($x_{2} = 1$), $x_{1} = 2$ и так далее.

В позиционной системе счисления важную роль имеет порядок цифр, то есть значение каждого числового знака (цифры) в записи числа зависит от его позиции (разряда).
