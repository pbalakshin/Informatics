\subsection{Перевод чисел из одной системы счисления в другую}
\label{subsec:notations_conversion}

Существуют три способа перевода из одной системы счисления в другую:

\begin{enumerate}
    \item Из десятичной системы счисления в систему счисления с основанием $N$
    \item Из системы счисления с основанием $N$ в десятичную систему счисления
    \item Из системы счисления с основанием $N$ в систему счисления с основанием $N^{k}$ и обратно, при условии $k \in \mathbb{N}$
\end{enumerate}

\subsubsection{Перевод числа из десятичной системы счисления в систему счисления с основанием \texorpdfstring{$N$}{N}}
\label{subsubsec:notation_conversion_10_to_n}

Чтобы перевести дробное число в систему счисления с основанием N необходимо разделить его на две части: целую и дробную, и каждую часть переводить отдельно.

\paragraph{Преобразования целой части числа}
Для перевода целой части числа из десятичной системы счисления в другую необходимо:

\begin{enumerate}
    \item Разделить целую часть десятичного числа на основание новой системы счисления;
    \item Записать остаток деления;
    \item Разделить получившийся результат деления (п.1) на основание новой системы счисления (при необходимости);
    \item Записать остаток деления;
    \item Повторять, пока целая часть десятичного числа не будет равна 0.
\end{enumerate}

Получившиеся в ходе деления остатки и есть цифры искомого числа в новой системе счисления. Записать остатки в обратном порядке (начиная с последнего полученного). Стоит заметить, что в данном способе очень удобно применять деление столбиком.

\bigskip

\example{1}
\task Перевести число $45_{10}$ в троичную систему счисления.
\solution Последовательно разделим $45_{10}$ на $3$, записывая остатки:

\begin{figure}[H]
    \includegraphics[scale=0.5]{3.2.1-example1}
\end{figure}

\begin{figure}[H]
\centering
\includegraphics[width=4cm]{1_2_1(1)}
\end{figure}

Полученные остатки: 0, 0, 2, 1. Записываем их в обратном порядке.

\answer $45_{10} = 1200_{3}$

\paragraph{Преобразования дробной части числа}
Для перевода дробной части числа из десятичной системы счисления в другую необходимо:
\begin{enumerate}
    \item Умножить дробную часть десятичного числа на основание новой системы счисления;
    \item Отделить и записать целую часть;
    \item Умножить дробную часть результата умножения (п.1) на основание новой системы счисления (при необходимости);
    \item Отделить и записать целую часть;
    \item Повторять, пока дробная часть десятичного числа не будет равна 0.
\end{enumerate}

Получившиеся в ходе умножения целые части и есть цифры искомого числа в новой системе счисления.
Записать целые части в прямом порядке (начиная с первого полученного). Первая записанная целая часть (0) идет в целую часть нового числа, а в дробную записываются полученные целые части, начиная со второй.
Стоит заметить, что в данном способе очень удобно применять умножение столбиком.

\bigskip
\example{2}
\task перевести число $0,625_{10}$ в четверичную систему счисления.
\solution умножим дробную часть $0,625_{10}$ на $4$, записывая целые части, пока не получим в дробной части 0:

\begin{figure}[H]
    \includegraphics[scale=0.5]{3.2.1-example2}
\end{figure}

\begin{figure}[H]
\centering
\includegraphics[width=1.7cm]{1_2_1(2)}
\end{figure}

\explain сначала умножаем 0,625 на 4, получаем 2,5. 2 записываем в целые части, а далее используем дробную часть - 0,5. Умножаем 0,5 на 4, получаем 2, записываем в целые части. Так как дробная часть равна 0, то перевод окончен.
Полученные целые части: 0, 2, 2. Первая полученная целая часть (0) идет в целую часть нового числа. Остальные (2, 2) в дробную часть.

\answer $0,625_{10} = 0,22_{4}$

\bigskip

\example{3}
\task перевести число $43,52_{10}$ в пятеричную систему счисления.
\solution разделим $43,52_{10}$ на две части: целую ($43_{10}$) и дробную ($0,52_{10}$). Переведем целую и дробную части по отдельности:

\begin{figure}[H]
    \includegraphics[scale=0.5]{3.2.1-example3}
\end{figure}

\begin{figure}[H]
\centering
\includegraphics[width=6cm]{1_2_1(3)}
\end{figure}

Полученные остатки (3, 1, 1) запишем в обратном порядке: $43_{10} = 113_{5}$.
Полученные целые части (0, 2, 3) запишем в прямом порядке: $0,52_{10} = 0,23_{5}$.
Объеденим полученные части

\answer $43,52_{10} = 113,23_{5}$

\subsubsection{Перевод числа из системы счисления с основанием \texorpdfstring{$N$}{N} в десятичную систему счисления}
\label{subsubsec:notation_conversion_n_to_10}

Формула перевода числа из системы счисления с основанием N в десятичную систему счисления это практически формула записи числа в позиционной системе счисления.

\[
X_{(10)} = \sum_{i=-m}^{n-1} x_{i} \times q^{i}
\]

\noindentГде:
\begin{description}[noitemsep]
    \item [$X_{(10)}$] --- запись числа в системе счисления с основанием $q$
    \item [$x_{i}$] --- натуральные числа меньше $q$, то есть цифры
    \item [$n$] --- число разрядов целой части
    \item [$m$] --- число разрядов дробной части
    \item [$q$] --- показатель системы счисления
\end{description}

\example{1}
\task Перевести число $1101,111_{2}$ в десятичную систему счисления.
\solution
% $1101,111_{2} = 1 \times 2^{3} + 1\times 2^{2} + 0\times 2^{1} + 1\times 2^{0} + 1\times 2^{-1} + 1\hm\times 2^{-2}\hm + 1\hm\times 2^{-3}\hm = 1\times 8 + 1\times 4 + 0\times 2 + 1\times 1 + 1\times 0,5 + 1\times 0,25 + 1\times 0,125\hm = 8 + 4 + 1 + 0,5 + 0,25 + 0,125 = 13,875_{10} $
\begin{flalign*}
1101,111_{2} & = 1 \times 2^{3} + 1 \times 2^{2} + 0 \times 2^{1} + 1 \times 2^{0} + 1 \times 2^{-1} +  1 \times 2^{-2} + 1 \times 2^{-3} \\
             & = 1 \times 8 + 1 \times 4 + 0 \times 2 + 1 \times 1 + 1 \times 0,5 + 1 \times 0,25 + 1 \times 0,125 \\
             & = 8 + 4 + 1 + 0,5 + 0,25 + 0,125 \\
             & = 13,875_{10}
\end{flalign*}

\answer $1101,111_{2} = 13,875_{10}$

\subsubsection{Перевод числа из системы счисления с основанием \texorpdfstring{$N$}{N} в систему счисления с основанием \texorpdfstring{$N^{k}$}{Nk} и обратно, при условии \texorpdfstring{$k \in \mathbb{N}$}{k in N}}
\label{subsubsec:notation_conversion_n_to_nk}

Если основание системы счисления первого числа является степенью основания системы счисления второго числа ($N = N^{k}$), при условии $k \in \mathbb{N}$, то можно использовать следующий алгоритм.

\paragraph{Преобразование $N \to N^{k}$}
\begin{enumerate}
    \item Дополнить число (записанное в системе счисления $N$) незначащими нулями так, чтобы количество цифр было кратно $k$ (если число дробное, то дополнить так, чтобы и в целой и в дробной частях количество цифр было кратно $k$).
    \item Разбить это число на группы по $k$ цифр, начиная от нуля (если число дробное, то целую часть разбивать, начиная от запятой в левую сторону, а дробную часть, начиная от запятой в правую сторону).
    \item Заменить каждую такую группу эквивалентным числом, записанным в системе $N^{k}$.
\end{enumerate}

\paragraph{Преобразование $N^{k} \to N$}

\begin{enumerate}
    \item Заменить каждую цифру числа, записанного в системе счисления $N^{k}$, эквивалентным набором из $k$ цифр системы счисления $N$.
\end{enumerate}

Рассмотрим данный метод на системах счисления с основанием $N = 2^{k}$. Для этого воспользуемся таблицей~\ref{tab:2k_bases}.

\begin{table}[H]
    \caption{Основания вида $2^{k}$}
    \label{tab:2k_bases}
    \centering
    \begin{tabular}{|c|c|c|c|}
    \hline
    Десятичная & Двоичная & Восмеричная & Шестнадцатиричная
    \\\hline
    0 & 0000 & 00 & 0 \\
    1 & 0001 & 01 & 1 \\
    2 & 0010 & 02 & 2 \\
    3 & 0011 & 03 & 3 \\
    4 & 0100 & 04 & 4 \\
    5 & 0101 & 05 & 5 \\
    6 & 0110 & 06 & 6 \\
    7 & 0111 & 07 & 7 \\
    8 & 1000 & 10 & 8 \\
    9 & 1001 & 11 & 9 \\
    10 & 1010 & 12 & A \\
    11 & 1011 & 13 & B \\
    12 & 1100 & 14 & C \\
    13 & 1101 & 15 & D \\
    14 & 1110 & 16 & E \\
    15 & 1111 & 17 & F \\
    \hline
    \end{tabular}
\end{table}

\example{1}

\task Пользуясь таблицей перевести число $1542,43_{8}$ в двоичную систему счисления

\solution По таблице находим чему равны цифры исходного числа в двоичной системе. $1_{8} = 001_{2}$, $5_{8} = 101_{2}$ (незначащие нули убираем, так как необходимо, чтобы количество цифр в эквивалентном наборе было равно степени $k$ из выражения $N = N^{k}$, где $N^{k}$ - исходная система счисления. В данном случае $k = 3$) и так далее. Заменяем каждую цифру числа эквивалентным набором.

\answer $1542,43_{8} = 001101100010,100011_{2}$

\example{2}

\task Пользуясь таблицей перевести число $11010,11_{2}$ в шестнадцатиричную систему счисления

 \solution Первым делом, добавим незначащие нули так, чтобы количество цифр было кратно $k$ (в данном случае $k = 4$). Так как число дробное, не забываем добавлять нули и в конце числа. Получим $00011010,1100_{2}$.
Теперь необходимо разбить число на группы по $k$ цифр (начинаем от запятой). Результат: $0001\ 1010\ ,\ 1100\ _{2}$.
Пользуясь таблицей, заменяем группы цифр эквивалентными числами, записанным в шестнадцатиричной системе счисления.

\answer $11010,11_{2} = 1A,C_{16}$.
