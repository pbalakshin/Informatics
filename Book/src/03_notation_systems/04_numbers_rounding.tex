% !TeX root = ../../main.tex
\subsection{Округление чисел}
\label{subsec:numbers_rounding}

Как происходит округление чисел в десятичной системе счисления?
Каждый учил в школе правила округления: 1, 2, 3, 4 округляется в меньшую сторону, а 5, 6, 7, 8 и 9 --- в большую.
Рассмотрим два примера:

\begin{figure}[H]
    \begin{minipage}[t]{.45\linewidth}
        \begin{tabular}{|c|c|c|}
            \multicolumn{3}{l}{Пример 1:} \\
            \hline
            & Число & Округление \\
            \cline{2-3}
            & 5,9 & 6,0 \\
            & 4,1 & 4,0 \\
            & 5,0 & 5,0 \\
            & 6,6 & 7,0 \\
            & 2,4 & 2,0 \\
            \hline
            Cумма & 24,0 & 24,0 \\
            \hline
        \end{tabular}
    \end{minipage}
    \hfill
    \begin{minipage}[t]{.45\linewidth}
        \begin{tabular}{|c|c|c|}
            \multicolumn{3}{l}{Пример 2:} \\
            \hline
            & Число & Округление \\
            \cline{2-3}
            & 5,5 & 6,0 \\
            & 3,5 & 4,0 \\
            & 2,5 & 3,0 \\
            & 8,5 & 9,0 \\
            & 1,5 & 2,0 \\
            \hline
            Cумма & 21,5 & 24,0 \\
            \hline
        \end{tabular}
    \end{minipage}
\end{figure}

В примере 1 мы видим, что сумма до и после округления одинакова.
Так случилось, потому что мы округляли то в большую сторону, то в меньшую, и в итоге количество округлений в большую сторону равно количеству округлений в меньшую.
Округление нам не помешало получить красивую сумму.

А теперь посмотрим на специально подобранный пример 2.
Разница сумм довольно большая.
Так получилось, потому что мы округляли всегда только в большую сторону, хотя мы округляли по правилам.

Если проделать такой эксперимент с большим количеством чисел (тысяча, две тысячи или даже больше), то ошибка будет небольшая, но она будет.

Работая с числами, у которых показатель системы счисления четный, мы натыкаемся на следующую проблему: у нас нечетное количество чисел для округления.
Разберем на примере десятичной системы счисления.
В ней всего 10 цифр - 0, 1, 2, 3, 4, 5, 6, 7, 8, 9.
Числа $X.0$ мы не округляем.
Остается 9 чисел: $X.1$, $X.2$, $X.3$, $X.4$ округляем в меньшую сторону (всего 4 числа), а $X.5$, $X.6$, $X.7$, $X.8$, $X.9$ - в большую (всего 5 чисел).
$X.5$ - середина между $X+1$ и $X$, однако мы округляем в пользу $X+1$, то есть в большую.
Таким образом, при работе с большим количеством чисел, мы всегда будем округлять в большую сторону чаще, чем в меньшую.
Отсюда и ошибка в итоговой сумме.

Чтобы этого избежать, некоторые программы при автоматическом округлении большого количества чисел округляют $X.5$ по очереди то в большую сторону, то в меньшую.

В системах счисления с нечетным основанием такой ошибки нет.
Возьмем, к примеру, пятиричную систему счисления.
Она содержит цифры 0, 1, 2, 3, 4.
Числа $X.0$ мы не округляем, остается 4 числа: $X.1$, $X.2$ округляем в меньшую сторону (всего 2 числа), а $X.3$, $X.4$ - в большую (всего 2 числа).
Таким образом, количество чисел, округленных в меньшую сторону, равно количеству чисел, округленных в большую.
