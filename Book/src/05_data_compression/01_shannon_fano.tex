
\subsubsection{Алгоритм Шеннона-Фано}
\label{subsubsec:shannon-fano-algo}

\begin{figure}[H]
\begin{wrapfigure}{l}{0.25\textwidth}
    \centering
    \vspace{-\intextsep}
    \includegraphics[width=0.25\textwidth]{person/fano-robert}
    \caption*{Роберт Фано\\1917 -- 2016}
\end{wrapfigure}

Алгоритм Шеннона-Фано --- один из первых алгоритмов сжатия.
Его сформулировали два ученых --- Клод Шеннон и Роберт Фано.
Алгоритм основан на частоте повторения.
Так, часто встречающийся символ кодируется кодом меньшей длины, а редко встречающийся --- кодом большей длины.
Коды, полученные при кодировании, префиксные, что позволяет декодировать любую последовательность.

\bigskip

Алгоритм Шеннона-Фано для некоторых последовательностей может сформировать неоптимальные коды.
\end{figure}

\bigskip
\paragraph{Описание алгоритма Шеннона-Фано}
\begin{enumerate}
    \item Символы входного (первичного) алфавита выписывают по убыванию вероятностей --- это корень будущего дерева;
    \item Строится дерево от корня к листьям. Находится середина, которая делит корень на два узла. Эти узлы (суммы вероятностей символов алфавита) примерно равны;
    \item Полученные узлы --- листья дерева. Левому узлу (с большей суммарной вероятностью) присваивается значение $1$, а правому --- $0$;
    \item Шаги 2--3 повторяются, пока в листьях дерева не останется один символ первичного алфавита.
    \item Символ входного (первичного) алфавита кодируется последовательностью нулей и единиц в соответствии с распределением их от корня к листьям (узлам).
\end{enumerate}

\example{1}
\task составить код Шеннона-Фано для последовательности $AAABCCCCCDEEEF$. Найти среднюю длину кодового слова.
\solution в последовательности $AAABCCCCCDEEEF$ алфавит состоит из 6 символов: A, B, C, D, E, F. Выпишем символы первичного алфавита по убыванию вероятностей:
\begin{table}[H]
\centering
\begin{tabular}{|r|c|c|c|c|c|c|}
\hline
\textbf{Символ} & C & A & E & B & D & F \\ \hline
\textbf{Вероятность} & $^5/_{14}$ & $^3/_{14}$ & $^3/_{14}$ & $^1/_{14}$ & $^1/_{14}$ & $^1/_{14}$ \\ \hline
\end{tabular}
\end{table}

\noindent Полученная последовательность $CAEDBF$ является корнем будущего дерева.

\noindent Построим дерево от корня к листьям:

\begin{table}[H]
    \centering
    \begin{tabular}{cccccc}
        \multicolumn{6}{c}{CAEBDF ($^5/_{14} + ^3/_{14} + ^3/_{14} + ^1/_{14} + ^1/_{14} + ^1/_{14}$)} \\
        % \multicolumn{6}{c}{CAEBDF ($\frac{5}{14} + \frac{3}{14} + \frac{3}{14} + \frac{1}{14} + \frac{1}{14} + \frac{1}{14}$)} \\
        \multicolumn{2}{c}{$\swarrow$} & \multicolumn{4}{c}{$\searrow$} \\
        \multicolumn{2}{c}{CA ($^5/_{14} + ^3/_{14}$)} & \multicolumn{4}{c}{EBDF ($^3/_{14} + ^1/_{14} + ^1/_{14} + ^1/_{14}$)} \\
        % \multicolumn{2}{c}{CA ($\frac{5}{14} + \frac{3}{14}$)} & \multicolumn{4}{c}{EBDF ($\frac{3}{14} + \frac{1}{14} + \frac{1}{14} + \frac{1}{14}$)} \\
        $\swarrow$ & $\searrow$ & \multicolumn{2}{c}{$\swarrow$} & \multicolumn{2}{c}{$\searrow$} \\
        C ($^5/_{14}$) & A ($^3/_{14}$) & \multicolumn{2}{c}{EB ($^3/_{14} + ^1/_{14}$)} & \multicolumn{2}{c}{DF ($^1/_{14} + ^1/_{14}$)} \\
         & & $\swarrow$ & $\searrow$ & $\swarrow$ & $\searrow$ \\
         & & E ($^3/_{14}$) & B ($^1/_{14}$) & D ($^1/_{14}$) & F($^1/_{14}$) \\
    \end{tabular}
\end{table}

\noindentПрисвоим левому символу (с большей вероятностью) значение $1$, а правому --- $0$:

\begin{table}[H]
    \centering
    \begin{tabular}{cccccc}
        \multicolumn{6}{c}{CAEBDF ($^5/_{14} + ^3/_{14} + ^3/_{14} + ^1/_{14} + ^1/_{14} + ^1/_{14}$)} \\
        \multicolumn{2}{c}{\textbf{[1]} $\swarrow$} & \multicolumn{4}{c}{$\searrow$ \textbf{[0]}} \\
        \multicolumn{2}{c}{CA ($^5/_{14} + ^3/_{14}$)} & \multicolumn{4}{c}{EBDF ($^3/_{14} + ^1/_{14} + ^1/_{14} + ^1/_{14}$)} \\
        \textbf{[1]} $\swarrow$ & $\searrow$ \textbf{[0]} & \multicolumn{2}{c}{\textbf{[1]} $\swarrow$} & \multicolumn{2}{c}{$\searrow$ \textbf{[0]}} \\
        C ($^5/_{14}$) & A ($^3/_{14}$) & \multicolumn{2}{c}{EB ($^3/_{14} + ^1/_{14}$)} & \multicolumn{2}{c}{DF ($^1/_{14} + ^1/_{14}$)} \\
         & & \textbf{[1]} $\swarrow$ & $\searrow$ \textbf{[0]} & \textbf{[1]} $\swarrow$ & $\searrow$ \textbf{[0]} \\
         & & E ($^3/_{14}$) & B ($^1/_{14}$) & D ($^1/_{14}$) & F($^1/_{14}$) \\
    \end{tabular}
\end{table}

\noindentПолучим следующую таблицу для кодировки:

\begin{table}[H]
\centering
\begin{tabular}{|r|c|c|c|c|c|c|}
\hline
\textbf{Символ} & C & A & E & B & D & F \\ \hline
% \textbf{Вероятность} & $^5/_{14}$ & $^3/_{14}$ & $^3/_{14}$ & $^1/_{14}$ & $^1/_{14}$ & $^1/_{14}$ \\ \hline
\textbf{Код} & 11 & 10 & 011 & 010 & 001 & 000 \\ \hline
\end{tabular}
\end{table}

\noindentИсходная последовательность AAABCCCCCDEEEF кодируется следующей: $10.10.10.010.11.11.11.11.11.001.011.011.011.000 - 34$ бита.

Средняя длина кодового слова: 
$$
2 \times \frac{5}{14} + 2 \times \frac{3}{14} + 3 \times \frac{3}{14} + 3 \times \frac{1}{14} + 3 \times \frac{1}{14} + 3 \times \frac{1}{14}  = \frac{34}{14} \approx 2,4
$$

