\subsection{Порядок хранения байт в памяти}

Существует несколько способов хранения байт в памяти:

\begin{itemize}
    \item \textbf{От старшего к младшему} (англ. \emph{big-endian}): $A_n,\dots,A_0$ запись начинается со старшего и заканчивается младшим. Этот порядок является стандартным для протоколов TCP/IP, он используется в заголовках пакетов данных и во многих протоколах более высокого уровня, разработанных для использования поверх TCP/IP. Поэтому, порядок байтов от старшего к младшему часто называют сетевым порядком байтов.
    \item \textbf{От младшего к старшему} (англ. \emph{little-endian}): $A_0,\dots,A_n$ запись начинается с младшего и заканчивается старшим. Этот порядок записи принят в памяти персональных компьютеров с x86-процессорами, в связи с чем иногда его называют интеловский порядок байт (по названию фирмы-создателя архитектуры x86).
    \item \textbf{Переключаемый порядок} (англ. \emph{bi-endian}). Многие процессоры могут работать и в порядке от младшего к старшему, и в обратном. Обычно порядок байтов выбирается программно во время инициализации операционной системы, но может быть выбран и аппаратно перемычками на материнской плате. В этом случае правильнее говорить о порядке байтов операционной системы.
    \item \textbf{Смешанный порядок} (англ. \emph{middle-endian}) иногда используется при работе с числами, длина которых превышает машинное слово. Число представляется последовательностью машинных слов, которые записываются в формате, естественном для данной архитектуры, но сами слова следуют в обратном порядке.
\end{itemize}

\begin{figure}[H]
    \begin{subfigure}{0.45\textwidth}
       \centering
        \includegraphics{10.5-little-endian}
        \caption{Little-Endian}
        \label{fig:rs-trigger-nand}
    \end{subfigure}
    \begin{subfigure}{0.45\textwidth}
        \centering
        \includegraphics{10.5-big-endian}
        \caption{Big-Endian}
        \label{fig:big-endian}
    \end{subfigure}
    
    \caption{Сравнение порядков от младшего к старшему и от старшего к младшему}
    \label{fig:endians}
\end{figure}
