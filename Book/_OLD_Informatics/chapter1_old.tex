\section{Единицы измерения объема данных}

В области цифровой и вычислительной техники двоичная система счисления (основанная на степени двойки) получила широкое распространение. В следствие этого стали употреблять двоичные приставки:
\\ $1 kB = 2^{10} B$ - $1$ килобайт равен $2^{10}$ байт
\\ $1 MB = 2^{20} B$ - $1$ мегабайт равен $2^{20}$ байт

Однако, такая система противоречит СИ, которая использует десятичные приставки (основанные на степени десяти):
\\ $1 k = 10^{3}$, $1 M = 10^{6}$



Поначалу это противоречие не было существенной проблемой.
Число $2^{10} = 1024$ достаточно близко к тысяче и при объемах памяти, измерявшихся килобайтами, ошибка была всего в 2,4\%.
Но по мере развития технологий, разница между "двоичным" и "десятичным" гигабайтом была в 7\%.

Тогда IEEE, Институт инженеров электротехники и электроники (англ. Institute of Electrical and Electronics Engineers), утвердил стандарт IEEE 1541-2002.
For example,
\begin{center}
Рекомендации стандарта IEEE 1541-2002
\end{center}
\begin{itemize}
\item Бит (bit) (символ 'b') - двоичный знак;
\item Байт (byte) (символ 'B') - равен 8 битам ($1B = 8b$); %($\qty{1}{\byte} = \qty{8}{\bit}$);
\item Киби (kibi) (символ 'Ki') - $2^{10} = 1024$;
\item Меби (mebi) (символ 'Mi') - $2^{20} = 1048576$;
\item Гиби (gibi) (символ 'Gi') - $2^{30} = 1073741824$;
\item Теби (tebi) (символ 'Ti') - $2^{40} = 1099511627776$;
\end{itemize}

Сегодня в кибибайтах, мебибайтах и т.д. измеряется память - оперативная память, жесткие диски, flash-накопители. Однако, скорость передачи данных измеряется в килобайтах и мегабайтах в секунду (например 512 kbps (kilobits per second) - 512 килобит в секунду).
Операционные системы считают по-разному. *nix системы (unix, linux) используют "двоичные" приставки, в то время как Windows использует приставки СИ.
