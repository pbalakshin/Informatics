% === НАСТРОЙКА CLEVEREF ПОД ГОСТ ===

% --- Разделы документа ---
\crefformat{section}{\S#2#1#3}             % §1
\crefformat{subsection}{\S#2#1#3}          % §1.1
\crefformat{subsubsection}{\S#2#1#3}       % §1.1.1
\crefformat{paragraph}{\S#2#1#3}           % §1.1.1.1

% --- Приложения ---
\crefformat{appendix}{прил.~#2#1#3}        % прил.~A
\crefformat{subappendix}{прил.~#2#1#3}     % прил.~A.1

% --- Рисунки и Таблицы ---
\crefformat{figure}{рис.~#2#1#3}           % рис.~1
\crefformat{subfigure}{рис.~#2#1#3}        % рис.~1a
\crefformat{table}{табл.~#2#1#3}           % табл.~1

% --- Формулы и Листинги ---
\crefformat{equation}{#2(#1)#3}            % (1) — ГОСТ требует в скобках
\crefformat{listing}{лист.~#2#1#3}         % лист.~1
\crefformat{algorithm}{алг.~#2#1#3}        % алг.~1

% --- Диапазоны (множественное число) ---
\crefrangeformat{section}{\S\S#3#1#4--#5#2#6}
\crefrangeformat{figure}{рис.~#3#1#4--#5#2#6}
\crefrangeformat{table}{табл.~#3#1#4--#5#2#6}
\crefrangeformat{equation}{#3(#1)#4--#5(#2)#6}

% \crefname{section}{раздел}{разделы}
% \crefname{subsection}{подраздел}{подразделы}
% \crefname{figure}{рис.}{рис.}
% \crefname{table}{табл.}{табл.}
% \crefname{equation}{формула}{формулы}

% --- Заглавная буква (для начала предложения) ---
\Crefname{section}{Раздел}{Разделы}
\Crefname{subsection}{Подраздел}{Подразделы}
\Crefname{figure}{Рисунок}{Рисунки}
\Crefname{table}{Таблица}{Таблицы}
\Crefname{equation}{Формула}{Формулы}
\Crefname{listing}{Листинг}{Листинги}
\Crefname{appendix}{Приложение}{Приложения}
