% From: https://github.com/AndreyAkinshin/Russian-Phd-LaTeX-Dissertation-Template/blob/6160ccd1679ab0ac3cfca8956913971044a7107c/common/styles.tex
% Используем короткое тире (endash) для ненумерованных списков (ГОСТ 2.105-95, пункт 4.1.7, требует дефиса, но так лучше смотрится)
\renewcommand{\labelitemi}{\normalfont\bfseries{--}}

% Перечисление строчными буквами латинского алфавита (ГОСТ 2.105-95, 4.1.7)
%\renewcommand{\theenumi}{\alph{enumi}}
%\renewcommand{\labelenumi}{\theenumi)}

% Перечисление строчными буквами русского алфавита (ГОСТ 2.105-95, 4.1.7)
\makeatletter
\AddEnumerateCounter{\asbuk}{\russian@alph}{щ}      % Управляем списками/перечислениями через пакет enumitem, а он 'не знает' про asbuk, потому 'учим' его
\makeatother
%\renewcommand{\theenumi}{\asbuk{enumi}} %первый уровень нумерации
%\renewcommand{\labelenumi}{\theenumi)} %первый уровень нумерации
\renewcommand{\theenumii}{\asbuk{enumii}} %второй уровень нумерации
\renewcommand{\labelenumii}{\theenumii)} %второй уровень нумерации
\renewcommand{\theenumiii}{\arabic{enumiii}} %третий уровень нумерации
\renewcommand{\labelenumiii}{\theenumiii)} %третий уровень нумерации

\setlist{
  nosep,%                                    % Единый стиль для всех списков (пакет enumitem), без дополнительных интервалов.
	labelindent=\parindent,
  leftmargin=*%            % Каждый пункт, подпункт и перечисление записывают с абзацного отступа (ГОСТ 2.105-95, 4.1.8)
}
