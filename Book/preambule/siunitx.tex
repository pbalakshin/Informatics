
% Partially based on Russian-Phd-LaTeX-Dissertation-Template but using version 3+
\usepackage[
	locale 								= DE,               % Запятая для десятичных (3,14)
	% Table 14: Output options for numbers.
	exponent-product 			= {\cdot},          % Знак умножения (1,5·10³)
	group-separator 			= {\,},             % Тонкий пробел для группировки
	group-minimum-digits 	= 4,                % Группировать от 4 цифр
	uncertainty-mode 			= separate,
	% separate-uncertainty  = true,
	% Table 15: Output options for lists, products and ranges of numbers and quantities.
	list-separator	 			= {\text{, }},      % Запятая с пробелом
	list-final-separator 	= {\text{ и }},     % «и» перед последним элементом
	list-pair-separator 	= {\text{ и }},     % Для пар чисел
	list-units 						= single,
	range-units 					= single,
	% range-phrase            = {\text{\ensuremath{-}}},
	range-phrase 					= {\text{--}},
	% Table 19: Unit output options
	per-mode 							= symbol,
	fraction-command 			= \sfrac            % красивые дроби могут не соответствовать ГОСТ
]{siunitx}

% === Пользовательские команды для информатики ===
\newcommand{\bits}{\bit}
\newcommand{\bytes}{\byte}
\newcommand{\bps}{\bit\per\second}
\newcommand{\Bps}{\byte\per\second}
\newcommand{\Hz}{\hertz}

\newcommand{\B}{\byte}
\newcommand{\KB}{\kilo\byte}
\newcommand{\MB}{\mega\byte}
\newcommand{\GB}{\giga\byte}
\newcommand{\TB}{\tera\byte}

\newcommand{\kbps}{\kilo\bit\per\second}
\newcommand{\mbps}{\mega\bit\per\second}
\newcommand{\gbps}{\giga\bit\per\second}
\newcommand{\tbps}{\tera\bit\per\second}
