\documentclass[a5paper,14pt]{report}
\usepackage{amsmath}
\usepackage{amssymb}
\usepackage[english, russian]{babel}
%\usepackage[cp1251]{inputenc}
\usepackage[T1,T2A]{fontenc} %added
\usepackage[utf8]{inputenc}
\usepackage[dvips]{graphicx}
\graphicspath{{images/}}
\usepackage{indentfirst}
\usepackage{multicol}
\usepackage{hhline}
\usepackage{multirow}
\usepackage{textcomp}
\usepackage{wrapfig}
\usepackage{color,colortbl}
\definecolor{Gray1}{gray}{0.7}
\definecolor{Gray2}{gray}{0.6}
\definecolor{Gray3}{gray}{0.5}
\definecolor{Gray4}{gray}{0.4}
\usepackage{geometry}
\geometry{left=1.5cm}
\geometry{right=1.5cm}
\geometry{top=1cm}
\geometry{bottom=2cm}
\newcommand*{\hm}[1]{#1\nobreak\discretionary{}{\hbox{$\mathsurround=0pt #1$}}{}}


\begin{document}

\begin{titlepage}
Балакшин П.В., Соснин В.В.  Информатика. – СПб: Университет ИТМО, 2018. – 122 с.
\\\\\textbf{TO REVIEW AND REWRITE!!!!} В пособии излагаются основные понятия, необходимые для более глубокого изучения компьютерной техники и систем в будущем. Рассматриваются основные принципы построения, функционирования и организации памяти ЭВМ. Предлагается изучить несколько пакетов, в том числе систему компьютерной верстки TeX, и некоторые варианты распознавания того, что данные были переданы с ошибкой.
\\\\\\\\Рекомендовано к печати Ученым советом факультета программной инженерии и компьютерной техники.
\vspace{5 cm}
\begin{flushright}
\includegraphics[width=5cm]{ITMO_log}
\end{flushright}
Университет ИТМО – ведущий вуз России в области информационных и фотонных технологий, один из немногих российских вузов, получивших в 2009 году статус национального исследовательского университета. С 2013 года Университет ИТМО – участник программы повышения конкурентоспособности российских университетов среди ведущих мировых научно-образовательных центров, известной как проект «5 в 100». Цель Университета ИТМО – становление исследовательского университета мирового уровня, предпринимательского по типу, ориентированного на интернационализацию всех направлений деятельности.
\begin{flushright}
\copyright Университет ИТМО, 2018
\\\copyright Балакшин П.В., Соснин В.В., 2018
\end{flushright}
\end{titlepage}

\tableofcontents

\newpage
%\textbf{О курсе\\}
\input{0_about_the_course}

\newpage
%\textbf{Введение в информатику\\}
\input{1_introduction_to_informatics}

\newpage
%\textbf{Основы теории информации\\}
\input{2_information_theory}


%\chapter{Единицы измерения объема данных}
%\input {chapter1}
%
%\chapter{Системы счисления}
%\input {chapter2}
%
%\chapter{Арифметика в ограниченной разрядной сетке}
%\input {chapter3}
%
%\chapter{Теория информации}
%\input {chapter4}
%
%\chapter{Сжатие данных}
%\input {chapter5}
%
%\chapter{Помехоустойчивое кодирование}
%\input {chapter6}
%
%\chapter{Алгебра логики}
%\input {chapter7}
%
%\chapter{Программное обеспечение}
%\input {chapter8}
%
%\chapter{Структура и принципы функционирования компьютера}
%\input {chapter9}
%
%\chapter{Организация хранения данных в ЭВМ}
%\input {chapter10}
%
%\chapter{Передача данных в компьютерных сетях}
%\input {chapter11}

\end{document}
